%!TEX root = ms.tex
\beginsupplement
\appendix
\pagenumbering{arabic}
\begin{center}
\textbf{\large Supplemental Materials \\
Selection of Regression Models under Linear Restrictions for Fixed and Random Designs}

Sen Tian, Clifford M. Hurvich, Jeffrey S. Simonoff
\end{center}

This document provides theoretical details of the theorems and lemmas in the paper. The complete simulation results can be viewed online\footnote{https://github.com/sentian/RAICc/blob/master/paper/simuresults_reduced.pdf}.




\section{Proof of Lemma \ref{thm:components_ekl_lr_fixedx}}
\begin{proof}
As is well known \citeponline[see, e.g.,][p.~122]{greene2003econometric},
\begin{equation}
  n\hat\sigma^2 = \lVert y-X\hat{\beta} \rVert_2^2 \sim \sigma_0^2 \chi^2(n-p+m),
  \label{eq:dist_rss}
\end{equation}
and 
\begin{equation}
\begin{aligned}
E_y\left(\hat{\beta} - \beta_0 \right) &= 0,\\
\text{Cov}_y\left(\hat{\beta} - \beta_0 \right) &= E\left[\left(\hat{\beta} - \beta_0 \right)\left(\hat{\beta} - \beta_0 \right)^T \right]\\
&= \sigma_0^2 \left\{ (X^T X)^{-1} - (X^T X)^{-1}R^T\left[ R(X^T X)^{-1}R^T \right]^{-1} R(X^T X)^{-1} \right\}.
\end{aligned}
\label{eq:dist_betahat}
\end{equation}
From \eqref{eq:dist_rss}, $1/\hat{\sigma}^2$ follows an inverse $\chi^2$ distribution and we have
\begin{equation*}
n \sigma_0^2 E_y\left[ \frac{1}{\hat{\sigma}^2} \right] = n\frac{n}{n-p+m-2}.
\end{equation*}
From \eqref{eq:dist_betahat}, we have
\begin{equation*}
E_y  \left [ (\hat \beta-\beta_0)^T X^T X (\hat \beta-\beta_0) \right ] 
= \text{Tr} \left[ X^T X \cdot \text{Cov}_y \left( \hat{\beta} - \beta_0 \right) \right] \\
= \sigma_0^2 (p-m).
\end{equation*}
We next show that $\hat{\sigma}^2$ and $(\hat \beta-\beta_0)^T X^T X (\hat \beta-\beta_0)$ are independent. Define an idempotent matrix $H_R=(X^T X)^{-1} R^T \left[ R (X^T X)^{-1} R^T \right]^{-1} R$. Recall that another two idempotent matrices are defined as $H=X(X^T X)^{-1} X^T$ and $H_Q = X H_R (X^T X)^{-1} X^T$, respectively. We have
\begin{equation*}
\begin{aligned}
y-X\hat{\beta}^f &= (I-H)\epsilon,\\
X\hat{\beta}^f - X\hat{\beta} &= XH_R(\hat{\beta}^f - \beta_0) =H_Q \epsilon,\\
X\hat{\beta} - X\beta_0 &= X(I-H_R)(\hat{\beta}^f - \beta_0) = (H-H_Q) \epsilon,
\end{aligned}
\end{equation*}
where we use the fact that $\hat{\beta}^f - \beta_0 = (X^T X)^{-1}X^T \epsilon$. Also since $HH_Q=H_QH=H_Q$, any two of the three idempotent symmetric matrices $I-H$, $H_Q$ and $H-H_Q$ have product zero. Then by the Craig's Theorem \citeponline{craig1943note} on the independence of two quadratic forms in a normal vector, 
\begin{equation*}
n\hat{\sigma}^2 = \lVert y-X\hat{\beta} \rVert_2^2 = \lVert y-X\hat{\beta}^f \rVert_2^2 + \lVert X\hat{\beta}^f - X\hat{\beta} \rVert_2^2 = \epsilon^T (I-H) \epsilon + \epsilon^T H_Q \epsilon
\end{equation*}
and
\begin{equation*}
\lVert X\hat{\beta} - X\beta_0 \rVert_2^2 = \epsilon^T (H-H_Q) \epsilon
\end{equation*}
are independent.
\end{proof}


\section{Proof of Theorem \ref{thm:EoptF_KL}}
\begin{proof}
By using Lemma \ref{thm:components_ekl_lr_fixedx}, the expected KL discrepancy can be derived as
\begin{equation*}
\begin{aligned}
E_y [\text{ErrF}_\text{KL} ] 
&= E_y \left\{ n \log (2\pi \hat\sigma^2) + \frac{1}{\hat\sigma^2}  (\hat\beta-\beta_0)^T X^T X (\hat\beta-\beta_0) + \frac{n\sigma_0^2}{\hat\sigma^2} \right\} \\
&= E_y \left [ n\log (2\pi \hat \sigma^2)\right ] +  (p-m) \frac{n}{n-p+m-2} + n \frac{n}{n-p+m-2}\\
&= E_y \left [ n\log (2\pi \hat \sigma^2) \right ] +  n \frac{n+p-m}{n-p+m-2}.
\end{aligned}
\end{equation*}
Recall that
\begin{equation*}
\text{errF}_\text{KL} = n\log(2\pi \hat\sigma^2) + n.
\end{equation*}
The expected optimism is then
\begin{equation*}
E_y(\text{opF}_\text{KL}) = E_y[\text{ErrF}_\text{KL} ]  -  E_y[\text{errF}_\text{KL} ] = n \frac{n+p-m}{n-p+m-2} - n.
\end{equation*}
\end{proof}

\section{Proof of Theorem \ref{thm:EoptF_SE}}
\begin{proof}
Using the expression of $\hat\beta$ \eqref{eq:betahat_sigmahatsq} and the definitions of $H$ and $H_Q$, we have
\begin{equation*}
\hat{\mu} = X\hat{\beta} = (H-H_Q)y + X(X^T X)^{-1} R^T \left[ R(X^TX)^{-1}R^T  \right]^{-1}r,
\end{equation*}
where the second term on the RHS is deterministic. Denote $h_i$ and ${h_Q}_i$ as the $i$-th rows of $H$ and $H_Q$, respectively. We then have
\begin{equation*}
\text{Cov}_y \left(\hat{\mu}_i, y_i \right) = \text{Cov}_y \left[ (h_i-{h_Q}_i)y, y_i  \right] = \text{Cov}_y \left[ (H_{ii}-{H_Q}_{ii})y_i, y_i \right] = \sigma_0^2 (H_{ii}-{H_Q}_{ii}).
\end{equation*}
Therefore, the covariance penalty \eqref{eq:EoptF_SE} can be derived as
\begin{equation*}
E_y (\text{optF}_\text{SE}) = 2 \sum_{i=1}^n \text{Cov}_y (\hat\mu_i,y_i) = 2 \sigma_0^2 \text{Tr}(H-H_Q) = 2 \sigma_0^2 (p-m).
\end{equation*}
\end{proof}

\section{Proof of Lemma \ref{thm:components_ekl_lr_randomx}}
\begin{proof}
Since $x_i$ are iid $\mathcal{N}(0,\Sigma_0)$, $X^T X \sim \mathcal{W}(\Sigma_0, n)$ and $ (X^T X)^{-1} \sim \mathcal{W}^{-1} (\Sigma_0^{-1}, n)$, where $\mathcal{W}$ and $\mathcal{W}^{-1}$ denotes a Wishart and an inverse Wishart distribution with $n$ degrees of freedom, respectively. We have $E(X^T X) = n\Sigma_0$ and $E((X^T X)^{-1}) = \Sigma_0^{-1} / (n-p-1)$. Hence,
\begin{equation*}
  E \left[ \text{Tr}(\hat \Sigma^{-1}\Sigma_0) \right] = E \left[ \text{Tr} (n (X^T X)^{-1} \Sigma_0) \right] = n \text{Tr}\left[ E\left( (X^T X)^{-1} \right) \Sigma_0\right] = \frac{np}{n-p-1}.
\end{equation*}
Define $H_S = X(X^T X)^{-1}(I-H_R)^T \Sigma_0 (I-H_R) (X^T X)^{-1} X^T$. Conditionally on $X$, the random variable $\hat{\sigma}^2$ and 
\begin{equation*}
(\hat{\beta}-\beta_0)^T \Sigma_0 (\hat{\beta}-\beta_0) = \epsilon^T H_S \epsilon
\end{equation*}
are independent by Craig's Theorem, since $H_S$ is symmetric and $H_S(I-H+H_Q)=0$. 
\iffalse
We also have
\begin{equation*}
n\sigma_0^2 E_{X,y} \left[ \frac{1}{\hat\sigma^2} \right] = n\sigma_0^2 E_X\left[ E_y \left(\frac{1}{\hat\sigma^2}\big| X \right) \right] = n \frac{n}{n-p+m-2},
\end{equation*}
where the last equality we use the result in Lemma \ref{thm:components_ekl_lr_fixedx}. 
\fi
In order to calculate $\displaystyle E_{X,y}  \left [ (\hat \beta-\beta_0)^T \Sigma_0 (\hat \beta-\beta_0) \right ]$, we transform the original basis of the problem. Denote $\tilde{R} = \big(\begin{smallmatrix}
  R\\
  R^c
\end{smallmatrix}\big)$, a ($p \times p$) matrix, where the rows of $R^c$ span the orthogonal complement of the row space of $R$. Hence $\tilde{R}$ has full rank. The true model now becomes
\begin{equation*}
y = X \beta_0 + \epsilon = \tilde{X} \tilde{\beta}_0 + \epsilon,
\end{equation*}
with restrictions $\tilde{M} \tilde{\beta}_0 = r$ where $\tilde{X} = X \tilde{R}^T$, $\tilde{\beta}_0 = \tilde{R}^{T^{-1}} \beta_0$ and $\tilde{M} = R \tilde{R}^T$. The approximating model is
\begin{equation*}
y = X \beta + u = \tilde{X} \tilde{\beta} + u,
\end{equation*}
with restrictions $\tilde{M} \tilde{\beta} = r$ where $\tilde{\beta} = \tilde{R}^{T^{-1}} \beta$. Denote $\hat{\tilde{\beta}}^f = (\tilde{X}^T \tilde{X})^{-1}\tilde{X}^Ty$ as the OLS estimator in the regression of $y$ on $\tilde{X}$. The restricted MLE is then
\begin{equation*}
\hat{\tilde{\beta}} = \hat{\tilde{\beta}}^f - \left( \tilde{X}^T \tilde{X} \right)^{-1} \tilde{M}^T \left[ \tilde{M} (\tilde{X}^T \tilde{X})^{-1} \tilde{M}^T \right]^{-1} \left( \tilde{M} \hat{\tilde{\beta}}^f - r \right),
\end{equation*}
and it can be easily verified that $\hat{\tilde{\beta}} = \tilde{R}^{T^{-1}} \hat{\beta}$. Denote $\tilde{X}_m$ and $\tilde{X}_{p-m}$ as the matrices containing the first $m$ and last $p-m$ columns of $\tilde{X}$, respectively. Let $\hat{\tilde{\beta}}_{m}$ and $\hat{\tilde{\beta}}_{p-m}$ be column vectors consisting of the first $m$ and last $p-m$ entries in $\hat{\tilde{\beta}}$, respectively. Also let $\tilde{\beta}_{0,m}$ and $\tilde{\beta}_{0,p-m}$ be column vectors consisting of the first $m$ and last $p-m$ entries in $\tilde{\beta}_0$, respectively. By using the formula for the inverse of partitioned matrices and some algebra, it can be shown that (details are given in Supplemental Material Section \ref{sec:derivation_betahattilde})
\begin{equation}
\begin{aligned}
\hat{\tilde{\beta}}_m &= \tilde{r},\\
\hat{\tilde{\beta}}_{p-m} &= \left( \tilde{X}_{p-m}^T \tilde{X}_{p-m} \right)^{-1} \left( \tilde{X}_{p-m}^Ty - \tilde{X}_{p-m}^T\tilde{X}_{m}\tilde{r} \right),
\end{aligned}
\label{eq:betahat_tilde_partition}
\end{equation}
where $\tilde{r} = (R R^T)^{-1}r$. The restriction $\tilde{M}\tilde{\beta}_0=r$ results in $\tilde{\beta}_{0,m} = \tilde{r}$. We then have 
\begin{equation*}
\hat{\tilde{\beta}}_{p-m} - \tilde{\beta}_{0,p-m} = \left( \tilde{X}_{p-m}^T \tilde{X}_{p-m} \right)^{-1} \tilde{X}_{p-m} ^T \epsilon,
\end{equation*}
and therefore 
\begin{equation*}
\hat{\tilde{\beta}}_{p-m} - \tilde{\beta}_{0,p-m} \big | \tilde{X} \sim \mathcal{N} \left(0, \sigma_0^2 \left( \tilde{X}_{p-m}^T \tilde{X}_{p-m} \right)^{-1} \right).
\end{equation*}
We also note that $\tilde{X}_{p-m} = X {R^c}^T$, and hence the rows $\tilde{x}_{p-m, i}$ of $\tilde{X}_{p-m}$ are independent and satisfy $\tilde{x}_{p-m, i} \sim \mathcal{N} \left( 0, R^c \Sigma_0 {R^c}^T \right)$. We then have that $\left( \tilde{X}_{p-m}^T \tilde{X}_{p-m} \right)^{-1}$ follows the inverse Wishart distribution $W^{-1}( R^c \Sigma_0 {R^c}^T, n)$. The expectation of the quadratic form can be derived as
\begin{equation*}
\begin{aligned}
E_{X,y}  \left [ (\hat \beta-\beta_0)^T \Sigma_0 (\hat \beta-\beta_0) \right ] 
&=  E_{\tilde{X},y}  \left [ \left( \hat{\tilde{\beta}} - \tilde{\beta}_0 \right)^T \tilde{R} \Sigma_0 \tilde{R}^T \left( \hat{\tilde{\beta}} - \tilde{\beta}_0 \right) \right ]\\
&=  E_{\tilde{X}} \left\{ E  \left [ \left( \hat{\tilde{\beta}}_{p-m} - \tilde{\beta}_{0,p-m} \right)^T R^c \Sigma_0 {R^c}^T \left(\hat{\tilde{\beta}}_{p-m} - \tilde{\beta}_{0,p-m} \right) \Big| \tilde{X} \right ]  \right\}\\
&= \sigma_0^2 \text{Tr} \left\{ R^c \Sigma_0 {R^c}^T E \left[ \left( \tilde{X}_{p-m}^T \tilde{X}_{p-m} \right)^{-1} \right] \right\} \\
&= \sigma_0^2 \frac{p-m}{n-p+m-1}.
\end{aligned}
\end{equation*}

\end{proof}


\section{Proof of Theorem \ref{thm:EoptR_KL}}
\begin{proof}
The expected KL can be derived as
\begin{equation*}
\begin{aligned}
&E_{X,y} (\text{ErrR}_\text{KL}) \\
&= E_{X,y} \left [ n \log (2\pi \hat\sigma^2) + \frac{n}{\hat\sigma^2}  (\hat\beta-\beta_0)^T \Sigma_0 (\hat\beta-\beta_0) + \frac{n\sigma_0^2}{\hat\sigma^2} \right ] + E \left [np \log(2\pi) + n \log |\hat\Sigma| + n \text{Tr}(\hat\Sigma^{-1}\Sigma_{0}) \right] \\
&= E_{X,y} \left [ n \log (2\pi \hat\sigma^2) \right ] + E_X \left[E\left(\frac{n}{\hat\sigma^2} \big | X \right)  E\left((\hat\beta-\beta_0)^T \Sigma_0 (\hat\beta-\beta_0) \big| X \right) + E\left(\frac{n\sigma_0^2}{\hat\sigma^2} \big| X \right) \right] + \\
 & \qquad E \left [np \log(2\pi) + n \log |\hat\Sigma| + n \text{Tr}(\hat\Sigma^{-1}\Sigma_{0}) \right] \\
&= E_{X,y} \left [ n\log (2\pi \hat \sigma^2) \right ] +  n \frac{n(n-1)}{(n-p+m-2)(n-p+m-1)} + E \left [ n\log |\hat \Sigma| \right ] + np\log(2\pi) + n \frac{np}{n-p-1},
\end{aligned}
\end{equation*}
where the second equality is based on Lemma \ref{thm:components_ekl_lr_randomx} for the independence of $\hat\sigma^2$ and $(\hat\beta-\beta_0)^T \Sigma_0 (\hat\beta-\beta_0)$ conditionally on $X$, and in the last equality we use results from Lemma \ref{thm:components_ekl_lr_fixedx} and \ref{thm:components_ekl_lr_randomx}. Since the training error is
\begin{equation*}
\text{errR}_\text{KL} = -2\log L(\hat\beta,\hat\sigma^2,\hat\Sigma|X,y) = \left [ n \log (2\pi \hat\sigma^2) + n \right ] + \left [np \log(2\pi) + n \log |\hat\Sigma| + np \right ],
\end{equation*}
the expected optimism can be obtained as
\begin{equation*}
\begin{aligned}
E_{X,y}(\text{optR}_\text{KL}) 
&= E_{X,y}[\text{ErrR}_\text{KL} ]  -  E_{X,y}[\text{errR}_\text{KL} ] \\
&= n \frac{n(n-1)}{(n-p+m-2)(n-p+m-1)} + n \frac{np}{n-p-1} - n(p+1).
\end{aligned}
\end{equation*}
\end{proof}


\section{Proof of Theorem \ref{thm:EoptR_SE}}
\begin{proof}

We first note from Theorem \ref{thm:EoptF_SE} that
\begin{equation*}
E_{X,y}(\text{optF}_\text{SE}) = E \left[ E(\text{optF}_\text{SE} | X ) \right] = 2 \sigma_0^2(p-m).
\end{equation*}
Based on formula 6 and proposition 1 in \citetonline{rosset2020fixed}, the expected optimism can be decomposed into 
\begin{equation*}
E_{X,y}(\text{optR}_\text{SE}) = E_{X,y}(\text{optF}_\text{SE}) + B^+ + V^+ = 2 \sigma_0^2(p-m) +  B^+ + V^+,
\end{equation*}
where $B^+$ and $V^+$ are the excess bias and excess variance of the fit. In particular, the excessed bias is defined as
\begin{equation*}
B^+ = E_{X,X^{(n)}} \big\lVert E( X^{(n)} \hat{\beta} \big | X, X^{(n)} ) - X^{(n)} \beta_0 \big\rVert_2^2 - E_X \big\lVert E( X \hat{\beta} \big | X ) - X \beta_0 \big\rVert_2^2.
\end{equation*}
Because of our assumption that the true model satisfies the restrictions, it follows that $\hat{\beta}$ is unbiased, and hence it is easy to see that $B^+=0$. Next, $V^+$ is defined as
\begin{equation*}
V^+ = E_{X,X^{(n)}} \left\{ \text{Tr} \left[ \text{Cov}\left( X^{(n)}\hat{\beta} \big | X, X^{(n)} \right) \right] \right\} - E_X \left\{ \text{Tr} \left[ \text{Cov}\left( X\hat{\beta} \big | X \right)  \right] \right\}.
\end{equation*}
The second term on the RHS is
\begin{equation*}
E_X \left\{  \text{Tr} \left[ \text{Cov} \left( X\hat{\beta} \big | X \right)  \right] \right\} = E_X \left\{ \text{Tr} \left[ \text{Cov} \left( (H-H_Q)y \big | X \right)  \right] \right\} = E\left\{  \sigma_0^2 \text{Tr} \left( H-H_Q  \right) \right\}= \sigma_0^2 (p-m).
\end{equation*}
The first term on the RHS is
\begin{equation*}
\begin{aligned}
& E_{X,X^{(n)}} \text{Tr} \left[ \text{Cov}\left( X^{(n)}\hat{\beta} \big | X, X^{(n)} \right) \right] \\
&= E_{X,X^{(n)}} \text{Tr} \left[ \text{Cov} \left( X^{(n)} (\hat{\beta} -\beta_0) \big | X, X^{(n)} \right) \right] \\
&= \text{Tr} \left\{ E_X\left[ \text{Cov} \left( \hat{\beta} -\beta_0 \big | X \right) \right] E \left ({X^{(n)}}^T X^{(n)}) \right] \right\}\\
&= n E_X \left\{ \text{Tr} \left[ \Sigma_0 \text{Cov} \left( \hat{\beta} - \beta_0  \big | X \right) \right] \right\}\\
&= n E_X \left\{ E\left[ \left(\hat{\beta} - \beta_0 \right)^T \Sigma_0 \left(\hat{\beta} - \beta_0 \right) \big| X \right] \right\} \\
&= n E_{X,y} \left[ \left(\hat{\beta} - \beta_0 \right)^T \Sigma_0 \left(\hat{\beta} - \beta_0 \right) \right]\\
&= n \sigma_0^2 \frac{p-m}{n-p+m-1},
\end{aligned}
\end{equation*}
where in the third equality we use the independence and identical distribution of $X$ and $X_0$, and $E(X^T X) = n\Sigma_0$, while in the last equality we use the result in Lemma \ref{thm:components_ekl_lr_randomx}. Combining the results together, we have
\begin{equation*}
V^+ = n\sigma_0^2 \frac{p-m}{n-p+m-1} - \sigma_0^2 (p-m) = \sigma_0^2\frac{(p-m)(p-m+1)}{n-p+m-1},
\end{equation*}
and 
\begin{equation*}
E_{X,y}(\text{optR}_\text{SE}) = 2\sigma_0^2 (p-m) + \sigma_0^2\frac{(p-m)(p-m+1)}{n-p+m-1}.
\end{equation*}

\end{proof}

\section{Derivation of the expression of \texorpdfstring{$\hat{\tilde{\beta}}$}{} in \eqref{eq:betahat_tilde_partition} }
\label{sec:derivation_betahattilde}
Denote $\tilde{H}_m = \tilde{X}_m \left(\tilde{X}_m^T \tilde{X}_m \right)^{-1} \tilde{X}_m^T$ and $\tilde{H}_{p-m} = \tilde{X}_{p-m} \left(\tilde{X}_{p-m}^T \tilde{X}_{p-m}\right)^{-1} \tilde{X}_{p-m}^T$. Then the partitioned form of the matrix $\tilde{X}^T \tilde{X}$ is given by
\begin{equation*}
\begin{aligned}
& \left(\tilde{X}^T \tilde{X}\right)^{-1} = 
\begin{bmatrix}
\tilde{X}_m^T \tilde{X}_m & \tilde{X}_m^T \tilde{X}_{p-m} \\
\tilde{X}_{p-m}^T \tilde{X}_m & \tilde{X}_{p-m}^T \tilde{X}_{p-m}
\end{bmatrix}^{-1} \\
&=
\begin{bmatrix}
\left[\tilde{X}_m^T (I-\tilde{H}_{p-m}) \tilde{X}_m \right]^{-1} & - \left[\tilde{X}_m^T (I-\tilde{H}_{p-m}) \tilde{X}_m \right]^{-1} \tilde{X}_m^T \tilde{X}_{p-m} \left(\tilde{X}_{p-m}^T \tilde{X}_{p-m}\right)^{-1}\\
- \left[\tilde{X}_{p-m}^T \left(I-\tilde{H}_m\right) \tilde{X}_{p-m} \right]^{-1} \tilde{X}_{p-m}^T \tilde{X}_m \left(\tilde{X}_m^T \tilde{X}_m \right)^{-1} & \left[\tilde{X}_{p-m}^T \left(I-\tilde{H}_m\right) \tilde{X}_{p-m} \right]^{-1}
\end{bmatrix},
\end{aligned}
\end{equation*}
and the partitioned form of $\hat{\tilde{\beta}}^f$ is given by
\begin{equation*}
\begin{aligned}
\hat{\tilde{\beta}}^f &= \left(\tilde{X}^T \tilde{X}\right)^{-1} \tilde{X}^T y \\
&= 
\begin{bmatrix}
\left[\tilde{X}_m^T \left(I-\tilde{H}_{p-m}\right) \tilde{X}_m \right]^{-1} \left( \tilde{X}_m^T y - \tilde{X}_m^T \tilde{H}_{p-m} y \right) \\
\left[\tilde{X}_{p-m}^T \left(I-\tilde{H}_m\right) \tilde{X}_{p-m} \right]^{-1} \left( \tilde{X}_{p-m}^T y - \tilde{X}_{p-m}^T \tilde{H}_m y \right)
\end{bmatrix}.
\end{aligned}
\end{equation*}
We also have 
\begin{equation*}
\begin{aligned}
& \left[\tilde{X}_{p-m}^T \left(I-\tilde{H}_m\right) \tilde{X}_{p-m} \right]^{-1} \tilde{X}_{p-m}^T \tilde{H}_m \left(I_n - \tilde{H}_{p-m}\right)\tilde{X}_m \\
&= \left[\tilde{X}_{p-m}^T \left(I-\tilde{H}_m\right) \tilde{X}_{p-m} \right]^{-1} \left( \tilde{X}_{p-m}^T \tilde{X}_m - \tilde{X}_{p-m}^T \tilde{H}_m\tilde{H}_{p-m}\tilde{X}_m \right)  \\
&= \left[\tilde{X}_{p-m}^T \left(I-\tilde{H}_m\right) \tilde{X}_{p-m} \right]^{-1} \tilde{X}_{p-m}^T \left(I-\tilde{H}_m\right) \tilde{X}_{p-m} \left(\tilde{X}_{p-m}^T \tilde{X}_{p-m} \right)^{-1} \tilde{X}_{p-m}^T \tilde{X}_m  \\
&= \left(\tilde{X}_{p-m}^T \tilde{X}_{p-m} \right)^{-1} \tilde{X}_{p-m}^T \tilde{X}_m.
\end{aligned}
\end{equation*}
Using this property and $\tilde{M} = R \tilde{R}^T = \big(\begin{smallmatrix}
  RR^T & 0
\end{smallmatrix}\big)$, we have
\begin{equation*}
\begin{aligned}
\left(\tilde{X}^T \tilde{X}\right)^{-1} \tilde{M}^T \left[ \tilde{M} \left( \tilde{X}^T \tilde{X} \right)^{-1} \tilde{M}^T \right]^{-1} &= 
\begin{bmatrix}
(RR^T)^{-1} \\
-\left(\tilde{X}_{p-m}^T \tilde{X}_{p-m} \right)^{-1} \tilde{X}_{p-m}^T \tilde{X}_m (RR^T)^{-1}
\end{bmatrix},
 \\ 
I_p - \left(\tilde{X}^T \tilde{X}\right)^{-1} \tilde{M}^T \left[ \tilde{M} \left( \tilde{X}^T \tilde{X} \right)^{-1} \tilde{M}^T \right]^{-1} \tilde{M} &= 
\begin{bmatrix}
0 & 0\\
\left(\tilde{X}_{p-m}^T \tilde{X}_{p-m} \right)^{-1} \tilde{X}_{p-m}^T \tilde{X}_m & I_{p-m}
\end{bmatrix}.
\end{aligned}
\end{equation*}
Therefore, \eqref{eq:betahat_tilde_partition} can be derived as
\begin{equation*}
\begin{aligned}
\hat{\tilde{\beta}} &= \left\{I_p - \left(\tilde{X}^T \tilde{X}\right)^{-1} \tilde{M}^T \left[ \tilde{M} \left( \tilde{X}^T \tilde{X} \right)^{-1} \tilde{M}^T \right]^{-1} \tilde{M} \right\}\hat{\tilde{\beta}}^f + \left\{\left(\tilde{X}^T \tilde{X}\right)^{-1} \tilde{M}^T \left[ \tilde{M} \left( \tilde{X}^T \tilde{X} \right)^{-1} \tilde{M}^T \right]^{-1} \right\}r\\
&=
\begin{bmatrix}
\tilde{r}\\
\left(\tilde{X}_{p-m}^T \tilde{X}_{p-m} \right)^{-1} \tilde{X}_{p-m}^T \left(y - \tilde{X}_m \tilde{r}\right)
\end{bmatrix}.
\end{aligned}
\end{equation*}


\iffalse
\section{\texorpdfstring{$\hat{\beta}$}{} for \texorpdfstring{$R=[0, I_{p-k}]$}{} }
We show that for $R=[0, I_{p-k}]$, $\hat{\beta}$ \eqref{eq:betahat} is the least squares coefficient vector of $y$ upon $X_k$ where $X_k$ involves the first $k$ columns of $X$. 

We simplify the notations. Denote $R=[0, I_2]$ where $0$ is an $p_2 \times p_1$ matrix and $I_2$ is an $p_2 \times p_2$ matrix, respectively. Also denote $X=[X_1, X_2]$ where $X_1$ is an $n \times p_1$ matrix and $X_2$ is an $n \times p_2$ matrix, respectively. The corresponding hat matrices are $H_1 = X_1 (X_1^T X_1)^{-1} X_1^T$ and $H_2 = X_2 (X_2^T X_2)^{-1} X_2^T$. 

First note that the partitioned form of the matrix $X^T X$ is given by
\begin{equation}
\begin{aligned}
(X^T X)^{-1} &= 
\begin{bmatrix}
X_1^T X_1 & X_1^T X_2 \\
X_2^T X_1 & X_2^T X_2
\end{bmatrix}^{-1} \\
&=
\begin{bmatrix}
[X_1^T (I-H_2) X_1 ]^{-1} & - [X_1^T (I-H_2) X_1 ]^{-1} X_1^T X_2 (X_2^T X_2)^{-1}\\
- [X_2^T (I-H_1) X_2 ]^{-1} X_2^T X_1 (X_1^T X_1)^{-1} & [X_2^T (I-H_1) X_2 ]^{-1}
\end{bmatrix}.
\end{aligned}
\end{equation}
Then we have
\begin{equation*}
I - H_R=
I - (X^T X)^{-1}(R^T X_2^T (I-H_1) X_2 R) =
\begin{bmatrix}
I_1 & [X_1^T (I-H_2) X_1 ]^{-1} X_1^T H_2 (I-H_1) X_2 \\
0 & 0
\end{bmatrix},
\end{equation*}
and
\begin{equation*}
z = (X^T X)^{-1} X^T y = 
\begin{bmatrix}
[X_1^T (I-H_2) X_1 ]^{-1} (X_1^T y - X_1^T H_2 y)\\
[X_2^T (I-H_1) X_2 ]^{-1} (X_2^T y - X_2^T H_1 y)
\end{bmatrix}.
\end{equation*}
Also it can be easily verified that $HH_1=H_1H=H_1$, $HH_2=H_2H=H_2$, and
\begin{equation*}
(I-H_1)X_2 [X_2^T (I-H_1) X_2 ]^{-1} X_2^T (I-H_1) = H - H_1.
\end{equation*}
Therefore, we have
\begin{equation*}
\begin{aligned}
  \hat{\beta} &= z + (X^T X)^{-1} R^T ( R(X^T X)^{-1} R^T)^{-1} (r-R z) \\
  &= \left[ I
   - (X^T X)^{-1}(R^T X_2^T (I-H_1) X_2 R) \right] z\\
  &= [X_1^T (I-H_2) X_1 ]^{-1} X_1^T \left\{ I-H_2+H_2(I-H_1)X_2 [X_2^T (I-H_1) X_2 ]^{-1} X_2^T (I-H_1)  \right\}y \\
  &= [X_1^T (I-H_2) X_1 ]^{-1} X_1^T \left\{ I-H_2H_1 \right\}y\\
  &= (X_1^T X_1)^{-1} X_1^T y,
\end{aligned}
\end{equation*}
which is the least squares estimates on $X_1$.

\section{Derivation of \texorpdfstring{\eqref{eq:firstterm_ekl_lr}}{} for \texorpdfstring{$R=[0, I_{p-k}]$}{}}
First note that
\begin{equation*}
\begin{aligned}
&[X_1^T (I-H_2) X_1 ]^{-1} - [X_1^T (I-H_2) X_1 ]^{-1} X_1^T H_2 (I-H_1) X_2 [X_2^T (I-H_1) X_2 ]^{-1} X_2^T X_1 (X_1^T X_1)^{-1}\\
&= [X_1^T (I-H_2) X_1 ]^{-1} \left( I - X_1^T H_2 X_1 (X_1^T X_1)^{-1} \right),\\
&= (X_1^T X_1)^{-1} 
\end{aligned}
\end{equation*}
Then from the partition of $(X^T X)^{-1}$ and the expression of $I-H_R$ shown in the above Section, it is easy to have
\begin{equation*}
(I-H_R) (X^T X)^{-1} = 
\begin{bmatrix}
(X_1^T X_1)^{-1}  & 0\\
0 & 0
\end{bmatrix}.
\end{equation*}
Therefore,
\begin{equation*}
\begin{aligned}
E_0  \left [ (\hat \beta-\beta_0)^T \Sigma_0 (\hat \beta-\beta_0) \right ] &= E_0 \left\{ E_0 \left [ (\hat \beta-\beta_0)^T \Sigma_0 (\hat \beta-\beta_0) \Big| X \right ]  \right\},\\
&= \sigma_0^2 \text{Tr} \left[ \Sigma_0 E_0 \left\{ (I-H_R) (X^T X)^{-1} \right\} \right], \\
&= \sigma_0^2 \frac{p_1}{n-p_1-1}.
\end{aligned}
\end{equation*}

\section{Cochrane's Theorem}
Let $\eta \sim \mathcal{N}(0, I)$ and let $H=\sum H_i$ be a sum of $k$ symmetric matrices with $\text{rank}(H)$=r and $\text{rank}(H_i)=r_i$ such that $H_i=H_i^2$ and $H_iH_j=0$ when $i \ne j$. Then $\eta^T H_i \eta \sim \chi^2(r_i)$, $i=1,\cdots,k$ are independent chi-square variates such that $\sum \eta^T H_i \eta = \eta^T H \eta \sim \chi^2(r)$ with $r=\sum r_i$.




\section{Using Adkin's result to derive \texorpdfstring{\eqref{eq:thirdterm_ekl_lr_randomx}}{} for any \texorpdfstring{$R$}{}}

\textcolor{red}{Unfinished}

The problem we have is 
\begin{equation}
y = X\beta_0 + \epsilon, \quad \text{s.t. } R\beta_0=r,
\end{equation}
where $R$ is a $m$ by $p$ matrix. From the eigen-decomposition of $X^T X$, we have $X^T X = P \Lambda P^T$. Denote $V = P \Lambda^{1/2} P^T$ and $V^{-1} = P \Lambda^{-1/2} P^T$. Also denote $H_1 = R V^{-1}$, and an orthogonal matrix $Q$ that contains the characteristic vectors of $H_1 ^T (H_1 H_1^T)^{-1} H_1$. Hence, $Z = XV^{-1}Q$ has orthonormal columns. Adkins shows that the problem can be transformed to
\begin{equation}
y = Z\theta_0 + \epsilon, \quad \text{s.t. } [I_m \quad 0]\theta_0=h,
\end{equation}
where $h$ is some constant. 

Denote $\hat{\theta}^f = Z^T y$ as the OLS estimator, 
\begin{equation}
\begin{aligned}
( \hat{\beta} - \beta_0 )^T \Sigma_0 ( \hat{\beta} - \beta_0 ) 
&= ( \hat{\theta} - \theta_0 )^T Q^T V^{-1} \Sigma_0 V^{-1} Q ( \hat{\theta} - \theta_0 ) \\
&= ( \hat{\theta}^f_{p-m} - \theta_{0,p-m} )^T [0 \quad I_{p-m}] Q^T V^{-1} \Sigma_0 V^{-1} Q [0 \quad I_{p-m}]^T ( \hat{\theta}^f_{p-m} - \theta_{0,p-m} ),
\end{aligned}
\end{equation}
where $\hat{\theta}^f_{p-m}$ is the last $p-m$ elements of $\hat{\theta}^f$ and $\theta_{0,p-m}$ is the last $p-m$ elements of $\theta_0$. Since $\hat{\theta}^f_{p-m} - \theta_{0,p-m} \sim \mathcal{N}(0, \sigma_0^2 I_{p-m})$, we have
\begin{equation}
\begin{aligned}
E_0 \left\{ ( \hat{\beta} - \beta_0 )^T \Sigma_0 ( \hat{\beta} - \beta_0 )  \right\} 
&= E_0 \left\{ E_0 \left[ ( \hat{\beta} - \beta_0 )^T \Sigma_0 ( \hat{\beta} - \beta_0 ) | X \right]  \right\},\\
&= \sigma_0^2 E_0\left\{ \text{Tr}\left( [0 \quad I_{p-m}] Q^T V^{-1} \Sigma_0 V^{-1} Q [0 \quad I_{p-m}]^T \right) \right\}
\end{aligned}
\end{equation}
Here $Q$ and $V$ are random. I verify in simulations that the matrix $E_0 ([0 \quad I_{p-m}] Q^T V^{-1} \Sigma_0 V^{-1} Q [0 \quad I_{p-m}]^T)$ is a diagonal matrix.

\fi

\bibliographystyleonline{chicago}
\bibliographyonline{raicc.bib}

\iffalse
% latex table generated in R 4.0.2 by xtable 1.8-4 package
% Fri Jun 26 10:43:48 2020
\begin{table}[ht]
\centering
\caption{Random X, Sparse-Ex1, n=40} 
\scalebox{0.52}{
\begin{tabular}{|c|c|c|cc|cc|cc|ccc|c||cc|cc|cc|ccc|c|}
  \toprule 
 \multicolumn{1}{|c}{} & \multicolumn{1}{c}{} &       & RAICc & AICc  & RC$_p$ & C$_p$ & 10F CV & LOO CV & S$_p$ & $\widehat{\text{C}}_p$ & GCV   & BIC   & RAICc & AICc  & RC$_p$ & C$_p$ & 10F CV & LOO CV & S$_p$ & $\widehat{\text{C}}_p$ & GCV   & BIC  \\
 \cmidrule{4-23}\multicolumn{1}{|c}{} & \multicolumn{1}{c}{} &       & \multicolumn{10}{c||}{RMSE}                                                   & \multicolumn{10}{c|}{Relative Risk}       \\
 \midrule\multirow{9}[6]{*}{hsnr} & \multirow{3}[2]{*}{$\rho=0$} & p=12 & 0.82 & 0.826 & 0.835 & 0.863 & 0.869 & 0.857 & 0.857 & 0.885 & 0.859 & 0.832 & 0.027 & 0.028 & 0.029 & 0.032 & 0.032 & 0.031 & 0.031 & 0.034 & 0.031 & 0.028 \\ 
   &  & p=20 & 0.82 & 0.828 & 0.848 & 0.916 & 0.9 & 0.887 & 0.882 & 0.987 & 0.885 & 0.84 & 0.027 & 0.028 & 0.031 & 0.041 & 0.037 & 0.036 & 0.035 & 0.052 & 0.035 & 0.029 \\ 
   &  & p=36 & 0.82 & 0.828 & 0.95 & 1.154 & 0.912 & 0.923 & 0.925 & 1.38 & 0.937 & 1.132 & 0.027 & 0.028 & 0.061 & 0.328 & 0.04 & 0.07 & 0.075 & 0.515 & 0.088 & 0.395 \\ 
  \cmidrule{2-23} & \multirow{3}[2]{*}{$\rho=0.5$} & p=12 & 1.27 & 1.279 & 1.288 & 1.318 & 1.319 & 1.309 & 1.304 & 1.339 & 1.306 & 1.296 & 0.028 & 0.029 & 0.03 & 0.033 & 0.032 & 0.032 & 0.032 & 0.034 & 0.032 & 0.03 \\ 
   &  & p=20 & 1.27 & 1.281 & 1.306 & 1.395 & 1.363 & 1.349 & 1.348 & 1.485 & 1.352 & 1.306 & 0.028 & 0.03 & 0.032 & 0.043 & 0.037 & 0.037 & 0.037 & 0.052 & 0.037 & 0.031 \\ 
   &  & p=36 & 1.27 & 1.281 & 1.456 & 1.744 & 1.375 & 1.399 & 1.4 & 2.058 & 1.422 & 1.721 & 0.028 & 0.03 & 0.061 & 0.324 & 0.04 & 0.072 & 0.074 & 0.511 & 0.087 & 0.382 \\ 
  \cmidrule{2-23} & \multirow{3}[2]{*}{$\rho=0.9$} & p=12 & 1.73 & 1.729 & 1.746 & 1.782 & 1.788 & 1.771 & 1.759 & 1.817 & 1.763 & 1.789 & 0.026 & 0.027 & 0.028 & 0.03 & 0.029 & 0.029 & 0.029 & 0.031 & 0.029 & 0.028 \\ 
   &  & p=20 & 1.73 & 1.726 & 1.778 & 1.88 & 1.841 & 1.833 & 1.817 & 1.996 & 1.828 & 1.799 & 0.026 & 0.026 & 0.03 & 0.038 & 0.034 & 0.034 & 0.033 & 0.047 & 0.034 & 0.029 \\ 
   &  & p=36 & 1.727 & 1.726 & 2.005 & 2.41 & 1.855 & 1.892 & 1.9 & 2.8 & 1.929 & 2.344 & 0.026 & 0.027 & 0.059 & 0.323 & 0.037 & 0.068 & 0.077 & 0.504 & 0.091 & 0.35 \\ 
  \midrule\multirow{9}[6]{*}{msnr} & \multirow{3}[2]{*}{$\rho=0$} & p=12 & 2.596 & 2.564 & 2.554 & 2.583 & 2.601 & 2.583 & 2.569 & 2.631 & 2.571 & 2.783 & 0.256 & 0.253 & 0.253 & 0.27 & 0.268 & 0.266 & 0.263 & 0.285 & 0.264 & 0.296 \\ 
   &  & p=20 & 2.596 & 2.566 & 2.596 & 2.717 & 2.669 & 2.648 & 2.625 & 2.888 & 2.637 & 2.797 & 0.256 & 0.253 & 0.271 & 0.336 & 0.299 & 0.297 & 0.29 & 0.426 & 0.296 & 0.302 \\ 
   &  & p=36 & 2.596 & 2.566 & 2.917 & 3.472 & 2.696 & 2.738 & 2.734 & 3.973 & 2.78 & 3.467 & 0.256 & 0.253 & 0.532 & 2.791 & 0.325 & 0.572 & 0.608 & 4.217 & 0.726 & 2.889 \\ 
  \cmidrule{2-23} & \multirow{3}[2]{*}{$\rho=0.5$} & p=12 & 3.688 & 3.672 & 3.658 & 3.736 & 3.739 & 3.725 & 3.696 & 3.787 & 3.692 & 3.962 & 0.227 & 0.23 & 0.232 & 0.253 & 0.248 & 0.25 & 0.243 & 0.266 & 0.243 & 0.263 \\ 
   &  & p=20 & 3.687 & 3.672 & 3.728 & 3.944 & 3.839 & 3.815 & 3.782 & 4.173 & 3.795 & 3.976 & 0.226 & 0.231 & 0.249 & 0.323 & 0.277 & 0.281 & 0.271 & 0.401 & 0.275 & 0.268 \\ 
   &  & p=36 & 3.687 & 3.672 & 4.226 & 5.029 & 3.874 & 3.966 & 3.926 & 5.808 & 3.991 & 4.934 & 0.226 & 0.231 & 0.502 & 2.711 & 0.301 & 0.58 & 0.576 & 4.155 & 0.682 & 2.715 \\ 
  \cmidrule{2-23} & \multirow{3}[2]{*}{$\rho=0.9$} & p=12 & 4.56 & 4.584 & 4.655 & 4.774 & 4.73 & 4.727 & 4.707 & 4.863 & 4.71 & 4.786 & 0.167 & 0.173 & 0.185 & 0.203 & 0.194 & 0.197 & 0.195 & 0.215 & 0.195 & 0.178 \\ 
   &  & p=20 & 4.561 & 4.592 & 4.744 & 5.097 & 4.842 & 4.888 & 4.841 & 5.34 & 4.864 & 4.78 & 0.168 & 0.174 & 0.2 & 0.274 & 0.219 & 0.229 & 0.224 & 0.331 & 0.229 & 0.178 \\ 
   &  & p=36 & 4.558 & 4.589 & 5.449 & 6.647 & 4.893 & 5.055 & 5.045 & 7.736 & 5.136 & 6.154 & 0.167 & 0.173 & 0.453 & 2.642 & 0.241 & 0.505 & 0.537 & 3.994 & 0.686 & 2.521 \\ 
  \midrule\multirow{9}[6]{*}{lsnr} & \multirow{3}[2]{*}{$\rho=0$} & p=12 & 5.119 & 5.14 & 5.168 & 5.263 & 5.24 & 5.207 & 5.178 & 5.326 & 5.196 & 5.471 & 0.901 & 0.938 & 0.963 & 1.045 & 1.007 & 1.009 & 0.988 & 1.097 & 0.997 & 0.975 \\ 
   &  & p=20 & 5.119 & 5.14 & 5.275 & 5.546 & 5.341 & 5.333 & 5.268 & 5.815 & 5.303 & 5.471 & 0.901 & 0.938 & 1.044 & 1.352 & 1.117 & 1.141 & 1.071 & 1.65 & 1.102 & 0.975 \\ 
   &  & p=36 & 5.119 & 5.14 & 5.988 & 7.279 & 5.394 & 5.514 & 5.483 & 8.329 & 5.593 & 6.851 & 0.901 & 0.938 & 2.343 & 13.386 & 1.224 & 2.454 & 2.498 & 19.618 & 2.99 & 11.93 \\ 
  \cmidrule{2-23} & \multirow{3}[2]{*}{$\rho=0.5$} & p=12 & 7.24 & 7.316 & 7.382 & 7.579 & 7.47 & 7.452 & 7.423 & 7.659 & 7.426 & 7.8 & 0.812 & 0.851 & 0.895 & 0.983 & 0.933 & 0.931 & 0.92 & 1.039 & 0.924 & 0.895 \\ 
   &  & p=20 & 7.24 & 7.316 & 7.518 & 8.033 & 7.632 & 7.651 & 7.607 & 8.372 & 7.623 & 7.793 & 0.812 & 0.852 & 0.958 & 1.307 & 1.039 & 1.069 & 1.036 & 1.57 & 1.05 & 0.891 \\ 
   &  & p=36 & 7.24 & 7.316 & 8.663 & 10.514 & 7.735 & 7.955 & 7.925 & 12.27 & 8.067 & 9.874 & 0.812 & 0.852 & 2.24 & 12.931 & 1.176 & 2.453 & 2.498 & 19.762 & 3.011 & 11.733 \\ 
  \cmidrule{2-23} & \multirow{3}[2]{*}{$\rho=0.9$} & p=12 & 8.023 & 8.144 & 8.462 & 8.79 & 8.797 & 8.788 & 8.582 & 9.065 & 8.607 & 8.418 & 0.545 & 0.57 & 0.646 & 0.742 & 0.719 & 0.723 & 0.686 & 0.809 & 0.691 & 0.575 \\ 
   &  & p=20 & 8.079 & 8.232 & 8.59 & 9.556 & 9.002 & 9.059 & 8.849 & 10.136 & 8.899 & 8.455 & 0.553 & 0.588 & 0.705 & 1.084 & 0.817 & 0.845 & 0.8 & 1.325 & 0.821 & 0.583 \\ 
   &  & p=36 & 8.055 & 8.231 & 10.605 & 13.436 & 9.153 & 9.427 & 9.302 & 15.9 & 9.621 & 11.652 & 0.55 & 0.586 & 2.035 & 12.898 & 0.941 & 2.172 & 2.235 & 19.216 & 3.029 & 11.282 \\ 
   \midrule 
 \multicolumn{1}{|c}{} & \multicolumn{1}{c}{} &       & \multicolumn{10}{c||}{Relative Test Error}                                    & \multicolumn{10}{c|}{Proportion of Variance Explained} \\
\midrule\multirow{9}[6]{*}{hsnr} & \multirow{3}[2]{*}{$\rho=0$} & p=12 & 1.233 & 1.239 & 1.248 & 1.273 & 1.273 & 1.264 & 1.267 & 1.29 & 1.268 & 1.243 & 0.871 & 0.871 & 0.87 & 0.868 & 0.867 & 0.868 & 0.868 & 0.866 & 0.868 & 0.87 \\ 
   &  & p=20 & 1.233 & 1.241 & 1.262 & 1.351 & 1.312 & 1.304 & 1.301 & 1.444 & 1.304 & 1.253 & 0.871 & 0.871 & 0.868 & 0.859 & 0.863 & 0.864 & 0.864 & 0.849 & 0.864 & 0.869 \\ 
   &  & p=36 & 1.233 & 1.241 & 1.514 & 3.866 & 1.344 & 1.587 & 1.636 & 5.49 & 1.734 & 4.429 & 0.871 & 0.871 & 0.842 & 0.602 & 0.86 & 0.833 & 0.829 & 0.435 & 0.818 & 0.543 \\ 
  \cmidrule{2-23} & \multirow{3}[2]{*}{$\rho=0.5$} & p=12 & 1.245 & 1.255 & 1.261 & 1.282 & 1.277 & 1.276 & 1.273 & 1.297 & 1.274 & 1.26 & 0.87 & 0.869 & 0.868 & 0.866 & 0.867 & 0.867 & 0.867 & 0.865 & 0.867 & 0.869 \\ 
   &  & p=20 & 1.245 & 1.256 & 1.275 & 1.363 & 1.315 & 1.315 & 1.313 & 1.445 & 1.316 & 1.269 & 0.87 & 0.869 & 0.867 & 0.858 & 0.863 & 0.863 & 0.863 & 0.849 & 0.863 & 0.868 \\ 
   &  & p=36 & 1.245 & 1.256 & 1.519 & 3.822 & 1.344 & 1.601 & 1.629 & 5.44 & 1.739 & 4.309 & 0.87 & 0.869 & 0.841 & 0.606 & 0.86 & 0.832 & 0.829 & 0.439 & 0.818 & 0.554 \\ 
  \cmidrule{2-23} & \multirow{3}[2]{*}{$\rho=0.9$} & p=12 & 1.226 & 1.23 & 1.239 & 1.257 & 1.252 & 1.25 & 1.247 & 1.27 & 1.249 & 1.244 & 0.872 & 0.871 & 0.871 & 0.869 & 0.869 & 0.869 & 0.87 & 0.867 & 0.87 & 0.87 \\ 
   &  & p=20 & 1.227 & 1.229 & 1.255 & 1.327 & 1.287 & 1.288 & 1.286 & 1.4 & 1.292 & 1.25 & 0.872 & 0.872 & 0.869 & 0.861 & 0.865 & 0.865 & 0.866 & 0.853 & 0.865 & 0.87 \\ 
   &  & p=36 & 1.225 & 1.229 & 1.498 & 3.76 & 1.317 & 1.559 & 1.631 & 5.327 & 1.747 & 3.985 & 0.872 & 0.872 & 0.843 & 0.607 & 0.863 & 0.835 & 0.827 & 0.446 & 0.815 & 0.583 \\ 
  \midrule\multirow{9}[6]{*}{msnr} & \multirow{3}[2]{*}{$\rho=0$} & p=12 & 1.262 & 1.258 & 1.258 & 1.276 & 1.273 & 1.271 & 1.269 & 1.292 & 1.27 & 1.303 & 0.379 & 0.381 & 0.381 & 0.372 & 0.374 & 0.375 & 0.376 & 0.365 & 0.375 & 0.359 \\ 
   &  & p=20 & 1.262 & 1.259 & 1.277 & 1.34 & 1.302 & 1.302 & 1.296 & 1.432 & 1.301 & 1.309 & 0.379 & 0.381 & 0.372 & 0.34 & 0.359 & 0.359 & 0.362 & 0.295 & 0.36 & 0.356 \\ 
   &  & p=36 & 1.262 & 1.259 & 1.533 & 3.872 & 1.33 & 1.566 & 1.61 & 5.326 & 1.719 & 3.961 & 0.379 & 0.381 & 0.244 & -0.894 & 0.345 & 0.226 & 0.206 & -1.606 & 0.15 & -0.941 \\ 
  \cmidrule{2-23} & \multirow{3}[2]{*}{$\rho=0.5$} & p=12 & 1.231 & 1.235 & 1.237 & 1.258 & 1.252 & 1.255 & 1.248 & 1.272 & 1.247 & 1.269 & 0.394 & 0.392 & 0.391 & 0.38 & 0.383 & 0.382 & 0.386 & 0.374 & 0.386 & 0.375 \\ 
   &  & p=20 & 1.231 & 1.235 & 1.255 & 1.326 & 1.28 & 1.284 & 1.275 & 1.405 & 1.279 & 1.274 & 0.394 & 0.392 & 0.382 & 0.346 & 0.369 & 0.367 & 0.372 & 0.307 & 0.37 & 0.373 \\ 
   &  & p=36 & 1.231 & 1.235 & 1.506 & 3.771 & 1.308 & 1.571 & 1.576 & 5.248 & 1.683 & 3.776 & 0.394 & 0.392 & 0.258 & -0.851 & 0.356 & 0.222 & 0.222 & -1.573 & 0.169 & -0.853 \\ 
  \cmidrule{2-23} & \multirow{3}[2]{*}{$\rho=0.9$} & p=12 & 1.17 & 1.177 & 1.189 & 1.206 & 1.197 & 1.2 & 1.198 & 1.219 & 1.198 & 1.183 & 0.423 & 0.42 & 0.414 & 0.405 & 0.409 & 0.408 & 0.409 & 0.399 & 0.409 & 0.417 \\ 
   &  & p=20 & 1.172 & 1.178 & 1.205 & 1.277 & 1.222 & 1.232 & 1.228 & 1.334 & 1.233 & 1.183 & 0.423 & 0.419 & 0.406 & 0.37 & 0.397 & 0.392 & 0.395 & 0.342 & 0.392 & 0.417 \\ 
   &  & p=36 & 1.171 & 1.178 & 1.453 & 3.66 & 1.249 & 1.492 & 1.529 & 5.022 & 1.663 & 3.526 & 0.423 & 0.419 & 0.282 & -0.808 & 0.385 & 0.259 & 0.242 & -1.478 & 0.172 & -0.745 \\ 
  \midrule\multirow{9}[6]{*}{lsnr} & \multirow{3}[2]{*}{$\rho=0$} & p=12 & 1.187 & 1.195 & 1.2 & 1.216 & 1.209 & 1.21 & 1.205 & 1.227 & 1.207 & 1.204 & 0.019 & 0.013 & 0.009 & -0.005 & 0.001 & 0.001 & 0.005 & -0.014 & 0.003 & 0.005 \\ 
   &  & p=20 & 1.187 & 1.195 & 1.217 & 1.277 & 1.23 & 1.235 & 1.222 & 1.336 & 1.228 & 1.204 & 0.019 & 0.013 & -0.005 & -0.055 & -0.016 & -0.021 & -0.01 & -0.104 & -0.014 & 0.005 \\ 
   &  & p=36 & 1.187 & 1.195 & 1.471 & 3.754 & 1.254 & 1.488 & 1.505 & 5.009 & 1.6 & 3.451 & 0.019 & 0.013 & -0.216 & -2.098 & -0.036 & -0.231 & -0.244 & -3.136 & -0.323 & -1.849 \\ 
  \cmidrule{2-23} & \multirow{3}[2]{*}{$\rho=0.5$} & p=12 & 1.168 & 1.176 & 1.185 & 1.202 & 1.192 & 1.193 & 1.19 & 1.213 & 1.19 & 1.187 & 0.035 & 0.028 & 0.021 & 0.007 & 0.014 & 0.014 & 0.017 & -0.003 & 0.016 & 0.019 \\ 
   &  & p=20 & 1.168 & 1.176 & 1.198 & 1.265 & 1.213 & 1.22 & 1.213 & 1.32 & 1.216 & 1.186 & 0.035 & 0.028 & 0.01 & -0.046 & -0.003 & -0.008 & -0.003 & -0.091 & -0.005 & 0.02 \\ 
   &  & p=36 & 1.168 & 1.176 & 1.452 & 3.644 & 1.244 & 1.487 & 1.501 & 5.041 & 1.604 & 3.404 & 0.035 & 0.028 & -0.201 & -2.01 & -0.028 & -0.231 & -0.242 & -3.163 & -0.327 & -1.812 \\ 
  \cmidrule{2-23} & \multirow{3}[2]{*}{$\rho=0.9$} & p=12 & 1.112 & 1.117 & 1.133 & 1.151 & 1.147 & 1.148 & 1.14 & 1.164 & 1.141 & 1.119 & 0.081 & 0.076 & 0.063 & 0.048 & 0.051 & 0.051 & 0.057 & 0.037 & 0.056 & 0.075 \\ 
   &  & p=20 & 1.114 & 1.121 & 1.144 & 1.219 & 1.166 & 1.172 & 1.163 & 1.266 & 1.168 & 1.121 & 0.079 & 0.073 & 0.054 & -0.008 & 0.036 & 0.031 & 0.038 & -0.048 & 0.034 & 0.073 \\ 
   &  & p=36 & 1.113 & 1.121 & 1.406 & 3.594 & 1.195 & 1.423 & 1.439 & 4.861 & 1.584 & 3.248 & 0.08 & 0.073 & -0.164 & -1.977 & 0.012 & -0.18 & -0.192 & -3.025 & -0.315 & -1.693 \\ 
   \midrule 
 \multicolumn{1}{|c}{} & \multicolumn{1}{c}{} &       & \multicolumn{10}{c||}{Kullback-Leibler Discrepancy}                                    & \multicolumn{10}{c|}{Number of Variables} \\
\midrule\multirow{9}[6]{*}{hsnr} & \multirow{3}[2]{*}{$\rho=0$} & p=12 & 2.741 & 2.763 & 2.788 & 2.845 & 2.835 & 2.828 & 2.833 & 2.891 & 2.836 & 2.759 & 6.007 & 6.144 & 6.271 & 6.614 & 6.625 & 6.555 & 6.545 & 6.954 & 6.564 & 6.041 \\ 
   &  & p=20 & 2.741 & 2.77 & 2.845 & 3.193 & 2.992 & 3.007 & 2.996 & 3.553 & 3.01 & 2.811 & 6.007 & 6.16 & 6.41 & 7.435 & 7.046 & 6.972 & 6.89 & 8.661 & 6.932 & 6.136 \\ 
   &  & p=36 & 2.741 & 2.77 & 5.3 & 132.746 & 4.028 & 47.066 & 54.433 & 168.88 & 62.386 & 161.072 & 6.007 & 6.16 & 8.308 & 13.578 & 7.297 & 7.83 & 7.904 & 19.221 & 8.198 & 14.005 \\ 
  \cmidrule{2-23} & \multirow{3}[2]{*}{$\rho=0.5$} & p=12 & 3.545 & 3.579 & 3.6 & 3.652 & 3.635 & 3.641 & 3.637 & 3.692 & 3.639 & 3.574 & 5.546 & 5.75 & 5.891 & 6.29 & 6.262 & 6.256 & 6.21 & 6.627 & 6.233 & 5.597 \\ 
   &  & p=20 & 3.545 & 3.586 & 3.655 & 4.014 & 3.795 & 3.827 & 3.821 & 4.332 & 3.838 & 3.63 & 5.546 & 5.762 & 6.031 & 7.113 & 6.671 & 6.653 & 6.611 & 8.294 & 6.676 & 5.687 \\ 
   &  & p=36 & 3.545 & 3.586 & 5.84 & 132.123 & 4.898 & 48.611 & 53.322 & 169.602 & 61.467 & 159.191 & 5.546 & 5.762 & 7.99 & 13.14 & 6.901 & 7.477 & 7.533 & 18.844 & 7.869 & 13.441 \\ 
  \cmidrule{2-23} & \multirow{3}[2]{*}{$\rho=0.9$} & p=12 & 4.194 & 4.216 & 4.244 & 4.292 & 4.277 & 4.274 & 4.271 & 4.328 & 4.275 & 4.211 & 4.479 & 4.722 & 4.922 & 5.336 & 5.281 & 5.244 & 5.216 & 5.682 & 5.251 & 4.334 \\ 
   &  & p=20 & 4.196 & 4.215 & 4.308 & 4.624 & 4.435 & 4.44 & 4.461 & 4.905 & 4.482 & 4.242 & 4.482 & 4.692 & 5.116 & 6.08 & 5.667 & 5.626 & 5.617 & 7.196 & 5.692 & 4.38 \\ 
   &  & p=36 & 4.195 & 4.217 & 6.548 & 129.469 & 7.318 & 48.141 & 54.996 & 166.869 & 64.994 & 152.924 & 4.488 & 4.699 & 7.079 & 12.372 & 5.883 & 6.352 & 6.577 & 17.845 & 6.933 & 11.699 \\ 
  \midrule\multirow{9}[6]{*}{msnr} & \multirow{3}[2]{*}{$\rho=0$} & p=12 & 4.878 & 4.894 & 4.912 & 4.966 & 4.943 & 4.95 & 4.949 & 5.011 & 4.952 & 4.913 & 4.293 & 4.577 & 4.829 & 5.326 & 5.153 & 5.221 & 5.176 & 5.739 & 5.216 & 3.971 \\ 
   &  & p=20 & 4.878 & 4.898 & 4.978 & 5.27 & 5.07 & 5.099 & 5.092 & 5.623 & 5.11 & 4.953 & 4.293 & 4.584 & 5.054 & 6.03 & 5.523 & 5.587 & 5.496 & 7.404 & 5.582 & 4.044 \\ 
   &  & p=36 & 4.878 & 4.898 & 7.426 & 134.786 & 6.061 & 48.783 & 56.268 & 169.214 & 63.366 & 152.886 & 4.293 & 4.584 & 6.981 & 12.379 & 5.732 & 6.364 & 6.455 & 17.998 & 6.842 & 11.374 \\ 
  \cmidrule{2-23} & \multirow{3}[2]{*}{$\rho=0.5$} & p=12 & 5.641 & 5.667 & 5.686 & 5.741 & 5.717 & 5.728 & 5.719 & 5.778 & 5.721 & 5.672 & 3.827 & 4.11 & 4.363 & 4.849 & 4.713 & 4.746 & 4.669 & 5.252 & 4.714 & 3.492 \\ 
   &  & p=20 & 5.641 & 5.668 & 5.748 & 6.049 & 5.846 & 5.877 & 5.87 & 6.363 & 5.892 & 5.702 & 3.833 & 4.11 & 4.563 & 5.591 & 5.075 & 5.078 & 4.999 & 6.857 & 5.099 & 3.546 \\ 
   &  & p=36 & 5.641 & 5.668 & 7.935 & 133.866 & 6.924 & 50.644 & 54.84 & 169.426 & 63.046 & 149.514 & 3.833 & 4.11 & 6.605 & 11.851 & 5.273 & 5.955 & 5.861 & 17.539 & 6.277 & 10.399 \\ 
  \cmidrule{2-23} & \multirow{3}[2]{*}{$\rho=0.9$} & p=12 & 6.245 & 6.273 & 6.306 & 6.351 & 6.323 & 6.331 & 6.329 & 6.382 & 6.331 & 6.252 & 2.527 & 2.774 & 3.071 & 3.467 & 3.362 & 3.345 & 3.304 & 3.805 & 3.325 & 2.115 \\ 
   &  & p=20 & 6.248 & 6.276 & 6.363 & 6.668 & 6.444 & 6.469 & 6.476 & 6.892 & 6.499 & 6.252 & 2.549 & 2.792 & 3.261 & 4.301 & 3.679 & 3.748 & 3.661 & 5.223 & 3.759 & 2.095 \\ 
   &  & p=36 & 6.248 & 6.276 & 8.617 & 130.727 & 9.279 & 50.23 & 55.323 & 165.097 & 66.126 & 145.663 & 2.539 & 2.796 & 5.51 & 10.741 & 3.898 & 4.465 & 4.556 & 15.98 & 4.918 & 8.557 \\ 
  \midrule\multirow{9}[6]{*}{lsnr} & \multirow{3}[2]{*}{$\rho=0$} & p=12 & 6.385 & 6.41 & 6.433 & 6.477 & 6.452 & 6.455 & 6.45 & 6.505 & 6.456 & 6.386 & 1.995 & 2.287 & 2.626 & 3.111 & 2.936 & 2.902 & 2.851 & 3.426 & 2.918 & 1.174 \\ 
   &  & p=20 & 6.385 & 6.41 & 6.494 & 6.753 & 6.542 & 6.578 & 6.552 & 6.999 & 6.57 & 6.386 & 1.995 & 2.287 & 2.914 & 3.869 & 3.222 & 3.236 & 3.082 & 4.856 & 3.186 & 1.174 \\ 
   &  & p=36 & 6.385 & 6.41 & 8.949 & 135.519 & 7.481 & 49.669 & 53.946 & 166.282 & 63.703 & 142.936 & 1.995 & 2.287 & 5.171 & 10.509 & 3.425 & 3.966 & 3.96 & 15.751 & 4.381 & 7.596 \\ 
  \cmidrule{2-23} & \multirow{3}[2]{*}{$\rho=0.5$} & p=12 & 7.165 & 7.186 & 7.218 & 7.262 & 7.235 & 7.24 & 7.234 & 7.293 & 7.236 & 7.167 & 1.927 & 2.154 & 2.51 & 2.932 & 2.744 & 2.716 & 2.684 & 3.237 & 2.731 & 1.16 \\ 
   &  & p=20 & 7.165 & 7.187 & 7.269 & 7.549 & 7.333 & 7.367 & 7.35 & 7.769 & 7.371 & 7.163 & 1.927 & 2.155 & 2.726 & 3.72 & 3.035 & 3.033 & 2.974 & 4.599 & 3.068 & 1.15 \\ 
   &  & p=36 & 7.165 & 7.187 & 9.466 & 134.173 & 8.434 & 50.408 & 54.602 & 168.59 & 63.814 & 142.66 & 1.927 & 2.155 & 5.029 & 10.23 & 3.307 & 3.828 & 3.845 & 15.698 & 4.269 & 7.434 \\ 
  \cmidrule{2-23} & \multirow{3}[2]{*}{$\rho=0.9$} & p=12 & 7.783 & 7.8 & 7.841 & 7.884 & 7.866 & 7.874 & 7.859 & 7.914 & 7.861 & 7.776 & 1.521 & 1.658 & 1.966 & 2.306 & 2.237 & 2.239 & 2.11 & 2.568 & 2.145 & 1.052 \\ 
   &  & p=20 & 7.786 & 7.808 & 7.891 & 8.181 & 7.957 & 7.983 & 7.977 & 8.362 & 7.994 & 7.779 & 1.524 & 1.664 & 2.099 & 3.108 & 2.473 & 2.517 & 2.369 & 3.836 & 2.452 & 1.067 \\ 
   &  & p=36 & 7.785 & 7.807 & 10.168 & 132.026 & 10.798 & 51.4 & 55.507 & 164.981 & 67.322 & 141.038 & 1.509 & 1.663 & 4.676 & 9.972 & 2.733 & 3.203 & 3.215 & 14.728 & 3.682 & 7.034 \\ 
   \bottomrule 
\end{tabular}
}
\end{table}

% latex table generated in R 4.0.2 by xtable 1.8-4 package
% Fri Jun 26 10:43:44 2020
\begin{table}[ht]
\centering
\caption{Fixed X, Sparse-Ex1, n=200} 
\scalebox{0.52}{
\begin{tabular}{|c|c|c|cc|cc|cc|ccc|c||cc|cc|cc|ccc|c|}
  \toprule 
 \multicolumn{1}{|c}{} & \multicolumn{1}{c}{} &       & RAICc & AICc  & RC$_p$ & C$_p$ & 10F CV & LOO CV & S$_p$ & $\widehat{\text{C}}_p$ & GCV   & BIC   & RAICc & AICc  & RC$_p$ & C$_p$ & 10F CV & LOO CV & S$_p$ & $\widehat{\text{C}}_p$ & GCV   & BIC  \\
 \cmidrule{4-23}\multicolumn{1}{|c}{} & \multicolumn{1}{c}{} &       & \multicolumn{10}{c||}{RMSE}                                                   & \multicolumn{10}{c|}{Relative Risk}       \\
 \midrule\multirow{9}[6]{*}{hsnr} & \multirow{3}[2]{*}{$\rho=0$} & p=12 & 0.382 & 0.385 & 0.386 & 0.389 & 0.388 & 0.389 & 0.388 & 0.392 & 0.388 & 0.348 & 0.005 & 0.005 & 0.005 & 0.005 & 0.005 & 0.005 & 0.005 & 0.005 & 0.005 & 0.004 \\ 
   &  & p=100 & 0.387 & 0.392 & 0.395 & 0.405 & 0.402 & 0.403 & 0.401 & 0.412 & 0.401 & 0.348 & 0.005 & 0.005 & 0.006 & 0.006 & 0.006 & 0.006 & 0.006 & 0.006 & 0.006 & 0.004 \\ 
   &  & p=196 & 0.387 & 0.392 & 0.646 & 0.901 & 0.402 & 0.403 & 0.401 & 1.095 & 0.403 & 0.613 & 0.005 & 0.005 & 0.04 & 1.733 & 0.006 & 0.006 & 0.006 & 2.918 & 0.009 & 1.388 \\ 
  \cmidrule{2-23} & \multirow{3}[2]{*}{$\rho=0.5$} & p=12 & 0.566 & 0.571 & 0.573 & 0.576 & 0.576 & 0.575 & 0.576 & 0.58 & 0.575 & 0.533 & 0.005 & 0.005 & 0.005 & 0.005 & 0.005 & 0.005 & 0.005 & 0.005 & 0.005 & 0.005 \\ 
   &  & p=100 & 0.575 & 0.581 & 0.587 & 0.603 & 0.599 & 0.596 & 0.592 & 0.613 & 0.592 & 0.533 & 0.005 & 0.005 & 0.006 & 0.006 & 0.006 & 0.006 & 0.006 & 0.006 & 0.006 & 0.005 \\ 
   &  & p=196 & 0.575 & 0.581 & 0.962 & 1.339 & 0.599 & 0.596 & 0.592 & 1.616 & 0.595 & 0.924 & 0.005 & 0.005 & 0.039 & 1.742 & 0.006 & 0.006 & 0.006 & 2.853 & 0.008 & 1.373 \\ 
  \cmidrule{2-23} & \multirow{3}[2]{*}{$\rho=0.9$} & p=12 & 0.803 & 0.805 & 0.807 & 0.809 & 0.81 & 0.811 & 0.809 & 0.813 & 0.809 & 0.82 & 0.005 & 0.005 & 0.005 & 0.005 & 0.005 & 0.005 & 0.005 & 0.005 & 0.005 & 0.005 \\ 
   &  & p=100 & 0.811 & 0.823 & 0.828 & 0.85 & 0.838 & 0.842 & 0.836 & 0.859 & 0.837 & 0.826 & 0.005 & 0.005 & 0.005 & 0.006 & 0.006 & 0.006 & 0.006 & 0.006 & 0.006 & 0.005 \\ 
   &  & p=196 & 0.811 & 0.823 & 1.354 & 1.873 & 0.838 & 0.841 & 0.836 & 2.222 & 0.844 & 1.348 & 0.005 & 0.005 & 0.038 & 1.681 & 0.006 & 0.006 & 0.006 & 2.765 & 0.012 & 1.313 \\ 
  \midrule\multirow{9}[6]{*}{msnr} & \multirow{3}[2]{*}{$\rho=0$} & p=12 & 1.148 & 1.155 & 1.159 & 1.166 & 1.168 & 1.166 & 1.164 & 1.173 & 1.164 & 1.2 & 0.046 & 0.046 & 0.047 & 0.047 & 0.047 & 0.047 & 0.047 & 0.048 & 0.047 & 0.049 \\ 
   &  & p=100 & 1.161 & 1.176 & 1.185 & 1.211 & 1.206 & 1.205 & 1.199 & 1.228 & 1.199 & 1.2 & 0.047 & 0.049 & 0.05 & 0.053 & 0.052 & 0.052 & 0.052 & 0.055 & 0.052 & 0.049 \\ 
   &  & p=196 & 1.161 & 1.176 & 1.927 & 2.67 & 1.206 & 1.205 & 1.199 & 3.21 & 1.205 & 1.939 & 0.047 & 0.049 & 0.341 & 14.731 & 0.052 & 0.052 & 0.052 & 24.807 & 0.076 & 11.72 \\ 
  \cmidrule{2-23} & \multirow{3}[2]{*}{$\rho=0.5$} & p=12 & 1.703 & 1.715 & 1.717 & 1.726 & 1.722 & 1.73 & 1.724 & 1.736 & 1.724 & 1.823 & 0.045 & 0.045 & 0.046 & 0.046 & 0.046 & 0.046 & 0.046 & 0.047 & 0.046 & 0.051 \\ 
   &  & p=100 & 1.729 & 1.746 & 1.757 & 1.794 & 1.779 & 1.786 & 1.769 & 1.821 & 1.77 & 1.823 & 0.047 & 0.048 & 0.049 & 0.052 & 0.051 & 0.051 & 0.05 & 0.055 & 0.05 & 0.051 \\ 
   &  & p=196 & 1.729 & 1.746 & 2.857 & 3.944 & 1.779 & 1.786 & 1.769 & 4.696 & 1.778 & 2.903 & 0.047 & 0.048 & 0.336 & 14.807 & 0.051 & 0.051 & 0.05 & 24.05 & 0.073 & 11.467 \\ 
  \cmidrule{2-23} & \multirow{3}[2]{*}{$\rho=0.9$} & p=12 & 2.269 & 2.281 & 2.288 & 2.3 & 2.298 & 2.294 & 2.292 & 2.303 & 2.292 & 2.465 & 0.04 & 0.041 & 0.041 & 0.041 & 0.041 & 0.041 & 0.041 & 0.041 & 0.041 & 0.046 \\ 
   &  & p=100 & 2.295 & 2.324 & 2.341 & 2.372 & 2.366 & 2.368 & 2.353 & 2.394 & 2.353 & 2.488 & 0.041 & 0.043 & 0.044 & 0.046 & 0.045 & 0.045 & 0.044 & 0.047 & 0.044 & 0.047 \\ 
   &  & p=196 & 2.295 & 2.324 & 3.919 & 5.413 & 2.366 & 2.368 & 2.353 & 6.354 & 2.373 & 3.964 & 0.041 & 0.043 & 0.322 & 14.286 & 0.045 & 0.045 & 0.044 & 23.324 & 0.096 & 11.043 \\ 
  \midrule\multirow{9}[6]{*}{lsnr} & \multirow{3}[2]{*}{$\rho=0$} & p=12 & 2.638 & 2.655 & 2.662 & 2.672 & 2.678 & 2.667 & 2.666 & 2.678 & 2.666 & 3.192 & 0.233 & 0.237 & 0.238 & 0.241 & 0.24 & 0.239 & 0.239 & 0.242 & 0.24 & 0.321 \\ 
   &  & p=100 & 2.656 & 2.691 & 2.721 & 2.765 & 2.736 & 2.73 & 2.735 & 2.782 & 2.736 & 3.192 & 0.239 & 0.247 & 0.255 & 0.27 & 0.257 & 0.258 & 0.26 & 0.275 & 0.261 & 0.321 \\ 
   &  & p=196 & 2.656 & 2.691 & 4.375 & 6.002 & 2.736 & 2.73 & 2.735 & 7.126 & 2.749 & 4.724 & 0.239 & 0.247 & 1.709 & 73.369 & 0.257 & 0.258 & 0.26 & 122.666 & 0.379 & 56.446 \\ 
  \cmidrule{2-23} & \multirow{3}[2]{*}{$\rho=0.5$} & p=12 & 3.731 & 3.747 & 3.759 & 3.769 & 3.789 & 3.782 & 3.764 & 3.793 & 3.767 & 4.426 & 0.211 & 0.213 & 0.214 & 0.216 & 0.217 & 0.217 & 0.215 & 0.219 & 0.216 & 0.281 \\ 
   &  & p=100 & 3.771 & 3.801 & 3.843 & 3.901 & 3.88 & 3.871 & 3.845 & 3.938 & 3.853 & 4.426 & 0.217 & 0.223 & 0.23 & 0.244 & 0.235 & 0.235 & 0.231 & 0.251 & 0.234 & 0.281 \\ 
   &  & p=196 & 3.771 & 3.801 & 6.361 & 8.781 & 3.88 & 3.871 & 3.845 & 10.354 & 3.871 & 6.73 & 0.217 & 0.223 & 1.67 & 74.005 & 0.235 & 0.235 & 0.231 & 119.624 & 0.347 & 55.333 \\ 
  \cmidrule{2-23} & \multirow{3}[2]{*}{$\rho=0.9$} & p=12 & 4.699 & 4.737 & 4.759 & 4.773 & 4.8 & 4.746 & 4.762 & 4.789 & 4.765 & 5.11 & 0.167 & 0.17 & 0.172 & 0.173 & 0.174 & 0.171 & 0.172 & 0.175 & 0.172 & 0.178 \\ 
   &  & p=100 & 4.761 & 4.809 & 4.897 & 4.957 & 4.953 & 4.881 & 4.89 & 5.026 & 4.89 & 5.133 & 0.173 & 0.177 & 0.187 & 0.195 & 0.193 & 0.185 & 0.186 & 0.205 & 0.186 & 0.178 \\ 
   &  & p=196 & 4.761 & 4.809 & 8.476 & 11.826 & 4.953 & 4.881 & 4.89 & 13.933 & 4.936 & 8.301 & 0.173 & 0.177 & 1.58 & 71.402 & 0.193 & 0.185 & 0.186 & 116.06 & 0.444 & 51.014 \\ 
   \midrule 
 \multicolumn{1}{|c}{} & \multicolumn{1}{c}{} &       & \multicolumn{10}{c||}{Relative Test Error}                                    & \multicolumn{10}{c|}{Proportion of Variance Explained} \\
\midrule\multirow{9}[6]{*}{hsnr} & \multirow{3}[2]{*}{$\rho=0$} & p=12 & 1.041 & 1.042 & 1.042 & 1.043 & 1.043 & 1.043 & 1.043 & 1.044 & 1.043 & 1.034 & 0.886 & 0.886 & 0.886 & 0.886 & 0.886 & 0.886 & 0.886 & 0.886 & 0.886 & 0.887 \\ 
   &  & p=100 & 1.043 & 1.044 & 1.046 & 1.049 & 1.048 & 1.048 & 1.048 & 1.051 & 1.048 & 1.034 & 0.886 & 0.886 & 0.885 & 0.885 & 0.885 & 0.885 & 0.885 & 0.885 & 0.885 & 0.887 \\ 
   &  & p=196 & 1.043 & 1.044 & 1.323 & 15.091 & 1.048 & 1.048 & 1.048 & 24.733 & 1.07 & 12.289 & 0.886 & 0.886 & 0.855 & -0.653 & 0.885 & 0.885 & 0.885 & -1.708 & 0.883 & -0.346 \\ 
  \cmidrule{2-23} & \multirow{3}[2]{*}{$\rho=0.5$} & p=12 & 1.041 & 1.042 & 1.042 & 1.043 & 1.043 & 1.043 & 1.043 & 1.044 & 1.043 & 1.037 & 0.887 & 0.887 & 0.887 & 0.887 & 0.887 & 0.887 & 0.887 & 0.887 & 0.887 & 0.888 \\ 
   &  & p=100 & 1.043 & 1.044 & 1.046 & 1.05 & 1.048 & 1.048 & 1.047 & 1.052 & 1.047 & 1.037 & 0.887 & 0.887 & 0.887 & 0.886 & 0.886 & 0.887 & 0.887 & 0.886 & 0.887 & 0.888 \\ 
   &  & p=196 & 1.043 & 1.044 & 1.323 & 15.346 & 1.048 & 1.048 & 1.047 & 24.5 & 1.069 & 12.307 & 0.887 & 0.887 & 0.857 & -0.662 & 0.886 & 0.887 & 0.887 & -1.653 & 0.884 & -0.333 \\ 
  \cmidrule{2-23} & \multirow{3}[2]{*}{$\rho=0.9$} & p=12 & 1.043 & 1.043 & 1.044 & 1.044 & 1.044 & 1.044 & 1.044 & 1.044 & 1.044 & 1.045 & 0.891 & 0.891 & 0.891 & 0.891 & 0.891 & 0.891 & 0.891 & 0.891 & 0.891 & 0.891 \\ 
   &  & p=100 & 1.044 & 1.046 & 1.046 & 1.05 & 1.048 & 1.049 & 1.048 & 1.052 & 1.048 & 1.046 & 0.89 & 0.889 & 0.889 & 0.889 & 0.889 & 0.889 & 0.889 & 0.889 & 0.889 & 0.889 \\ 
   &  & p=196 & 1.044 & 1.046 & 1.322 & 15.214 & 1.048 & 1.048 & 1.048 & 24.374 & 1.099 & 12.103 & 0.89 & 0.889 & 0.86 & -0.609 & 0.889 & 0.889 & 0.889 & -1.578 & 0.884 & -0.28 \\ 
  \midrule\multirow{9}[6]{*}{msnr} & \multirow{3}[2]{*}{$\rho=0$} & p=12 & 1.044 & 1.044 & 1.045 & 1.045 & 1.045 & 1.045 & 1.045 & 1.046 & 1.045 & 1.047 & 0.467 & 0.466 & 0.466 & 0.466 & 0.466 & 0.466 & 0.466 & 0.466 & 0.466 & 0.465 \\ 
   &  & p=100 & 1.045 & 1.047 & 1.048 & 1.051 & 1.05 & 1.05 & 1.05 & 1.053 & 1.05 & 1.047 & 0.466 & 0.465 & 0.464 & 0.463 & 0.464 & 0.463 & 0.464 & 0.462 & 0.464 & 0.465 \\ 
   &  & p=196 & 1.045 & 1.047 & 1.326 & 15.094 & 1.05 & 1.05 & 1.05 & 24.734 & 1.072 & 12.213 & 0.466 & 0.465 & 0.322 & -6.714 & 0.464 & 0.463 & 0.464 & -11.64 & 0.452 & -5.242 \\ 
  \cmidrule{2-23} & \multirow{3}[2]{*}{$\rho=0.5$} & p=12 & 1.043 & 1.044 & 1.044 & 1.045 & 1.044 & 1.045 & 1.045 & 1.045 & 1.045 & 1.049 & 0.47 & 0.47 & 0.47 & 0.469 & 0.47 & 0.469 & 0.469 & 0.469 & 0.469 & 0.467 \\ 
   &  & p=100 & 1.045 & 1.046 & 1.047 & 1.051 & 1.049 & 1.05 & 1.049 & 1.053 & 1.049 & 1.049 & 0.469 & 0.468 & 0.468 & 0.466 & 0.467 & 0.467 & 0.467 & 0.465 & 0.467 & 0.467 \\ 
   &  & p=196 & 1.045 & 1.046 & 1.325 & 15.347 & 1.049 & 1.05 & 1.049 & 24.302 & 1.07 & 12.111 & 0.469 & 0.468 & 0.327 & -6.795 & 0.467 & 0.467 & 0.467 & -11.343 & 0.456 & -5.151 \\ 
  \cmidrule{2-23} & \multirow{3}[2]{*}{$\rho=0.9$} & p=12 & 1.04 & 1.041 & 1.041 & 1.042 & 1.041 & 1.041 & 1.041 & 1.042 & 1.041 & 1.046 & 0.482 & 0.482 & 0.482 & 0.482 & 0.482 & 0.482 & 0.482 & 0.482 & 0.482 & 0.479 \\ 
   &  & p=100 & 1.041 & 1.042 & 1.043 & 1.046 & 1.045 & 1.045 & 1.044 & 1.047 & 1.044 & 1.047 & 0.478 & 0.477 & 0.477 & 0.476 & 0.476 & 0.476 & 0.477 & 0.475 & 0.477 & 0.475 \\ 
   &  & p=196 & 1.041 & 1.042 & 1.32 & 15.208 & 1.045 & 1.045 & 1.044 & 24.196 & 1.095 & 11.983 & 0.478 & 0.477 & 0.338 & -6.625 & 0.476 & 0.476 & 0.477 & -11.131 & 0.451 & -5.008 \\ 
  \midrule\multirow{9}[6]{*}{lsnr} & \multirow{3}[2]{*}{$\rho=0$} & p=12 & 1.045 & 1.045 & 1.046 & 1.046 & 1.046 & 1.046 & 1.046 & 1.046 & 1.046 & 1.061 & 0.123 & 0.123 & 0.122 & 0.122 & 0.122 & 0.122 & 0.122 & 0.122 & 0.122 & 0.109 \\ 
   &  & p=100 & 1.046 & 1.047 & 1.049 & 1.052 & 1.049 & 1.049 & 1.05 & 1.053 & 1.05 & 1.061 & 0.122 & 0.121 & 0.12 & 0.117 & 0.119 & 0.119 & 0.119 & 0.116 & 0.119 & 0.109 \\ 
   &  & p=196 & 1.046 & 1.047 & 1.327 & 15.039 & 1.049 & 1.049 & 1.05 & 24.472 & 1.073 & 11.801 & 0.122 & 0.121 & -0.114 & -11.624 & 0.119 & 0.119 & 0.119 & -19.541 & 0.1 & -8.905 \\ 
  \cmidrule{2-23} & \multirow{3}[2]{*}{$\rho=0.5$} & p=12 & 1.041 & 1.041 & 1.042 & 1.042 & 1.042 & 1.042 & 1.042 & 1.042 & 1.042 & 1.054 & 0.128 & 0.128 & 0.128 & 0.127 & 0.127 & 0.127 & 0.127 & 0.127 & 0.127 & 0.117 \\ 
   &  & p=100 & 1.042 & 1.043 & 1.045 & 1.047 & 1.046 & 1.046 & 1.045 & 1.049 & 1.045 & 1.054 & 0.127 & 0.126 & 0.125 & 0.123 & 0.124 & 0.124 & 0.125 & 0.122 & 0.124 & 0.117 \\ 
   &  & p=196 & 1.042 & 1.043 & 1.324 & 15.341 & 1.046 & 1.046 & 1.045 & 24.181 & 1.067 & 11.723 & 0.127 & 0.126 & -0.109 & -11.851 & 0.124 & 0.124 & 0.125 & -19.256 & 0.106 & -8.82 \\ 
  \cmidrule{2-23} & \multirow{3}[2]{*}{$\rho=0.9$} & p=12 & 1.034 & 1.034 & 1.035 & 1.035 & 1.035 & 1.035 & 1.035 & 1.035 & 1.035 & 1.036 & 0.14 & 0.139 & 0.139 & 0.139 & 0.139 & 0.139 & 0.139 & 0.138 & 0.139 & 0.138 \\ 
   &  & p=100 & 1.034 & 1.035 & 1.037 & 1.039 & 1.038 & 1.037 & 1.037 & 1.041 & 1.037 & 1.035 & 0.137 & 0.137 & 0.135 & 0.134 & 0.134 & 0.135 & 0.135 & 0.132 & 0.135 & 0.136 \\ 
   &  & p=196 & 1.034 & 1.035 & 1.314 & 15.202 & 1.038 & 1.037 & 1.037 & 24.084 & 1.088 & 11.147 & 0.137 & 0.137 & -0.096 & -11.68 & 0.134 & 0.135 & 0.135 & -19.089 & 0.092 & -8.298 \\ 
   \midrule 
 \multicolumn{1}{|c}{} & \multicolumn{1}{c}{} &       & \multicolumn{10}{c||}{Kullback-Leibler Discrepancy}                                    & \multicolumn{10}{c|}{Number of Variables} \\
\midrule\multirow{9}[6]{*}{hsnr} & \multirow{3}[2]{*}{$\rho=0$} & p=12 & 2.499 & 2.5 & 2.5 & 2.501 & 2.5 & 2.501 & 2.501 & 2.502 & 2.501 & 2.491 & 6.508 & 6.562 & 6.595 & 6.657 & 6.7 & 6.66 & 6.639 & 6.726 & 6.64 & 6.032 \\ 
   &  & p=100 & 2.501 & 2.502 & 2.504 & 2.508 & 2.506 & 2.507 & 2.506 & 2.511 & 2.506 & 2.491 & 6.628 & 6.728 & 6.867 & 7.144 & 7.101 & 7.068 & 6.993 & 7.323 & 6.994 & 6.032 \\ 
   &  & p=196 & 2.501 & 2.502 & 2.947 & 106.918 & 2.506 & 2.507 & 2.506 & 146.314 & 3.751 & 109.825 & 6.628 & 6.728 & 25.908 & 59.763 & 7.101 & 7.068 & 6.993 & 84.379 & 7.18 & 35.067 \\ 
  \cmidrule{2-23} & \multirow{3}[2]{*}{$\rho=0.5$} & p=12 & 3.285 & 3.286 & 3.286 & 3.287 & 3.287 & 3.287 & 3.287 & 3.288 & 3.287 & 3.28 & 6.454 & 6.521 & 6.554 & 6.604 & 6.612 & 6.602 & 6.601 & 6.679 & 6.603 & 5.969 \\ 
   &  & p=100 & 3.287 & 3.289 & 3.291 & 3.295 & 3.294 & 3.293 & 3.292 & 3.298 & 3.292 & 3.28 & 6.588 & 6.68 & 6.825 & 7.144 & 7.104 & 7 & 6.906 & 7.36 & 6.907 & 5.969 \\ 
   &  & p=196 & 3.287 & 3.289 & 3.729 & 107.479 & 3.294 & 3.293 & 3.292 & 147.138 & 4.278 & 110.52 & 6.588 & 6.68 & 25.903 & 59.568 & 7.104 & 7 & 6.906 & 83.866 & 7.092 & 35.006 \\ 
  \cmidrule{2-23} & \multirow{3}[2]{*}{$\rho=0.9$} & p=12 & 3.928 & 3.928 & 3.929 & 3.929 & 3.929 & 3.929 & 3.929 & 3.93 & 3.929 & 3.928 & 5.928 & 5.975 & 6.001 & 6.066 & 6.073 & 6.055 & 6.059 & 6.126 & 6.059 & 5.073 \\ 
   &  & p=100 & 3.944 & 3.945 & 3.946 & 3.951 & 3.949 & 3.95 & 3.948 & 3.954 & 3.948 & 3.943 & 5.969 & 6.121 & 6.208 & 6.541 & 6.431 & 6.434 & 6.343 & 6.735 & 6.352 & 5.067 \\ 
   &  & p=196 & 3.944 & 3.945 & 4.364 & 107.035 & 3.949 & 3.949 & 3.948 & 145.521 & 7.697 & 110.69 & 5.969 & 6.121 & 25.512 & 59.042 & 6.431 & 6.42 & 6.343 & 82.188 & 6.723 & 33.861 \\ 
  \midrule\multirow{9}[6]{*}{msnr} & \multirow{3}[2]{*}{$\rho=0$} & p=12 & 4.641 & 4.642 & 4.642 & 4.643 & 4.642 & 4.643 & 4.643 & 4.644 & 4.643 & 4.642 & 6.316 & 6.378 & 6.419 & 6.491 & 6.486 & 6.489 & 6.47 & 6.563 & 6.472 & 5.506 \\ 
   &  & p=100 & 4.643 & 4.645 & 4.646 & 4.65 & 4.648 & 4.649 & 4.648 & 4.653 & 4.648 & 4.642 & 6.422 & 6.556 & 6.701 & 6.983 & 6.885 & 6.891 & 6.832 & 7.16 & 6.834 & 5.506 \\ 
   &  & p=196 & 4.643 & 4.645 & 5.089 & 109.06 & 4.648 & 4.649 & 4.648 & 148.456 & 5.893 & 111.673 & 6.422 & 6.556 & 25.733 & 59.545 & 6.885 & 6.891 & 6.832 & 84.273 & 7.02 & 34.421 \\ 
  \cmidrule{2-23} & \multirow{3}[2]{*}{$\rho=0.5$} & p=12 & 5.427 & 5.427 & 5.428 & 5.428 & 5.428 & 5.428 & 5.428 & 5.429 & 5.428 & 5.43 & 5.837 & 5.925 & 5.946 & 6.015 & 6.02 & 6.024 & 5.997 & 6.104 & 6 & 4.82 \\ 
   &  & p=100 & 5.429 & 5.43 & 5.432 & 5.435 & 5.433 & 5.435 & 5.433 & 5.438 & 5.433 & 5.43 & 5.979 & 6.091 & 6.177 & 6.489 & 6.426 & 6.405 & 6.283 & 6.722 & 6.291 & 4.82 \\ 
   &  & p=196 & 5.429 & 5.43 & 5.871 & 109.619 & 5.433 & 5.435 & 5.433 & 148.719 & 6.419 & 112.06 & 5.979 & 6.091 & 25.38 & 58.999 & 6.426 & 6.405 & 6.283 & 82.563 & 6.476 & 33.641 \\ 
  \cmidrule{2-23} & \multirow{3}[2]{*}{$\rho=0.9$} & p=12 & 6.065 & 6.066 & 6.066 & 6.067 & 6.066 & 6.066 & 6.066 & 6.067 & 6.066 & 6.069 & 4.679 & 4.756 & 4.792 & 4.884 & 4.855 & 4.864 & 4.837 & 4.95 & 4.846 & 3.37 \\ 
   &  & p=100 & 6.081 & 6.082 & 6.083 & 6.086 & 6.085 & 6.085 & 6.084 & 6.088 & 6.084 & 6.084 & 4.699 & 4.859 & 4.953 & 5.194 & 5.154 & 5.129 & 5.042 & 5.333 & 5.042 & 3.344 \\ 
   &  & p=196 & 6.081 & 6.082 & 6.502 & 109.168 & 6.085 & 6.085 & 6.084 & 147.222 & 9.832 & 112.179 & 4.699 & 4.859 & 24.552 & 57.875 & 5.154 & 5.129 & 5.042 & 80.429 & 5.419 & 32.029 \\ 
  \midrule\multirow{9}[6]{*}{lsnr} & \multirow{3}[2]{*}{$\rho=0$} & p=12 & 6.251 & 6.252 & 6.253 & 6.253 & 6.253 & 6.253 & 6.253 & 6.254 & 6.253 & 6.267 & 5.185 & 5.283 & 5.353 & 5.434 & 5.35 & 5.384 & 5.407 & 5.49 & 5.416 & 3.315 \\ 
   &  & p=100 & 6.253 & 6.255 & 6.257 & 6.26 & 6.257 & 6.258 & 6.258 & 6.262 & 6.258 & 6.267 & 5.266 & 5.45 & 5.622 & 5.902 & 5.645 & 5.688 & 5.734 & 6.035 & 5.743 & 3.315 \\ 
   &  & p=196 & 6.253 & 6.255 & 6.7 & 110.476 & 6.257 & 6.258 & 6.258 & 149.554 & 7.503 & 112.013 & 5.266 & 5.45 & 24.881 & 58.363 & 5.645 & 5.688 & 5.734 & 82.556 & 5.929 & 31.797 \\ 
  \cmidrule{2-23} & \multirow{3}[2]{*}{$\rho=0.5$} & p=12 & 7.034 & 7.034 & 7.035 & 7.035 & 7.035 & 7.035 & 7.035 & 7.036 & 7.035 & 7.046 & 4.591 & 4.692 & 4.728 & 4.794 & 4.839 & 4.815 & 4.767 & 4.886 & 4.781 & 2.911 \\ 
   &  & p=100 & 7.035 & 7.037 & 7.038 & 7.042 & 7.04 & 7.04 & 7.038 & 7.043 & 7.039 & 7.046 & 4.697 & 4.851 & 5.028 & 5.301 & 5.154 & 5.114 & 5.028 & 5.43 & 5.085 & 2.911 \\ 
   &  & p=196 & 7.035 & 7.037 & 7.479 & 111.222 & 7.04 & 7.04 & 7.038 & 150.027 & 8.025 & 112.353 & 4.697 & 4.851 & 24.445 & 57.919 & 5.154 & 5.114 & 5.028 & 81.218 & 5.272 & 31.252 \\ 
  \cmidrule{2-23} & \multirow{3}[2]{*}{$\rho=0.9$} & p=12 & 7.669 & 7.669 & 7.67 & 7.67 & 7.67 & 7.67 & 7.67 & 7.671 & 7.67 & 7.671 & 3.125 & 3.208 & 3.28 & 3.353 & 3.352 & 3.296 & 3.302 & 3.417 & 3.317 & 1.585 \\ 
   &  & p=100 & 7.684 & 7.685 & 7.687 & 7.689 & 7.689 & 7.687 & 7.688 & 7.692 & 7.688 & 7.685 & 3.158 & 3.258 & 3.454 & 3.62 & 3.643 & 3.48 & 3.473 & 3.866 & 3.475 & 1.605 \\ 
   &  & p=196 & 7.684 & 7.685 & 8.107 & 110.773 & 7.689 & 7.687 & 7.688 & 148.518 & 11.436 & 110.896 & 3.158 & 3.258 & 23.412 & 56.787 & 3.643 & 3.48 & 3.473 & 79.01 & 3.856 & 28.793 \\ 
   \bottomrule 
\end{tabular}
}
\end{table}

% latex table generated in R 4.0.2 by xtable 1.8-4 package
% Fri Jun 26 10:43:44 2020
\begin{table}[ht]
\centering
\caption{Fixed X, Sparse-Ex1, n=1000} 
\scalebox{0.52}{
\begin{tabular}{|c|c|c|cc|cc|cc|ccc|c||cc|cc|cc|ccc|c|}
  \toprule 
 \multicolumn{1}{|c}{} & \multicolumn{1}{c}{} &       & RAICc & AICc  & RC$_p$ & C$_p$ & 10F CV & LOO CV & S$_p$ & $\widehat{\text{C}}_p$ & GCV   & BIC   & RAICc & AICc  & RC$_p$ & C$_p$ & 10F CV & LOO CV & S$_p$ & $\widehat{\text{C}}_p$ & GCV   & BIC  \\
 \cmidrule{4-23}\multicolumn{1}{|c}{} & \multicolumn{1}{c}{} &       & \multicolumn{10}{c||}{RMSE}                                                   & \multicolumn{10}{c|}{Relative Risk}       \\
 \midrule\multirow{9}[6]{*}{hsnr} & \multirow{3}[2]{*}{$\rho=0$} & p=12 & 0.165 & 0.166 & 0.166 & 0.166 & 0.166 & 0.166 & 0.166 & 0.166 & 0.166 & 0.146 & 0.001 & 0.001 & 0.001 & 0.001 & 0.001 & 0.001 & 0.001 & 0.001 & 0.001 & 0.001 \\ 
   &  & p=500 & 0.17 & 0.171 & 0.171 & 0.172 & 0.173 & 0.173 & 0.172 & 0.174 & 0.172 & 0.146 & 0.001 & 0.001 & 0.001 & 0.001 & 0.001 & 0.001 & 0.001 & 0.001 & 0.001 & 0.001 \\ 
   &  & p=996 & 0.17 & 0.171 & 0.425 & 0.654 & 0.173 & 0.173 & 0.172 & 0.811 & 0.172 & 0.325 & 0.001 & 0.001 & 0.023 & 5.236 & 0.001 & 0.001 & 0.001 & 9.175 & 0.001 & 2.892 \\ 
  \cmidrule{2-23} & \multirow{3}[2]{*}{$\rho=0.5$} & p=12 & 0.247 & 0.247 & 0.247 & 0.248 & 0.248 & 0.248 & 0.248 & 0.248 & 0.248 & 0.218 & 0.001 & 0.001 & 0.001 & 0.001 & 0.001 & 0.001 & 0.001 & 0.001 & 0.001 & 0.001 \\ 
   &  & p=500 & 0.254 & 0.256 & 0.256 & 0.257 & 0.256 & 0.256 & 0.256 & 0.257 & 0.256 & 0.218 & 0.001 & 0.001 & 0.001 & 0.001 & 0.001 & 0.001 & 0.001 & 0.001 & 0.001 & 0.001 \\ 
   &  & p=996 & 0.254 & 0.256 & 0.635 & 0.98 & 0.256 & 0.256 & 0.256 & 1.217 & 0.256 & 0.486 & 0.001 & 0.001 & 0.023 & 5.253 & 0.001 & 0.001 & 0.001 & 9.477 & 0.001 & 2.906 \\ 
  \cmidrule{2-23} & \multirow{3}[2]{*}{$\rho=0.9$} & p=12 & 0.352 & 0.353 & 0.353 & 0.354 & 0.356 & 0.354 & 0.354 & 0.355 & 0.354 & 0.353 & 0.001 & 0.001 & 0.001 & 0.001 & 0.001 & 0.001 & 0.001 & 0.001 & 0.001 & 0.001 \\ 
   &  & p=500 & 0.365 & 0.367 & 0.367 & 0.368 & 0.37 & 0.367 & 0.368 & 0.369 & 0.368 & 0.354 & 0.001 & 0.001 & 0.001 & 0.001 & 0.001 & 0.001 & 0.001 & 0.001 & 0.001 & 0.001 \\ 
   &  & p=996 & 0.365 & 0.367 & 0.903 & 1.392 & 0.37 & 0.367 & 0.368 & 1.731 & 0.368 & 0.73 & 0.001 & 0.001 & 0.024 & 5.462 & 0.001 & 0.001 & 0.001 & 9.574 & 0.001 & 2.964 \\ 
  \midrule\multirow{9}[6]{*}{msnr} & \multirow{3}[2]{*}{$\rho=0$} & p=12 & 0.482 & 0.483 & 0.483 & 0.484 & 0.484 & 0.484 & 0.484 & 0.485 & 0.484 & 0.426 & 0.008 & 0.008 & 0.008 & 0.008 & 0.008 & 0.008 & 0.008 & 0.008 & 0.008 & 0.006 \\ 
   &  & p=500 & 0.497 & 0.499 & 0.499 & 0.503 & 0.503 & 0.503 & 0.503 & 0.506 & 0.503 & 0.426 & 0.008 & 0.008 & 0.008 & 0.009 & 0.009 & 0.009 & 0.009 & 0.009 & 0.009 & 0.006 \\ 
   &  & p=996 & 0.497 & 0.499 & 1.239 & 1.908 & 0.503 & 0.503 & 0.503 & 2.365 & 0.503 & 0.948 & 0.008 & 0.008 & 0.198 & 44.507 & 0.009 & 0.009 & 0.009 & 77.985 & 0.009 & 24.582 \\ 
  \cmidrule{2-23} & \multirow{3}[2]{*}{$\rho=0.5$} & p=12 & 0.726 & 0.727 & 0.727 & 0.729 & 0.735 & 0.729 & 0.728 & 0.73 & 0.728 & 0.729 & 0.008 & 0.008 & 0.008 & 0.008 & 0.008 & 0.008 & 0.008 & 0.008 & 0.008 & 0.008 \\ 
   &  & p=500 & 0.748 & 0.751 & 0.753 & 0.756 & 0.758 & 0.754 & 0.754 & 0.757 & 0.754 & 0.73 & 0.008 & 0.009 & 0.009 & 0.009 & 0.009 & 0.009 & 0.009 & 0.009 & 0.009 & 0.008 \\ 
   &  & p=996 & 0.748 & 0.751 & 1.863 & 2.867 & 0.758 & 0.754 & 0.754 & 3.554 & 0.754 & 1.501 & 0.008 & 0.009 & 0.199 & 44.651 & 0.009 & 0.009 & 0.009 & 80.555 & 0.009 & 24.704 \\ 
  \cmidrule{2-23} & \multirow{3}[2]{*}{$\rho=0.9$} & p=12 & 1.037 & 1.039 & 1.041 & 1.042 & 1.047 & 1.043 & 1.042 & 1.043 & 1.042 & 1.137 & 0.008 & 0.008 & 0.008 & 0.008 & 0.008 & 0.008 & 0.008 & 0.008 & 0.008 & 0.009 \\ 
   &  & p=500 & 1.07 & 1.075 & 1.075 & 1.081 & 1.078 & 1.08 & 1.081 & 1.083 & 1.082 & 1.139 & 0.009 & 0.009 & 0.009 & 0.009 & 0.009 & 0.009 & 0.009 & 0.009 & 0.009 & 0.009 \\ 
   &  & p=996 & 1.07 & 1.075 & 2.647 & 4.073 & 1.078 & 1.08 & 1.081 & 5.05 & 1.082 & 2.225 & 0.009 & 0.009 & 0.203 & 46.429 & 0.009 & 0.009 & 0.009 & 81.378 & 0.009 & 25.193 \\ 
  \midrule\multirow{9}[6]{*}{lsnr} & \multirow{3}[2]{*}{$\rho=0$} & p=12 & 1.103 & 1.105 & 1.105 & 1.107 & 1.114 & 1.107 & 1.107 & 1.109 & 1.107 & 1.222 & 0.04 & 0.04 & 0.04 & 0.04 & 0.04 & 0.04 & 0.04 & 0.04 & 0.04 & 0.048 \\ 
   &  & p=500 & 1.135 & 1.141 & 1.141 & 1.149 & 1.153 & 1.149 & 1.148 & 1.154 & 1.148 & 1.222 & 0.043 & 0.044 & 0.044 & 0.044 & 0.045 & 0.045 & 0.044 & 0.045 & 0.044 & 0.048 \\ 
   &  & p=996 & 1.135 & 1.141 & 2.805 & 4.299 & 1.153 & 1.149 & 1.148 & 5.292 & 1.148 & 2.353 & 0.043 & 0.044 & 0.991 & 222.537 & 0.045 & 0.045 & 0.044 & 389.172 & 0.044 & 122.439 \\ 
  \cmidrule{2-23} & \multirow{3}[2]{*}{$\rho=0.5$} & p=12 & 1.65 & 1.651 & 1.652 & 1.654 & 1.67 & 1.653 & 1.653 & 1.655 & 1.654 & 1.887 & 0.039 & 0.039 & 0.039 & 0.039 & 0.04 & 0.039 & 0.039 & 0.04 & 0.039 & 0.052 \\ 
   &  & p=500 & 1.694 & 1.701 & 1.707 & 1.713 & 1.72 & 1.703 & 1.702 & 1.708 & 1.702 & 1.886 & 0.043 & 0.043 & 0.043 & 0.044 & 0.044 & 0.043 & 0.043 & 0.043 & 0.043 & 0.052 \\ 
   &  & p=996 & 1.694 & 1.701 & 4.199 & 6.446 & 1.72 & 1.703 & 1.702 & 7.955 & 1.702 & 3.573 & 0.043 & 0.043 & 0.996 & 223.258 & 0.044 & 0.043 & 0.043 & 402.772 & 0.043 & 123.04 \\ 
  \cmidrule{2-23} & \multirow{3}[2]{*}{$\rho=0.9$} & p=12 & 2.288 & 2.29 & 2.291 & 2.293 & 2.291 & 2.291 & 2.292 & 2.293 & 2.292 & 2.761 & 0.039 & 0.039 & 0.039 & 0.039 & 0.039 & 0.039 & 0.039 & 0.039 & 0.039 & 0.056 \\ 
   &  & p=500 & 2.326 & 2.34 & 2.339 & 2.349 & 2.35 & 2.355 & 2.347 & 2.362 & 2.347 & 2.762 & 0.041 & 0.042 & 0.042 & 0.042 & 0.042 & 0.042 & 0.042 & 0.043 & 0.042 & 0.057 \\ 
   &  & p=996 & 2.326 & 2.34 & 5.907 & 9.086 & 2.35 & 2.355 & 2.347 & 11.219 & 2.347 & 5.107 & 0.041 & 0.042 & 1.014 & 232.144 & 0.042 & 0.042 & 0.042 & 406.246 & 0.042 & 124.468 \\ 
   \midrule 
 \multicolumn{1}{|c}{} & \multicolumn{1}{c}{} &       & \multicolumn{10}{c||}{Relative Test Error}                                    & \multicolumn{10}{c|}{Proportion of Variance Explained} \\
\midrule\multirow{9}[6]{*}{hsnr} & \multirow{3}[2]{*}{$\rho=0$} & p=12 & 1.008 & 1.008 & 1.008 & 1.008 & 1.008 & 1.008 & 1.008 & 1.008 & 1.008 & 1.006 & 0.9 & 0.9 & 0.9 & 0.9 & 0.9 & 0.9 & 0.9 & 0.9 & 0.9 & 0.9 \\ 
   &  & p=500 & 1.009 & 1.009 & 1.009 & 1.009 & 1.009 & 1.009 & 1.009 & 1.009 & 1.009 & 1.006 & 0.9 & 0.9 & 0.9 & 0.9 & 0.9 & 0.9 & 0.9 & 0.9 & 0.9 & 0.9 \\ 
   &  & p=996 & 1.009 & 1.009 & 1.212 & 48.611 & 1.009 & 1.009 & 1.009 & 84.423 & 1.009 & 27.297 & 0.9 & 0.9 & 0.88 & -3.816 & 0.9 & 0.9 & 0.9 & -7.365 & 0.9 & -1.705 \\ 
  \cmidrule{2-23} & \multirow{3}[2]{*}{$\rho=0.5$} & p=12 & 1.008 & 1.008 & 1.008 & 1.008 & 1.008 & 1.008 & 1.008 & 1.008 & 1.008 & 1.006 & 0.9 & 0.9 & 0.9 & 0.9 & 0.9 & 0.9 & 0.9 & 0.9 & 0.9 & 0.9 \\ 
   &  & p=500 & 1.009 & 1.009 & 1.009 & 1.009 & 1.009 & 1.009 & 1.009 & 1.009 & 1.009 & 1.006 & 0.9 & 0.9 & 0.9 & 0.9 & 0.9 & 0.9 & 0.9 & 0.9 & 0.9 & 0.9 \\ 
   &  & p=996 & 1.009 & 1.009 & 1.212 & 48.52 & 1.009 & 1.009 & 1.009 & 86.73 & 1.009 & 27.29 & 0.9 & 0.9 & 0.879 & -3.83 & 0.9 & 0.9 & 0.9 & -7.633 & 0.9 & -1.716 \\ 
  \cmidrule{2-23} & \multirow{3}[2]{*}{$\rho=0.9$} & p=12 & 1.008 & 1.008 & 1.008 & 1.008 & 1.008 & 1.008 & 1.008 & 1.008 & 1.008 & 1.008 & 0.898 & 0.898 & 0.898 & 0.898 & 0.898 & 0.898 & 0.898 & 0.898 & 0.898 & 0.898 \\ 
   &  & p=500 & 1.009 & 1.009 & 1.009 & 1.009 & 1.009 & 1.009 & 1.009 & 1.009 & 1.009 & 1.008 & 0.898 & 0.898 & 0.898 & 0.898 & 0.898 & 0.898 & 0.898 & 0.898 & 0.898 & 0.898 \\ 
   &  & p=996 & 1.009 & 1.009 & 1.212 & 49.481 & 1.009 & 1.009 & 1.009 & 85.975 & 1.009 & 27.306 & 0.898 & 0.898 & 0.877 & -4.01 & 0.898 & 0.898 & 0.898 & -7.706 & 0.898 & -1.765 \\ 
  \midrule\multirow{9}[6]{*}{msnr} & \multirow{3}[2]{*}{$\rho=0$} & p=12 & 1.008 & 1.008 & 1.008 & 1.008 & 1.008 & 1.008 & 1.008 & 1.008 & 1.008 & 1.006 & 0.513 & 0.513 & 0.513 & 0.513 & 0.513 & 0.513 & 0.513 & 0.513 & 0.513 & 0.514 \\ 
   &  & p=500 & 1.009 & 1.009 & 1.009 & 1.009 & 1.009 & 1.009 & 1.009 & 1.009 & 1.009 & 1.006 & 0.513 & 0.512 & 0.512 & 0.512 & 0.512 & 0.512 & 0.512 & 0.512 & 0.512 & 0.514 \\ 
   &  & p=996 & 1.009 & 1.009 & 1.212 & 48.611 & 1.009 & 1.009 & 1.009 & 84.423 & 1.009 & 27.297 & 0.513 & 0.512 & 0.415 & -22.486 & 0.512 & 0.512 & 0.512 & -39.789 & 0.512 & -12.189 \\ 
  \cmidrule{2-23} & \multirow{3}[2]{*}{$\rho=0.5$} & p=12 & 1.008 & 1.008 & 1.008 & 1.008 & 1.008 & 1.008 & 1.008 & 1.008 & 1.008 & 1.008 & 0.512 & 0.512 & 0.512 & 0.512 & 0.512 & 0.512 & 0.512 & 0.512 & 0.512 & 0.512 \\ 
   &  & p=500 & 1.009 & 1.009 & 1.009 & 1.009 & 1.009 & 1.009 & 1.009 & 1.009 & 1.009 & 1.008 & 0.511 & 0.511 & 0.511 & 0.511 & 0.511 & 0.511 & 0.511 & 0.511 & 0.511 & 0.512 \\ 
   &  & p=996 & 1.009 & 1.009 & 1.212 & 48.52 & 1.009 & 1.009 & 1.009 & 86.73 & 1.009 & 27.292 & 0.511 & 0.511 & 0.413 & -22.505 & 0.511 & 0.511 & 0.511 & -41.015 & 0.511 & -12.221 \\ 
  \cmidrule{2-23} & \multirow{3}[2]{*}{$\rho=0.9$} & p=12 & 1.008 & 1.008 & 1.008 & 1.008 & 1.008 & 1.008 & 1.008 & 1.008 & 1.008 & 1.01 & 0.507 & 0.507 & 0.507 & 0.507 & 0.507 & 0.507 & 0.507 & 0.507 & 0.507 & 0.506 \\ 
   &  & p=500 & 1.009 & 1.009 & 1.009 & 1.009 & 1.009 & 1.009 & 1.009 & 1.009 & 1.009 & 1.01 & 0.506 & 0.506 & 0.506 & 0.506 & 0.506 & 0.506 & 0.506 & 0.506 & 0.506 & 0.506 \\ 
   &  & p=996 & 1.009 & 1.009 & 1.212 & 49.481 & 1.009 & 1.009 & 1.009 & 85.975 & 1.009 & 27.307 & 0.506 & 0.506 & 0.407 & -23.206 & 0.506 & 0.506 & 0.506 & -41.058 & 0.506 & -12.358 \\ 
  \midrule\multirow{9}[6]{*}{lsnr} & \multirow{3}[2]{*}{$\rho=0$} & p=12 & 1.008 & 1.008 & 1.008 & 1.009 & 1.009 & 1.009 & 1.009 & 1.009 & 1.009 & 1.01 & 0.169 & 0.169 & 0.169 & 0.169 & 0.169 & 0.169 & 0.169 & 0.169 & 0.169 & 0.168 \\ 
   &  & p=500 & 1.009 & 1.009 & 1.009 & 1.01 & 1.01 & 1.01 & 1.01 & 1.01 & 1.01 & 1.01 & 0.169 & 0.169 & 0.169 & 0.168 & 0.168 & 0.168 & 0.168 & 0.168 & 0.168 & 0.168 \\ 
   &  & p=996 & 1.009 & 1.009 & 1.212 & 48.611 & 1.01 & 1.01 & 1.01 & 84.263 & 1.01 & 27.196 & 0.169 & 0.169 & 0.002 & -39.044 & 0.168 & 0.168 & 0.168 & -68.412 & 0.168 & -21.403 \\ 
  \cmidrule{2-23} & \multirow{3}[2]{*}{$\rho=0.5$} & p=12 & 1.008 & 1.008 & 1.008 & 1.008 & 1.009 & 1.008 & 1.008 & 1.008 & 1.008 & 1.011 & 0.169 & 0.169 & 0.169 & 0.169 & 0.168 & 0.169 & 0.169 & 0.169 & 0.169 & 0.166 \\ 
   &  & p=500 & 1.009 & 1.009 & 1.009 & 1.009 & 1.009 & 1.009 & 1.009 & 1.009 & 1.009 & 1.011 & 0.168 & 0.168 & 0.168 & 0.168 & 0.168 & 0.168 & 0.168 & 0.168 & 0.168 & 0.166 \\ 
   &  & p=996 & 1.009 & 1.009 & 1.212 & 48.521 & 1.009 & 1.009 & 1.009 & 86.73 & 1.009 & 27.189 & 0.168 & 0.168 & 0.001 & -39.005 & 0.168 & 0.168 & 0.168 & -70.51 & 0.168 & -21.418 \\ 
  \cmidrule{2-23} & \multirow{3}[2]{*}{$\rho=0.9$} & p=12 & 1.008 & 1.008 & 1.008 & 1.008 & 1.008 & 1.008 & 1.008 & 1.008 & 1.008 & 1.012 & 0.166 & 0.166 & 0.166 & 0.166 & 0.166 & 0.166 & 0.166 & 0.166 & 0.166 & 0.163 \\ 
   &  & p=500 & 1.009 & 1.009 & 1.009 & 1.009 & 1.009 & 1.009 & 1.009 & 1.009 & 1.009 & 1.012 & 0.166 & 0.166 & 0.166 & 0.165 & 0.165 & 0.165 & 0.166 & 0.165 & 0.166 & 0.163 \\ 
   &  & p=996 & 1.009 & 1.009 & 1.212 & 49.481 & 1.009 & 1.009 & 1.009 & 85.841 & 1.009 & 26.994 & 0.166 & 0.166 & -0.002 & -39.933 & 0.165 & 0.165 & 0.166 & -70.011 & 0.166 & -21.33 \\ 
   \midrule 
 \multicolumn{1}{|c}{} & \multicolumn{1}{c}{} &       & \multicolumn{10}{c||}{Kullback-Leibler Discrepancy}                                    & \multicolumn{10}{c|}{Number of Variables} \\
\midrule\multirow{9}[6]{*}{hsnr} & \multirow{3}[2]{*}{$\rho=0$} & p=12 & 2.35 & 2.35 & 2.35 & 2.35 & 2.35 & 2.35 & 2.35 & 2.35 & 2.35 & 2.348 & 6.643 & 6.658 & 6.659 & 6.677 & 6.72 & 6.68 & 6.672 & 6.698 & 6.673 & 6.01 \\ 
   &  & p=500 & 2.351 & 2.351 & 2.351 & 2.351 & 2.351 & 2.351 & 2.351 & 2.351 & 2.351 & 2.348 & 6.96 & 7.02 & 7.018 & 7.101 & 7.174 & 7.117 & 7.098 & 7.169 & 7.099 & 6.01 \\ 
   &  & p=996 & 2.351 & 2.351 & 2.628 & 389.985 & 2.351 & 2.351 & 2.351 & 588.931 & 2.351 & 370.266 & 6.96 & 7.02 & 95.117 & 262.275 & 7.174 & 7.117 & 7.098 & 360.193 & 7.099 & 104.015 \\ 
  \cmidrule{2-23} & \multirow{3}[2]{*}{$\rho=0.5$} & p=12 & 3.154 & 3.154 & 3.154 & 3.154 & 3.154 & 3.154 & 3.154 & 3.154 & 3.154 & 3.152 & 6.617 & 6.631 & 6.637 & 6.653 & 6.708 & 6.654 & 6.647 & 6.663 & 6.647 & 6.005 \\ 
   &  & p=500 & 3.155 & 3.155 & 3.155 & 3.155 & 3.155 & 3.155 & 3.155 & 3.155 & 3.155 & 3.152 & 6.943 & 6.992 & 7.01 & 7.053 & 7.104 & 7.024 & 7.023 & 7.057 & 7.023 & 6.005 \\ 
   &  & p=996 & 3.155 & 3.155 & 3.432 & 393.113 & 3.155 & 3.155 & 3.155 & 598.289 & 3.155 & 371.06 & 6.943 & 6.992 & 95.072 & 263.028 & 7.104 & 7.024 & 7.023 & 362.184 & 7.023 & 104.007 \\ 
  \cmidrule{2-23} & \multirow{3}[2]{*}{$\rho=0.9$} & p=12 & 3.856 & 3.856 & 3.856 & 3.856 & 3.856 & 3.856 & 3.856 & 3.856 & 3.856 & 3.856 & 6.555 & 6.569 & 6.592 & 6.609 & 6.649 & 6.6 & 6.609 & 6.621 & 6.609 & 5.789 \\ 
   &  & p=500 & 3.856 & 3.856 & 3.856 & 3.856 & 3.856 & 3.856 & 3.856 & 3.856 & 3.856 & 3.855 & 6.941 & 6.99 & 6.999 & 7.022 & 7.091 & 7.011 & 7.025 & 7.051 & 7.025 & 5.782 \\ 
   &  & p=996 & 3.856 & 3.856 & 4.134 & 400.505 & 3.856 & 3.856 & 3.856 & 593.378 & 3.856 & 371.717 & 6.941 & 6.99 & 94.879 & 262.796 & 7.091 & 7.011 & 7.025 & 362.135 & 7.025 & 103.799 \\ 
  \midrule\multirow{9}[6]{*}{msnr} & \multirow{3}[2]{*}{$\rho=0$} & p=12 & 4.49 & 4.49 & 4.49 & 4.49 & 4.49 & 4.49 & 4.49 & 4.49 & 4.49 & 4.488 & 6.643 & 6.658 & 6.659 & 6.677 & 6.72 & 6.68 & 6.672 & 6.698 & 6.673 & 6.009 \\ 
   &  & p=500 & 4.491 & 4.491 & 4.491 & 4.491 & 4.491 & 4.491 & 4.491 & 4.491 & 4.491 & 4.488 & 6.96 & 7.02 & 7.018 & 7.101 & 7.174 & 7.117 & 7.098 & 7.169 & 7.099 & 6.009 \\ 
   &  & p=996 & 4.491 & 4.491 & 4.768 & 392.126 & 4.491 & 4.491 & 4.491 & 591.071 & 4.491 & 372.406 & 6.96 & 7.02 & 95.117 & 262.275 & 7.174 & 7.117 & 7.098 & 360.193 & 7.099 & 104.014 \\ 
  \cmidrule{2-23} & \multirow{3}[2]{*}{$\rho=0.5$} & p=12 & 5.294 & 5.294 & 5.294 & 5.294 & 5.294 & 5.294 & 5.294 & 5.294 & 5.294 & 5.294 & 6.568 & 6.585 & 6.591 & 6.607 & 6.657 & 6.604 & 6.601 & 6.617 & 6.601 & 5.756 \\ 
   &  & p=500 & 5.295 & 5.295 & 5.295 & 5.295 & 5.295 & 5.295 & 5.295 & 5.295 & 5.295 & 5.294 & 6.894 & 6.943 & 6.969 & 7.012 & 7.056 & 6.98 & 6.974 & 7.008 & 6.974 & 5.755 \\ 
   &  & p=996 & 5.295 & 5.295 & 5.573 & 395.253 & 5.295 & 5.295 & 5.295 & 600.429 & 5.295 & 373.202 & 6.894 & 6.943 & 95.029 & 262.985 & 7.056 & 6.98 & 6.974 & 362.151 & 6.974 & 103.78 \\ 
  \cmidrule{2-23} & \multirow{3}[2]{*}{$\rho=0.9$} & p=12 & 5.996 & 5.996 & 5.996 & 5.996 & 5.996 & 5.996 & 5.996 & 5.996 & 5.996 & 5.997 & 5.889 & 5.912 & 5.929 & 5.941 & 5.927 & 5.948 & 5.94 & 5.949 & 5.941 & 4.607 \\ 
   &  & p=500 & 5.996 & 5.996 & 5.996 & 5.996 & 5.996 & 5.996 & 5.996 & 5.996 & 5.996 & 5.997 & 6.245 & 6.299 & 6.288 & 6.377 & 6.341 & 6.367 & 6.379 & 6.408 & 6.384 & 4.592 \\ 
   &  & p=996 & 5.996 & 5.996 & 6.274 & 402.645 & 5.996 & 5.996 & 5.996 & 595.518 & 5.996 & 373.858 & 6.245 & 6.299 & 94.332 & 262.293 & 6.341 & 6.367 & 6.379 & 361.715 & 6.384 & 102.727 \\ 
  \midrule\multirow{9}[6]{*}{lsnr} & \multirow{3}[2]{*}{$\rho=0$} & p=12 & 6.1 & 6.1 & 6.1 & 6.1 & 6.1 & 6.1 & 6.1 & 6.1 & 6.1 & 6.101 & 6.488 & 6.499 & 6.506 & 6.529 & 6.565 & 6.527 & 6.524 & 6.552 & 6.526 & 5.342 \\ 
   &  & p=500 & 6.1 & 6.101 & 6.101 & 6.101 & 6.101 & 6.101 & 6.101 & 6.101 & 6.101 & 6.101 & 6.796 & 6.857 & 6.857 & 6.945 & 6.999 & 6.953 & 6.939 & 7.002 & 6.941 & 5.342 \\ 
   &  & p=996 & 6.1 & 6.101 & 6.378 & 393.735 & 6.101 & 6.101 & 6.101 & 591.88 & 6.101 & 372.177 & 6.796 & 6.857 & 94.952 & 262.1 & 6.999 & 6.953 & 6.939 & 359.11 & 6.941 & 102.416 \\ 
  \cmidrule{2-23} & \multirow{3}[2]{*}{$\rho=0.5$} & p=12 & 6.904 & 6.904 & 6.904 & 6.904 & 6.904 & 6.904 & 6.904 & 6.904 & 6.904 & 6.906 & 6.083 & 6.098 & 6.113 & 6.125 & 6.141 & 6.122 & 6.12 & 6.143 & 6.125 & 4.631 \\ 
   &  & p=500 & 6.904 & 6.904 & 6.904 & 6.905 & 6.905 & 6.904 & 6.904 & 6.905 & 6.904 & 6.906 & 6.391 & 6.448 & 6.489 & 6.532 & 6.52 & 6.465 & 6.456 & 6.499 & 6.456 & 4.632 \\ 
   &  & p=996 & 6.904 & 6.904 & 7.182 & 396.863 & 6.905 & 6.904 & 6.904 & 602.039 & 6.904 & 372.972 & 6.391 & 6.448 & 94.55 & 262.534 & 6.52 & 6.465 & 6.456 & 361.817 & 6.456 & 101.768 \\ 
  \cmidrule{2-23} & \multirow{3}[2]{*}{$\rho=0.9$} & p=12 & 7.605 & 7.605 & 7.605 & 7.605 & 7.605 & 7.605 & 7.605 & 7.605 & 7.605 & 7.608 & 4.874 & 4.898 & 4.908 & 4.925 & 4.909 & 4.92 & 4.92 & 4.938 & 4.92 & 2.948 \\ 
   &  & p=500 & 7.605 & 7.605 & 7.605 & 7.605 & 7.605 & 7.605 & 7.605 & 7.605 & 7.605 & 7.608 & 5.147 & 5.224 & 5.205 & 5.266 & 5.236 & 5.288 & 5.264 & 5.347 & 5.264 & 2.945 \\ 
   &  & p=996 & 7.605 & 7.605 & 7.883 & 404.255 & 7.605 & 7.605 & 7.605 & 596.673 & 7.605 & 371.682 & 5.147 & 5.224 & 93.577 & 261.488 & 5.236 & 5.288 & 5.264 & 360.038 & 5.264 & 99.272 \\ 
   \bottomrule 
\end{tabular}
}
\end{table}

% latex table generated in R 4.0.2 by xtable 1.8-4 package
% Fri Jun 26 10:43:45 2020
\begin{table}[ht]
\centering
\caption{Fixed X, Sparse-Ex2, n=40} 
\scalebox{0.52}{
\begin{tabular}{|c|c|c|cc|cc|cc|ccc|c||cc|cc|cc|ccc|c|}
  \toprule 
 \multicolumn{1}{|c}{} & \multicolumn{1}{c}{} &       & RAICc & AICc  & RC$_p$ & C$_p$ & 10F CV & LOO CV & S$_p$ & $\widehat{\text{C}}_p$ & GCV   & BIC   & RAICc & AICc  & RC$_p$ & C$_p$ & 10F CV & LOO CV & S$_p$ & $\widehat{\text{C}}_p$ & GCV   & BIC  \\
 \cmidrule{4-23}\multicolumn{1}{|c}{} & \multicolumn{1}{c}{} &       & \multicolumn{10}{c||}{RMSE}                                                   & \multicolumn{10}{c|}{Relative Risk}       \\
 \midrule\multirow{9}[6]{*}{hsnr} & \multirow{3}[2]{*}{$\rho=0$} & p=12 & 0.985 & 1.016 & 1.033 & 1.088 & 1.082 & 1.076 & 1.081 & 1.124 & 1.084 & 1.005 & 0.021 & 0.023 & 0.023 & 0.026 & 0.026 & 0.026 & 0.026 & 0.028 & 0.026 & 0.022 \\ 
   &  & p=20 & 0.985 & 1.019 & 1.055 & 1.157 & 1.121 & 1.123 & 1.127 & 1.251 & 1.132 & 1.017 & 0.021 & 0.023 & 0.025 & 0.035 & 0.03 & 0.031 & 0.031 & 0.044 & 0.032 & 0.023 \\ 
   &  & p=36 & 0.985 & 1.019 & 1.185 & 1.442 & 1.136 & 1.164 & 1.162 & 1.721 & 1.191 & 1.4 & 0.021 & 0.023 & 0.046 & 0.207 & 0.034 & 0.048 & 0.048 & 0.33 & 0.061 & 0.245 \\ 
  \cmidrule{2-23} & \multirow{3}[2]{*}{$\rho=0.5$} & p=12 & 1.421 & 1.461 & 1.486 & 1.556 & 1.553 & 1.551 & 1.541 & 1.62 & 1.547 & 1.449 & 0.02 & 0.022 & 0.023 & 0.025 & 0.025 & 0.025 & 0.025 & 0.028 & 0.025 & 0.021 \\ 
   &  & p=20 & 1.423 & 1.464 & 1.511 & 1.66 & 1.609 & 1.616 & 1.608 & 1.804 & 1.619 & 1.46 & 0.02 & 0.022 & 0.024 & 0.034 & 0.029 & 0.03 & 0.03 & 0.043 & 0.031 & 0.022 \\ 
   &  & p=36 & 1.423 & 1.464 & 1.701 & 2.071 & 1.631 & 1.669 & 1.659 & 2.476 & 1.702 & 2.008 & 0.02 & 0.022 & 0.044 & 0.198 & 0.033 & 0.047 & 0.047 & 0.315 & 0.06 & 0.237 \\ 
  \cmidrule{2-23} & \multirow{3}[2]{*}{$\rho=0.9$} & p=12 & 2.056 & 2.082 & 2.098 & 2.164 & 2.171 & 2.147 & 2.14 & 2.227 & 2.146 & 2.097 & 0.023 & 0.024 & 0.025 & 0.026 & 0.026 & 0.026 & 0.026 & 0.028 & 0.026 & 0.025 \\ 
   &  & p=20 & 2.055 & 2.088 & 2.137 & 2.312 & 2.249 & 2.244 & 2.232 & 2.455 & 2.241 & 2.111 & 0.023 & 0.024 & 0.026 & 0.036 & 0.031 & 0.031 & 0.031 & 0.043 & 0.032 & 0.025 \\ 
   &  & p=36 & 2.054 & 2.09 & 2.403 & 2.882 & 2.276 & 2.322 & 2.302 & 3.355 & 2.364 & 2.816 & 0.023 & 0.024 & 0.045 & 0.197 & 0.034 & 0.048 & 0.047 & 0.306 & 0.06 & 0.229 \\ 
  \midrule\multirow{9}[6]{*}{msnr} & \multirow{3}[2]{*}{$\rho=0$} & p=12 & 2.94 & 2.995 & 3.032 & 3.182 & 3.188 & 3.155 & 3.165 & 3.284 & 3.171 & 3.245 & 0.191 & 0.198 & 0.203 & 0.226 & 0.229 & 0.222 & 0.224 & 0.243 & 0.224 & 0.245 \\ 
   &  & p=20 & 2.94 & 3.005 & 3.1 & 3.384 & 3.3 & 3.292 & 3.3 & 3.653 & 3.311 & 3.281 & 0.191 & 0.201 & 0.22 & 0.303 & 0.266 & 0.267 & 0.269 & 0.377 & 0.271 & 0.257 \\ 
   &  & p=36 & 2.94 & 3.005 & 3.619 & 4.337 & 3.345 & 3.412 & 3.403 & 5.021 & 3.483 & 4.322 & 0.191 & 0.201 & 0.424 & 1.783 & 0.295 & 0.413 & 0.412 & 2.804 & 0.523 & 2.101 \\ 
  \cmidrule{2-23} & \multirow{3}[2]{*}{$\rho=0.5$} & p=12 & 4.514 & 4.512 & 4.538 & 4.66 & 4.68 & 4.678 & 4.634 & 4.795 & 4.638 & 4.859 & 0.211 & 0.211 & 0.214 & 0.228 & 0.229 & 0.228 & 0.224 & 0.243 & 0.224 & 0.255 \\ 
   &  & p=20 & 4.519 & 4.52 & 4.618 & 4.976 & 4.832 & 4.86 & 4.824 & 5.311 & 4.841 & 4.886 & 0.212 & 0.213 & 0.228 & 0.305 & 0.263 & 0.27 & 0.269 & 0.37 & 0.272 & 0.262 \\ 
   &  & p=36 & 4.519 & 4.52 & 5.39 & 6.35 & 4.895 & 5.01 & 4.969 & 7.239 & 5.082 & 6.299 & 0.212 & 0.213 & 0.424 & 1.721 & 0.29 & 0.41 & 0.413 & 2.665 & 0.52 & 1.98 \\ 
  \cmidrule{2-23} & \multirow{3}[2]{*}{$\rho=0.9$} & p=12 & 6.274 & 6.216 & 6.209 & 6.312 & 6.373 & 6.297 & 6.26 & 6.456 & 6.264 & 6.834 & 0.199 & 0.2 & 0.202 & 0.215 & 0.215 & 0.213 & 0.209 & 0.229 & 0.21 & 0.232 \\ 
   &  & p=20 & 6.289 & 6.214 & 6.31 & 6.736 & 6.545 & 6.529 & 6.482 & 7.075 & 6.505 & 6.858 & 0.2 & 0.201 & 0.216 & 0.292 & 0.245 & 0.254 & 0.246 & 0.342 & 0.249 & 0.236 \\ 
   &  & p=36 & 6.29 & 6.229 & 7.309 & 8.637 & 6.623 & 6.733 & 6.671 & 9.692 & 6.835 & 8.598 & 0.2 & 0.201 & 0.389 & 1.667 & 0.263 & 0.384 & 0.374 & 2.554 & 0.473 & 1.828 \\ 
  \midrule\multirow{9}[6]{*}{lsnr} & \multirow{3}[2]{*}{$\rho=0$} & p=12 & 7.677 & 7.649 & 7.652 & 7.758 & 7.755 & 7.717 & 7.732 & 7.837 & 7.74 & 7.608 & 1.13 & 1.146 & 1.174 & 1.229 & 1.215 & 1.207 & 1.209 & 1.272 & 1.213 & 1.099 \\ 
   &  & p=20 & 7.677 & 7.652 & 7.761 & 8.139 & 7.925 & 7.926 & 7.911 & 8.494 & 7.927 & 7.621 & 1.13 & 1.149 & 1.24 & 1.568 & 1.35 & 1.374 & 1.345 & 1.852 & 1.352 & 1.108 \\ 
   &  & p=36 & 7.677 & 7.652 & 8.464 & 9.997 & 7.979 & 8.155 & 8.098 & 11.332 & 8.203 & 9.347 & 1.13 & 1.149 & 2.103 & 8.878 & 1.447 & 2.069 & 1.981 & 13.479 & 2.374 & 8.18 \\ 
  \cmidrule{2-23} & \multirow{3}[2]{*}{$\rho=0.5$} & p=12 & 10.871 & 10.864 & 10.855 & 10.958 & 10.938 & 10.933 & 10.926 & 11.069 & 10.926 & 10.929 & 1.076 & 1.094 & 1.113 & 1.156 & 1.137 & 1.143 & 1.144 & 1.202 & 1.146 & 1.08 \\ 
   &  & p=20 & 10.871 & 10.877 & 11 & 11.479 & 11.17 & 11.224 & 11.183 & 11.995 & 11.223 & 10.943 & 1.076 & 1.102 & 1.178 & 1.485 & 1.269 & 1.3 & 1.271 & 1.734 & 1.297 & 1.089 \\ 
   &  & p=36 & 10.871 & 10.877 & 11.966 & 14.21 & 11.244 & 11.498 & 11.428 & 16.121 & 11.642 & 13.342 & 1.076 & 1.102 & 1.972 & 8.465 & 1.339 & 1.951 & 1.865 & 12.946 & 2.38 & 7.689 \\ 
  \cmidrule{2-23} & \multirow{3}[2]{*}{$\rho=0.9$} & p=12 & 12.638 & 12.752 & 12.942 & 13.304 & 12.992 & 13.168 & 13.131 & 13.541 & 13.154 & 13.306 & 0.726 & 0.764 & 0.815 & 0.887 & 0.826 & 0.864 & 0.854 & 0.945 & 0.86 & 0.799 \\ 
   &  & p=20 & 12.611 & 12.77 & 13.23 & 14.076 & 13.319 & 13.5 & 13.43 & 14.802 & 13.545 & 13.299 & 0.721 & 0.769 & 0.89 & 1.218 & 0.95 & 0.997 & 0.969 & 1.442 & 1.007 & 0.802 \\ 
   &  & p=36 & 12.613 & 12.77 & 14.976 & 18.072 & 13.363 & 13.902 & 13.831 & 20.805 & 14.012 & 16.771 & 0.721 & 0.768 & 1.683 & 8.083 & 0.982 & 1.509 & 1.58 & 12.15 & 1.767 & 7.158 \\ 
   \midrule 
 \multicolumn{1}{|c}{} & \multicolumn{1}{c}{} &       & \multicolumn{10}{c||}{Relative Test Error}                                    & \multicolumn{10}{c|}{Proportion of Variance Explained} \\
\midrule\multirow{9}[6]{*}{hsnr} & \multirow{3}[2]{*}{$\rho=0$} & p=12 & 1.229 & 1.247 & 1.256 & 1.288 & 1.287 & 1.281 & 1.284 & 1.311 & 1.286 & 1.241 & 0.897 & 0.896 & 0.895 & 0.892 & 0.892 & 0.893 & 0.892 & 0.89 & 0.892 & 0.896 \\ 
   &  & p=20 & 1.229 & 1.251 & 1.278 & 1.387 & 1.333 & 1.34 & 1.342 & 1.484 & 1.346 & 1.255 & 0.897 & 0.895 & 0.893 & 0.884 & 0.888 & 0.888 & 0.888 & 0.876 & 0.887 & 0.895 \\ 
   &  & p=36 & 1.229 & 1.251 & 1.507 & 3.263 & 1.372 & 1.527 & 1.526 & 4.606 & 1.67 & 3.682 & 0.897 & 0.895 & 0.874 & 0.727 & 0.885 & 0.872 & 0.872 & 0.614 & 0.86 & 0.692 \\ 
  \cmidrule{2-23} & \multirow{3}[2]{*}{$\rho=0.5$} & p=12 & 1.227 & 1.244 & 1.255 & 1.284 & 1.283 & 1.281 & 1.278 & 1.311 & 1.28 & 1.24 & 0.899 & 0.898 & 0.897 & 0.895 & 0.895 & 0.895 & 0.895 & 0.893 & 0.895 & 0.898 \\ 
   &  & p=20 & 1.228 & 1.247 & 1.273 & 1.386 & 1.33 & 1.338 & 1.338 & 1.485 & 1.344 & 1.251 & 0.899 & 0.898 & 0.896 & 0.886 & 0.891 & 0.89 & 0.89 & 0.878 & 0.89 & 0.898 \\ 
   &  & p=36 & 1.228 & 1.247 & 1.496 & 3.221 & 1.366 & 1.524 & 1.529 & 4.527 & 1.67 & 3.657 & 0.899 & 0.898 & 0.877 & 0.736 & 0.888 & 0.875 & 0.875 & 0.629 & 0.863 & 0.7 \\ 
  \cmidrule{2-23} & \multirow{3}[2]{*}{$\rho=0.9$} & p=12 & 1.268 & 1.275 & 1.282 & 1.301 & 1.303 & 1.295 & 1.293 & 1.321 & 1.295 & 1.281 & 0.898 & 0.897 & 0.897 & 0.895 & 0.895 & 0.896 & 0.896 & 0.894 & 0.896 & 0.897 \\ 
   &  & p=20 & 1.268 & 1.28 & 1.302 & 1.412 & 1.355 & 1.36 & 1.355 & 1.488 & 1.361 & 1.291 & 0.898 & 0.897 & 0.895 & 0.886 & 0.891 & 0.891 & 0.891 & 0.88 & 0.891 & 0.896 \\ 
   &  & p=36 & 1.267 & 1.28 & 1.518 & 3.253 & 1.384 & 1.545 & 1.539 & 4.495 & 1.691 & 3.615 & 0.898 & 0.897 & 0.878 & 0.738 & 0.889 & 0.876 & 0.876 & 0.638 & 0.864 & 0.709 \\ 
  \midrule\multirow{9}[6]{*}{msnr} & \multirow{3}[2]{*}{$\rho=0$} & p=12 & 1.245 & 1.255 & 1.261 & 1.291 & 1.295 & 1.286 & 1.288 & 1.313 & 1.289 & 1.316 & 0.455 & 0.451 & 0.448 & 0.436 & 0.434 & 0.438 & 0.437 & 0.426 & 0.436 & 0.425 \\ 
   &  & p=20 & 1.245 & 1.259 & 1.283 & 1.389 & 1.342 & 1.344 & 1.346 & 1.485 & 1.349 & 1.33 & 0.455 & 0.449 & 0.439 & 0.392 & 0.413 & 0.412 & 0.411 & 0.35 & 0.41 & 0.418 \\ 
   &  & p=36 & 1.245 & 1.259 & 1.546 & 3.294 & 1.38 & 1.532 & 1.53 & 4.607 & 1.673 & 3.704 & 0.455 & 0.449 & 0.324 & -0.441 & 0.397 & 0.33 & 0.331 & -1.015 & 0.268 & -0.62 \\ 
  \cmidrule{2-23} & \multirow{3}[2]{*}{$\rho=0.5$} & p=12 & 1.278 & 1.278 & 1.283 & 1.3 & 1.302 & 1.301 & 1.295 & 1.32 & 1.296 & 1.336 & 0.448 & 0.449 & 0.447 & 0.439 & 0.438 & 0.439 & 0.441 & 0.431 & 0.441 & 0.423 \\ 
   &  & p=20 & 1.279 & 1.281 & 1.301 & 1.401 & 1.346 & 1.355 & 1.354 & 1.488 & 1.358 & 1.346 & 0.448 & 0.447 & 0.439 & 0.395 & 0.419 & 0.415 & 0.416 & 0.358 & 0.414 & 0.419 \\ 
   &  & p=36 & 1.279 & 1.281 & 1.558 & 3.268 & 1.382 & 1.54 & 1.545 & 4.513 & 1.685 & 3.609 & 0.448 & 0.447 & 0.328 & -0.41 & 0.404 & 0.335 & 0.334 & -0.947 & 0.273 & -0.557 \\ 
  \cmidrule{2-23} & \multirow{3}[2]{*}{$\rho=0.9$} & p=12 & 1.268 & 1.269 & 1.272 & 1.289 & 1.29 & 1.286 & 1.282 & 1.309 & 1.282 & 1.312 & 0.459 & 0.459 & 0.457 & 0.45 & 0.45 & 0.452 & 0.453 & 0.442 & 0.453 & 0.441 \\ 
   &  & p=20 & 1.27 & 1.271 & 1.291 & 1.393 & 1.329 & 1.341 & 1.331 & 1.46 & 1.335 & 1.317 & 0.459 & 0.458 & 0.45 & 0.406 & 0.433 & 0.428 & 0.432 & 0.377 & 0.431 & 0.438 \\ 
   &  & p=36 & 1.269 & 1.271 & 1.524 & 3.243 & 1.354 & 1.516 & 1.503 & 4.436 & 1.637 & 3.459 & 0.459 & 0.458 & 0.35 & -0.383 & 0.423 & 0.354 & 0.359 & -0.891 & 0.302 & -0.475 \\ 
  \midrule\multirow{9}[6]{*}{lsnr} & \multirow{3}[2]{*}{$\rho=0$} & p=12 & 1.291 & 1.295 & 1.302 & 1.316 & 1.313 & 1.311 & 1.311 & 1.327 & 1.312 & 1.283 & -0.027 & -0.03 & -0.036 & -0.047 & -0.044 & -0.042 & -0.043 & -0.056 & -0.044 & -0.02 \\ 
   &  & p=20 & 1.291 & 1.296 & 1.319 & 1.404 & 1.347 & 1.354 & 1.346 & 1.477 & 1.348 & 1.285 & -0.027 & -0.03 & -0.049 & -0.116 & -0.072 & -0.077 & -0.071 & -0.174 & -0.072 & -0.022 \\ 
   &  & p=36 & 1.291 & 1.296 & 1.541 & 3.284 & 1.372 & 1.532 & 1.51 & 4.468 & 1.611 & 3.105 & -0.027 & -0.03 & -0.226 & -1.612 & -0.091 & -0.219 & -0.201 & -2.554 & -0.281 & -1.469 \\ 
  \cmidrule{2-23} & \multirow{3}[2]{*}{$\rho=0.5$} & p=12 & 1.284 & 1.288 & 1.293 & 1.305 & 1.3 & 1.301 & 1.301 & 1.317 & 1.302 & 1.285 & -0.016 & -0.02 & -0.024 & -0.033 & -0.029 & -0.03 & -0.03 & -0.042 & -0.031 & -0.017 \\ 
   &  & p=20 & 1.284 & 1.29 & 1.311 & 1.391 & 1.334 & 1.343 & 1.335 & 1.457 & 1.342 & 1.287 & -0.016 & -0.021 & -0.037 & -0.101 & -0.056 & -0.063 & -0.057 & -0.153 & -0.062 & -0.018 \\ 
   &  & p=36 & 1.284 & 1.29 & 1.52 & 3.231 & 1.353 & 1.514 & 1.492 & 4.412 & 1.627 & 3.027 & -0.016 & -0.021 & -0.203 & -1.557 & -0.071 & -0.198 & -0.18 & -2.492 & -0.288 & -1.395 \\ 
  \cmidrule{2-23} & \multirow{3}[2]{*}{$\rho=0.9$} & p=12 & 1.195 & 1.206 & 1.219 & 1.239 & 1.222 & 1.233 & 1.23 & 1.254 & 1.231 & 1.215 & 0.058 & 0.05 & 0.039 & 0.024 & 0.037 & 0.029 & 0.031 & 0.012 & 0.03 & 0.043 \\ 
   &  & p=20 & 1.194 & 1.207 & 1.239 & 1.328 & 1.256 & 1.268 & 1.261 & 1.388 & 1.271 & 1.216 & 0.059 & 0.049 & 0.023 & -0.046 & 0.011 & 0.001 & 0.007 & -0.094 & -0.001 & 0.042 \\ 
   &  & p=36 & 1.194 & 1.207 & 1.453 & 3.175 & 1.264 & 1.406 & 1.425 & 4.269 & 1.475 & 2.926 & 0.059 & 0.049 & -0.145 & -1.502 & 0.004 & -0.108 & -0.123 & -2.364 & -0.163 & -1.306 \\ 
   \midrule 
 \multicolumn{1}{|c}{} & \multicolumn{1}{c}{} &       & \multicolumn{10}{c||}{Kullback-Leibler Discrepancy}                                    & \multicolumn{10}{c|}{Number of Variables} \\
\midrule\multirow{9}[6]{*}{hsnr} & \multirow{3}[2]{*}{$\rho=0$} & p=12 & 3.144 & 3.178 & 3.192 & 3.242 & 3.233 & 3.229 & 3.236 & 3.276 & 3.238 & 3.173 & 6.148 & 6.319 & 6.418 & 6.815 & 6.868 & 6.758 & 6.758 & 7.138 & 6.778 & 6.274 \\ 
   &  & p=20 & 3.144 & 3.186 & 3.244 & 3.459 & 3.346 & 3.378 & 3.376 & 3.647 & 3.385 & 3.218 & 6.148 & 6.343 & 6.624 & 7.65 & 7.301 & 7.289 & 7.241 & 8.775 & 7.29 & 6.391 \\ 
   &  & p=36 & 3.144 & 3.186 & 5.239 & 21.118 & 3.474 & 8.63 & 9.569 & 27.251 & 11.117 & 26.12 & 6.148 & 6.343 & 8.522 & 13.668 & 7.526 & 8.022 & 7.939 & 18.913 & 8.414 & 14.278 \\ 
  \cmidrule{2-23} & \multirow{3}[2]{*}{$\rho=0.5$} & p=12 & 3.87 & 3.902 & 3.918 & 3.964 & 3.958 & 3.957 & 3.955 & 4.004 & 3.958 & 3.897 & 6.154 & 6.31 & 6.417 & 6.78 & 6.866 & 6.774 & 6.694 & 7.16 & 6.721 & 6.271 \\ 
   &  & p=20 & 3.873 & 3.907 & 3.966 & 4.188 & 4.068 & 4.1 & 4.104 & 4.376 & 4.112 & 3.932 & 6.162 & 6.322 & 6.591 & 7.657 & 7.299 & 7.296 & 7.203 & 8.808 & 7.261 & 6.349 \\ 
   &  & p=36 & 3.873 & 3.907 & 5.92 & 21.579 & 4.203 & 9.376 & 10.344 & 27.875 & 11.658 & 26.724 & 6.162 & 6.322 & 8.502 & 13.627 & 7.524 & 7.969 & 7.91 & 18.97 & 8.362 & 14.214 \\ 
  \cmidrule{2-23} & \multirow{3}[2]{*}{$\rho=0.9$} & p=12 & 4.492 & 4.52 & 4.532 & 4.575 & 4.57 & 4.564 & 4.563 & 4.61 & 4.566 & 4.515 & 5.879 & 6.088 & 6.191 & 6.614 & 6.643 & 6.528 & 6.498 & 7.017 & 6.539 & 5.95 \\ 
   &  & p=20 & 4.49 & 4.524 & 4.578 & 4.807 & 4.683 & 4.706 & 4.704 & 4.963 & 4.714 & 4.541 & 5.88 & 6.104 & 6.369 & 7.504 & 7.111 & 7.087 & 7.004 & 8.528 & 7.061 & 6.018 \\ 
   &  & p=36 & 4.49 & 4.526 & 6.451 & 22.223 & 4.88 & 9.837 & 10.885 & 28.535 & 12.412 & 27.206 & 5.878 & 6.115 & 8.235 & 13.484 & 7.318 & 7.851 & 7.735 & 18.884 & 8.29 & 13.799 \\ 
  \midrule\multirow{9}[6]{*}{msnr} & \multirow{3}[2]{*}{$\rho=0$} & p=12 & 5.293 & 5.323 & 5.335 & 5.384 & 5.378 & 5.372 & 5.379 & 5.417 & 5.38 & 5.351 & 6.059 & 6.272 & 6.392 & 6.803 & 6.823 & 6.734 & 6.74 & 7.133 & 6.766 & 5.857 \\ 
   &  & p=20 & 5.293 & 5.331 & 5.388 & 5.601 & 5.49 & 5.521 & 5.519 & 5.788 & 5.527 & 5.397 & 6.059 & 6.296 & 6.596 & 7.638 & 7.256 & 7.265 & 7.223 & 8.77 & 7.278 & 5.974 \\ 
   &  & p=36 & 5.293 & 5.331 & 7.399 & 23.274 & 5.618 & 10.773 & 11.711 & 29.391 & 13.259 & 28.219 & 6.059 & 6.296 & 8.313 & 13.501 & 7.481 & 7.998 & 7.921 & 18.91 & 8.402 & 13.893 \\ 
  \cmidrule{2-23} & \multirow{3}[2]{*}{$\rho=0.5$} & p=12 & 6.037 & 6.06 & 6.072 & 6.111 & 6.104 & 6.105 & 6.103 & 6.146 & 6.105 & 6.083 & 5.784 & 6.036 & 6.178 & 6.574 & 6.619 & 6.523 & 6.474 & 6.985 & 6.504 & 5.621 \\ 
   &  & p=20 & 6.04 & 6.065 & 6.122 & 6.336 & 6.213 & 6.246 & 6.251 & 6.512 & 6.258 & 6.117 & 5.792 & 6.049 & 6.359 & 7.449 & 7.058 & 7.035 & 6.985 & 8.609 & 7.047 & 5.698 \\ 
   &  & p=36 & 6.04 & 6.065 & 8.091 & 23.743 & 6.348 & 11.521 & 12.491 & 29.976 & 13.805 & 28.566 & 5.792 & 6.049 & 8.086 & 13.329 & 7.286 & 7.71 & 7.695 & 18.776 & 8.158 & 13.413 \\ 
  \cmidrule{2-23} & \multirow{3}[2]{*}{$\rho=0.9$} & p=12 & 6.627 & 6.644 & 6.657 & 6.695 & 6.687 & 6.687 & 6.683 & 6.729 & 6.685 & 6.663 & 4.474 & 4.784 & 5.037 & 5.499 & 5.406 & 5.434 & 5.359 & 5.956 & 5.386 & 4.1 \\ 
   &  & p=20 & 6.629 & 6.646 & 6.701 & 6.918 & 6.783 & 6.818 & 6.803 & 7.065 & 6.811 & 6.686 & 4.494 & 4.807 & 5.2 & 6.414 & 5.774 & 5.95 & 5.825 & 7.509 & 5.903 & 4.172 \\ 
   &  & p=36 & 6.63 & 6.648 & 8.592 & 24.31 & 6.969 & 11.87 & 12.962 & 30.559 & 14.062 & 28.56 & 4.494 & 4.821 & 7.116 & 12.436 & 5.973 & 6.663 & 6.532 & 18.064 & 7.037 & 11.738 \\ 
  \midrule\multirow{9}[6]{*}{lsnr} & \multirow{3}[2]{*}{$\rho=0$} & p=12 & 6.905 & 6.934 & 6.96 & 6.993 & 6.975 & 6.979 & 6.982 & 7.02 & 6.984 & 6.877 & 2 & 2.687 & 3.516 & 4.149 & 3.765 & 3.877 & 3.843 & 4.606 & 3.917 & 0.868 \\ 
   &  & p=20 & 6.905 & 6.938 & 7.006 & 7.193 & 7.069 & 7.107 & 7.099 & 7.359 & 7.104 & 6.896 & 2 & 2.692 & 3.612 & 4.995 & 4.16 & 4.364 & 4.233 & 6.274 & 4.322 & 0.907 \\ 
   &  & p=36 & 6.905 & 6.938 & 8.993 & 24.862 & 7.167 & 12.337 & 13.146 & 30.773 & 14.252 & 27.66 & 2 & 2.692 & 5.803 & 11.218 & 4.339 & 5.131 & 4.93 & 16.947 & 5.32 & 8.123 \\ 
  \cmidrule{2-23} & \multirow{3}[2]{*}{$\rho=0.5$} & p=12 & 7.625 & 7.648 & 7.673 & 7.703 & 7.682 & 7.693 & 7.693 & 7.733 & 7.696 & 7.602 & 2.225 & 2.756 & 3.436 & 3.965 & 3.649 & 3.799 & 3.737 & 4.453 & 3.827 & 1.018 \\ 
   &  & p=20 & 7.625 & 7.654 & 7.719 & 7.905 & 7.769 & 7.816 & 7.808 & 8.055 & 7.823 & 7.619 & 2.225 & 2.779 & 3.6 & 4.908 & 4.039 & 4.264 & 4.14 & 6.022 & 4.275 & 1.047 \\ 
   &  & p=36 & 7.625 & 7.654 & 9.662 & 25.296 & 7.868 & 13.031 & 13.793 & 31.367 & 15.022 & 27.948 & 2.225 & 2.779 & 5.72 & 11.083 & 4.202 & 4.939 & 4.76 & 16.765 & 5.298 & 7.924 \\ 
  \cmidrule{2-23} & \multirow{3}[2]{*}{$\rho=0.9$} & p=12 & 8.165 & 8.182 & 8.211 & 8.249 & 8.218 & 8.236 & 8.231 & 8.276 & 8.236 & 8.163 & 2.014 & 2.305 & 2.747 & 3.242 & 2.96 & 3.06 & 3.052 & 3.649 & 3.116 & 1.176 \\ 
   &  & p=20 & 8.163 & 8.185 & 8.26 & 8.446 & 8.304 & 8.335 & 8.328 & 8.58 & 8.346 & 8.173 & 1.993 & 2.318 & 2.982 & 4.107 & 3.35 & 3.459 & 3.411 & 5.144 & 3.564 & 1.172 \\ 
   &  & p=36 & 8.163 & 8.185 & 10.132 & 25.847 & 8.44 & 13.124 & 14.328 & 31.793 & 14.938 & 28.262 & 1.995 & 2.322 & 5.308 & 10.713 & 3.406 & 4.162 & 4.098 & 16.011 & 4.403 & 7.76 \\ 
   \bottomrule 
\end{tabular}
}
\end{table}

% latex table generated in R 4.0.2 by xtable 1.8-4 package
% Fri Jun 26 10:43:45 2020
\begin{table}[ht]
\centering
\caption{Fixed X, Sparse-Ex2, n=200} 
\scalebox{0.52}{
\begin{tabular}{|c|c|c|cc|cc|cc|ccc|c||cc|cc|cc|ccc|c|}
  \toprule 
 \multicolumn{1}{|c}{} & \multicolumn{1}{c}{} &       & RAICc & AICc  & RC$_p$ & C$_p$ & 10F CV & LOO CV & S$_p$ & $\widehat{\text{C}}_p$ & GCV   & BIC   & RAICc & AICc  & RC$_p$ & C$_p$ & 10F CV & LOO CV & S$_p$ & $\widehat{\text{C}}_p$ & GCV   & BIC  \\
 \cmidrule{4-23}\multicolumn{1}{|c}{} & \multicolumn{1}{c}{} &       & \multicolumn{10}{c||}{RMSE}                                                   & \multicolumn{10}{c|}{Relative Risk}       \\
 \midrule\multirow{9}[6]{*}{hsnr} & \multirow{3}[2]{*}{$\rho=0$} & p=12 & 0.554 & 0.558 & 0.561 & 0.565 & 0.564 & 0.565 & 0.564 & 0.57 & 0.564 & 0.505 & 0.005 & 0.005 & 0.005 & 0.006 & 0.005 & 0.006 & 0.006 & 0.006 & 0.006 & 0.004 \\ 
   &  & p=100 & 0.562 & 0.568 & 0.574 & 0.588 & 0.584 & 0.585 & 0.582 & 0.598 & 0.582 & 0.505 & 0.006 & 0.006 & 0.006 & 0.006 & 0.006 & 0.006 & 0.006 & 0.007 & 0.006 & 0.004 \\ 
   &  & p=196 & 0.562 & 0.568 & 0.937 & 1.309 & 0.584 & 0.585 & 0.582 & 1.589 & 0.585 & 0.89 & 0.006 & 0.006 & 0.042 & 1.813 & 0.006 & 0.006 & 0.006 & 3.053 & 0.009 & 1.452 \\ 
  \cmidrule{2-23} & \multirow{3}[2]{*}{$\rho=0.5$} & p=12 & 0.807 & 0.811 & 0.816 & 0.821 & 0.822 & 0.819 & 0.821 & 0.828 & 0.821 & 0.738 & 0.005 & 0.005 & 0.005 & 0.005 & 0.005 & 0.005 & 0.005 & 0.006 & 0.005 & 0.004 \\ 
   &  & p=100 & 0.819 & 0.828 & 0.837 & 0.858 & 0.853 & 0.851 & 0.844 & 0.875 & 0.846 & 0.738 & 0.006 & 0.006 & 0.006 & 0.006 & 0.006 & 0.006 & 0.006 & 0.007 & 0.006 & 0.004 \\ 
   &  & p=196 & 0.819 & 0.828 & 1.371 & 1.913 & 0.853 & 0.851 & 0.844 & 2.313 & 0.85 & 1.303 & 0.006 & 0.006 & 0.042 & 1.855 & 0.006 & 0.006 & 0.006 & 3.039 & 0.009 & 1.466 \\ 
  \cmidrule{2-23} & \multirow{3}[2]{*}{$\rho=0.9$} & p=12 & 1.097 & 1.104 & 1.109 & 1.116 & 1.109 & 1.114 & 1.115 & 1.122 & 1.115 & 1.006 & 0.005 & 0.005 & 0.005 & 0.005 & 0.005 & 0.005 & 0.005 & 0.005 & 0.005 & 0.004 \\ 
   &  & p=100 & 1.106 & 1.125 & 1.13 & 1.167 & 1.151 & 1.157 & 1.153 & 1.178 & 1.155 & 1.006 & 0.005 & 0.006 & 0.006 & 0.006 & 0.006 & 0.006 & 0.006 & 0.006 & 0.006 & 0.004 \\ 
   &  & p=196 & 1.106 & 1.125 & 1.869 & 2.595 & 1.151 & 1.157 & 1.153 & 3.108 & 1.165 & 1.768 & 0.005 & 0.006 & 0.041 & 1.82 & 0.006 & 0.006 & 0.006 & 3 & 0.013 & 1.446 \\ 
  \midrule\multirow{9}[6]{*}{msnr} & \multirow{3}[2]{*}{$\rho=0$} & p=12 & 1.616 & 1.628 & 1.634 & 1.647 & 1.644 & 1.646 & 1.644 & 1.66 & 1.644 & 1.473 & 0.045 & 0.046 & 0.046 & 0.047 & 0.046 & 0.047 & 0.047 & 0.048 & 0.047 & 0.037 \\ 
   &  & p=100 & 1.638 & 1.657 & 1.674 & 1.715 & 1.701 & 1.706 & 1.696 & 1.742 & 1.696 & 1.473 & 0.047 & 0.049 & 0.05 & 0.054 & 0.052 & 0.053 & 0.052 & 0.056 & 0.052 & 0.037 \\ 
   &  & p=196 & 1.638 & 1.657 & 2.732 & 3.815 & 1.701 & 1.706 & 1.696 & 4.633 & 1.704 & 2.594 & 0.047 & 0.049 & 0.353 & 15.407 & 0.052 & 0.053 & 0.052 & 25.948 & 0.077 & 12.343 \\ 
  \cmidrule{2-23} & \multirow{3}[2]{*}{$\rho=0.5$} & p=12 & 2.356 & 2.366 & 2.379 & 2.396 & 2.399 & 2.389 & 2.393 & 2.413 & 2.395 & 2.174 & 0.045 & 0.046 & 0.046 & 0.047 & 0.047 & 0.046 & 0.047 & 0.047 & 0.047 & 0.039 \\ 
   &  & p=100 & 2.389 & 2.416 & 2.44 & 2.501 & 2.491 & 2.482 & 2.462 & 2.552 & 2.467 & 2.174 & 0.047 & 0.048 & 0.05 & 0.054 & 0.053 & 0.053 & 0.051 & 0.057 & 0.052 & 0.039 \\ 
   &  & p=196 & 2.389 & 2.416 & 3.999 & 5.58 & 2.491 & 2.482 & 2.462 & 6.743 & 2.478 & 3.817 & 0.047 & 0.048 & 0.354 & 15.763 & 0.053 & 0.053 & 0.051 & 25.834 & 0.076 & 12.46 \\ 
  \cmidrule{2-23} & \multirow{3}[2]{*}{$\rho=0.9$} & p=12 & 3.317 & 3.328 & 3.341 & 3.354 & 3.347 & 3.361 & 3.349 & 3.366 & 3.349 & 3.453 & 0.047 & 0.048 & 0.048 & 0.049 & 0.048 & 0.049 & 0.048 & 0.049 & 0.048 & 0.051 \\ 
   &  & p=100 & 3.338 & 3.392 & 3.405 & 3.497 & 3.458 & 3.483 & 3.457 & 3.526 & 3.464 & 3.451 & 0.049 & 0.051 & 0.051 & 0.056 & 0.053 & 0.055 & 0.053 & 0.057 & 0.054 & 0.051 \\ 
   &  & p=196 & 3.338 & 3.392 & 5.546 & 7.643 & 3.458 & 3.483 & 3.457 & 9.08 & 3.492 & 5.561 & 0.049 & 0.051 & 0.35 & 15.467 & 0.053 & 0.055 & 0.053 & 25.456 & 0.11 & 12.042 \\ 
  \midrule\multirow{9}[6]{*}{lsnr} & \multirow{3}[2]{*}{$\rho=0$} & p=12 & 3.639 & 3.664 & 3.676 & 3.705 & 3.704 & 3.702 & 3.695 & 3.733 & 3.696 & 4.327 & 0.23 & 0.233 & 0.235 & 0.238 & 0.237 & 0.238 & 0.237 & 0.241 & 0.237 & 0.341 \\ 
   &  & p=100 & 3.683 & 3.727 & 3.762 & 3.854 & 3.83 & 3.834 & 3.81 & 3.913 & 3.811 & 4.327 & 0.239 & 0.246 & 0.254 & 0.271 & 0.264 & 0.266 & 0.263 & 0.282 & 0.263 & 0.341 \\ 
   &  & p=196 & 3.683 & 3.727 & 6.25 & 8.655 & 3.83 & 3.834 & 3.81 & 10.373 & 3.829 & 6.695 & 0.239 & 0.246 & 1.787 & 77.054 & 0.264 & 0.266 & 0.263 & 129.744 & 0.387 & 61.378 \\ 
  \cmidrule{2-23} & \multirow{3}[2]{*}{$\rho=0.5$} & p=12 & 5.471 & 5.481 & 5.513 & 5.543 & 5.545 & 5.528 & 5.535 & 5.576 & 5.54 & 6.658 & 0.243 & 0.244 & 0.246 & 0.25 & 0.248 & 0.248 & 0.249 & 0.252 & 0.249 & 0.363 \\ 
   &  & p=100 & 5.541 & 5.597 & 5.644 & 5.752 & 5.724 & 5.725 & 5.683 & 5.863 & 5.692 & 6.658 & 0.252 & 0.258 & 0.266 & 0.281 & 0.274 & 0.277 & 0.271 & 0.297 & 0.272 & 0.363 \\ 
   &  & p=196 & 5.541 & 5.597 & 9.256 & 12.744 & 5.724 & 5.725 & 5.683 & 15.081 & 5.719 & 10.038 & 0.252 & 0.258 & 1.8 & 78.839 & 0.274 & 0.277 & 0.271 & 128.539 & 0.393 & 61.966 \\ 
  \cmidrule{2-23} & \multirow{3}[2]{*}{$\rho=0.9$} & p=12 & 7.545 & 7.557 & 7.575 & 7.583 & 7.588 & 7.548 & 7.586 & 7.592 & 7.585 & 9.362 & 0.238 & 0.239 & 0.241 & 0.242 & 0.241 & 0.24 & 0.242 & 0.243 & 0.242 & 0.329 \\ 
   &  & p=100 & 7.562 & 7.629 & 7.684 & 7.795 & 7.769 & 7.728 & 7.712 & 7.868 & 7.725 & 9.362 & 0.241 & 0.248 & 0.254 & 0.269 & 0.26 & 0.261 & 0.256 & 0.276 & 0.259 & 0.329 \\ 
   &  & p=196 & 7.562 & 7.629 & 12.643 & 17.313 & 7.769 & 7.728 & 7.712 & 20.168 & 7.787 & 13.645 & 0.241 & 0.248 & 1.755 & 77.057 & 0.26 & 0.261 & 0.256 & 126.394 & 0.537 & 58.637 \\ 
   \midrule 
 \multicolumn{1}{|c}{} & \multicolumn{1}{c}{} &       & \multicolumn{10}{c||}{Relative Test Error}                                    & \multicolumn{10}{c|}{Proportion of Variance Explained} \\
\midrule\multirow{9}[6]{*}{hsnr} & \multirow{3}[2]{*}{$\rho=0$} & p=12 & 1.041 & 1.042 & 1.042 & 1.043 & 1.043 & 1.043 & 1.043 & 1.044 & 1.043 & 1.034 & 0.881 & 0.881 & 0.881 & 0.881 & 0.881 & 0.881 & 0.881 & 0.881 & 0.881 & 0.882 \\ 
   &  & p=100 & 1.043 & 1.044 & 1.046 & 1.049 & 1.048 & 1.048 & 1.048 & 1.051 & 1.048 & 1.034 & 0.881 & 0.881 & 0.881 & 0.88 & 0.881 & 0.881 & 0.881 & 0.88 & 0.881 & 0.882 \\ 
   &  & p=196 & 1.043 & 1.044 & 1.323 & 15.091 & 1.048 & 1.048 & 1.048 & 24.733 & 1.07 & 12.289 & 0.881 & 0.881 & 0.849 & -0.72 & 0.881 & 0.881 & 0.881 & -1.819 & 0.878 & -0.401 \\ 
  \cmidrule{2-23} & \multirow{3}[2]{*}{$\rho=0.5$} & p=12 & 1.041 & 1.041 & 1.042 & 1.043 & 1.042 & 1.042 & 1.042 & 1.043 & 1.043 & 1.034 & 0.881 & 0.881 & 0.881 & 0.881 & 0.881 & 0.881 & 0.881 & 0.881 & 0.881 & 0.882 \\ 
   &  & p=100 & 1.043 & 1.044 & 1.046 & 1.049 & 1.048 & 1.048 & 1.047 & 1.052 & 1.047 & 1.034 & 0.881 & 0.88 & 0.88 & 0.88 & 0.88 & 0.88 & 0.88 & 0.88 & 0.88 & 0.882 \\ 
   &  & p=196 & 1.043 & 1.044 & 1.322 & 15.346 & 1.048 & 1.048 & 1.047 & 24.511 & 1.069 & 12.338 & 0.881 & 0.88 & 0.849 & -0.757 & 0.88 & 0.88 & 0.88 & -1.806 & 0.878 & -0.412 \\ 
  \cmidrule{2-23} & \multirow{3}[2]{*}{$\rho=0.9$} & p=12 & 1.041 & 1.042 & 1.042 & 1.043 & 1.042 & 1.042 & 1.042 & 1.043 & 1.042 & 1.035 & 0.882 & 0.882 & 0.882 & 0.882 & 0.882 & 0.882 & 0.882 & 0.882 & 0.882 & 0.883 \\ 
   &  & p=100 & 1.042 & 1.044 & 1.045 & 1.049 & 1.047 & 1.048 & 1.047 & 1.051 & 1.048 & 1.035 & 0.882 & 0.882 & 0.882 & 0.881 & 0.882 & 0.881 & 0.882 & 0.881 & 0.881 & 0.883 \\ 
   &  & p=196 & 1.042 & 1.044 & 1.32 & 15.263 & 1.047 & 1.048 & 1.047 & 24.518 & 1.099 & 12.334 & 0.882 & 0.882 & 0.851 & -0.727 & 0.882 & 0.881 & 0.882 & -1.774 & 0.876 & -0.395 \\ 
  \midrule\multirow{9}[6]{*}{msnr} & \multirow{3}[2]{*}{$\rho=0$} & p=12 & 1.041 & 1.042 & 1.042 & 1.043 & 1.043 & 1.043 & 1.043 & 1.044 & 1.043 & 1.034 & 0.456 & 0.456 & 0.456 & 0.455 & 0.455 & 0.455 & 0.455 & 0.455 & 0.455 & 0.46 \\ 
   &  & p=100 & 1.043 & 1.044 & 1.046 & 1.049 & 1.048 & 1.048 & 1.048 & 1.051 & 1.048 & 1.034 & 0.455 & 0.455 & 0.454 & 0.452 & 0.453 & 0.453 & 0.453 & 0.451 & 0.453 & 0.46 \\ 
   &  & p=196 & 1.043 & 1.044 & 1.323 & 15.091 & 1.048 & 1.048 & 1.048 & 24.733 & 1.07 & 12.289 & 0.455 & 0.455 & 0.309 & -6.882 & 0.453 & 0.453 & 0.453 & -11.918 & 0.441 & -5.419 \\ 
  \cmidrule{2-23} & \multirow{3}[2]{*}{$\rho=0.5$} & p=12 & 1.041 & 1.041 & 1.042 & 1.043 & 1.042 & 1.042 & 1.042 & 1.043 & 1.043 & 1.035 & 0.455 & 0.455 & 0.455 & 0.454 & 0.454 & 0.454 & 0.454 & 0.454 & 0.454 & 0.458 \\ 
   &  & p=100 & 1.043 & 1.044 & 1.046 & 1.049 & 1.048 & 1.048 & 1.047 & 1.052 & 1.047 & 1.035 & 0.454 & 0.453 & 0.453 & 0.451 & 0.451 & 0.451 & 0.452 & 0.449 & 0.452 & 0.458 \\ 
   &  & p=196 & 1.043 & 1.044 & 1.323 & 15.346 & 1.048 & 1.048 & 1.047 & 24.511 & 1.069 & 12.339 & 0.454 & 0.453 & 0.308 & -7.034 & 0.451 & 0.451 & 0.452 & -11.833 & 0.44 & -5.46 \\ 
  \cmidrule{2-23} & \multirow{3}[2]{*}{$\rho=0.9$} & p=12 & 1.044 & 1.044 & 1.044 & 1.045 & 1.044 & 1.045 & 1.045 & 1.045 & 1.045 & 1.047 & 0.457 & 0.457 & 0.457 & 0.456 & 0.457 & 0.456 & 0.457 & 0.456 & 0.457 & 0.455 \\ 
   &  & p=100 & 1.045 & 1.047 & 1.047 & 1.051 & 1.049 & 1.051 & 1.049 & 1.053 & 1.05 & 1.047 & 0.456 & 0.455 & 0.455 & 0.453 & 0.454 & 0.453 & 0.454 & 0.452 & 0.454 & 0.455 \\ 
   &  & p=196 & 1.045 & 1.047 & 1.322 & 15.264 & 1.049 & 1.051 & 1.049 & 24.475 & 1.101 & 12.105 & 0.456 & 0.455 & 0.312 & -6.941 & 0.454 & 0.453 & 0.454 & -11.733 & 0.427 & -5.297 \\ 
  \midrule\multirow{9}[6]{*}{lsnr} & \multirow{3}[2]{*}{$\rho=0$} & p=12 & 1.042 & 1.043 & 1.043 & 1.044 & 1.043 & 1.043 & 1.043 & 1.044 & 1.043 & 1.062 & 0.119 & 0.119 & 0.118 & 0.118 & 0.118 & 0.118 & 0.118 & 0.117 & 0.118 & 0.102 \\ 
   &  & p=100 & 1.044 & 1.045 & 1.046 & 1.05 & 1.048 & 1.049 & 1.048 & 1.052 & 1.048 & 1.062 & 0.118 & 0.117 & 0.115 & 0.113 & 0.114 & 0.113 & 0.114 & 0.111 & 0.114 & 0.102 \\ 
   &  & p=196 & 1.044 & 1.045 & 1.327 & 15.095 & 1.048 & 1.049 & 1.048 & 24.733 & 1.071 & 12.227 & 0.118 & 0.117 & -0.122 & -11.761 & 0.114 & 0.113 & 0.114 & -19.908 & 0.095 & -9.337 \\ 
  \cmidrule{2-23} & \multirow{3}[2]{*}{$\rho=0.5$} & p=12 & 1.044 & 1.044 & 1.045 & 1.045 & 1.045 & 1.045 & 1.045 & 1.046 & 1.045 & 1.066 & 0.117 & 0.116 & 0.116 & 0.116 & 0.116 & 0.116 & 0.116 & 0.115 & 0.116 & 0.098 \\ 
   &  & p=100 & 1.046 & 1.047 & 1.048 & 1.051 & 1.05 & 1.05 & 1.049 & 1.054 & 1.05 & 1.066 & 0.115 & 0.114 & 0.113 & 0.111 & 0.112 & 0.111 & 0.112 & 0.108 & 0.112 & 0.098 \\ 
   &  & p=196 & 1.046 & 1.047 & 1.328 & 15.35 & 1.05 & 1.05 & 1.049 & 24.396 & 1.071 & 12.279 & 0.115 & 0.114 & -0.123 & -11.986 & 0.112 & 0.111 & 0.112 & -19.639 & 0.094 & -9.388 \\ 
  \cmidrule{2-23} & \multirow{3}[2]{*}{$\rho=0.9$} & p=12 & 1.044 & 1.044 & 1.044 & 1.045 & 1.044 & 1.044 & 1.045 & 1.045 & 1.045 & 1.061 & 0.119 & 0.119 & 0.118 & 0.118 & 0.118 & 0.118 & 0.118 & 0.118 & 0.118 & 0.105 \\ 
   &  & p=100 & 1.044 & 1.046 & 1.047 & 1.05 & 1.048 & 1.048 & 1.047 & 1.051 & 1.048 & 1.061 & 0.118 & 0.117 & 0.116 & 0.114 & 0.115 & 0.115 & 0.116 & 0.113 & 0.115 & 0.105 \\ 
   &  & p=196 & 1.044 & 1.046 & 1.324 & 15.212 & 1.048 & 1.048 & 1.047 & 24.312 & 1.099 & 11.815 & 0.118 & 0.117 & -0.118 & -11.843 & 0.115 & 0.115 & 0.116 & -19.526 & 0.072 & -8.975 \\ 
   \midrule 
 \multicolumn{1}{|c}{} & \multicolumn{1}{c}{} &       & \multicolumn{10}{c||}{Kullback-Leibler Discrepancy}                                    & \multicolumn{10}{c|}{Number of Variables} \\
\midrule\multirow{9}[6]{*}{hsnr} & \multirow{3}[2]{*}{$\rho=0$} & p=12 & 3.244 & 3.245 & 3.245 & 3.246 & 3.246 & 3.246 & 3.246 & 3.247 & 3.246 & 3.236 & 6.508 & 6.562 & 6.595 & 6.657 & 6.7 & 6.66 & 6.639 & 6.726 & 6.64 & 6.032 \\ 
   &  & p=100 & 3.246 & 3.248 & 3.25 & 3.254 & 3.252 & 3.253 & 3.252 & 3.256 & 3.252 & 3.236 & 6.628 & 6.728 & 6.867 & 7.144 & 7.101 & 7.068 & 6.993 & 7.323 & 6.994 & 6.032 \\ 
   &  & p=196 & 3.246 & 3.248 & 3.692 & 107.663 & 3.252 & 3.253 & 3.252 & 147.06 & 4.497 & 110.571 & 6.628 & 6.728 & 25.908 & 59.763 & 7.101 & 7.068 & 6.993 & 84.379 & 7.18 & 35.067 \\ 
  \cmidrule{2-23} & \multirow{3}[2]{*}{$\rho=0.5$} & p=12 & 4 & 4 & 4.001 & 4.002 & 4.001 & 4.001 & 4.002 & 4.002 & 4.002 & 3.992 & 6.491 & 6.521 & 6.565 & 6.625 & 6.675 & 6.607 & 6.616 & 6.687 & 6.623 & 6.033 \\ 
   &  & p=100 & 4.002 & 4.003 & 4.005 & 4.009 & 4.008 & 4.008 & 4.007 & 4.013 & 4.007 & 3.992 & 6.612 & 6.708 & 6.843 & 7.134 & 7.119 & 7.034 & 6.924 & 7.369 & 6.943 & 6.033 \\ 
   &  & p=196 & 4.002 & 4.003 & 4.443 & 108.194 & 4.008 & 4.008 & 4.007 & 147.889 & 4.993 & 111.537 & 6.612 & 6.708 & 25.921 & 59.65 & 7.119 & 7.034 & 6.924 & 84.047 & 7.128 & 35.255 \\ 
  \cmidrule{2-23} & \multirow{3}[2]{*}{$\rho=0.9$} & p=12 & 4.609 & 4.61 & 4.61 & 4.611 & 4.61 & 4.611 & 4.611 & 4.611 & 4.611 & 4.602 & 6.497 & 6.544 & 6.581 & 6.629 & 6.641 & 6.628 & 6.624 & 6.671 & 6.624 & 6.02 \\ 
   &  & p=100 & 4.611 & 4.613 & 4.613 & 4.619 & 4.616 & 4.618 & 4.616 & 4.621 & 4.617 & 4.602 & 6.579 & 6.736 & 6.789 & 7.176 & 7.073 & 7.078 & 6.983 & 7.273 & 7.022 & 6.018 \\ 
   &  & p=196 & 4.611 & 4.613 & 5.031 & 107.89 & 4.616 & 4.618 & 4.616 & 146.532 & 8.365 & 112.263 & 6.579 & 6.736 & 26.084 & 59.637 & 7.073 & 7.078 & 6.983 & 83.068 & 7.393 & 35.245 \\ 
  \midrule\multirow{9}[6]{*}{msnr} & \multirow{3}[2]{*}{$\rho=0$} & p=12 & 5.384 & 5.385 & 5.385 & 5.386 & 5.386 & 5.386 & 5.386 & 5.387 & 5.386 & 5.376 & 6.508 & 6.562 & 6.595 & 6.657 & 6.7 & 6.66 & 6.639 & 6.726 & 6.64 & 6.032 \\ 
   &  & p=100 & 5.386 & 5.388 & 5.39 & 5.394 & 5.392 & 5.393 & 5.392 & 5.396 & 5.392 & 5.376 & 6.628 & 6.728 & 6.867 & 7.144 & 7.101 & 7.068 & 6.993 & 7.323 & 6.994 & 6.032 \\ 
   &  & p=196 & 5.386 & 5.388 & 5.832 & 109.803 & 5.392 & 5.393 & 5.392 & 149.2 & 6.637 & 112.711 & 6.628 & 6.728 & 25.908 & 59.763 & 7.101 & 7.068 & 6.993 & 84.379 & 7.18 & 35.067 \\ 
  \cmidrule{2-23} & \multirow{3}[2]{*}{$\rho=0.5$} & p=12 & 6.14 & 6.14 & 6.141 & 6.142 & 6.142 & 6.142 & 6.142 & 6.143 & 6.142 & 6.133 & 6.489 & 6.52 & 6.564 & 6.624 & 6.672 & 6.606 & 6.615 & 6.686 & 6.622 & 6.016 \\ 
   &  & p=100 & 6.142 & 6.143 & 6.145 & 6.149 & 6.148 & 6.148 & 6.147 & 6.153 & 6.147 & 6.133 & 6.61 & 6.707 & 6.842 & 7.133 & 7.116 & 7.033 & 6.923 & 7.368 & 6.942 & 6.016 \\ 
   &  & p=196 & 6.142 & 6.143 & 6.584 & 110.334 & 6.148 & 6.148 & 6.147 & 150.029 & 7.133 & 113.677 & 6.61 & 6.707 & 25.919 & 59.648 & 7.116 & 7.033 & 6.923 & 84.047 & 7.127 & 35.24 \\ 
  \cmidrule{2-23} & \multirow{3}[2]{*}{$\rho=0.9$} & p=12 & 6.751 & 6.752 & 6.752 & 6.753 & 6.752 & 6.753 & 6.752 & 6.753 & 6.752 & 6.752 & 6.114 & 6.164 & 6.2 & 6.274 & 6.233 & 6.277 & 6.262 & 6.324 & 6.262 & 5.178 \\ 
   &  & p=100 & 6.752 & 6.755 & 6.755 & 6.76 & 6.758 & 6.76 & 6.758 & 6.762 & 6.759 & 6.752 & 6.166 & 6.345 & 6.387 & 6.766 & 6.642 & 6.708 & 6.604 & 6.907 & 6.643 & 5.179 \\ 
   &  & p=196 & 6.752 & 6.755 & 7.173 & 110.03 & 6.758 & 6.76 & 6.758 & 148.473 & 10.507 & 113.5 & 6.166 & 6.345 & 25.669 & 59.149 & 6.642 & 6.708 & 6.604 & 82.493 & 7.014 & 33.953 \\ 
  \midrule\multirow{9}[6]{*}{lsnr} & \multirow{3}[2]{*}{$\rho=0$} & p=12 & 6.994 & 6.995 & 6.995 & 6.996 & 6.996 & 6.996 & 6.996 & 6.997 & 6.996 & 7.014 & 6.478 & 6.54 & 6.575 & 6.637 & 6.675 & 6.642 & 6.62 & 6.707 & 6.621 & 4.955 \\ 
   &  & p=100 & 6.996 & 6.998 & 7 & 7.004 & 7.002 & 7.003 & 7.002 & 7.006 & 7.002 & 7.014 & 6.602 & 6.708 & 6.848 & 7.126 & 7.079 & 7.051 & 6.975 & 7.305 & 6.976 & 4.955 \\ 
   &  & p=196 & 6.996 & 6.998 & 7.445 & 111.416 & 7.002 & 7.003 & 7.002 & 150.81 & 8.247 & 114.043 & 6.602 & 6.708 & 25.764 & 59.634 & 7.079 & 7.051 & 6.975 & 84.367 & 7.162 & 33.92 \\ 
  \cmidrule{2-23} & \multirow{3}[2]{*}{$\rho=0.5$} & p=12 & 7.752 & 7.752 & 7.753 & 7.754 & 7.753 & 7.753 & 7.753 & 7.754 & 7.754 & 7.772 & 6.224 & 6.262 & 6.311 & 6.375 & 6.382 & 6.352 & 6.368 & 6.452 & 6.375 & 4.694 \\ 
   &  & p=100 & 7.754 & 7.755 & 7.757 & 7.761 & 7.759 & 7.76 & 7.758 & 7.764 & 7.759 & 7.772 & 6.338 & 6.459 & 6.569 & 6.856 & 6.78 & 6.769 & 6.671 & 7.1 & 6.69 & 4.694 \\ 
   &  & p=196 & 7.754 & 7.755 & 8.198 & 111.947 & 7.759 & 7.76 & 7.758 & 151.213 & 8.745 & 115.009 & 6.338 & 6.459 & 25.586 & 59.252 & 6.78 & 6.769 & 6.671 & 83.144 & 6.875 & 33.914 \\ 
  \cmidrule{2-23} & \multirow{3}[2]{*}{$\rho=0.9$} & p=12 & 8.361 & 8.361 & 8.362 & 8.362 & 8.362 & 8.362 & 8.362 & 8.362 & 8.362 & 8.378 & 4.854 & 4.941 & 4.997 & 5.086 & 5.028 & 5.058 & 5.081 & 5.138 & 5.083 & 2.592 \\ 
   &  & p=100 & 8.362 & 8.363 & 8.365 & 8.368 & 8.366 & 8.366 & 8.365 & 8.37 & 8.366 & 8.378 & 4.895 & 5.056 & 5.199 & 5.495 & 5.374 & 5.372 & 5.29 & 5.667 & 5.327 & 2.592 \\ 
   &  & p=196 & 8.362 & 8.363 & 8.784 & 111.451 & 8.366 & 8.366 & 8.365 & 149.768 & 12.114 & 113.836 & 4.895 & 5.056 & 24.607 & 57.972 & 5.374 & 5.372 & 5.29 & 81.065 & 5.698 & 31.021 \\ 
   \bottomrule 
\end{tabular}
}
\end{table}

% latex table generated in R 4.0.2 by xtable 1.8-4 package
% Fri Jun 26 10:43:50 2020
\begin{table}[ht]
\centering
\caption{Random X, Sparse-Ex2, n=1000} 
\scalebox{0.52}{
\begin{tabular}{|c|c|c|cc|cc|cc|ccc|c||cc|cc|cc|ccc|c|}
  \toprule 
 \multicolumn{1}{|c}{} & \multicolumn{1}{c}{} &       & RAICc & AICc  & RC$_p$ & C$_p$ & 10F CV & LOO CV & S$_p$ & $\widehat{\text{C}}_p$ & GCV   & BIC   & RAICc & AICc  & RC$_p$ & C$_p$ & 10F CV & LOO CV & S$_p$ & $\widehat{\text{C}}_p$ & GCV   & BIC  \\
 \cmidrule{4-23}\multicolumn{1}{|c}{} & \multicolumn{1}{c}{} &       & \multicolumn{10}{c||}{RMSE}                                                   & \multicolumn{10}{c|}{Relative Risk}       \\
 \midrule\multirow{9}[6]{*}{hsnr} & \multirow{3}[2]{*}{$\rho=0$} & p=12 & 0.239 & 0.239 & 0.239 & 0.24 & 0.239 & 0.24 & 0.24 & 0.24 & 0.24 & 0.211 & 0.001 & 0.001 & 0.001 & 0.001 & 0.001 & 0.001 & 0.001 & 0.001 & 0.001 & 0.001 \\ 
   &  & p=500 & 0.245 & 0.246 & 0.249 & 0.249 & 0.247 & 0.248 & 0.247 & 0.248 & 0.248 & 0.211 & 0.001 & 0.001 & 0.001 & 0.001 & 0.001 & 0.001 & 0.001 & 0.001 & 0.001 & 0.001 \\ 
   &  & p=996 & 0.245 & 0.246 & 0.63 & 0.994 & 0.247 & 0.248 & 0.247 & 1.233 & 0.248 & 0.471 & 0.001 & 0.001 & 0.026 & 6.81 & 0.001 & 0.001 & 0.001 & 12.404 & 0.001 & 3.809 \\ 
  \cmidrule{2-23} & \multirow{3}[2]{*}{$\rho=0.5$} & p=12 & 0.348 & 0.349 & 0.349 & 0.349 & 0.348 & 0.35 & 0.349 & 0.35 & 0.349 & 0.308 & 0.001 & 0.001 & 0.001 & 0.001 & 0.001 & 0.001 & 0.001 & 0.001 & 0.001 & 0.001 \\ 
   &  & p=500 & 0.358 & 0.359 & 0.363 & 0.365 & 0.36 & 0.361 & 0.361 & 0.362 & 0.361 & 0.308 & 0.001 & 0.001 & 0.001 & 0.001 & 0.001 & 0.001 & 0.001 & 0.001 & 0.001 & 0.001 \\ 
   &  & p=996 & 0.358 & 0.359 & 0.914 & 1.447 & 0.36 & 0.361 & 0.361 & 1.794 & 0.361 & 0.685 & 0.001 & 0.001 & 0.026 & 6.783 & 0.001 & 0.001 & 0.001 & 11.988 & 0.001 & 3.813 \\ 
  \cmidrule{2-23} & \multirow{3}[2]{*}{$\rho=0.9$} & p=12 & 0.475 & 0.476 & 0.476 & 0.477 & 0.475 & 0.477 & 0.477 & 0.477 & 0.477 & 0.42 & 0.001 & 0.001 & 0.001 & 0.001 & 0.001 & 0.001 & 0.001 & 0.001 & 0.001 & 0.001 \\ 
   &  & p=500 & 0.486 & 0.489 & 0.494 & 0.498 & 0.489 & 0.489 & 0.49 & 0.492 & 0.49 & 0.421 & 0.001 & 0.001 & 0.001 & 0.001 & 0.001 & 0.001 & 0.001 & 0.001 & 0.001 & 0.001 \\ 
   &  & p=996 & 0.486 & 0.489 & 1.25 & 1.973 & 0.489 & 0.489 & 0.49 & 2.447 & 0.49 & 0.936 & 0.001 & 0.001 & 0.026 & 6.674 & 0.001 & 0.001 & 0.001 & 11.982 & 0.001 & 3.8 \\ 
  \midrule\multirow{9}[6]{*}{msnr} & \multirow{3}[2]{*}{$\rho=0$} & p=12 & 0.697 & 0.697 & 0.698 & 0.699 & 0.698 & 0.7 & 0.699 & 0.699 & 0.699 & 0.617 & 0.008 & 0.008 & 0.008 & 0.008 & 0.008 & 0.008 & 0.008 & 0.008 & 0.008 & 0.006 \\ 
   &  & p=500 & 0.715 & 0.716 & 0.725 & 0.727 & 0.72 & 0.722 & 0.721 & 0.724 & 0.722 & 0.617 & 0.008 & 0.008 & 0.009 & 0.009 & 0.009 & 0.009 & 0.009 & 0.009 & 0.009 & 0.006 \\ 
   &  & p=996 & 0.715 & 0.716 & 1.837 & 2.898 & 0.72 & 0.722 & 0.721 & 3.596 & 0.722 & 1.372 & 0.008 & 0.008 & 0.222 & 57.889 & 0.009 & 0.009 & 0.009 & 105.436 & 0.009 & 32.38 \\ 
  \cmidrule{2-23} & \multirow{3}[2]{*}{$\rho=0.5$} & p=12 & 1.016 & 1.018 & 1.018 & 1.018 & 1.015 & 1.019 & 1.018 & 1.02 & 1.018 & 0.899 & 0.008 & 0.008 & 0.008 & 0.008 & 0.008 & 0.008 & 0.008 & 0.008 & 0.008 & 0.006 \\ 
   &  & p=500 & 1.043 & 1.046 & 1.06 & 1.063 & 1.049 & 1.054 & 1.052 & 1.056 & 1.052 & 0.899 & 0.008 & 0.008 & 0.009 & 0.009 & 0.009 & 0.009 & 0.009 & 0.009 & 0.009 & 0.006 \\ 
   &  & p=996 & 1.043 & 1.046 & 2.665 & 4.217 & 1.049 & 1.054 & 1.052 & 5.231 & 1.052 & 1.998 & 0.008 & 0.008 & 0.221 & 57.659 & 0.009 & 0.009 & 0.009 & 101.899 & 0.009 & 32.414 \\ 
  \cmidrule{2-23} & \multirow{3}[2]{*}{$\rho=0.9$} & p=12 & 1.389 & 1.39 & 1.391 & 1.395 & 1.388 & 1.395 & 1.395 & 1.395 & 1.395 & 1.3 & 0.008 & 0.008 & 0.008 & 0.008 & 0.008 & 0.008 & 0.008 & 0.008 & 0.008 & 0.007 \\ 
   &  & p=500 & 1.419 & 1.43 & 1.442 & 1.452 & 1.429 & 1.43 & 1.432 & 1.438 & 1.432 & 1.302 & 0.008 & 0.009 & 0.009 & 0.009 & 0.009 & 0.009 & 0.009 & 0.009 & 0.009 & 0.007 \\ 
   &  & p=996 & 1.419 & 1.43 & 3.649 & 5.758 & 1.429 & 1.43 & 1.432 & 7.137 & 1.432 & 2.793 & 0.008 & 0.009 & 0.223 & 56.732 & 0.009 & 0.009 & 0.009 & 101.848 & 0.009 & 32.305 \\ 
  \midrule\multirow{9}[6]{*}{lsnr} & \multirow{3}[2]{*}{$\rho=0$} & p=12 & 1.558 & 1.559 & 1.561 & 1.563 & 1.56 & 1.565 & 1.563 & 1.564 & 1.563 & 1.379 & 0.039 & 0.039 & 0.039 & 0.039 & 0.039 & 0.039 & 0.039 & 0.039 & 0.039 & 0.03 \\ 
   &  & p=500 & 1.598 & 1.602 & 1.62 & 1.625 & 1.611 & 1.615 & 1.611 & 1.619 & 1.614 & 1.379 & 0.042 & 0.042 & 0.043 & 0.044 & 0.043 & 0.043 & 0.043 & 0.043 & 0.043 & 0.03 \\ 
   &  & p=996 & 1.598 & 1.602 & 4.108 & 6.48 & 1.611 & 1.615 & 1.611 & 8.041 & 1.614 & 3.068 & 0.042 & 0.042 & 1.108 & 289.446 & 0.043 & 0.043 & 0.043 & 527.182 & 0.043 & 161.899 \\ 
  \cmidrule{2-23} & \multirow{3}[2]{*}{$\rho=0.5$} & p=12 & 2.272 & 2.275 & 2.276 & 2.277 & 2.269 & 2.279 & 2.277 & 2.281 & 2.277 & 2.035 & 0.039 & 0.039 & 0.039 & 0.039 & 0.039 & 0.039 & 0.039 & 0.039 & 0.039 & 0.031 \\ 
   &  & p=500 & 2.333 & 2.338 & 2.37 & 2.377 & 2.346 & 2.356 & 2.352 & 2.361 & 2.352 & 2.035 & 0.042 & 0.042 & 0.044 & 0.044 & 0.043 & 0.043 & 0.043 & 0.044 & 0.043 & 0.031 \\ 
   &  & p=996 & 2.333 & 2.338 & 5.96 & 9.432 & 2.346 & 2.356 & 2.352 & 11.696 & 2.352 & 4.486 & 0.042 & 0.042 & 1.104 & 288.297 & 0.043 & 0.043 & 0.043 & 509.496 & 0.043 & 162.07 \\ 
  \cmidrule{2-23} & \multirow{3}[2]{*}{$\rho=0.9$} & p=12 & 3.171 & 3.176 & 3.177 & 3.18 & 3.217 & 3.189 & 3.18 & 3.179 & 3.18 & 3.559 & 0.04 & 0.041 & 0.041 & 0.041 & 0.041 & 0.041 & 0.041 & 0.041 & 0.041 & 0.05 \\ 
   &  & p=500 & 3.254 & 3.267 & 3.304 & 3.33 & 3.296 & 3.28 & 3.272 & 3.289 & 3.272 & 3.559 & 0.043 & 0.044 & 0.045 & 0.046 & 0.044 & 0.044 & 0.044 & 0.045 & 0.044 & 0.05 \\ 
   &  & p=996 & 3.254 & 3.267 & 8.257 & 12.968 & 3.296 & 3.28 & 3.272 & 16.003 & 3.272 & 6.727 & 0.043 & 0.044 & 1.115 & 283.661 & 0.044 & 0.044 & 0.044 & 509.239 & 0.044 & 157.615 \\ 
   \midrule 
 \multicolumn{1}{|c}{} & \multicolumn{1}{c}{} &       & \multicolumn{10}{c||}{Relative Test Error}                                    & \multicolumn{10}{c|}{Proportion of Variance Explained} \\
\midrule\multirow{9}[6]{*}{hsnr} & \multirow{3}[2]{*}{$\rho=0$} & p=12 & 1.008 & 1.008 & 1.008 & 1.008 & 1.008 & 1.008 & 1.008 & 1.008 & 1.008 & 1.006 & 0.894 & 0.894 & 0.894 & 0.894 & 0.894 & 0.894 & 0.894 & 0.894 & 0.894 & 0.894 \\ 
   &  & p=500 & 1.008 & 1.008 & 1.009 & 1.009 & 1.009 & 1.009 & 1.009 & 1.009 & 1.009 & 1.006 & 0.894 & 0.894 & 0.894 & 0.894 & 0.894 & 0.894 & 0.894 & 0.894 & 0.894 & 0.894 \\ 
   &  & p=996 & 1.008 & 1.008 & 1.222 & 58.89 & 1.009 & 1.009 & 1.009 & 106.503 & 1.009 & 33.197 & 0.894 & 0.894 & 0.872 & -5.198 & 0.894 & 0.894 & 0.894 & -10.203 & 0.894 & -2.511 \\ 
  \cmidrule{2-23} & \multirow{3}[2]{*}{$\rho=0.5$} & p=12 & 1.008 & 1.008 & 1.008 & 1.008 & 1.008 & 1.008 & 1.008 & 1.008 & 1.008 & 1.006 & 0.894 & 0.894 & 0.894 & 0.894 & 0.894 & 0.894 & 0.894 & 0.894 & 0.894 & 0.894 \\ 
   &  & p=500 & 1.008 & 1.009 & 1.009 & 1.009 & 1.009 & 1.009 & 1.009 & 1.009 & 1.009 & 1.006 & 0.894 & 0.894 & 0.894 & 0.894 & 0.894 & 0.894 & 0.894 & 0.894 & 0.894 & 0.894 \\ 
   &  & p=996 & 1.008 & 1.009 & 1.221 & 58.571 & 1.009 & 1.009 & 1.009 & 102.753 & 1.009 & 33.19 & 0.894 & 0.894 & 0.872 & -5.173 & 0.894 & 0.894 & 0.894 & -9.828 & 0.894 & -2.514 \\ 
  \cmidrule{2-23} & \multirow{3}[2]{*}{$\rho=0.9$} & p=12 & 1.008 & 1.008 & 1.008 & 1.008 & 1.008 & 1.008 & 1.008 & 1.008 & 1.008 & 1.006 & 0.894 & 0.894 & 0.894 & 0.894 & 0.894 & 0.894 & 0.894 & 0.894 & 0.894 & 0.894 \\ 
   &  & p=500 & 1.008 & 1.008 & 1.009 & 1.009 & 1.008 & 1.008 & 1.009 & 1.009 & 1.009 & 1.006 & 0.894 & 0.894 & 0.894 & 0.894 & 0.894 & 0.894 & 0.894 & 0.894 & 0.894 & 0.894 \\ 
   &  & p=996 & 1.008 & 1.008 & 1.223 & 57.707 & 1.008 & 1.008 & 1.009 & 102.803 & 1.009 & 33.184 & 0.894 & 0.894 & 0.871 & -5.076 & 0.894 & 0.894 & 0.894 & -9.824 & 0.894 & -2.504 \\ 
  \midrule\multirow{9}[6]{*}{msnr} & \multirow{3}[2]{*}{$\rho=0$} & p=12 & 1.008 & 1.008 & 1.008 & 1.008 & 1.008 & 1.008 & 1.008 & 1.008 & 1.008 & 1.006 & 0.497 & 0.497 & 0.497 & 0.497 & 0.497 & 0.497 & 0.497 & 0.497 & 0.497 & 0.498 \\ 
   &  & p=500 & 1.008 & 1.008 & 1.009 & 1.009 & 1.009 & 1.009 & 1.009 & 1.009 & 1.009 & 1.006 & 0.496 & 0.496 & 0.496 & 0.496 & 0.496 & 0.496 & 0.496 & 0.496 & 0.496 & 0.498 \\ 
   &  & p=996 & 1.008 & 1.008 & 1.222 & 58.89 & 1.009 & 1.009 & 1.009 & 106.503 & 1.009 & 33.197 & 0.496 & 0.496 & 0.39 & -28.432 & 0.496 & 0.496 & 0.496 & -52.211 & 0.496 & -15.636 \\ 
  \cmidrule{2-23} & \multirow{3}[2]{*}{$\rho=0.5$} & p=12 & 1.008 & 1.008 & 1.008 & 1.008 & 1.008 & 1.008 & 1.008 & 1.008 & 1.008 & 1.006 & 0.497 & 0.497 & 0.497 & 0.497 & 0.497 & 0.497 & 0.497 & 0.497 & 0.497 & 0.497 \\ 
   &  & p=500 & 1.008 & 1.009 & 1.009 & 1.009 & 1.009 & 1.009 & 1.009 & 1.009 & 1.009 & 1.006 & 0.496 & 0.496 & 0.496 & 0.496 & 0.496 & 0.496 & 0.496 & 0.496 & 0.496 & 0.497 \\ 
   &  & p=996 & 1.008 & 1.009 & 1.221 & 58.571 & 1.009 & 1.009 & 1.009 & 102.753 & 1.009 & 33.19 & 0.496 & 0.496 & 0.39 & -28.294 & 0.496 & 0.496 & 0.496 & -50.389 & 0.496 & -15.642 \\ 
  \cmidrule{2-23} & \multirow{3}[2]{*}{$\rho=0.9$} & p=12 & 1.008 & 1.008 & 1.008 & 1.008 & 1.008 & 1.008 & 1.008 & 1.008 & 1.008 & 1.007 & 0.496 & 0.496 & 0.496 & 0.496 & 0.496 & 0.496 & 0.496 & 0.496 & 0.496 & 0.497 \\ 
   &  & p=500 & 1.008 & 1.009 & 1.009 & 1.009 & 1.009 & 1.009 & 1.009 & 1.009 & 1.009 & 1.007 & 0.496 & 0.496 & 0.496 & 0.496 & 0.496 & 0.496 & 0.496 & 0.496 & 0.496 & 0.497 \\ 
   &  & p=996 & 1.008 & 1.009 & 1.223 & 57.707 & 1.009 & 1.009 & 1.009 & 102.803 & 1.009 & 33.185 & 0.496 & 0.496 & 0.389 & -27.846 & 0.496 & 0.496 & 0.496 & -50.388 & 0.496 & -15.614 \\ 
  \midrule\multirow{9}[6]{*}{lsnr} & \multirow{3}[2]{*}{$\rho=0$} & p=12 & 1.008 & 1.008 & 1.008 & 1.008 & 1.008 & 1.008 & 1.008 & 1.008 & 1.008 & 1.006 & 0.161 & 0.161 & 0.161 & 0.16 & 0.161 & 0.16 & 0.16 & 0.16 & 0.16 & 0.162 \\ 
   &  & p=500 & 1.008 & 1.008 & 1.009 & 1.009 & 1.009 & 1.009 & 1.009 & 1.009 & 1.009 & 1.006 & 0.16 & 0.16 & 0.16 & 0.16 & 0.16 & 0.16 & 0.16 & 0.16 & 0.16 & 0.162 \\ 
   &  & p=996 & 1.008 & 1.008 & 1.222 & 58.89 & 1.009 & 1.009 & 1.009 & 106.503 & 1.009 & 33.197 & 0.16 & 0.16 & -0.018 & -48.062 & 0.16 & 0.16 & 0.16 & -87.721 & 0.16 & -26.682 \\ 
  \cmidrule{2-23} & \multirow{3}[2]{*}{$\rho=0.5$} & p=12 & 1.008 & 1.008 & 1.008 & 1.008 & 1.008 & 1.008 & 1.008 & 1.008 & 1.008 & 1.006 & 0.16 & 0.16 & 0.16 & 0.16 & 0.161 & 0.16 & 0.16 & 0.16 & 0.16 & 0.162 \\ 
   &  & p=500 & 1.008 & 1.009 & 1.009 & 1.009 & 1.009 & 1.009 & 1.009 & 1.009 & 1.009 & 1.006 & 0.16 & 0.16 & 0.16 & 0.16 & 0.16 & 0.16 & 0.16 & 0.16 & 0.16 & 0.162 \\ 
   &  & p=996 & 1.008 & 1.009 & 1.221 & 58.571 & 1.009 & 1.009 & 1.009 & 102.753 & 1.009 & 33.19 & 0.16 & 0.16 & -0.017 & -47.809 & 0.16 & 0.16 & 0.16 & -84.626 & 0.16 & -26.681 \\ 
  \cmidrule{2-23} & \multirow{3}[2]{*}{$\rho=0.9$} & p=12 & 1.008 & 1.008 & 1.008 & 1.008 & 1.008 & 1.008 & 1.008 & 1.008 & 1.008 & 1.01 & 0.16 & 0.16 & 0.16 & 0.16 & 0.16 & 0.16 & 0.16 & 0.16 & 0.16 & 0.159 \\ 
   &  & p=500 & 1.009 & 1.009 & 1.009 & 1.009 & 1.009 & 1.009 & 1.009 & 1.009 & 1.009 & 1.01 & 0.16 & 0.16 & 0.159 & 0.159 & 0.16 & 0.16 & 0.16 & 0.16 & 0.16 & 0.159 \\ 
   &  & p=996 & 1.009 & 1.009 & 1.224 & 57.707 & 1.009 & 1.009 & 1.009 & 102.803 & 1.009 & 32.371 & 0.16 & 0.16 & -0.019 & -47.08 & 0.16 & 0.16 & 0.16 & -84.653 & 0.16 & -25.989 \\ 
   \midrule 
 \multicolumn{1}{|c}{} & \multicolumn{1}{c}{} &       & \multicolumn{10}{c||}{Kullback-Leibler Discrepancy}                                    & \multicolumn{10}{c|}{Number of Variables} \\
\midrule\multirow{9}[6]{*}{hsnr} & \multirow{3}[2]{*}{$\rho=0$} & p=12 & 3.115 & 3.115 & 3.115 & 3.115 & 3.115 & 3.115 & 3.115 & 3.115 & 3.115 & 3.113 & 6.617 & 6.62 & 6.631 & 6.639 & 6.68 & 6.653 & 6.639 & 6.645 & 6.639 & 6.011 \\ 
   &  & p=500 & 3.116 & 3.116 & 3.116 & 3.116 & 3.116 & 3.116 & 3.116 & 3.116 & 3.116 & 3.113 & 6.861 & 6.886 & 7.005 & 7.047 & 7.025 & 6.961 & 6.938 & 6.999 & 6.957 & 6.011 \\ 
   &  & p=996 & 3.116 & 3.116 & 3.912 & 76905.805 & 3.116 & 3.116 & 3.116 & 107437.34 & 3.116 & 68394.068 & 6.861 & 6.886 & 96.618 & 275.033 & 7.025 & 6.961 & 6.938 & 379.786 & 6.957 & 103.022 \\ 
  \cmidrule{2-23} & \multirow{3}[2]{*}{$\rho=0.5$} & p=12 & 3.866 & 3.866 & 3.866 & 3.866 & 3.866 & 3.866 & 3.866 & 3.866 & 3.866 & 3.865 & 6.629 & 6.64 & 6.641 & 6.645 & 6.675 & 6.656 & 6.645 & 6.66 & 6.645 & 6.012 \\ 
   &  & p=500 & 3.867 & 3.867 & 3.867 & 3.868 & 3.867 & 3.867 & 3.867 & 3.867 & 3.867 & 3.865 & 6.889 & 6.914 & 7.071 & 7.108 & 7.03 & 6.983 & 6.972 & 7.022 & 6.972 & 6.012 \\ 
   &  & p=996 & 3.867 & 3.867 & 4.66 & 75602.015 & 3.867 & 3.867 & 3.867 & 104378.88 & 3.867 & 68383.572 & 6.889 & 6.914 & 96.203 & 274.935 & 7.03 & 6.983 & 6.972 & 378.791 & 6.972 & 103.021 \\ 
  \cmidrule{2-23} & \multirow{3}[2]{*}{$\rho=0.9$} & p=12 & 4.489 & 4.489 & 4.489 & 4.489 & 4.489 & 4.489 & 4.489 & 4.489 & 4.489 & 4.487 & 6.613 & 6.623 & 6.629 & 6.649 & 6.669 & 6.649 & 6.647 & 6.649 & 6.647 & 6.011 \\ 
   &  & p=500 & 4.489 & 4.489 & 4.49 & 4.49 & 4.489 & 4.489 & 4.489 & 4.49 & 4.489 & 4.487 & 6.822 & 6.905 & 7.041 & 7.127 & 6.976 & 6.905 & 6.925 & 6.967 & 6.925 & 6.014 \\ 
   &  & p=996 & 4.489 & 4.489 & 5.314 & 75583.261 & 4.489 & 4.489 & 4.489 & 106170.983 & 4.489 & 68386.457 & 6.822 & 6.905 & 96.161 & 274.518 & 6.976 & 6.905 & 6.925 & 378.841 & 6.925 & 103.03 \\ 
  \midrule\multirow{9}[6]{*}{msnr} & \multirow{3}[2]{*}{$\rho=0$} & p=12 & 5.255 & 5.255 & 5.255 & 5.255 & 5.255 & 5.256 & 5.255 & 5.256 & 5.255 & 5.254 & 6.617 & 6.62 & 6.631 & 6.639 & 6.68 & 6.653 & 6.639 & 6.645 & 6.639 & 6.011 \\ 
   &  & p=500 & 5.256 & 5.256 & 5.256 & 5.256 & 5.256 & 5.256 & 5.256 & 5.256 & 5.256 & 5.254 & 6.861 & 6.886 & 7.005 & 7.047 & 7.025 & 6.961 & 6.938 & 6.999 & 6.957 & 6.011 \\ 
   &  & p=996 & 5.256 & 5.256 & 6.052 & 76907.945 & 5.256 & 5.256 & 5.256 & 107439.48 & 5.256 & 68396.208 & 6.861 & 6.886 & 96.618 & 275.033 & 7.025 & 6.961 & 6.938 & 379.786 & 6.957 & 103.022 \\ 
  \cmidrule{2-23} & \multirow{3}[2]{*}{$\rho=0.5$} & p=12 & 6.006 & 6.007 & 6.007 & 6.007 & 6.006 & 6.007 & 6.007 & 6.007 & 6.007 & 6.005 & 6.629 & 6.64 & 6.641 & 6.645 & 6.675 & 6.656 & 6.645 & 6.66 & 6.645 & 6.012 \\ 
   &  & p=500 & 6.007 & 6.007 & 6.008 & 6.008 & 6.007 & 6.007 & 6.007 & 6.007 & 6.007 & 6.005 & 6.889 & 6.914 & 7.071 & 7.108 & 7.03 & 6.983 & 6.972 & 7.022 & 6.972 & 6.012 \\ 
   &  & p=996 & 6.007 & 6.007 & 6.8 & 75604.155 & 6.007 & 6.007 & 6.007 & 104381.02 & 6.007 & 68385.712 & 6.889 & 6.914 & 96.203 & 274.935 & 7.03 & 6.983 & 6.972 & 378.791 & 6.972 & 103.021 \\ 
  \cmidrule{2-23} & \multirow{3}[2]{*}{$\rho=0.9$} & p=12 & 6.629 & 6.629 & 6.629 & 6.629 & 6.629 & 6.629 & 6.629 & 6.629 & 6.629 & 6.628 & 6.602 & 6.612 & 6.618 & 6.638 & 6.663 & 6.642 & 6.636 & 6.638 & 6.636 & 5.927 \\ 
   &  & p=500 & 6.629 & 6.63 & 6.63 & 6.63 & 6.63 & 6.63 & 6.63 & 6.63 & 6.63 & 6.628 & 6.81 & 6.893 & 7.013 & 7.099 & 6.97 & 6.898 & 6.913 & 6.955 & 6.913 & 5.931 \\ 
   &  & p=996 & 6.629 & 6.63 & 7.454 & 75585.401 & 6.63 & 6.63 & 6.63 & 106173.123 & 6.63 & 68388.598 & 6.81 & 6.893 & 96.149 & 274.506 & 6.97 & 6.898 & 6.913 & 378.831 & 6.913 & 102.957 \\ 
  \midrule\multirow{9}[6]{*}{lsnr} & \multirow{3}[2]{*}{$\rho=0$} & p=12 & 6.865 & 6.865 & 6.865 & 6.865 & 6.865 & 6.865 & 6.865 & 6.865 & 6.865 & 6.863 & 6.617 & 6.62 & 6.631 & 6.639 & 6.68 & 6.653 & 6.639 & 6.645 & 6.639 & 6.011 \\ 
   &  & p=500 & 6.866 & 6.866 & 6.866 & 6.866 & 6.866 & 6.866 & 6.866 & 6.866 & 6.866 & 6.863 & 6.861 & 6.886 & 7.005 & 7.047 & 7.025 & 6.961 & 6.938 & 6.999 & 6.957 & 6.011 \\ 
   &  & p=996 & 6.866 & 6.866 & 7.662 & 76909.555 & 6.866 & 6.866 & 6.866 & 107441.089 & 6.866 & 68397.817 & 6.861 & 6.886 & 96.618 & 275.033 & 7.025 & 6.961 & 6.938 & 379.786 & 6.957 & 103.022 \\ 
  \cmidrule{2-23} & \multirow{3}[2]{*}{$\rho=0.5$} & p=12 & 7.616 & 7.616 & 7.616 & 7.616 & 7.616 & 7.616 & 7.616 & 7.616 & 7.616 & 7.614 & 6.629 & 6.64 & 6.641 & 6.645 & 6.675 & 6.656 & 6.645 & 6.66 & 6.645 & 5.997 \\ 
   &  & p=500 & 7.617 & 7.617 & 7.617 & 7.617 & 7.617 & 7.617 & 7.617 & 7.617 & 7.617 & 7.614 & 6.889 & 6.914 & 7.071 & 7.108 & 7.03 & 6.983 & 6.972 & 7.022 & 6.972 & 5.997 \\ 
   &  & p=996 & 7.617 & 7.617 & 8.409 & 75605.765 & 7.617 & 7.617 & 7.617 & 104382.629 & 7.617 & 68387.321 & 6.889 & 6.914 & 96.202 & 274.934 & 7.03 & 6.983 & 6.972 & 378.791 & 6.972 & 103.01 \\ 
  \cmidrule{2-23} & \multirow{3}[2]{*}{$\rho=0.9$} & p=12 & 8.239 & 8.239 & 8.239 & 8.239 & 8.239 & 8.239 & 8.239 & 8.239 & 8.239 & 8.24 & 6.262 & 6.287 & 6.287 & 6.304 & 6.286 & 6.306 & 6.301 & 6.305 & 6.301 & 5.013 \\ 
   &  & p=500 & 8.239 & 8.239 & 8.24 & 8.24 & 8.239 & 8.239 & 8.239 & 8.239 & 8.239 & 8.24 & 6.507 & 6.561 & 6.716 & 6.823 & 6.565 & 6.594 & 6.586 & 6.656 & 6.586 & 5.013 \\ 
   &  & p=996 & 8.239 & 8.239 & 9.064 & 75587.011 & 8.239 & 8.239 & 8.239 & 106174.732 & 8.239 & 67638.755 & 6.507 & 6.561 & 95.876 & 274.236 & 6.565 & 6.594 & 6.586 & 378.646 & 6.586 & 99.162 \\ 
   \bottomrule 
\end{tabular}
}
\end{table}

%% latex table generated in R 4.0.2 by xtable 1.8-4 package
% Fri Jun 26 10:43:45 2020
\begin{table}[ht]
\centering
\caption{Fixed X, Sparse-Ex3, n=40} 
\scalebox{0.52}{
\begin{tabular}{|c|c|c|cc|cc|cc|ccc|c||cc|cc|cc|ccc|c|}
  \toprule 
 \multicolumn{1}{|c}{} & \multicolumn{1}{c}{} &       & RAICc & AICc  & RC$_p$ & C$_p$ & 10F CV & LOO CV & S$_p$ & $\widehat{\text{C}}_p$ & GCV   & BIC   & RAICc & AICc  & RC$_p$ & C$_p$ & 10F CV & LOO CV & S$_p$ & $\widehat{\text{C}}_p$ & GCV   & BIC  \\
 \cmidrule{4-23}\multicolumn{1}{|c}{} & \multicolumn{1}{c}{} &       & \multicolumn{10}{c||}{RMSE}                                                   & \multicolumn{10}{c|}{Relative Risk}       \\
 \midrule\multirow{9}[6]{*}{hsnr} & \multirow{3}[2]{*}{$\rho=0$} & p=12 & 0.282 & 0.291 & 0.295 & 0.311 & 0.309 & 0.308 & 0.309 & 0.322 & 0.31 & 0.287 & 0.02 & 0.022 & 0.022 & 0.025 & 0.025 & 0.025 & 0.025 & 0.027 & 0.025 & 0.021 \\ 
   &  & p=20 & 0.282 & 0.292 & 0.302 & 0.331 & 0.321 & 0.321 & 0.322 & 0.358 & 0.324 & 0.291 & 0.02 & 0.022 & 0.024 & 0.034 & 0.029 & 0.03 & 0.03 & 0.042 & 0.03 & 0.022 \\ 
   &  & p=36 & 0.282 & 0.292 & 0.339 & 0.412 & 0.325 & 0.333 & 0.332 & 0.492 & 0.341 & 0.4 & 0.02 & 0.022 & 0.044 & 0.197 & 0.032 & 0.046 & 0.046 & 0.315 & 0.058 & 0.234 \\ 
  \cmidrule{2-23} & \multirow{3}[2]{*}{$\rho=0.5$} & p=12 & 0.304 & 0.313 & 0.318 & 0.334 & 0.332 & 0.333 & 0.331 & 0.347 & 0.332 & 0.309 & 0.024 & 0.026 & 0.027 & 0.029 & 0.029 & 0.029 & 0.029 & 0.032 & 0.029 & 0.025 \\ 
   &  & p=20 & 0.304 & 0.314 & 0.323 & 0.355 & 0.344 & 0.347 & 0.345 & 0.387 & 0.348 & 0.312 & 0.024 & 0.026 & 0.028 & 0.039 & 0.034 & 0.035 & 0.034 & 0.049 & 0.035 & 0.026 \\ 
   &  & p=36 & 0.304 & 0.314 & 0.364 & 0.444 & 0.349 & 0.359 & 0.357 & 0.529 & 0.365 & 0.429 & 0.024 & 0.026 & 0.052 & 0.23 & 0.038 & 0.053 & 0.053 & 0.364 & 0.068 & 0.271 \\ 
  \cmidrule{2-23} & \multirow{3}[2]{*}{$\rho=0.9$} & p=12 & 0.339 & 0.348 & 0.355 & 0.374 & 0.37 & 0.371 & 0.369 & 0.387 & 0.37 & 0.343 & 0.028 & 0.03 & 0.032 & 0.036 & 0.035 & 0.035 & 0.035 & 0.039 & 0.035 & 0.029 \\ 
   &  & p=20 & 0.339 & 0.35 & 0.359 & 0.397 & 0.383 & 0.388 & 0.385 & 0.432 & 0.386 & 0.348 & 0.028 & 0.031 & 0.034 & 0.048 & 0.041 & 0.043 & 0.042 & 0.062 & 0.043 & 0.031 \\ 
   &  & p=36 & 0.339 & 0.35 & 0.405 & 0.496 & 0.387 & 0.402 & 0.399 & 0.592 & 0.406 & 0.478 & 0.028 & 0.031 & 0.064 & 0.287 & 0.045 & 0.068 & 0.067 & 0.456 & 0.082 & 0.337 \\ 
  \midrule\multirow{9}[6]{*}{msnr} & \multirow{3}[2]{*}{$\rho=0$} & p=12 & 0.901 & 0.897 & 0.897 & 0.93 & 0.946 & 0.924 & 0.922 & 0.953 & 0.925 & 0.986 & 0.212 & 0.21 & 0.208 & 0.226 & 0.234 & 0.221 & 0.222 & 0.239 & 0.223 & 0.265 \\ 
   &  & p=20 & 0.901 & 0.9 & 0.92 & 0.99 & 0.977 & 0.961 & 0.959 & 1.057 & 0.963 & 0.995 & 0.212 & 0.213 & 0.227 & 0.301 & 0.268 & 0.264 & 0.264 & 0.366 & 0.267 & 0.275 \\ 
   &  & p=36 & 0.901 & 0.9 & 1.082 & 1.268 & 0.99 & 0.996 & 0.987 & 1.443 & 1.011 & 1.276 & 0.212 & 0.213 & 0.43 & 1.715 & 0.297 & 0.403 & 0.391 & 2.679 & 0.501 & 1.978 \\ 
  \cmidrule{2-23} & \multirow{3}[2]{*}{$\rho=0.5$} & p=12 & 0.975 & 0.976 & 0.975 & 1.006 & 1.021 & 1.004 & 0.995 & 1.028 & 0.998 & 1.066 & 0.251 & 0.252 & 0.251 & 0.267 & 0.274 & 0.266 & 0.261 & 0.281 & 0.262 & 0.304 \\ 
   &  & p=20 & 0.975 & 0.977 & 0.992 & 1.067 & 1.052 & 1.045 & 1.035 & 1.141 & 1.043 & 1.074 & 0.251 & 0.253 & 0.267 & 0.349 & 0.312 & 0.314 & 0.309 & 0.426 & 0.317 & 0.313 \\ 
   &  & p=36 & 0.975 & 0.977 & 1.162 & 1.368 & 1.066 & 1.08 & 1.069 & 1.552 & 1.092 & 1.369 & 0.251 & 0.253 & 0.497 & 1.999 & 0.344 & 0.472 & 0.468 & 3.099 & 0.593 & 2.283 \\ 
  \cmidrule{2-23} & \multirow{3}[2]{*}{$\rho=0.9$} & p=12 & 1.114 & 1.098 & 1.104 & 1.132 & 1.15 & 1.13 & 1.119 & 1.155 & 1.122 & 1.214 & 0.304 & 0.298 & 0.304 & 0.329 & 0.332 & 0.324 & 0.319 & 0.347 & 0.321 & 0.356 \\ 
   &  & p=20 & 1.114 & 1.102 & 1.127 & 1.199 & 1.184 & 1.178 & 1.167 & 1.28 & 1.17 & 1.225 & 0.304 & 0.302 & 0.325 & 0.429 & 0.377 & 0.386 & 0.384 & 0.532 & 0.387 & 0.37 \\ 
   &  & p=36 & 1.114 & 1.102 & 1.314 & 1.537 & 1.195 & 1.219 & 1.205 & 1.74 & 1.225 & 1.551 & 0.304 & 0.302 & 0.608 & 2.487 & 0.407 & 0.603 & 0.583 & 3.884 & 0.716 & 2.869 \\ 
  \midrule\multirow{9}[6]{*}{lsnr} & \multirow{3}[2]{*}{$\rho=0$} & p=12 & 2.13 & 2.138 & 2.141 & 2.175 & 2.163 & 2.165 & 2.169 & 2.202 & 2.172 & 2.163 & 1.019 & 1.051 & 1.067 & 1.126 & 1.107 & 1.105 & 1.113 & 1.173 & 1.12 & 1.049 \\ 
   &  & p=20 & 2.13 & 2.14 & 2.167 & 2.28 & 2.209 & 2.223 & 2.215 & 2.373 & 2.222 & 2.167 & 1.019 & 1.054 & 1.131 & 1.441 & 1.227 & 1.256 & 1.228 & 1.671 & 1.242 & 1.058 \\ 
   &  & p=36 & 2.13 & 2.14 & 2.385 & 2.817 & 2.227 & 2.283 & 2.272 & 3.209 & 2.302 & 2.651 & 1.019 & 1.054 & 1.972 & 8.408 & 1.326 & 1.86 & 1.828 & 12.862 & 2.209 & 7.651 \\ 
  \cmidrule{2-23} & \multirow{3}[2]{*}{$\rho=0.5$} & p=12 & 2.295 & 2.302 & 2.291 & 2.337 & 2.313 & 2.319 & 2.323 & 2.365 & 2.326 & 2.325 & 1.101 & 1.144 & 1.169 & 1.261 & 1.21 & 1.227 & 1.23 & 1.312 & 1.238 & 1.094 \\ 
   &  & p=20 & 2.295 & 2.304 & 2.322 & 2.451 & 2.363 & 2.383 & 2.375 & 2.552 & 2.384 & 2.328 & 1.101 & 1.151 & 1.241 & 1.615 & 1.35 & 1.401 & 1.371 & 1.886 & 1.397 & 1.101 \\ 
   &  & p=36 & 2.295 & 2.304 & 2.566 & 3.037 & 2.383 & 2.452 & 2.434 & 3.445 & 2.469 & 2.85 & 1.101 & 1.151 & 2.245 & 9.788 & 1.47 & 2.153 & 2.039 & 14.842 & 2.496 & 8.723 \\ 
  \cmidrule{2-23} & \multirow{3}[2]{*}{$\rho=0.9$} & p=12 & 2.542 & 2.542 & 2.537 & 2.596 & 2.552 & 2.562 & 2.565 & 2.628 & 2.572 & 2.579 & 1.232 & 1.276 & 1.329 & 1.482 & 1.373 & 1.388 & 1.388 & 1.55 & 1.408 & 1.17 \\ 
   &  & p=20 & 2.542 & 2.548 & 2.575 & 2.727 & 2.606 & 2.632 & 2.627 & 2.841 & 2.641 & 2.583 & 1.232 & 1.294 & 1.436 & 1.935 & 1.543 & 1.593 & 1.569 & 2.282 & 1.611 & 1.18 \\ 
   &  & p=36 & 2.542 & 2.548 & 2.853 & 3.391 & 2.629 & 2.703 & 2.691 & 3.83 & 2.73 & 3.164 & 1.232 & 1.294 & 2.708 & 12.144 & 1.695 & 2.409 & 2.407 & 18.385 & 2.921 & 10.625 \\ 
   \midrule 
 \multicolumn{1}{|c}{} & \multicolumn{1}{c}{} &       & \multicolumn{10}{c||}{Relative Test Error}                                    & \multicolumn{10}{c|}{Proportion of Variance Explained} \\
\midrule\multirow{9}[6]{*}{hsnr} & \multirow{3}[2]{*}{$\rho=0$} & p=12 & 1.229 & 1.247 & 1.256 & 1.288 & 1.287 & 1.281 & 1.284 & 1.311 & 1.286 & 1.241 & 0.901 & 0.9 & 0.899 & 0.897 & 0.897 & 0.897 & 0.897 & 0.895 & 0.897 & 0.9 \\ 
   &  & p=20 & 1.229 & 1.251 & 1.278 & 1.387 & 1.333 & 1.34 & 1.342 & 1.484 & 1.346 & 1.255 & 0.901 & 0.9 & 0.897 & 0.889 & 0.893 & 0.892 & 0.892 & 0.881 & 0.892 & 0.899 \\ 
   &  & p=36 & 1.229 & 1.251 & 1.507 & 3.263 & 1.372 & 1.527 & 1.526 & 4.606 & 1.67 & 3.682 & 0.901 & 0.9 & 0.879 & 0.738 & 0.89 & 0.877 & 0.878 & 0.63 & 0.866 & 0.704 \\ 
  \cmidrule{2-23} & \multirow{3}[2]{*}{$\rho=0.5$} & p=12 & 1.236 & 1.252 & 1.262 & 1.29 & 1.288 & 1.288 & 1.283 & 1.315 & 1.286 & 1.245 & 0.886 & 0.885 & 0.884 & 0.881 & 0.881 & 0.881 & 0.882 & 0.879 & 0.882 & 0.885 \\ 
   &  & p=20 & 1.236 & 1.254 & 1.277 & 1.385 & 1.333 & 1.343 & 1.34 & 1.487 & 1.35 & 1.256 & 0.886 & 0.885 & 0.882 & 0.872 & 0.877 & 0.876 & 0.877 & 0.863 & 0.876 & 0.884 \\ 
   &  & p=36 & 1.236 & 1.254 & 1.512 & 3.271 & 1.371 & 1.527 & 1.525 & 4.589 & 1.67 & 3.675 & 0.886 & 0.885 & 0.861 & 0.699 & 0.874 & 0.859 & 0.86 & 0.577 & 0.846 & 0.662 \\ 
  \cmidrule{2-23} & \multirow{3}[2]{*}{$\rho=0.9$} & p=12 & 1.223 & 1.238 & 1.25 & 1.286 & 1.277 & 1.278 & 1.275 & 1.311 & 1.278 & 1.232 & 0.862 & 0.861 & 0.859 & 0.855 & 0.856 & 0.856 & 0.857 & 0.853 & 0.856 & 0.861 \\ 
   &  & p=20 & 1.223 & 1.242 & 1.264 & 1.378 & 1.32 & 1.337 & 1.335 & 1.485 & 1.339 & 1.246 & 0.862 & 0.86 & 0.858 & 0.845 & 0.851 & 0.85 & 0.85 & 0.833 & 0.849 & 0.86 \\ 
   &  & p=36 & 1.223 & 1.242 & 1.502 & 3.265 & 1.352 & 1.539 & 1.526 & 4.597 & 1.645 & 3.66 & 0.862 & 0.86 & 0.831 & 0.633 & 0.848 & 0.827 & 0.828 & 0.483 & 0.815 & 0.588 \\ 
  \midrule\multirow{9}[6]{*}{msnr} & \multirow{3}[2]{*}{$\rho=0$} & p=12 & 1.286 & 1.283 & 1.281 & 1.304 & 1.316 & 1.298 & 1.299 & 1.322 & 1.3 & 1.357 & 0.452 & 0.454 & 0.454 & 0.445 & 0.44 & 0.447 & 0.447 & 0.437 & 0.446 & 0.422 \\ 
   &  & p=20 & 1.286 & 1.287 & 1.306 & 1.406 & 1.362 & 1.355 & 1.355 & 1.493 & 1.359 & 1.37 & 0.452 & 0.452 & 0.444 & 0.401 & 0.42 & 0.423 & 0.423 & 0.364 & 0.421 & 0.416 \\ 
   &  & p=36 & 1.286 & 1.287 & 1.58 & 3.312 & 1.4 & 1.543 & 1.527 & 4.612 & 1.676 & 3.667 & 0.452 & 0.452 & 0.327 & -0.411 & 0.404 & 0.343 & 0.35 & -0.964 & 0.286 & -0.562 \\ 
  \cmidrule{2-23} & \multirow{3}[2]{*}{$\rho=0.5$} & p=12 & 1.291 & 1.292 & 1.292 & 1.31 & 1.318 & 1.309 & 1.303 & 1.326 & 1.304 & 1.353 & 0.402 & 0.402 & 0.402 & 0.394 & 0.39 & 0.394 & 0.397 & 0.386 & 0.396 & 0.374 \\ 
   &  & p=20 & 1.291 & 1.294 & 1.31 & 1.405 & 1.361 & 1.364 & 1.358 & 1.495 & 1.368 & 1.363 & 0.402 & 0.401 & 0.394 & 0.349 & 0.37 & 0.368 & 0.371 & 0.308 & 0.367 & 0.369 \\ 
   &  & p=36 & 1.291 & 1.294 & 1.576 & 3.319 & 1.399 & 1.548 & 1.543 & 4.595 & 1.687 & 3.648 & 0.402 & 0.401 & 0.27 & -0.536 & 0.352 & 0.283 & 0.285 & -1.127 & 0.219 & -0.689 \\ 
  \cmidrule{2-23} & \multirow{3}[2]{*}{$\rho=0.9$} & p=12 & 1.282 & 1.276 & 1.282 & 1.305 & 1.308 & 1.3 & 1.296 & 1.323 & 1.298 & 1.331 & 0.335 & 0.338 & 0.335 & 0.323 & 0.322 & 0.326 & 0.328 & 0.314 & 0.327 & 0.31 \\ 
   &  & p=20 & 1.282 & 1.28 & 1.302 & 1.398 & 1.35 & 1.359 & 1.356 & 1.494 & 1.359 & 1.343 & 0.335 & 0.336 & 0.325 & 0.275 & 0.3 & 0.295 & 0.297 & 0.225 & 0.295 & 0.303 \\ 
   &  & p=36 & 1.282 & 1.28 & 1.564 & 3.308 & 1.378 & 1.559 & 1.541 & 4.605 & 1.665 & 3.662 & 0.335 & 0.336 & 0.189 & -0.716 & 0.285 & 0.191 & 0.201 & -1.388 & 0.137 & -0.9 \\ 
  \midrule\multirow{9}[6]{*}{lsnr} & \multirow{3}[2]{*}{$\rho=0$} & p=12 & 1.275 & 1.283 & 1.288 & 1.304 & 1.299 & 1.298 & 1.3 & 1.316 & 1.302 & 1.283 & -0.004 & -0.011 & -0.014 & -0.027 & -0.023 & -0.022 & -0.024 & -0.037 & -0.025 & -0.01 \\ 
   &  & p=20 & 1.275 & 1.284 & 1.305 & 1.388 & 1.331 & 1.339 & 1.331 & 1.451 & 1.335 & 1.285 & -0.004 & -0.012 & -0.028 & -0.094 & -0.048 & -0.054 & -0.048 & -0.142 & -0.051 & -0.012 \\ 
   &  & p=36 & 1.275 & 1.284 & 1.532 & 3.267 & 1.358 & 1.502 & 1.493 & 4.468 & 1.595 & 3.063 & -0.004 & -0.012 & -0.207 & -1.573 & -0.069 & -0.183 & -0.176 & -2.519 & -0.257 & -1.413 \\ 
  \cmidrule{2-23} & \multirow{3}[2]{*}{$\rho=0.5$} & p=12 & 1.255 & 1.265 & 1.271 & 1.293 & 1.281 & 1.285 & 1.285 & 1.304 & 1.287 & 1.254 & -0.019 & -0.027 & -0.032 & -0.049 & -0.04 & -0.043 & -0.043 & -0.059 & -0.045 & -0.018 \\ 
   &  & p=20 & 1.255 & 1.267 & 1.288 & 1.375 & 1.313 & 1.325 & 1.318 & 1.438 & 1.324 & 1.255 & -0.019 & -0.028 & -0.045 & -0.116 & -0.066 & -0.075 & -0.07 & -0.167 & -0.075 & -0.019 \\ 
   &  & p=36 & 1.255 & 1.267 & 1.521 & 3.271 & 1.341 & 1.499 & 1.473 & 4.443 & 1.579 & 3.024 & -0.019 & -0.028 & -0.234 & -1.655 & -0.088 & -0.217 & -0.196 & -2.607 & -0.282 & -1.454 \\ 
  \cmidrule{2-23} & \multirow{3}[2]{*}{$\rho=0.9$} & p=12 & 1.229 & 1.237 & 1.247 & 1.275 & 1.255 & 1.258 & 1.258 & 1.288 & 1.261 & 1.217 & -0.036 & -0.043 & -0.051 & -0.076 & -0.058 & -0.061 & -0.061 & -0.086 & -0.064 & -0.027 \\ 
   &  & p=20 & 1.229 & 1.24 & 1.267 & 1.359 & 1.286 & 1.296 & 1.291 & 1.424 & 1.299 & 1.219 & -0.036 & -0.046 & -0.068 & -0.146 & -0.085 & -0.093 & -0.089 & -0.201 & -0.096 & -0.028 \\ 
   &  & p=36 & 1.229 & 1.24 & 1.503 & 3.254 & 1.315 & 1.447 & 1.447 & 4.413 & 1.542 & 2.972 & -0.036 & -0.046 & -0.267 & -1.745 & -0.109 & -0.221 & -0.22 & -2.722 & -0.301 & -1.507 \\ 
   \midrule 
 \multicolumn{1}{|c}{} & \multicolumn{1}{c}{} &       & \multicolumn{10}{c||}{Kullback-Leibler Discrepancy}                                    & \multicolumn{10}{c|}{Number of Variables} \\
\midrule\multirow{9}[6]{*}{hsnr} & \multirow{3}[2]{*}{$\rho=0$} & p=12 & 0.64 & 0.674 & 0.689 & 0.739 & 0.73 & 0.725 & 0.733 & 0.772 & 0.735 & 0.669 & 6.148 & 6.319 & 6.418 & 6.815 & 6.868 & 6.758 & 6.758 & 7.138 & 6.778 & 6.274 \\ 
   &  & p=20 & 0.64 & 0.682 & 0.741 & 0.955 & 0.842 & 0.875 & 0.873 & 1.143 & 0.881 & 0.715 & 6.148 & 6.343 & 6.624 & 7.65 & 7.301 & 7.289 & 7.241 & 8.775 & 7.29 & 6.391 \\ 
   &  & p=36 & 0.64 & 0.682 & 2.736 & 18.615 & 0.97 & 6.127 & 7.065 & 24.747 & 8.614 & 23.616 & 6.148 & 6.343 & 8.522 & 13.668 & 7.526 & 8.022 & 7.939 & 18.913 & 8.414 & 14.278 \\ 
  \cmidrule{2-23} & \multirow{3}[2]{*}{$\rho=0.5$} & p=12 & 0.792 & 0.823 & 0.84 & 0.889 & 0.879 & 0.88 & 0.877 & 0.926 & 0.881 & 0.812 & 6.149 & 6.314 & 6.418 & 6.813 & 6.846 & 6.797 & 6.714 & 7.164 & 6.745 & 6.24 \\ 
   &  & p=20 & 0.792 & 0.834 & 0.88 & 1.107 & 0.993 & 1.027 & 1.02 & 1.293 & 1.036 & 0.849 & 6.149 & 6.333 & 6.569 & 7.636 & 7.271 & 7.304 & 7.199 & 8.799 & 7.292 & 6.33 \\ 
   &  & p=36 & 0.792 & 0.834 & 2.882 & 18.802 & 1.123 & 6.27 & 7.229 & 24.872 & 8.748 & 23.717 & 6.149 & 6.333 & 8.502 & 13.647 & 7.503 & 8.027 & 7.927 & 18.825 & 8.378 & 14.176 \\ 
  \cmidrule{2-23} & \multirow{3}[2]{*}{$\rho=0.9$} & p=12 & 1.013 & 1.041 & 1.061 & 1.116 & 1.098 & 1.105 & 1.097 & 1.154 & 1.103 & 1.033 & 6.147 & 6.295 & 6.421 & 6.844 & 6.847 & 6.802 & 6.699 & 7.199 & 6.741 & 6.232 \\ 
   &  & p=20 & 1.013 & 1.057 & 1.101 & 1.329 & 1.206 & 1.256 & 1.246 & 1.515 & 1.252 & 1.084 & 6.147 & 6.332 & 6.558 & 7.635 & 7.249 & 7.335 & 7.201 & 8.808 & 7.243 & 6.355 \\ 
   &  & p=36 & 1.013 & 1.057 & 3.1 & 19.028 & 1.316 & 6.619 & 7.516 & 25.097 & 8.753 & 23.889 & 6.147 & 6.332 & 8.502 & 13.672 & 7.434 & 8.116 & 7.969 & 18.856 & 8.311 & 14.163 \\ 
  \midrule\multirow{9}[6]{*}{msnr} & \multirow{3}[2]{*}{$\rho=0$} & p=12 & 2.813 & 2.836 & 2.845 & 2.889 & 2.886 & 2.876 & 2.882 & 2.919 & 2.883 & 2.87 & 5.81 & 6.08 & 6.25 & 6.691 & 6.646 & 6.607 & 6.641 & 7.037 & 6.661 & 5.603 \\ 
   &  & p=20 & 2.813 & 2.844 & 2.898 & 3.107 & 2.996 & 3.024 & 3.02 & 3.289 & 3.029 & 2.915 & 5.81 & 6.104 & 6.433 & 7.525 & 7.075 & 7.133 & 7.118 & 8.684 & 7.169 & 5.713 \\ 
   &  & p=36 & 2.813 & 2.844 & 4.914 & 20.781 & 3.124 & 8.276 & 9.091 & 26.891 & 10.575 & 25.567 & 5.81 & 6.104 & 8.122 & 13.362 & 7.3 & 7.866 & 7.787 & 18.868 & 8.265 & 13.551 \\ 
  \cmidrule{2-23} & \multirow{3}[2]{*}{$\rho=0.5$} & p=12 & 2.965 & 2.988 & 2.999 & 3.042 & 3.037 & 3.033 & 3.029 & 3.072 & 3.032 & 3.017 & 5.788 & 6.024 & 6.196 & 6.639 & 6.581 & 6.619 & 6.549 & 7.054 & 6.581 & 5.541 \\ 
   &  & p=20 & 2.965 & 2.999 & 3.041 & 3.259 & 3.148 & 3.179 & 3.171 & 3.437 & 3.187 & 3.054 & 5.788 & 6.044 & 6.347 & 7.471 & 6.991 & 7.137 & 7.039 & 8.691 & 7.131 & 5.632 \\ 
   &  & p=36 & 2.965 & 2.999 & 5.06 & 20.97 & 3.278 & 8.422 & 9.38 & 27.015 & 10.898 & 25.616 & 5.788 & 6.044 & 8.109 & 13.311 & 7.224 & 7.861 & 7.768 & 18.756 & 8.223 & 13.414 \\ 
  \cmidrule{2-23} & \multirow{3}[2]{*}{$\rho=0.9$} & p=12 & 3.194 & 3.206 & 3.225 & 3.269 & 3.259 & 3.26 & 3.253 & 3.301 & 3.257 & 3.242 & 5.658 & 5.912 & 6.126 & 6.601 & 6.503 & 6.537 & 6.456 & 7.021 & 6.5 & 5.407 \\ 
   &  & p=20 & 3.194 & 3.222 & 3.268 & 3.483 & 3.366 & 3.411 & 3.401 & 3.66 & 3.406 & 3.286 & 5.658 & 5.949 & 6.249 & 7.405 & 6.9 & 7.083 & 6.971 & 8.63 & 7.008 & 5.517 \\ 
   &  & p=36 & 3.194 & 3.222 & 5.284 & 21.197 & 3.473 & 8.767 & 9.615 & 27.241 & 10.907 & 25.881 & 5.658 & 5.949 & 8.001 & 13.301 & 7.069 & 7.85 & 7.719 & 18.754 & 8.086 & 13.36 \\ 
  \midrule\multirow{9}[6]{*}{lsnr} & \multirow{3}[2]{*}{$\rho=0$} & p=12 & 4.383 & 4.41 & 4.433 & 4.47 & 4.449 & 4.456 & 4.46 & 4.498 & 4.463 & 4.362 & 2.091 & 2.577 & 3.188 & 3.821 & 3.514 & 3.542 & 3.578 & 4.312 & 3.658 & 0.988 \\ 
   &  & p=20 & 4.383 & 4.415 & 4.483 & 4.667 & 4.538 & 4.576 & 4.565 & 4.818 & 4.575 & 4.38 & 2.091 & 2.593 & 3.432 & 4.669 & 3.876 & 3.995 & 3.939 & 5.867 & 4.045 & 1.029 \\ 
   &  & p=36 & 4.383 & 4.415 & 6.472 & 22.323 & 4.639 & 9.644 & 10.615 & 28.253 & 11.724 & 24.815 & 2.091 & 2.593 & 5.665 & 11.044 & 4.071 & 4.747 & 4.66 & 16.683 & 5.059 & 7.973 \\ 
  \cmidrule{2-23} & \multirow{3}[2]{*}{$\rho=0.5$} & p=12 & 4.533 & 4.561 & 4.582 & 4.622 & 4.596 & 4.607 & 4.608 & 4.648 & 4.611 & 4.516 & 2.053 & 2.587 & 3.247 & 3.907 & 3.545 & 3.629 & 3.599 & 4.313 & 3.685 & 1.022 \\ 
   &  & p=20 & 4.533 & 4.567 & 4.625 & 4.816 & 4.687 & 4.729 & 4.717 & 4.963 & 4.725 & 4.531 & 2.053 & 2.604 & 3.438 & 4.706 & 3.908 & 4.082 & 3.966 & 5.837 & 4.082 & 1.05 \\ 
   &  & p=36 & 4.533 & 4.567 & 6.622 & 22.536 & 4.787 & 9.956 & 10.743 & 28.354 & 11.715 & 24.912 & 2.053 & 2.604 & 5.672 & 11.124 & 4.124 & 4.85 & 4.646 & 16.56 & 5.061 & 7.955 \\ 
  \cmidrule{2-23} & \multirow{3}[2]{*}{$\rho=0.9$} & p=12 & 4.753 & 4.775 & 4.799 & 4.845 & 4.811 & 4.818 & 4.82 & 4.869 & 4.825 & 4.741 & 2.13 & 2.56 & 3.14 & 3.846 & 3.477 & 3.439 & 3.418 & 4.234 & 3.51 & 1.043 \\ 
   &  & p=20 & 4.753 & 4.789 & 4.842 & 5.044 & 4.901 & 4.933 & 4.925 & 5.18 & 4.941 & 4.756 & 2.13 & 2.604 & 3.401 & 4.705 & 3.825 & 3.885 & 3.777 & 5.789 & 3.918 & 1.065 \\ 
   &  & p=36 & 4.753 & 4.789 & 6.843 & 22.75 & 5.002 & 10.041 & 10.952 & 28.503 & 11.885 & 25.045 & 2.13 & 2.604 & 5.698 & 11.107 & 4.04 & 4.603 & 4.463 & 16.412 & 4.844 & 7.879 \\ 
   \bottomrule 
\end{tabular}
}
\end{table}

%% latex table generated in R 4.0.2 by xtable 1.8-4 package
% Fri Jun 26 10:43:46 2020
\begin{table}[ht]
\centering
\caption{Fixed X, Sparse-Ex3, n=200} 
\scalebox{0.52}{
\begin{tabular}{|c|c|c|cc|cc|cc|ccc|c||cc|cc|cc|ccc|c|}
  \toprule 
 \multicolumn{1}{|c}{} & \multicolumn{1}{c}{} &       & RAICc & AICc  & RC$_p$ & C$_p$ & 10F CV & LOO CV & S$_p$ & $\widehat{\text{C}}_p$ & GCV   & BIC   & RAICc & AICc  & RC$_p$ & C$_p$ & 10F CV & LOO CV & S$_p$ & $\widehat{\text{C}}_p$ & GCV   & BIC  \\
 \cmidrule{4-23}\multicolumn{1}{|c}{} & \multicolumn{1}{c}{} &       & \multicolumn{10}{c||}{RMSE}                                                   & \multicolumn{10}{c|}{Relative Risk}       \\
 \midrule\multirow{9}[6]{*}{hsnr} & \multirow{3}[2]{*}{$\rho=0$} & p=12 & 0.16 & 0.162 & 0.162 & 0.164 & 0.163 & 0.163 & 0.163 & 0.165 & 0.163 & 0.146 & 0.005 & 0.005 & 0.005 & 0.005 & 0.005 & 0.005 & 0.005 & 0.005 & 0.005 & 0.004 \\ 
   &  & p=100 & 0.163 & 0.164 & 0.166 & 0.17 & 0.169 & 0.169 & 0.168 & 0.173 & 0.168 & 0.146 & 0.005 & 0.006 & 0.006 & 0.006 & 0.006 & 0.006 & 0.006 & 0.006 & 0.006 & 0.004 \\ 
   &  & p=196 & 0.163 & 0.164 & 0.271 & 0.379 & 0.169 & 0.169 & 0.168 & 0.46 & 0.169 & 0.257 & 0.005 & 0.006 & 0.041 & 1.771 & 0.006 & 0.006 & 0.006 & 2.982 & 0.009 & 1.419 \\ 
  \cmidrule{2-23} & \multirow{3}[2]{*}{$\rho=0.5$} & p=12 & 0.158 & 0.159 & 0.159 & 0.161 & 0.16 & 0.16 & 0.16 & 0.162 & 0.161 & 0.145 & 0.005 & 0.005 & 0.005 & 0.005 & 0.005 & 0.005 & 0.005 & 0.005 & 0.005 & 0.004 \\ 
   &  & p=100 & 0.159 & 0.162 & 0.163 & 0.167 & 0.166 & 0.166 & 0.165 & 0.17 & 0.166 & 0.145 & 0.005 & 0.005 & 0.005 & 0.006 & 0.006 & 0.006 & 0.006 & 0.006 & 0.006 & 0.004 \\ 
   &  & p=196 & 0.159 & 0.162 & 0.267 & 0.373 & 0.166 & 0.166 & 0.165 & 0.45 & 0.166 & 0.254 & 0.005 & 0.005 & 0.039 & 1.709 & 0.006 & 0.006 & 0.006 & 2.86 & 0.008 & 1.363 \\ 
  \cmidrule{2-23} & \multirow{3}[2]{*}{$\rho=0.9$} & p=12 & 0.155 & 0.157 & 0.157 & 0.158 & 0.159 & 0.158 & 0.158 & 0.159 & 0.158 & 0.143 & 0.005 & 0.005 & 0.005 & 0.005 & 0.005 & 0.005 & 0.005 & 0.005 & 0.005 & 0.004 \\ 
   &  & p=100 & 0.157 & 0.16 & 0.161 & 0.165 & 0.164 & 0.165 & 0.163 & 0.168 & 0.164 & 0.143 & 0.005 & 0.005 & 0.005 & 0.006 & 0.005 & 0.006 & 0.005 & 0.006 & 0.005 & 0.004 \\ 
   &  & p=196 & 0.157 & 0.16 & 0.263 & 0.367 & 0.164 & 0.165 & 0.163 & 0.444 & 0.164 & 0.251 & 0.005 & 0.005 & 0.038 & 1.66 & 0.005 & 0.006 & 0.005 & 2.805 & 0.008 & 1.336 \\ 
  \midrule\multirow{9}[6]{*}{msnr} & \multirow{3}[2]{*}{$\rho=0$} & p=12 & 0.468 & 0.471 & 0.473 & 0.477 & 0.476 & 0.476 & 0.476 & 0.48 & 0.476 & 0.427 & 0.044 & 0.045 & 0.045 & 0.046 & 0.045 & 0.046 & 0.046 & 0.047 & 0.046 & 0.037 \\ 
   &  & p=100 & 0.474 & 0.48 & 0.484 & 0.496 & 0.492 & 0.494 & 0.491 & 0.504 & 0.491 & 0.427 & 0.046 & 0.047 & 0.049 & 0.052 & 0.051 & 0.052 & 0.051 & 0.055 & 0.051 & 0.037 \\ 
   &  & p=196 & 0.474 & 0.48 & 0.791 & 1.104 & 0.492 & 0.494 & 0.491 & 1.341 & 0.493 & 0.751 & 0.046 & 0.047 & 0.345 & 15.052 & 0.051 & 0.052 & 0.051 & 25.351 & 0.075 & 12.059 \\ 
  \cmidrule{2-23} & \multirow{3}[2]{*}{$\rho=0.5$} & p=12 & 0.459 & 0.463 & 0.465 & 0.469 & 0.467 & 0.468 & 0.467 & 0.471 & 0.468 & 0.423 & 0.042 & 0.043 & 0.043 & 0.044 & 0.043 & 0.043 & 0.043 & 0.044 & 0.044 & 0.035 \\ 
   &  & p=100 & 0.465 & 0.472 & 0.474 & 0.487 & 0.484 & 0.484 & 0.482 & 0.497 & 0.483 & 0.423 & 0.043 & 0.045 & 0.046 & 0.05 & 0.048 & 0.049 & 0.048 & 0.053 & 0.048 & 0.035 \\ 
   &  & p=196 & 0.465 & 0.472 & 0.779 & 1.089 & 0.484 & 0.484 & 0.482 & 1.313 & 0.485 & 0.74 & 0.043 & 0.045 & 0.333 & 14.523 & 0.048 & 0.049 & 0.048 & 24.306 & 0.072 & 11.585 \\ 
  \cmidrule{2-23} & \multirow{3}[2]{*}{$\rho=0.9$} & p=12 & 0.453 & 0.457 & 0.459 & 0.461 & 0.463 & 0.461 & 0.461 & 0.464 & 0.461 & 0.418 & 0.04 & 0.04 & 0.041 & 0.041 & 0.041 & 0.041 & 0.041 & 0.042 & 0.041 & 0.033 \\ 
   &  & p=100 & 0.459 & 0.467 & 0.47 & 0.481 & 0.478 & 0.48 & 0.476 & 0.489 & 0.477 & 0.418 & 0.041 & 0.043 & 0.044 & 0.047 & 0.046 & 0.047 & 0.046 & 0.05 & 0.046 & 0.033 \\ 
   &  & p=196 & 0.459 & 0.467 & 0.768 & 1.071 & 0.478 & 0.48 & 0.476 & 1.293 & 0.479 & 0.731 & 0.041 & 0.043 & 0.323 & 14.107 & 0.046 & 0.047 & 0.046 & 23.843 & 0.069 & 11.357 \\ 
  \midrule\multirow{9}[6]{*}{lsnr} & \multirow{3}[2]{*}{$\rho=0$} & p=12 & 1.085 & 1.09 & 1.095 & 1.1 & 1.104 & 1.102 & 1.098 & 1.108 & 1.098 & 1.409 & 0.237 & 0.24 & 0.242 & 0.244 & 0.244 & 0.245 & 0.243 & 0.247 & 0.243 & 0.395 \\ 
   &  & p=100 & 1.098 & 1.109 & 1.121 & 1.147 & 1.139 & 1.139 & 1.131 & 1.159 & 1.131 & 1.409 & 0.247 & 0.253 & 0.261 & 0.278 & 0.27 & 0.272 & 0.268 & 0.286 & 0.268 & 0.395 \\ 
   &  & p=196 & 1.098 & 1.109 & 1.838 & 2.535 & 1.139 & 1.139 & 1.131 & 3.017 & 1.136 & 2.062 & 0.247 & 0.253 & 1.756 & 75.29 & 0.27 & 0.272 & 0.268 & 126.763 & 0.389 & 60.014 \\ 
  \cmidrule{2-23} & \multirow{3}[2]{*}{$\rho=0.5$} & p=12 & 1.062 & 1.066 & 1.069 & 1.075 & 1.08 & 1.079 & 1.071 & 1.079 & 1.073 & 1.386 & 0.224 & 0.226 & 0.228 & 0.23 & 0.231 & 0.231 & 0.228 & 0.232 & 0.229 & 0.383 \\ 
   &  & p=100 & 1.072 & 1.085 & 1.094 & 1.119 & 1.118 & 1.114 & 1.104 & 1.135 & 1.105 & 1.386 & 0.231 & 0.238 & 0.244 & 0.262 & 0.256 & 0.256 & 0.252 & 0.273 & 0.252 & 0.383 \\ 
   &  & p=196 & 1.072 & 1.085 & 1.81 & 2.493 & 1.118 & 1.114 & 1.104 & 2.948 & 1.11 & 2.034 & 0.231 & 0.238 & 1.694 & 72.642 & 0.256 & 0.256 & 0.252 & 121.536 & 0.369 & 58.105 \\ 
  \cmidrule{2-23} & \multirow{3}[2]{*}{$\rho=0.9$} & p=12 & 1.047 & 1.053 & 1.057 & 1.061 & 1.07 & 1.06 & 1.059 & 1.063 & 1.059 & 1.349 & 0.212 & 0.215 & 0.217 & 0.219 & 0.222 & 0.218 & 0.218 & 0.22 & 0.218 & 0.363 \\ 
   &  & p=100 & 1.06 & 1.076 & 1.081 & 1.103 & 1.1 & 1.101 & 1.092 & 1.118 & 1.093 & 1.349 & 0.22 & 0.229 & 0.234 & 0.248 & 0.242 & 0.247 & 0.242 & 0.26 & 0.242 & 0.363 \\ 
   &  & p=196 & 1.06 & 1.076 & 1.782 & 2.451 & 1.1 & 1.101 & 1.092 & 2.906 & 1.098 & 1.987 & 0.22 & 0.229 & 1.643 & 70.559 & 0.242 & 0.247 & 0.242 & 119.22 & 0.356 & 56.434 \\ 
   \midrule 
 \multicolumn{1}{|c}{} & \multicolumn{1}{c}{} &       & \multicolumn{10}{c||}{Relative Test Error}                                    & \multicolumn{10}{c|}{Proportion of Variance Explained} \\
\midrule\multirow{9}[6]{*}{hsnr} & \multirow{3}[2]{*}{$\rho=0$} & p=12 & 1.041 & 1.042 & 1.042 & 1.043 & 1.043 & 1.043 & 1.043 & 1.044 & 1.043 & 1.034 & 0.884 & 0.884 & 0.884 & 0.884 & 0.884 & 0.884 & 0.884 & 0.883 & 0.884 & 0.885 \\ 
   &  & p=100 & 1.043 & 1.044 & 1.046 & 1.049 & 1.048 & 1.048 & 1.048 & 1.051 & 1.048 & 1.034 & 0.884 & 0.883 & 0.883 & 0.883 & 0.883 & 0.883 & 0.883 & 0.883 & 0.883 & 0.885 \\ 
   &  & p=196 & 1.043 & 1.044 & 1.323 & 15.091 & 1.048 & 1.048 & 1.048 & 24.733 & 1.07 & 12.289 & 0.884 & 0.883 & 0.852 & -0.685 & 0.883 & 0.883 & 0.883 & -1.761 & 0.881 & -0.372 \\ 
  \cmidrule{2-23} & \multirow{3}[2]{*}{$\rho=0.5$} & p=12 & 1.041 & 1.041 & 1.042 & 1.042 & 1.042 & 1.042 & 1.042 & 1.043 & 1.042 & 1.034 & 0.887 & 0.887 & 0.887 & 0.887 & 0.887 & 0.887 & 0.887 & 0.887 & 0.887 & 0.888 \\ 
   &  & p=100 & 1.042 & 1.044 & 1.045 & 1.048 & 1.047 & 1.047 & 1.047 & 1.051 & 1.047 & 1.034 & 0.887 & 0.887 & 0.887 & 0.887 & 0.887 & 0.887 & 0.887 & 0.886 & 0.887 & 0.888 \\ 
   &  & p=196 & 1.042 & 1.044 & 1.323 & 15.091 & 1.047 & 1.047 & 1.047 & 24.583 & 1.07 & 12.24 & 0.887 & 0.887 & 0.857 & -0.632 & 0.887 & 0.887 & 0.887 & -1.658 & 0.884 & -0.324 \\ 
  \cmidrule{2-23} & \multirow{3}[2]{*}{$\rho=0.9$} & p=12 & 1.04 & 1.04 & 1.041 & 1.041 & 1.041 & 1.041 & 1.041 & 1.042 & 1.041 & 1.033 & 0.89 & 0.89 & 0.89 & 0.89 & 0.89 & 0.89 & 0.89 & 0.89 & 0.89 & 0.891 \\ 
   &  & p=100 & 1.041 & 1.043 & 1.044 & 1.047 & 1.046 & 1.047 & 1.046 & 1.05 & 1.046 & 1.033 & 0.89 & 0.89 & 0.89 & 0.89 & 0.89 & 0.89 & 0.89 & 0.889 & 0.89 & 0.891 \\ 
   &  & p=196 & 1.041 & 1.043 & 1.322 & 15.09 & 1.046 & 1.047 & 1.046 & 24.815 & 1.069 & 12.344 & 0.89 & 0.89 & 0.861 & -0.59 & 0.89 & 0.89 & 0.89 & -1.615 & 0.887 & -0.301 \\ 
  \midrule\multirow{9}[6]{*}{msnr} & \multirow{3}[2]{*}{$\rho=0$} & p=12 & 1.041 & 1.042 & 1.042 & 1.043 & 1.043 & 1.043 & 1.043 & 1.044 & 1.043 & 1.034 & 0.462 & 0.462 & 0.462 & 0.461 & 0.462 & 0.461 & 0.461 & 0.461 & 0.461 & 0.466 \\ 
   &  & p=100 & 1.043 & 1.044 & 1.046 & 1.049 & 1.048 & 1.048 & 1.048 & 1.051 & 1.048 & 1.034 & 0.461 & 0.461 & 0.46 & 0.458 & 0.459 & 0.459 & 0.459 & 0.457 & 0.459 & 0.466 \\ 
   &  & p=196 & 1.043 & 1.044 & 1.323 & 15.091 & 1.048 & 1.048 & 1.048 & 24.733 & 1.07 & 12.29 & 0.461 & 0.461 & 0.317 & -6.794 & 0.459 & 0.459 & 0.459 & -11.774 & 0.447 & -5.347 \\ 
  \cmidrule{2-23} & \multirow{3}[2]{*}{$\rho=0.5$} & p=12 & 1.041 & 1.041 & 1.042 & 1.042 & 1.042 & 1.042 & 1.042 & 1.043 & 1.042 & 1.034 & 0.472 & 0.471 & 0.471 & 0.471 & 0.471 & 0.471 & 0.471 & 0.471 & 0.471 & 0.475 \\ 
   &  & p=100 & 1.042 & 1.044 & 1.045 & 1.048 & 1.047 & 1.047 & 1.047 & 1.051 & 1.047 & 1.034 & 0.471 & 0.47 & 0.47 & 0.468 & 0.469 & 0.469 & 0.469 & 0.467 & 0.469 & 0.475 \\ 
   &  & p=196 & 1.042 & 1.044 & 1.323 & 15.091 & 1.047 & 1.047 & 1.047 & 24.583 & 1.07 & 12.24 & 0.471 & 0.47 & 0.329 & -6.66 & 0.469 & 0.469 & 0.469 & -11.477 & 0.457 & -5.213 \\ 
  \cmidrule{2-23} & \multirow{3}[2]{*}{$\rho=0.9$} & p=12 & 1.04 & 1.04 & 1.041 & 1.041 & 1.041 & 1.041 & 1.041 & 1.042 & 1.041 & 1.033 & 0.48 & 0.48 & 0.479 & 0.479 & 0.479 & 0.479 & 0.479 & 0.479 & 0.479 & 0.483 \\ 
   &  & p=100 & 1.041 & 1.043 & 1.044 & 1.047 & 1.046 & 1.047 & 1.046 & 1.05 & 1.046 & 1.033 & 0.479 & 0.478 & 0.478 & 0.476 & 0.477 & 0.476 & 0.477 & 0.475 & 0.477 & 0.483 \\ 
   &  & p=196 & 1.041 & 1.043 & 1.322 & 15.09 & 1.046 & 1.047 & 1.046 & 24.815 & 1.069 & 12.344 & 0.479 & 0.478 & 0.338 & -6.549 & 0.477 & 0.476 & 0.477 & -11.415 & 0.465 & -5.176 \\ 
  \midrule\multirow{9}[6]{*}{lsnr} & \multirow{3}[2]{*}{$\rho=0$} & p=12 & 1.044 & 1.045 & 1.045 & 1.046 & 1.046 & 1.046 & 1.045 & 1.046 & 1.045 & 1.074 & 0.12 & 0.12 & 0.12 & 0.119 & 0.119 & 0.119 & 0.119 & 0.119 & 0.119 & 0.095 \\ 
   &  & p=100 & 1.046 & 1.047 & 1.049 & 1.052 & 1.05 & 1.051 & 1.05 & 1.054 & 1.05 & 1.074 & 0.119 & 0.118 & 0.117 & 0.114 & 0.115 & 0.115 & 0.115 & 0.113 & 0.115 & 0.095 \\ 
   &  & p=196 & 1.046 & 1.047 & 1.329 & 15.097 & 1.05 & 1.051 & 1.05 & 24.734 & 1.073 & 12.237 & 0.119 & 0.118 & -0.119 & -11.716 & 0.115 & 0.115 & 0.115 & -19.833 & 0.096 & -9.307 \\ 
  \cmidrule{2-23} & \multirow{3}[2]{*}{$\rho=0.5$} & p=12 & 1.043 & 1.044 & 1.044 & 1.045 & 1.045 & 1.045 & 1.044 & 1.045 & 1.044 & 1.074 & 0.126 & 0.126 & 0.126 & 0.125 & 0.125 & 0.125 & 0.125 & 0.125 & 0.125 & 0.1 \\ 
   &  & p=100 & 1.045 & 1.046 & 1.047 & 1.051 & 1.05 & 1.05 & 1.049 & 1.053 & 1.049 & 1.074 & 0.125 & 0.124 & 0.123 & 0.12 & 0.121 & 0.121 & 0.122 & 0.118 & 0.121 & 0.1 \\ 
   &  & p=196 & 1.045 & 1.046 & 1.329 & 15.096 & 1.05 & 1.05 & 1.049 & 24.584 & 1.072 & 12.275 & 0.125 & 0.124 & -0.113 & -11.643 & 0.121 & 0.121 & 0.122 & -19.589 & 0.102 & -9.28 \\ 
  \cmidrule{2-23} & \multirow{3}[2]{*}{$\rho=0.9$} & p=12 & 1.042 & 1.043 & 1.043 & 1.044 & 1.044 & 1.044 & 1.044 & 1.044 & 1.044 & 1.073 & 0.131 & 0.131 & 0.13 & 0.13 & 0.13 & 0.13 & 0.13 & 0.13 & 0.13 & 0.106 \\ 
   &  & p=100 & 1.044 & 1.046 & 1.047 & 1.05 & 1.048 & 1.049 & 1.048 & 1.052 & 1.048 & 1.073 & 0.13 & 0.128 & 0.128 & 0.125 & 0.126 & 0.125 & 0.126 & 0.123 & 0.126 & 0.106 \\ 
   &  & p=196 & 1.044 & 1.046 & 1.328 & 15.095 & 1.048 & 1.049 & 1.048 & 24.816 & 1.071 & 12.273 & 0.13 & 0.128 & -0.107 & -11.582 & 0.126 & 0.125 & 0.126 & -19.684 & 0.107 & -9.23 \\ 
   \midrule 
 \multicolumn{1}{|c}{} & \multicolumn{1}{c}{} &       & \multicolumn{10}{c||}{Kullback-Leibler Discrepancy}                                    & \multicolumn{10}{c|}{Number of Variables} \\
\midrule\multirow{9}[6]{*}{hsnr} & \multirow{3}[2]{*}{$\rho=0$} & p=12 & 0.764 & 0.765 & 0.765 & 0.766 & 0.766 & 0.766 & 0.766 & 0.767 & 0.766 & 0.756 & 6.508 & 6.562 & 6.595 & 6.657 & 6.7 & 6.66 & 6.639 & 6.726 & 6.64 & 6.032 \\ 
   &  & p=100 & 0.766 & 0.768 & 0.77 & 0.774 & 0.772 & 0.773 & 0.772 & 0.776 & 0.772 & 0.756 & 6.628 & 6.728 & 6.867 & 7.144 & 7.101 & 7.068 & 6.993 & 7.323 & 6.994 & 6.032 \\ 
   &  & p=196 & 0.766 & 0.768 & 1.212 & 105.183 & 0.772 & 0.773 & 0.772 & 144.58 & 2.017 & 108.091 & 6.628 & 6.728 & 25.908 & 59.763 & 7.101 & 7.068 & 6.993 & 84.379 & 7.18 & 35.067 \\ 
  \cmidrule{2-23} & \multirow{3}[2]{*}{$\rho=0.5$} & p=12 & 0.728 & 0.729 & 0.73 & 0.73 & 0.73 & 0.73 & 0.73 & 0.731 & 0.73 & 0.721 & 6.464 & 6.529 & 6.552 & 6.625 & 6.651 & 6.61 & 6.597 & 6.672 & 6.615 & 6.033 \\ 
   &  & p=100 & 0.73 & 0.732 & 0.733 & 0.737 & 0.736 & 0.736 & 0.736 & 0.741 & 0.736 & 0.721 & 6.56 & 6.703 & 6.77 & 7.084 & 7.074 & 6.988 & 6.944 & 7.322 & 6.957 & 6.033 \\ 
   &  & p=196 & 0.73 & 0.732 & 1.177 & 105.147 & 0.736 & 0.736 & 0.736 & 144.347 & 1.981 & 108.028 & 6.56 & 6.703 & 25.926 & 59.793 & 7.074 & 6.988 & 6.944 & 83.869 & 7.143 & 35.067 \\ 
  \cmidrule{2-23} & \multirow{3}[2]{*}{$\rho=0.9$} & p=12 & 0.699 & 0.7 & 0.701 & 0.701 & 0.701 & 0.701 & 0.701 & 0.702 & 0.701 & 0.692 & 6.441 & 6.507 & 6.538 & 6.587 & 6.676 & 6.584 & 6.578 & 6.63 & 6.579 & 6.025 \\ 
   &  & p=100 & 0.701 & 0.703 & 0.705 & 0.709 & 0.707 & 0.708 & 0.707 & 0.711 & 0.707 & 0.692 & 6.55 & 6.705 & 6.802 & 7.075 & 7.043 & 7.022 & 6.936 & 7.264 & 6.953 & 6.025 \\ 
   &  & p=196 & 0.701 & 0.703 & 1.148 & 105.118 & 0.707 & 0.708 & 0.707 & 144.556 & 1.952 & 108.319 & 6.55 & 6.705 & 25.926 & 59.713 & 7.043 & 7.022 & 6.936 & 83.901 & 7.139 & 35.249 \\ 
  \midrule\multirow{9}[6]{*}{msnr} & \multirow{3}[2]{*}{$\rho=0$} & p=12 & 2.904 & 2.905 & 2.905 & 2.906 & 2.906 & 2.906 & 2.906 & 2.907 & 2.906 & 2.896 & 6.508 & 6.562 & 6.595 & 6.657 & 6.7 & 6.66 & 6.639 & 6.726 & 6.64 & 6.03 \\ 
   &  & p=100 & 2.906 & 2.908 & 2.91 & 2.914 & 2.912 & 2.913 & 2.912 & 2.916 & 2.912 & 2.896 & 6.628 & 6.728 & 6.867 & 7.144 & 7.101 & 7.068 & 6.993 & 7.323 & 6.994 & 6.03 \\ 
   &  & p=196 & 2.906 & 2.908 & 3.352 & 107.323 & 2.912 & 2.913 & 2.912 & 146.72 & 4.157 & 110.231 & 6.628 & 6.728 & 25.908 & 59.763 & 7.101 & 7.068 & 6.993 & 84.379 & 7.18 & 35.065 \\ 
  \cmidrule{2-23} & \multirow{3}[2]{*}{$\rho=0.5$} & p=12 & 2.869 & 2.869 & 2.87 & 2.871 & 2.87 & 2.87 & 2.87 & 2.871 & 2.87 & 2.861 & 6.464 & 6.529 & 6.552 & 6.625 & 6.651 & 6.61 & 6.597 & 6.672 & 6.615 & 6.031 \\ 
   &  & p=100 & 2.87 & 2.872 & 2.873 & 2.878 & 2.876 & 2.876 & 2.876 & 2.881 & 2.876 & 2.861 & 6.56 & 6.703 & 6.77 & 7.084 & 7.074 & 6.988 & 6.944 & 7.322 & 6.957 & 6.031 \\ 
   &  & p=196 & 2.87 & 2.872 & 3.317 & 107.287 & 2.876 & 2.876 & 2.876 & 146.487 & 4.121 & 110.169 & 6.56 & 6.703 & 25.926 & 59.793 & 7.074 & 6.988 & 6.944 & 83.869 & 7.143 & 35.066 \\ 
  \cmidrule{2-23} & \multirow{3}[2]{*}{$\rho=0.9$} & p=12 & 2.84 & 2.84 & 2.841 & 2.841 & 2.841 & 2.841 & 2.841 & 2.842 & 2.841 & 2.832 & 6.441 & 6.507 & 6.538 & 6.587 & 6.676 & 6.584 & 6.578 & 6.63 & 6.579 & 6.025 \\ 
   &  & p=100 & 2.841 & 2.844 & 2.845 & 2.849 & 2.847 & 2.848 & 2.847 & 2.852 & 2.847 & 2.832 & 6.55 & 6.705 & 6.802 & 7.075 & 7.043 & 7.022 & 6.936 & 7.264 & 6.953 & 6.025 \\ 
   &  & p=196 & 2.841 & 2.844 & 3.288 & 107.258 & 2.847 & 2.848 & 2.847 & 146.696 & 4.092 & 110.459 & 6.55 & 6.705 & 25.926 & 59.713 & 7.043 & 7.022 & 6.936 & 83.901 & 7.139 & 35.249 \\ 
  \midrule\multirow{9}[6]{*}{lsnr} & \multirow{3}[2]{*}{$\rho=0$} & p=12 & 4.517 & 4.517 & 4.518 & 4.518 & 4.518 & 4.518 & 4.518 & 4.519 & 4.518 & 4.545 & 6.304 & 6.373 & 6.414 & 6.481 & 6.493 & 6.463 & 6.465 & 6.553 & 6.466 & 4.471 \\ 
   &  & p=100 & 4.518 & 4.52 & 4.522 & 4.526 & 4.524 & 4.525 & 4.524 & 4.528 & 4.524 & 4.545 & 6.42 & 6.551 & 6.683 & 6.963 & 6.887 & 6.865 & 6.827 & 7.147 & 6.828 & 4.471 \\ 
   &  & p=196 & 4.518 & 4.52 & 4.967 & 108.938 & 4.524 & 4.525 & 4.524 & 148.331 & 5.769 & 111.571 & 6.42 & 6.551 & 25.638 & 59.456 & 6.887 & 6.865 & 6.827 & 84.267 & 7.014 & 33.529 \\ 
  \cmidrule{2-23} & \multirow{3}[2]{*}{$\rho=0.5$} & p=12 & 4.481 & 4.481 & 4.481 & 4.482 & 4.482 & 4.482 & 4.482 & 4.482 & 4.482 & 4.51 & 6.28 & 6.356 & 6.382 & 6.463 & 6.454 & 6.423 & 6.435 & 6.517 & 6.448 & 4.497 \\ 
   &  & p=100 & 4.482 & 4.484 & 4.485 & 4.489 & 4.488 & 4.488 & 4.487 & 4.492 & 4.487 & 4.51 & 6.368 & 6.523 & 6.585 & 6.917 & 6.888 & 6.806 & 6.787 & 7.177 & 6.8 & 4.497 \\ 
   &  & p=196 & 4.482 & 4.484 & 4.932 & 108.901 & 4.488 & 4.488 & 4.487 & 148.097 & 5.732 & 111.812 & 6.368 & 6.523 & 25.654 & 59.485 & 6.888 & 6.806 & 6.787 & 83.78 & 6.986 & 33.734 \\ 
  \cmidrule{2-23} & \multirow{3}[2]{*}{$\rho=0.9$} & p=12 & 4.452 & 4.452 & 4.453 & 4.453 & 4.453 & 4.453 & 4.453 & 4.453 & 4.453 & 4.48 & 6.245 & 6.325 & 6.357 & 6.428 & 6.482 & 6.413 & 6.402 & 6.485 & 6.411 & 4.559 \\ 
   &  & p=100 & 4.453 & 4.455 & 4.456 & 4.46 & 4.459 & 4.46 & 4.458 & 4.463 & 4.459 & 4.48 & 6.355 & 6.524 & 6.634 & 6.892 & 6.826 & 6.859 & 6.753 & 7.121 & 6.778 & 4.559 \\ 
   &  & p=196 & 4.453 & 4.455 & 4.903 & 108.872 & 4.459 & 4.46 & 4.458 & 148.306 & 5.703 & 111.782 & 6.355 & 6.524 & 25.633 & 59.439 & 6.826 & 6.859 & 6.753 & 83.814 & 6.964 & 33.789 \\ 
   \bottomrule 
\end{tabular}
}
\end{table}

%% latex table generated in R 4.0.2 by xtable 1.8-4 package
% Fri Jun 26 10:43:51 2020
\begin{table}[ht]
\centering
\caption{Random X, Sparse-Ex3, n=1000} 
\scalebox{0.52}{
\begin{tabular}{|c|c|c|cc|cc|cc|ccc|c||cc|cc|cc|ccc|c|}
  \toprule 
 \multicolumn{1}{|c}{} & \multicolumn{1}{c}{} &       & RAICc & AICc  & RC$_p$ & C$_p$ & 10F CV & LOO CV & S$_p$ & $\widehat{\text{C}}_p$ & GCV   & BIC   & RAICc & AICc  & RC$_p$ & C$_p$ & 10F CV & LOO CV & S$_p$ & $\widehat{\text{C}}_p$ & GCV   & BIC  \\
 \cmidrule{4-23}\multicolumn{1}{|c}{} & \multicolumn{1}{c}{} &       & \multicolumn{10}{c||}{RMSE}                                                   & \multicolumn{10}{c|}{Relative Risk}       \\
 \midrule\multirow{9}[6]{*}{hsnr} & \multirow{3}[2]{*}{$\rho=0$} & p=12 & 0.07 & 0.07 & 0.07 & 0.07 & 0.07 & 0.07 & 0.07 & 0.07 & 0.07 & 0.062 & 0.001 & 0.001 & 0.001 & 0.001 & 0.001 & 0.001 & 0.001 & 0.001 & 0.001 & 0.001 \\ 
   &  & p=500 & 0.072 & 0.072 & 0.073 & 0.073 & 0.072 & 0.073 & 0.072 & 0.073 & 0.073 & 0.062 & 0.001 & 0.001 & 0.001 & 0.001 & 0.001 & 0.001 & 0.001 & 0.001 & 0.001 & 0.001 \\ 
   &  & p=996 & 0.072 & 0.072 & 0.184 & 0.291 & 0.072 & 0.073 & 0.072 & 0.361 & 0.073 & 0.138 & 0.001 & 0.001 & 0.026 & 6.795 & 0.001 & 0.001 & 0.001 & 12.379 & 0.001 & 3.79 \\ 
  \cmidrule{2-23} & \multirow{3}[2]{*}{$\rho=0.5$} & p=12 & 0.07 & 0.07 & 0.07 & 0.07 & 0.07 & 0.07 & 0.07 & 0.07 & 0.07 & 0.062 & 0.001 & 0.001 & 0.001 & 0.001 & 0.001 & 0.001 & 0.001 & 0.001 & 0.001 & 0.001 \\ 
   &  & p=500 & 0.072 & 0.072 & 0.072 & 0.073 & 0.072 & 0.072 & 0.072 & 0.072 & 0.072 & 0.062 & 0.001 & 0.001 & 0.001 & 0.001 & 0.001 & 0.001 & 0.001 & 0.001 & 0.001 & 0.001 \\ 
   &  & p=996 & 0.072 & 0.072 & 0.185 & 0.291 & 0.072 & 0.072 & 0.072 & 0.36 & 0.072 & 0.138 & 0.001 & 0.001 & 0.026 & 6.806 & 0.001 & 0.001 & 0.001 & 12.326 & 0.001 & 3.806 \\ 
  \cmidrule{2-23} & \multirow{3}[2]{*}{$\rho=0.9$} & p=12 & 0.07 & 0.07 & 0.07 & 0.07 & 0.07 & 0.07 & 0.07 & 0.07 & 0.07 & 0.062 & 0.001 & 0.001 & 0.001 & 0.001 & 0.001 & 0.001 & 0.001 & 0.001 & 0.001 & 0.001 \\ 
   &  & p=500 & 0.071 & 0.072 & 0.072 & 0.073 & 0.072 & 0.072 & 0.072 & 0.073 & 0.072 & 0.062 & 0.001 & 0.001 & 0.001 & 0.001 & 0.001 & 0.001 & 0.001 & 0.001 & 0.001 & 0.001 \\ 
   &  & p=996 & 0.071 & 0.072 & 0.185 & 0.292 & 0.072 & 0.072 & 0.072 & 0.36 & 0.072 & 0.138 & 0.001 & 0.001 & 0.026 & 6.822 & 0.001 & 0.001 & 0.001 & 12.334 & 0.001 & 3.821 \\ 
  \midrule\multirow{9}[6]{*}{msnr} & \multirow{3}[2]{*}{$\rho=0$} & p=12 & 0.204 & 0.204 & 0.204 & 0.205 & 0.204 & 0.205 & 0.205 & 0.205 & 0.205 & 0.181 & 0.008 & 0.008 & 0.008 & 0.008 & 0.008 & 0.008 & 0.008 & 0.008 & 0.008 & 0.006 \\ 
   &  & p=500 & 0.209 & 0.21 & 0.212 & 0.213 & 0.211 & 0.212 & 0.211 & 0.212 & 0.211 & 0.181 & 0.008 & 0.008 & 0.009 & 0.009 & 0.009 & 0.009 & 0.009 & 0.009 & 0.009 & 0.006 \\ 
   &  & p=996 & 0.209 & 0.21 & 0.538 & 0.848 & 0.211 & 0.212 & 0.211 & 1.053 & 0.211 & 0.401 & 0.008 & 0.008 & 0.221 & 57.756 & 0.009 & 0.009 & 0.009 & 105.22 & 0.009 & 32.212 \\ 
  \cmidrule{2-23} & \multirow{3}[2]{*}{$\rho=0.5$} & p=12 & 0.203 & 0.204 & 0.204 & 0.204 & 0.205 & 0.204 & 0.204 & 0.204 & 0.204 & 0.18 & 0.008 & 0.008 & 0.008 & 0.008 & 0.008 & 0.008 & 0.008 & 0.008 & 0.008 & 0.006 \\ 
   &  & p=500 & 0.209 & 0.21 & 0.211 & 0.212 & 0.211 & 0.211 & 0.21 & 0.211 & 0.21 & 0.18 & 0.008 & 0.008 & 0.009 & 0.009 & 0.009 & 0.009 & 0.008 & 0.009 & 0.008 & 0.006 \\ 
   &  & p=996 & 0.209 & 0.21 & 0.538 & 0.848 & 0.211 & 0.211 & 0.21 & 1.049 & 0.21 & 0.401 & 0.008 & 0.008 & 0.221 & 57.85 & 0.009 & 0.009 & 0.008 & 104.772 & 0.008 & 32.349 \\ 
  \cmidrule{2-23} & \multirow{3}[2]{*}{$\rho=0.9$} & p=12 & 0.203 & 0.203 & 0.203 & 0.204 & 0.203 & 0.204 & 0.203 & 0.204 & 0.204 & 0.181 & 0.008 & 0.008 & 0.008 & 0.008 & 0.008 & 0.008 & 0.008 & 0.008 & 0.008 & 0.006 \\ 
   &  & p=500 & 0.208 & 0.209 & 0.21 & 0.212 & 0.211 & 0.211 & 0.21 & 0.212 & 0.21 & 0.181 & 0.008 & 0.008 & 0.009 & 0.009 & 0.009 & 0.009 & 0.008 & 0.009 & 0.008 & 0.006 \\ 
   &  & p=996 & 0.208 & 0.209 & 0.54 & 0.85 & 0.211 & 0.211 & 0.21 & 1.049 & 0.21 & 0.402 & 0.008 & 0.008 & 0.222 & 57.983 & 0.009 & 0.009 & 0.008 & 104.836 & 0.008 & 32.482 \\ 
  \midrule\multirow{9}[6]{*}{lsnr} & \multirow{3}[2]{*}{$\rho=0$} & p=12 & 0.456 & 0.456 & 0.457 & 0.458 & 0.457 & 0.458 & 0.458 & 0.458 & 0.458 & 0.404 & 0.039 & 0.039 & 0.039 & 0.039 & 0.039 & 0.039 & 0.039 & 0.039 & 0.039 & 0.03 \\ 
   &  & p=500 & 0.468 & 0.469 & 0.475 & 0.476 & 0.472 & 0.473 & 0.472 & 0.474 & 0.473 & 0.404 & 0.042 & 0.042 & 0.043 & 0.044 & 0.043 & 0.043 & 0.043 & 0.043 & 0.043 & 0.03 \\ 
   &  & p=996 & 0.468 & 0.469 & 1.203 & 1.897 & 0.472 & 0.473 & 0.472 & 2.354 & 0.473 & 0.898 & 0.042 & 0.042 & 1.106 & 288.778 & 0.043 & 0.043 & 0.043 & 526.102 & 0.043 & 161.058 \\ 
  \cmidrule{2-23} & \multirow{3}[2]{*}{$\rho=0.5$} & p=12 & 0.455 & 0.455 & 0.456 & 0.456 & 0.457 & 0.457 & 0.456 & 0.457 & 0.456 & 0.404 & 0.039 & 0.039 & 0.039 & 0.039 & 0.039 & 0.039 & 0.039 & 0.039 & 0.039 & 0.03 \\ 
   &  & p=500 & 0.468 & 0.469 & 0.472 & 0.474 & 0.472 & 0.471 & 0.47 & 0.472 & 0.47 & 0.404 & 0.042 & 0.042 & 0.043 & 0.044 & 0.043 & 0.043 & 0.042 & 0.043 & 0.042 & 0.03 \\ 
   &  & p=996 & 0.468 & 0.469 & 1.203 & 1.896 & 0.472 & 0.471 & 0.47 & 2.345 & 0.47 & 0.897 & 0.042 & 0.042 & 1.107 & 289.249 & 0.043 & 0.043 & 0.042 & 523.862 & 0.042 & 161.744 \\ 
  \cmidrule{2-23} & \multirow{3}[2]{*}{$\rho=0.9$} & p=12 & 0.453 & 0.454 & 0.454 & 0.455 & 0.455 & 0.456 & 0.455 & 0.456 & 0.455 & 0.404 & 0.038 & 0.038 & 0.038 & 0.039 & 0.039 & 0.039 & 0.039 & 0.039 & 0.039 & 0.03 \\ 
   &  & p=500 & 0.466 & 0.468 & 0.47 & 0.475 & 0.472 & 0.472 & 0.47 & 0.474 & 0.47 & 0.404 & 0.041 & 0.042 & 0.043 & 0.044 & 0.043 & 0.043 & 0.042 & 0.043 & 0.042 & 0.03 \\ 
   &  & p=996 & 0.466 & 0.468 & 1.208 & 1.901 & 0.472 & 0.472 & 0.47 & 2.346 & 0.47 & 0.899 & 0.041 & 0.042 & 1.108 & 289.917 & 0.043 & 0.043 & 0.042 & 524.181 & 0.042 & 162.41 \\ 
   \midrule 
 \multicolumn{1}{|c}{} & \multicolumn{1}{c}{} &       & \multicolumn{10}{c||}{Relative Test Error}                                    & \multicolumn{10}{c|}{Proportion of Variance Explained} \\
\midrule\multirow{9}[6]{*}{hsnr} & \multirow{3}[2]{*}{$\rho=0$} & p=12 & 1.008 & 1.008 & 1.008 & 1.008 & 1.008 & 1.008 & 1.008 & 1.008 & 1.008 & 1.006 & 0.894 & 0.894 & 0.894 & 0.894 & 0.894 & 0.894 & 0.894 & 0.894 & 0.894 & 0.894 \\ 
   &  & p=500 & 1.008 & 1.008 & 1.009 & 1.009 & 1.009 & 1.009 & 1.009 & 1.009 & 1.009 & 1.006 & 0.894 & 0.894 & 0.894 & 0.894 & 0.894 & 0.894 & 0.894 & 0.894 & 0.894 & 0.894 \\ 
   &  & p=996 & 1.008 & 1.008 & 1.222 & 58.89 & 1.009 & 1.009 & 1.009 & 106.503 & 1.009 & 33.197 & 0.894 & 0.894 & 0.872 & -5.185 & 0.894 & 0.894 & 0.894 & -10.182 & 0.894 & -2.495 \\ 
  \cmidrule{2-23} & \multirow{3}[2]{*}{$\rho=0.5$} & p=12 & 1.008 & 1.008 & 1.008 & 1.008 & 1.008 & 1.008 & 1.008 & 1.008 & 1.008 & 1.006 & 0.894 & 0.894 & 0.894 & 0.894 & 0.894 & 0.894 & 0.894 & 0.894 & 0.894 & 0.894 \\ 
   &  & p=500 & 1.008 & 1.008 & 1.009 & 1.009 & 1.009 & 1.009 & 1.008 & 1.009 & 1.008 & 1.006 & 0.894 & 0.894 & 0.894 & 0.894 & 0.894 & 0.894 & 0.894 & 0.894 & 0.894 & 0.894 \\ 
   &  & p=996 & 1.008 & 1.008 & 1.222 & 58.889 & 1.009 & 1.009 & 1.008 & 106.067 & 1.008 & 33.197 & 0.894 & 0.894 & 0.872 & -5.194 & 0.894 & 0.894 & 0.894 & -10.135 & 0.894 & -2.508 \\ 
  \cmidrule{2-23} & \multirow{3}[2]{*}{$\rho=0.9$} & p=12 & 1.008 & 1.008 & 1.008 & 1.008 & 1.008 & 1.008 & 1.008 & 1.008 & 1.008 & 1.006 & 0.894 & 0.894 & 0.894 & 0.894 & 0.894 & 0.894 & 0.894 & 0.894 & 0.894 & 0.894 \\ 
   &  & p=500 & 1.008 & 1.008 & 1.009 & 1.009 & 1.009 & 1.009 & 1.009 & 1.009 & 1.009 & 1.006 & 0.894 & 0.894 & 0.894 & 0.894 & 0.894 & 0.894 & 0.894 & 0.894 & 0.894 & 0.894 \\ 
   &  & p=996 & 1.008 & 1.008 & 1.222 & 58.89 & 1.009 & 1.009 & 1.009 & 106.067 & 1.009 & 33.197 & 0.894 & 0.894 & 0.872 & -5.206 & 0.894 & 0.894 & 0.894 & -10.141 & 0.894 & -2.52 \\ 
  \midrule\multirow{9}[6]{*}{msnr} & \multirow{3}[2]{*}{$\rho=0$} & p=12 & 1.008 & 1.008 & 1.008 & 1.008 & 1.008 & 1.008 & 1.008 & 1.008 & 1.008 & 1.006 & 0.496 & 0.496 & 0.496 & 0.496 & 0.496 & 0.496 & 0.496 & 0.496 & 0.496 & 0.497 \\ 
   &  & p=500 & 1.008 & 1.008 & 1.009 & 1.009 & 1.009 & 1.009 & 1.009 & 1.009 & 1.009 & 1.006 & 0.496 & 0.496 & 0.496 & 0.496 & 0.496 & 0.496 & 0.496 & 0.496 & 0.496 & 0.497 \\ 
   &  & p=996 & 1.008 & 1.008 & 1.222 & 58.89 & 1.009 & 1.009 & 1.009 & 106.503 & 1.009 & 33.197 & 0.496 & 0.496 & 0.39 & -28.398 & 0.496 & 0.496 & 0.496 & -52.155 & 0.496 & -15.594 \\ 
  \cmidrule{2-23} & \multirow{3}[2]{*}{$\rho=0.5$} & p=12 & 1.008 & 1.008 & 1.008 & 1.008 & 1.008 & 1.008 & 1.008 & 1.008 & 1.008 & 1.006 & 0.497 & 0.497 & 0.497 & 0.497 & 0.497 & 0.497 & 0.497 & 0.497 & 0.497 & 0.497 \\ 
   &  & p=500 & 1.008 & 1.008 & 1.009 & 1.009 & 1.009 & 1.009 & 1.008 & 1.009 & 1.008 & 1.006 & 0.496 & 0.496 & 0.496 & 0.496 & 0.496 & 0.496 & 0.496 & 0.496 & 0.496 & 0.497 \\ 
   &  & p=996 & 1.008 & 1.008 & 1.222 & 58.889 & 1.009 & 1.009 & 1.008 & 106.067 & 1.008 & 33.197 & 0.496 & 0.496 & 0.39 & -28.421 & 0.496 & 0.496 & 0.496 & -51.932 & 0.496 & -15.628 \\ 
  \cmidrule{2-23} & \multirow{3}[2]{*}{$\rho=0.9$} & p=12 & 1.008 & 1.008 & 1.008 & 1.008 & 1.008 & 1.008 & 1.008 & 1.008 & 1.008 & 1.006 & 0.497 & 0.497 & 0.497 & 0.497 & 0.497 & 0.497 & 0.497 & 0.497 & 0.497 & 0.498 \\ 
   &  & p=500 & 1.008 & 1.008 & 1.009 & 1.009 & 1.009 & 1.009 & 1.009 & 1.009 & 1.009 & 1.006 & 0.496 & 0.496 & 0.496 & 0.496 & 0.496 & 0.496 & 0.496 & 0.496 & 0.496 & 0.498 \\ 
   &  & p=996 & 1.008 & 1.008 & 1.222 & 58.89 & 1.009 & 1.009 & 1.009 & 106.067 & 1.009 & 33.197 & 0.496 & 0.496 & 0.39 & -28.453 & 0.496 & 0.496 & 0.496 & -51.946 & 0.496 & -15.66 \\ 
  \midrule\multirow{9}[6]{*}{lsnr} & \multirow{3}[2]{*}{$\rho=0$} & p=12 & 1.008 & 1.008 & 1.008 & 1.008 & 1.008 & 1.008 & 1.008 & 1.008 & 1.008 & 1.006 & 0.16 & 0.16 & 0.16 & 0.16 & 0.16 & 0.16 & 0.16 & 0.16 & 0.16 & 0.162 \\ 
   &  & p=500 & 1.008 & 1.008 & 1.009 & 1.009 & 1.009 & 1.009 & 1.009 & 1.009 & 1.009 & 1.006 & 0.16 & 0.16 & 0.16 & 0.16 & 0.16 & 0.16 & 0.16 & 0.16 & 0.16 & 0.162 \\ 
   &  & p=996 & 1.008 & 1.008 & 1.222 & 58.89 & 1.009 & 1.009 & 1.009 & 106.503 & 1.009 & 33.197 & 0.16 & 0.16 & -0.018 & -48.044 & 0.16 & 0.16 & 0.16 & -87.69 & 0.16 & -26.659 \\ 
  \cmidrule{2-23} & \multirow{3}[2]{*}{$\rho=0.5$} & p=12 & 1.008 & 1.008 & 1.008 & 1.008 & 1.008 & 1.008 & 1.008 & 1.008 & 1.008 & 1.006 & 0.161 & 0.161 & 0.161 & 0.161 & 0.16 & 0.16 & 0.161 & 0.16 & 0.161 & 0.162 \\ 
   &  & p=500 & 1.008 & 1.008 & 1.009 & 1.009 & 1.009 & 1.009 & 1.008 & 1.009 & 1.008 & 1.006 & 0.16 & 0.16 & 0.16 & 0.16 & 0.16 & 0.16 & 0.16 & 0.16 & 0.16 & 0.162 \\ 
   &  & p=996 & 1.008 & 1.008 & 1.222 & 58.889 & 1.009 & 1.009 & 1.008 & 106.067 & 1.008 & 33.197 & 0.16 & 0.16 & -0.018 & -48.056 & 0.16 & 0.16 & 0.16 & -87.323 & 0.16 & -26.677 \\ 
  \cmidrule{2-23} & \multirow{3}[2]{*}{$\rho=0.9$} & p=12 & 1.008 & 1.008 & 1.008 & 1.008 & 1.008 & 1.008 & 1.008 & 1.008 & 1.008 & 1.006 & 0.161 & 0.161 & 0.161 & 0.161 & 0.161 & 0.161 & 0.161 & 0.161 & 0.161 & 0.162 \\ 
   &  & p=500 & 1.008 & 1.008 & 1.009 & 1.009 & 1.009 & 1.009 & 1.009 & 1.009 & 1.009 & 1.006 & 0.16 & 0.16 & 0.16 & 0.16 & 0.16 & 0.16 & 0.16 & 0.16 & 0.16 & 0.162 \\ 
   &  & p=996 & 1.008 & 1.008 & 1.222 & 58.89 & 1.009 & 1.009 & 1.009 & 106.067 & 1.009 & 33.197 & 0.16 & 0.16 & -0.018 & -48.074 & 0.16 & 0.16 & 0.16 & -87.33 & 0.16 & -26.695 \\ 
   \midrule 
 \multicolumn{1}{|c}{} & \multicolumn{1}{c}{} &       & \multicolumn{10}{c||}{Kullback-Leibler Discrepancy}                                    & \multicolumn{10}{c|}{Number of Variables} \\
\midrule\multirow{9}[6]{*}{hsnr} & \multirow{3}[2]{*}{$\rho=0$} & p=12 & 0.659 & 0.659 & 0.659 & 0.659 & 0.659 & 0.659 & 0.659 & 0.659 & 0.659 & 0.657 & 6.617 & 6.62 & 6.631 & 6.639 & 6.68 & 6.653 & 6.639 & 6.645 & 6.639 & 6.011 \\ 
   &  & p=500 & 0.66 & 0.66 & 0.66 & 0.66 & 0.66 & 0.66 & 0.66 & 0.66 & 0.66 & 0.657 & 6.861 & 6.886 & 7.005 & 7.047 & 7.025 & 6.961 & 6.938 & 6.999 & 6.957 & 6.011 \\ 
   &  & p=996 & 0.66 & 0.66 & 1.456 & 76903.349 & 0.66 & 0.66 & 0.66 & 107434.883 & 0.66 & 68391.611 & 6.861 & 6.886 & 96.618 & 275.033 & 7.025 & 6.961 & 6.938 & 379.786 & 6.957 & 103.022 \\ 
  \cmidrule{2-23} & \multirow{3}[2]{*}{$\rho=0.5$} & p=12 & 0.659 & 0.659 & 0.659 & 0.659 & 0.659 & 0.659 & 0.659 & 0.659 & 0.659 & 0.657 & 6.598 & 6.609 & 6.616 & 6.622 & 6.703 & 6.638 & 6.622 & 6.636 & 6.622 & 6.011 \\ 
   &  & p=500 & 0.66 & 0.66 & 0.66 & 0.66 & 0.66 & 0.66 & 0.66 & 0.66 & 0.66 & 0.657 & 6.859 & 6.883 & 6.974 & 7.031 & 7.054 & 6.925 & 6.902 & 6.967 & 6.902 & 6.011 \\ 
   &  & p=996 & 0.66 & 0.66 & 1.456 & 76903.349 & 0.66 & 0.66 & 0.66 & 107212.014 & 0.66 & 68391.611 & 6.859 & 6.883 & 96.635 & 274.995 & 7.054 & 6.925 & 6.902 & 378.725 & 6.902 & 103.02 \\ 
  \cmidrule{2-23} & \multirow{3}[2]{*}{$\rho=0.9$} & p=12 & 0.658 & 0.658 & 0.658 & 0.659 & 0.658 & 0.659 & 0.659 & 0.659 & 0.659 & 0.657 & 6.569 & 6.585 & 6.585 & 6.606 & 6.656 & 6.621 & 6.598 & 6.62 & 6.607 & 6.01 \\ 
   &  & p=500 & 0.659 & 0.659 & 0.659 & 0.66 & 0.659 & 0.659 & 0.659 & 0.66 & 0.659 & 0.657 & 6.823 & 6.863 & 6.931 & 7.058 & 7.054 & 6.963 & 6.914 & 7.003 & 6.923 & 6.01 \\ 
   &  & p=996 & 0.659 & 0.659 & 1.456 & 76903.349 & 0.659 & 0.659 & 0.659 & 107212.013 & 0.659 & 68391.611 & 6.823 & 6.863 & 96.769 & 275.168 & 7.054 & 6.963 & 6.914 & 378.769 & 6.923 & 103.02 \\ 
  \midrule\multirow{9}[6]{*}{msnr} & \multirow{3}[2]{*}{$\rho=0$} & p=12 & 2.799 & 2.799 & 2.799 & 2.799 & 2.799 & 2.799 & 2.799 & 2.799 & 2.799 & 2.797 & 6.617 & 6.62 & 6.631 & 6.639 & 6.68 & 6.653 & 6.639 & 6.645 & 6.639 & 6.011 \\ 
   &  & p=500 & 2.8 & 2.8 & 2.8 & 2.8 & 2.8 & 2.8 & 2.8 & 2.8 & 2.8 & 2.797 & 6.861 & 6.886 & 7.005 & 7.047 & 7.025 & 6.961 & 6.938 & 6.999 & 6.957 & 6.011 \\ 
   &  & p=996 & 2.8 & 2.8 & 3.596 & 76905.489 & 2.8 & 2.8 & 2.8 & 107437.024 & 2.8 & 68393.751 & 6.861 & 6.886 & 96.618 & 275.033 & 7.025 & 6.961 & 6.938 & 379.786 & 6.957 & 103.022 \\ 
  \cmidrule{2-23} & \multirow{3}[2]{*}{$\rho=0.5$} & p=12 & 2.799 & 2.799 & 2.799 & 2.799 & 2.799 & 2.799 & 2.799 & 2.799 & 2.799 & 2.797 & 6.598 & 6.609 & 6.616 & 6.622 & 6.703 & 6.638 & 6.622 & 6.636 & 6.622 & 6.011 \\ 
   &  & p=500 & 2.8 & 2.8 & 2.8 & 2.8 & 2.8 & 2.8 & 2.8 & 2.8 & 2.8 & 2.797 & 6.859 & 6.883 & 6.974 & 7.031 & 7.054 & 6.925 & 6.902 & 6.967 & 6.902 & 6.011 \\ 
   &  & p=996 & 2.8 & 2.8 & 3.596 & 76905.489 & 2.8 & 2.8 & 2.8 & 107214.154 & 2.8 & 68393.751 & 6.859 & 6.883 & 96.635 & 274.995 & 7.054 & 6.925 & 6.902 & 378.725 & 6.902 & 103.02 \\ 
  \cmidrule{2-23} & \multirow{3}[2]{*}{$\rho=0.9$} & p=12 & 2.799 & 2.799 & 2.799 & 2.799 & 2.799 & 2.799 & 2.799 & 2.799 & 2.799 & 2.797 & 6.569 & 6.585 & 6.585 & 6.606 & 6.656 & 6.621 & 6.598 & 6.62 & 6.607 & 6.01 \\ 
   &  & p=500 & 2.799 & 2.799 & 2.799 & 2.8 & 2.799 & 2.799 & 2.799 & 2.8 & 2.799 & 2.797 & 6.823 & 6.863 & 6.931 & 7.058 & 7.054 & 6.963 & 6.914 & 7.003 & 6.923 & 6.01 \\ 
   &  & p=996 & 2.799 & 2.799 & 3.596 & 76905.489 & 2.799 & 2.799 & 2.799 & 107214.153 & 2.799 & 68393.751 & 6.823 & 6.863 & 96.769 & 275.168 & 7.054 & 6.963 & 6.914 & 378.769 & 6.923 & 103.02 \\ 
  \midrule\multirow{9}[6]{*}{lsnr} & \multirow{3}[2]{*}{$\rho=0$} & p=12 & 4.409 & 4.409 & 4.409 & 4.409 & 4.409 & 4.409 & 4.409 & 4.409 & 4.409 & 4.407 & 6.617 & 6.62 & 6.631 & 6.639 & 6.68 & 6.653 & 6.639 & 6.645 & 6.639 & 6.011 \\ 
   &  & p=500 & 4.409 & 4.409 & 4.41 & 4.41 & 4.409 & 4.41 & 4.409 & 4.41 & 4.41 & 4.407 & 6.861 & 6.886 & 7.005 & 7.047 & 7.025 & 6.961 & 6.938 & 6.999 & 6.957 & 6.011 \\ 
   &  & p=996 & 4.409 & 4.409 & 5.206 & 76907.098 & 4.409 & 4.41 & 4.409 & 107438.633 & 4.41 & 68395.361 & 6.861 & 6.886 & 96.618 & 275.033 & 7.025 & 6.961 & 6.938 & 379.786 & 6.957 & 103.022 \\ 
  \cmidrule{2-23} & \multirow{3}[2]{*}{$\rho=0.5$} & p=12 & 4.408 & 4.408 & 4.408 & 4.408 & 4.408 & 4.408 & 4.408 & 4.408 & 4.408 & 4.407 & 6.598 & 6.609 & 6.616 & 6.622 & 6.703 & 6.638 & 6.622 & 6.636 & 6.622 & 6.01 \\ 
   &  & p=500 & 4.409 & 4.409 & 4.409 & 4.409 & 4.409 & 4.409 & 4.409 & 4.409 & 4.409 & 4.407 & 6.859 & 6.883 & 6.974 & 7.031 & 7.054 & 6.925 & 6.902 & 6.967 & 6.902 & 6.01 \\ 
   &  & p=996 & 4.409 & 4.409 & 5.205 & 76907.098 & 4.409 & 4.409 & 4.409 & 107215.763 & 4.409 & 68395.361 & 6.859 & 6.883 & 96.635 & 274.995 & 7.054 & 6.925 & 6.902 & 378.725 & 6.902 & 103.019 \\ 
  \cmidrule{2-23} & \multirow{3}[2]{*}{$\rho=0.9$} & p=12 & 4.408 & 4.408 & 4.408 & 4.408 & 4.408 & 4.408 & 4.408 & 4.408 & 4.408 & 4.406 & 6.569 & 6.585 & 6.585 & 6.606 & 6.656 & 6.621 & 6.598 & 6.62 & 6.607 & 6.01 \\ 
   &  & p=500 & 4.409 & 4.409 & 4.409 & 4.409 & 4.409 & 4.409 & 4.409 & 4.409 & 4.409 & 4.406 & 6.823 & 6.863 & 6.931 & 7.058 & 7.054 & 6.963 & 6.914 & 7.003 & 6.923 & 6.01 \\ 
   &  & p=996 & 4.409 & 4.409 & 5.205 & 76907.098 & 4.409 & 4.409 & 4.409 & 107215.763 & 4.409 & 68395.36 & 6.823 & 6.863 & 96.769 & 275.168 & 7.054 & 6.963 & 6.914 & 378.769 & 6.923 & 103.02 \\ 
   \bottomrule 
\end{tabular}
}
\end{table}

%% latex table generated in R 4.0.2 by xtable 1.8-4 package
% Fri Jun 26 10:43:46 2020
\begin{table}[ht]
\centering
\caption{Fixed X, Sparse-Ex4, n=40} 
\scalebox{0.52}{
\begin{tabular}{|c|c|c|cc|cc|cc|ccc|c||cc|cc|cc|ccc|c|}
  \toprule 
 \multicolumn{1}{|c}{} & \multicolumn{1}{c}{} &       & RAICc & AICc  & RC$_p$ & C$_p$ & 10F CV & LOO CV & S$_p$ & $\widehat{\text{C}}_p$ & GCV   & BIC   & RAICc & AICc  & RC$_p$ & C$_p$ & 10F CV & LOO CV & S$_p$ & $\widehat{\text{C}}_p$ & GCV   & BIC  \\
 \cmidrule{4-23}\multicolumn{1}{|c}{} & \multicolumn{1}{c}{} &       & \multicolumn{10}{c||}{RMSE}                                                   & \multicolumn{10}{c|}{Relative Risk}       \\
 \midrule\multirow{9}[6]{*}{hsnr} & \multirow{3}[2]{*}{$\rho=0$} & p=12 & 1.082 & 1.116 & 1.135 & 1.195 & 1.189 & 1.182 & 1.187 & 1.235 & 1.191 & 1.104 & 0.025 & 0.027 & 0.028 & 0.032 & 0.032 & 0.031 & 0.031 & 0.034 & 0.032 & 0.027 \\ 
   &  & p=20 & 1.082 & 1.12 & 1.159 & 1.271 & 1.231 & 1.234 & 1.238 & 1.374 & 1.243 & 1.118 & 0.025 & 0.028 & 0.031 & 0.043 & 0.037 & 0.037 & 0.038 & 0.053 & 0.038 & 0.028 \\ 
   &  & p=36 & 1.082 & 1.12 & 1.301 & 1.584 & 1.248 & 1.279 & 1.277 & 1.891 & 1.308 & 1.538 & 0.025 & 0.028 & 0.056 & 0.25 & 0.041 & 0.058 & 0.058 & 0.398 & 0.074 & 0.296 \\ 
  \cmidrule{2-23} & \multirow{3}[2]{*}{$\rho=0.5$} & p=12 & 0.765 & 0.789 & 0.802 & 0.845 & 0.84 & 0.836 & 0.839 & 0.873 & 0.842 & 0.781 & 0.025 & 0.027 & 0.028 & 0.032 & 0.032 & 0.031 & 0.031 & 0.034 & 0.032 & 0.027 \\ 
   &  & p=20 & 0.765 & 0.792 & 0.819 & 0.899 & 0.871 & 0.872 & 0.875 & 0.972 & 0.879 & 0.79 & 0.025 & 0.028 & 0.031 & 0.043 & 0.037 & 0.037 & 0.038 & 0.053 & 0.038 & 0.028 \\ 
   &  & p=36 & 0.765 & 0.792 & 0.92 & 1.12 & 0.882 & 0.904 & 0.903 & 1.337 & 0.925 & 1.087 & 0.025 & 0.028 & 0.056 & 0.25 & 0.041 & 0.058 & 0.058 & 0.398 & 0.074 & 0.296 \\ 
  \cmidrule{2-23} & \multirow{3}[2]{*}{$\rho=0.9$} & p=12 & 0.342 & 0.353 & 0.359 & 0.378 & 0.376 & 0.374 & 0.375 & 0.391 & 0.377 & 0.349 & 0.025 & 0.027 & 0.028 & 0.032 & 0.032 & 0.031 & 0.031 & 0.034 & 0.032 & 0.027 \\ 
   &  & p=20 & 0.342 & 0.354 & 0.366 & 0.402 & 0.389 & 0.39 & 0.391 & 0.435 & 0.393 & 0.353 & 0.025 & 0.028 & 0.031 & 0.043 & 0.037 & 0.037 & 0.038 & 0.053 & 0.038 & 0.028 \\ 
   &  & p=36 & 0.342 & 0.354 & 0.411 & 0.501 & 0.395 & 0.404 & 0.404 & 0.598 & 0.414 & 0.486 & 0.025 & 0.028 & 0.056 & 0.25 & 0.041 & 0.058 & 0.058 & 0.398 & 0.074 & 0.296 \\ 
  \midrule\multirow{9}[6]{*}{msnr} & \multirow{3}[2]{*}{$\rho=0$} & p=12 & 3.276 & 3.325 & 3.371 & 3.505 & 3.526 & 3.477 & 3.493 & 3.618 & 3.504 & 3.569 & 0.241 & 0.248 & 0.255 & 0.275 & 0.282 & 0.271 & 0.274 & 0.296 & 0.276 & 0.292 \\ 
   &  & p=20 & 3.276 & 3.336 & 3.43 & 3.737 & 3.65 & 3.622 & 3.639 & 4.022 & 3.656 & 3.608 & 0.241 & 0.252 & 0.273 & 0.37 & 0.326 & 0.323 & 0.328 & 0.457 & 0.332 & 0.305 \\ 
   &  & p=36 & 3.276 & 3.336 & 4.019 & 4.755 & 3.698 & 3.755 & 3.753 & 5.518 & 3.844 & 4.762 & 0.241 & 0.252 & 0.521 & 2.15 & 0.362 & 0.5 & 0.5 & 3.376 & 0.636 & 2.551 \\ 
  \cmidrule{2-23} & \multirow{3}[2]{*}{$\rho=0.5$} & p=12 & 2.299 & 2.335 & 2.358 & 2.469 & 2.476 & 2.447 & 2.457 & 2.551 & 2.464 & 2.5 & 0.236 & 0.243 & 0.247 & 0.273 & 0.278 & 0.267 & 0.27 & 0.294 & 0.272 & 0.284 \\ 
   &  & p=20 & 2.299 & 2.343 & 2.405 & 2.628 & 2.563 & 2.549 & 2.56 & 2.836 & 2.572 & 2.527 & 0.236 & 0.246 & 0.266 & 0.366 & 0.321 & 0.32 & 0.324 & 0.455 & 0.328 & 0.297 \\ 
   &  & p=36 & 2.299 & 2.343 & 2.813 & 3.35 & 2.598 & 2.644 & 2.64 & 3.9 & 2.705 & 3.352 & 0.236 & 0.246 & 0.511 & 2.145 & 0.357 & 0.497 & 0.496 & 3.383 & 0.632 & 2.556 \\ 
  \cmidrule{2-23} & \multirow{3}[2]{*}{$\rho=0.9$} & p=12 & 1.024 & 1.042 & 1.053 & 1.104 & 1.106 & 1.094 & 1.098 & 1.141 & 1.102 & 1.115 & 0.233 & 0.241 & 0.246 & 0.272 & 0.277 & 0.267 & 0.27 & 0.294 & 0.272 & 0.281 \\ 
   &  & p=20 & 1.024 & 1.046 & 1.074 & 1.175 & 1.145 & 1.14 & 1.145 & 1.268 & 1.15 & 1.127 & 0.233 & 0.244 & 0.265 & 0.365 & 0.321 & 0.32 & 0.324 & 0.455 & 0.328 & 0.294 \\ 
   &  & p=36 & 1.024 & 1.046 & 1.254 & 1.497 & 1.161 & 1.182 & 1.18 & 1.744 & 1.209 & 1.497 & 0.233 & 0.244 & 0.508 & 2.144 & 0.357 & 0.496 & 0.496 & 3.383 & 0.631 & 2.554 \\ 
  \midrule\multirow{9}[6]{*}{lsnr} & \multirow{3}[2]{*}{$\rho=0$} & p=12 & 8.458 & 8.428 & 8.417 & 8.547 & 8.505 & 8.518 & 8.513 & 8.621 & 8.498 & 8.506 & 1.306 & 1.324 & 1.38 & 1.469 & 1.42 & 1.424 & 1.431 & 1.516 & 1.435 & 1.2 \\ 
   &  & p=20 & 8.458 & 8.441 & 8.505 & 8.962 & 8.675 & 8.734 & 8.711 & 9.283 & 8.713 & 8.527 & 1.306 & 1.333 & 1.441 & 1.867 & 1.57 & 1.608 & 1.598 & 2.154 & 1.615 & 1.218 \\ 
   &  & p=36 & 8.458 & 8.441 & 9.267 & 10.921 & 8.73 & 8.915 & 8.897 & 12.368 & 8.993 & 10.351 & 1.306 & 1.333 & 2.466 & 10.573 & 1.682 & 2.216 & 2.337 & 16.202 & 2.848 & 9.784 \\ 
  \cmidrule{2-23} & \multirow{3}[2]{*}{$\rho=0.5$} & p=12 & 6.037 & 5.993 & 5.978 & 6.082 & 6.047 & 6.055 & 6.059 & 6.136 & 6.053 & 6.023 & 1.317 & 1.332 & 1.391 & 1.492 & 1.435 & 1.436 & 1.449 & 1.535 & 1.456 & 1.199 \\ 
   &  & p=20 & 6.037 & 5.999 & 6.041 & 6.372 & 6.17 & 6.2 & 6.195 & 6.598 & 6.205 & 6.041 & 1.317 & 1.339 & 1.454 & 1.88 & 1.588 & 1.614 & 1.616 & 2.173 & 1.64 & 1.22 \\ 
   &  & p=36 & 6.037 & 5.999 & 6.571 & 7.738 & 6.211 & 6.333 & 6.324 & 8.77 & 6.41 & 7.335 & 1.317 & 1.339 & 2.476 & 10.58 & 1.705 & 2.23 & 2.354 & 16.229 & 2.888 & 9.802 \\ 
  \cmidrule{2-23} & \multirow{3}[2]{*}{$\rho=0.9$} & p=12 & 2.703 & 2.68 & 2.677 & 2.716 & 2.709 & 2.714 & 2.715 & 2.744 & 2.711 & 2.687 & 1.306 & 1.322 & 1.387 & 1.485 & 1.435 & 1.439 & 1.449 & 1.535 & 1.457 & 1.177 \\ 
   &  & p=20 & 2.703 & 2.683 & 2.708 & 2.853 & 2.761 & 2.779 & 2.777 & 2.952 & 2.779 & 2.695 & 1.306 & 1.329 & 1.451 & 1.883 & 1.581 & 1.619 & 1.619 & 2.174 & 1.641 & 1.198 \\ 
   &  & p=36 & 2.703 & 2.683 & 2.939 & 3.458 & 2.781 & 2.838 & 2.835 & 3.929 & 2.869 & 3.279 & 1.306 & 1.329 & 2.473 & 10.578 & 1.709 & 2.233 & 2.357 & 16.292 & 2.887 & 9.833 \\ 
   \midrule 
 \multicolumn{1}{|c}{} & \multicolumn{1}{c}{} &       & \multicolumn{10}{c||}{Relative Test Error}                                    & \multicolumn{10}{c|}{Proportion of Variance Explained} \\
\midrule\multirow{9}[6]{*}{hsnr} & \multirow{3}[2]{*}{$\rho=0$} & p=12 & 1.229 & 1.247 & 1.256 & 1.288 & 1.287 & 1.281 & 1.284 & 1.311 & 1.286 & 1.241 & 0.878 & 0.876 & 0.875 & 0.872 & 0.872 & 0.873 & 0.872 & 0.87 & 0.872 & 0.877 \\ 
   &  & p=20 & 1.229 & 1.251 & 1.278 & 1.387 & 1.333 & 1.34 & 1.342 & 1.484 & 1.346 & 1.255 & 0.878 & 0.876 & 0.873 & 0.862 & 0.867 & 0.867 & 0.867 & 0.853 & 0.866 & 0.875 \\ 
   &  & p=36 & 1.229 & 1.251 & 1.507 & 3.263 & 1.372 & 1.527 & 1.526 & 4.606 & 1.67 & 3.682 & 0.878 & 0.876 & 0.85 & 0.676 & 0.864 & 0.848 & 0.848 & 0.542 & 0.834 & 0.634 \\ 
  \cmidrule{2-23} & \multirow{3}[2]{*}{$\rho=0.5$} & p=12 & 1.229 & 1.247 & 1.256 & 1.288 & 1.287 & 1.281 & 1.284 & 1.311 & 1.286 & 1.241 & 0.878 & 0.876 & 0.875 & 0.872 & 0.872 & 0.873 & 0.872 & 0.87 & 0.872 & 0.877 \\ 
   &  & p=20 & 1.229 & 1.251 & 1.278 & 1.387 & 1.333 & 1.34 & 1.342 & 1.484 & 1.346 & 1.255 & 0.878 & 0.876 & 0.873 & 0.862 & 0.867 & 0.867 & 0.867 & 0.853 & 0.866 & 0.875 \\ 
   &  & p=36 & 1.229 & 1.251 & 1.507 & 3.263 & 1.372 & 1.527 & 1.526 & 4.606 & 1.67 & 3.682 & 0.878 & 0.876 & 0.85 & 0.676 & 0.864 & 0.848 & 0.848 & 0.542 & 0.834 & 0.634 \\ 
  \cmidrule{2-23} & \multirow{3}[2]{*}{$\rho=0.9$} & p=12 & 1.229 & 1.247 & 1.256 & 1.288 & 1.287 & 1.281 & 1.284 & 1.311 & 1.286 & 1.241 & 0.878 & 0.876 & 0.875 & 0.872 & 0.872 & 0.873 & 0.872 & 0.87 & 0.872 & 0.877 \\ 
   &  & p=20 & 1.229 & 1.251 & 1.278 & 1.387 & 1.333 & 1.34 & 1.342 & 1.484 & 1.346 & 1.255 & 0.878 & 0.876 & 0.873 & 0.862 & 0.867 & 0.867 & 0.867 & 0.853 & 0.866 & 0.875 \\ 
   &  & p=36 & 1.229 & 1.251 & 1.507 & 3.263 & 1.372 & 1.527 & 1.526 & 4.606 & 1.67 & 3.682 & 0.878 & 0.876 & 0.85 & 0.676 & 0.864 & 0.848 & 0.848 & 0.542 & 0.834 & 0.634 \\ 
  \midrule\multirow{9}[6]{*}{msnr} & \multirow{3}[2]{*}{$\rho=0$} & p=12 & 1.256 & 1.265 & 1.272 & 1.293 & 1.301 & 1.288 & 1.292 & 1.316 & 1.294 & 1.311 & 0.392 & 0.388 & 0.385 & 0.374 & 0.371 & 0.376 & 0.374 & 0.363 & 0.374 & 0.365 \\ 
   &  & p=20 & 1.256 & 1.269 & 1.291 & 1.395 & 1.347 & 1.345 & 1.35 & 1.488 & 1.354 & 1.325 & 0.392 & 0.386 & 0.375 & 0.325 & 0.348 & 0.349 & 0.347 & 0.28 & 0.345 & 0.359 \\ 
   &  & p=36 & 1.256 & 1.269 & 1.555 & 3.292 & 1.385 & 1.533 & 1.534 & 4.6 & 1.678 & 3.72 & 0.392 & 0.386 & 0.247 & -0.593 & 0.329 & 0.258 & 0.258 & -1.226 & 0.188 & -0.8 \\ 
  \cmidrule{2-23} & \multirow{3}[2]{*}{$\rho=0.5$} & p=12 & 1.252 & 1.259 & 1.263 & 1.291 & 1.296 & 1.285 & 1.288 & 1.313 & 1.29 & 1.303 & 0.394 & 0.391 & 0.389 & 0.375 & 0.373 & 0.378 & 0.377 & 0.364 & 0.376 & 0.369 \\ 
   &  & p=20 & 1.252 & 1.263 & 1.284 & 1.39 & 1.343 & 1.341 & 1.346 & 1.485 & 1.35 & 1.317 & 0.394 & 0.389 & 0.379 & 0.327 & 0.35 & 0.351 & 0.349 & 0.281 & 0.347 & 0.363 \\ 
   &  & p=36 & 1.252 & 1.263 & 1.545 & 3.287 & 1.381 & 1.53 & 1.529 & 4.607 & 1.673 & 3.725 & 0.394 & 0.389 & 0.252 & -0.591 & 0.332 & 0.26 & 0.26 & -1.23 & 0.19 & -0.803 \\ 
  \cmidrule{2-23} & \multirow{3}[2]{*}{$\rho=0.9$} & p=12 & 1.248 & 1.257 & 1.262 & 1.29 & 1.295 & 1.285 & 1.288 & 1.314 & 1.29 & 1.299 & 0.396 & 0.392 & 0.389 & 0.375 & 0.373 & 0.378 & 0.377 & 0.364 & 0.376 & 0.371 \\ 
   &  & p=20 & 1.248 & 1.26 & 1.283 & 1.39 & 1.342 & 1.341 & 1.346 & 1.485 & 1.35 & 1.314 & 0.396 & 0.39 & 0.379 & 0.327 & 0.35 & 0.351 & 0.349 & 0.281 & 0.347 & 0.364 \\ 
   &  & p=36 & 1.248 & 1.26 & 1.542 & 3.286 & 1.38 & 1.529 & 1.529 & 4.607 & 1.673 & 3.723 & 0.396 & 0.39 & 0.254 & -0.59 & 0.332 & 0.26 & 0.26 & -1.23 & 0.19 & -0.802 \\ 
  \midrule\multirow{9}[6]{*}{lsnr} & \multirow{3}[2]{*}{$\rho=0$} & p=12 & 1.279 & 1.282 & 1.294 & 1.313 & 1.303 & 1.304 & 1.305 & 1.323 & 1.306 & 1.256 & -0.054 & -0.057 & -0.067 & -0.082 & -0.074 & -0.075 & -0.076 & -0.091 & -0.076 & -0.035 \\ 
   &  & p=20 & 1.279 & 1.284 & 1.307 & 1.398 & 1.335 & 1.343 & 1.341 & 1.459 & 1.344 & 1.26 & -0.054 & -0.059 & -0.078 & -0.152 & -0.1 & -0.107 & -0.105 & -0.203 & -0.108 & -0.038 \\ 
   &  & p=36 & 1.279 & 1.284 & 1.526 & 3.255 & 1.359 & 1.473 & 1.498 & 4.455 & 1.607 & 3.086 & -0.054 & -0.059 & -0.258 & -1.683 & -0.12 & -0.214 & -0.235 & -2.672 & -0.325 & -1.544 \\ 
  \cmidrule{2-23} & \multirow{3}[2]{*}{$\rho=0.5$} & p=12 & 1.281 & 1.284 & 1.297 & 1.318 & 1.306 & 1.306 & 1.309 & 1.327 & 1.311 & 1.256 & -0.056 & -0.058 & -0.069 & -0.086 & -0.076 & -0.077 & -0.079 & -0.094 & -0.08 & -0.035 \\ 
   &  & p=20 & 1.281 & 1.286 & 1.31 & 1.401 & 1.339 & 1.344 & 1.345 & 1.463 & 1.35 & 1.26 & -0.056 & -0.06 & -0.08 & -0.155 & -0.103 & -0.108 & -0.108 & -0.206 & -0.112 & -0.039 \\ 
   &  & p=36 & 1.281 & 1.286 & 1.528 & 3.256 & 1.364 & 1.476 & 1.502 & 4.461 & 1.616 & 3.09 & -0.056 & -0.06 & -0.259 & -1.684 & -0.124 & -0.216 & -0.238 & -2.677 & -0.332 & -1.547 \\ 
  \cmidrule{2-23} & \multirow{3}[2]{*}{$\rho=0.9$} & p=12 & 1.279 & 1.282 & 1.296 & 1.317 & 1.306 & 1.307 & 1.309 & 1.327 & 1.311 & 1.251 & -0.054 & -0.057 & -0.068 & -0.085 & -0.076 & -0.077 & -0.079 & -0.094 & -0.08 & -0.031 \\ 
   &  & p=20 & 1.279 & 1.283 & 1.309 & 1.401 & 1.337 & 1.345 & 1.345 & 1.464 & 1.35 & 1.256 & -0.054 & -0.058 & -0.079 & -0.155 & -0.102 & -0.109 & -0.109 & -0.206 & -0.113 & -0.035 \\ 
   &  & p=36 & 1.279 & 1.283 & 1.527 & 3.256 & 1.365 & 1.476 & 1.503 & 4.474 & 1.616 & 3.097 & -0.054 & -0.058 & -0.259 & -1.683 & -0.125 & -0.217 & -0.238 & -2.688 & -0.332 & -1.553 \\ 
   \midrule 
 \multicolumn{1}{|c}{} & \multicolumn{1}{c}{} &       & \multicolumn{10}{c||}{Kullback-Leibler Discrepancy}                                    & \multicolumn{10}{c|}{Number of Variables} \\
\midrule\multirow{9}[6]{*}{hsnr} & \multirow{3}[2]{*}{$\rho=0$} & p=12 & 3.332 & 3.366 & 3.38 & 3.43 & 3.421 & 3.417 & 3.424 & 3.464 & 3.426 & 3.361 & 6.148 & 6.319 & 6.418 & 6.815 & 6.868 & 6.758 & 6.758 & 7.138 & 6.778 & 6.274 \\ 
   &  & p=20 & 3.332 & 3.374 & 3.432 & 3.647 & 3.534 & 3.566 & 3.564 & 3.834 & 3.573 & 3.406 & 6.148 & 6.343 & 6.624 & 7.65 & 7.301 & 7.289 & 7.241 & 8.775 & 7.29 & 6.391 \\ 
   &  & p=36 & 3.332 & 3.374 & 5.427 & 21.306 & 3.662 & 8.818 & 9.757 & 27.438 & 11.305 & 26.308 & 6.148 & 6.343 & 8.522 & 13.668 & 7.526 & 8.022 & 7.939 & 18.913 & 8.414 & 14.278 \\ 
  \cmidrule{2-23} & \multirow{3}[2]{*}{$\rho=0.5$} & p=12 & 2.639 & 2.672 & 2.687 & 2.737 & 2.728 & 2.724 & 2.731 & 2.771 & 2.733 & 2.667 & 6.148 & 6.319 & 6.418 & 6.815 & 6.868 & 6.758 & 6.758 & 7.138 & 6.778 & 6.274 \\ 
   &  & p=20 & 2.639 & 2.68 & 2.739 & 2.954 & 2.841 & 2.873 & 2.871 & 3.141 & 2.88 & 2.713 & 6.148 & 6.343 & 6.624 & 7.65 & 7.301 & 7.289 & 7.241 & 8.775 & 7.29 & 6.391 \\ 
   &  & p=36 & 2.639 & 2.68 & 4.734 & 20.613 & 2.969 & 8.125 & 9.063 & 26.745 & 10.612 & 25.615 & 6.148 & 6.343 & 8.522 & 13.668 & 7.526 & 8.022 & 7.939 & 18.913 & 8.414 & 14.278 \\ 
  \cmidrule{2-23} & \multirow{3}[2]{*}{$\rho=0.9$} & p=12 & 1.029 & 1.063 & 1.077 & 1.128 & 1.119 & 1.114 & 1.122 & 1.161 & 1.123 & 1.058 & 6.148 & 6.319 & 6.418 & 6.815 & 6.868 & 6.758 & 6.758 & 7.138 & 6.778 & 6.274 \\ 
   &  & p=20 & 1.029 & 1.071 & 1.129 & 1.344 & 1.231 & 1.264 & 1.262 & 1.532 & 1.27 & 1.103 & 6.148 & 6.343 & 6.624 & 7.65 & 7.301 & 7.289 & 7.241 & 8.775 & 7.29 & 6.391 \\ 
   &  & p=36 & 1.029 & 1.071 & 3.124 & 19.003 & 1.359 & 6.516 & 7.454 & 25.136 & 9.002 & 24.005 & 6.148 & 6.343 & 8.522 & 13.668 & 7.526 & 8.022 & 7.939 & 18.913 & 8.414 & 14.278 \\ 
  \midrule\multirow{9}[6]{*}{msnr} & \multirow{3}[2]{*}{$\rho=0$} & p=12 & 5.487 & 5.515 & 5.528 & 5.574 & 5.569 & 5.561 & 5.569 & 5.606 & 5.571 & 5.541 & 6.027 & 6.242 & 6.35 & 6.79 & 6.805 & 6.719 & 6.72 & 7.118 & 6.74 & 5.886 \\ 
   &  & p=20 & 5.487 & 5.523 & 5.579 & 5.789 & 5.681 & 5.703 & 5.708 & 5.977 & 5.717 & 5.586 & 6.027 & 6.266 & 6.569 & 7.602 & 7.238 & 7.226 & 7.205 & 8.76 & 7.254 & 6.003 \\ 
   &  & p=36 & 5.487 & 5.523 & 7.592 & 23.462 & 5.809 & 10.958 & 11.901 & 29.559 & 13.449 & 28.422 & 6.027 & 6.266 & 8.301 & 13.526 & 7.463 & 7.972 & 7.904 & 18.874 & 8.379 & 13.896 \\ 
  \cmidrule{2-23} & \multirow{3}[2]{*}{$\rho=0.5$} & p=12 & 4.79 & 4.819 & 4.831 & 4.879 & 4.873 & 4.866 & 4.873 & 4.912 & 4.875 & 4.843 & 6.052 & 6.265 & 6.387 & 6.804 & 6.832 & 6.74 & 6.744 & 7.127 & 6.764 & 5.917 \\ 
   &  & p=20 & 4.79 & 4.827 & 4.882 & 5.095 & 4.985 & 5.008 & 5.013 & 5.282 & 5.021 & 4.889 & 6.052 & 6.289 & 6.6 & 7.639 & 7.265 & 7.247 & 7.229 & 8.769 & 7.278 & 6.034 \\ 
   &  & p=36 & 4.79 & 4.827 & 6.894 & 22.766 & 5.113 & 10.262 & 11.205 & 28.886 & 12.753 & 27.76 & 6.052 & 6.289 & 8.33 & 13.539 & 7.49 & 7.993 & 7.928 & 18.909 & 8.403 & 13.95 \\ 
  \cmidrule{2-23} & \multirow{3}[2]{*}{$\rho=0.9$} & p=12 & 3.179 & 3.208 & 3.221 & 3.269 & 3.263 & 3.256 & 3.264 & 3.302 & 3.265 & 3.233 & 6.062 & 6.274 & 6.394 & 6.805 & 6.837 & 6.741 & 6.744 & 7.126 & 6.764 & 5.921 \\ 
   &  & p=20 & 3.179 & 3.216 & 3.272 & 3.486 & 3.375 & 3.398 & 3.403 & 3.673 & 3.412 & 3.278 & 6.062 & 6.298 & 6.604 & 7.64 & 7.27 & 7.248 & 7.229 & 8.769 & 7.278 & 6.038 \\ 
   &  & p=36 & 3.179 & 3.216 & 5.284 & 21.157 & 3.503 & 8.653 & 9.595 & 27.276 & 11.144 & 26.15 & 6.062 & 6.298 & 8.339 & 13.54 & 7.495 & 7.994 & 7.929 & 18.909 & 8.404 & 13.952 \\ 
  \midrule\multirow{9}[6]{*}{lsnr} & \multirow{3}[2]{*}{$\rho=0$} & p=12 & 7.097 & 7.112 & 7.138 & 7.18 & 7.157 & 7.162 & 7.169 & 7.201 & 7.17 & 7.079 & 2.11 & 2.575 & 3.361 & 4.103 & 3.665 & 3.682 & 3.747 & 4.456 & 3.842 & 0.971 \\ 
   &  & p=20 & 7.097 & 7.12 & 7.181 & 7.381 & 7.249 & 7.28 & 7.278 & 7.525 & 7.284 & 7.101 & 2.11 & 2.602 & 3.491 & 4.884 & 4.048 & 4.142 & 4.15 & 6.037 & 4.278 & 1.031 \\ 
   &  & p=36 & 7.097 & 7.12 & 9.173 & 25 & 7.341 & 12.025 & 13.138 & 30.888 & 14.41 & 27.428 & 2.11 & 2.602 & 5.792 & 11.159 & 4.223 & 4.759 & 4.786 & 16.74 & 5.245 & 7.967 \\ 
  \cmidrule{2-23} & \multirow{3}[2]{*}{$\rho=0.5$} & p=12 & 6.407 & 6.423 & 6.451 & 6.491 & 6.469 & 6.471 & 6.48 & 6.513 & 6.481 & 6.387 & 2.057 & 2.582 & 3.412 & 4.133 & 3.7 & 3.676 & 3.755 & 4.499 & 3.855 & 0.966 \\ 
   &  & p=20 & 6.407 & 6.43 & 6.49 & 6.694 & 6.561 & 6.585 & 6.59 & 6.836 & 6.598 & 6.411 & 2.057 & 2.603 & 3.489 & 4.95 & 4.099 & 4.112 & 4.175 & 6.076 & 4.316 & 1.039 \\ 
   &  & p=36 & 6.407 & 6.43 & 8.483 & 24.311 & 6.656 & 11.344 & 12.448 & 30.204 & 13.746 & 26.784 & 2.057 & 2.603 & 5.828 & 11.191 & 4.287 & 4.756 & 4.799 & 16.796 & 5.315 & 7.978 \\ 
  \cmidrule{2-23} & \multirow{3}[2]{*}{$\rho=0.9$} & p=12 & 4.799 & 4.814 & 4.843 & 4.883 & 4.862 & 4.864 & 4.873 & 4.905 & 4.875 & 4.776 & 2.054 & 2.592 & 3.454 & 4.205 & 3.721 & 3.722 & 3.79 & 4.55 & 3.912 & 0.932 \\ 
   &  & p=20 & 4.799 & 4.821 & 4.885 & 5.086 & 4.95 & 4.977 & 4.984 & 5.227 & 4.991 & 4.8 & 2.054 & 2.613 & 3.512 & 4.984 & 4.103 & 4.147 & 4.215 & 6.109 & 4.37 & 1.005 \\ 
   &  & p=36 & 4.799 & 4.821 & 6.873 & 22.704 & 5.047 & 9.736 & 10.842 & 28.614 & 12.139 & 25.21 & 2.054 & 2.613 & 5.824 & 11.236 & 4.3 & 4.784 & 4.839 & 16.87 & 5.37 & 7.988 \\ 
   \bottomrule 
\end{tabular}
}
\end{table}

%% latex table generated in R 4.0.2 by xtable 1.8-4 package
% Fri Jun 26 10:43:51 2020
\begin{table}[ht]
\centering
\caption{Random X, Sparse-Ex4, n=200} 
\scalebox{0.52}{
\begin{tabular}{|c|c|c|cc|cc|cc|ccc|c||cc|cc|cc|ccc|c|}
  \toprule 
 \multicolumn{1}{|c}{} & \multicolumn{1}{c}{} &       & RAICc & AICc  & RC$_p$ & C$_p$ & 10F CV & LOO CV & S$_p$ & $\widehat{\text{C}}_p$ & GCV   & BIC   & RAICc & AICc  & RC$_p$ & C$_p$ & 10F CV & LOO CV & S$_p$ & $\widehat{\text{C}}_p$ & GCV   & BIC  \\
 \cmidrule{4-23}\multicolumn{1}{|c}{} & \multicolumn{1}{c}{} &       & \multicolumn{10}{c||}{RMSE}                                                   & \multicolumn{10}{c|}{Relative Risk}       \\
 \midrule\multirow{9}[6]{*}{hsnr} & \multirow{3}[2]{*}{$\rho=0$} & p=12 & 0.534 & 0.539 & 0.541 & 0.547 & 0.548 & 0.544 & 0.546 & 0.549 & 0.546 & 0.494 & 0.005 & 0.005 & 0.005 & 0.005 & 0.005 & 0.005 & 0.005 & 0.005 & 0.005 & 0.004 \\ 
   &  & p=100 & 0.539 & 0.545 & 0.548 & 0.565 & 0.565 & 0.557 & 0.557 & 0.578 & 0.558 & 0.494 & 0.005 & 0.005 & 0.005 & 0.006 & 0.005 & 0.005 & 0.005 & 0.006 & 0.005 & 0.004 \\ 
   &  & p=196 & 0.539 & 0.545 & 0.852 & 1.181 & 0.565 & 0.557 & 0.557 & 1.449 & 0.558 & 0.791 & 0.005 & 0.005 & 0.029 & 1.343 & 0.005 & 0.005 & 0.005 & 2.346 & 0.005 & 0.894 \\ 
  \cmidrule{2-23} & \multirow{3}[2]{*}{$\rho=0.5$} & p=12 & 0.377 & 0.381 & 0.383 & 0.387 & 0.388 & 0.385 & 0.386 & 0.388 & 0.386 & 0.349 & 0.005 & 0.005 & 0.005 & 0.005 & 0.005 & 0.005 & 0.005 & 0.005 & 0.005 & 0.004 \\ 
   &  & p=100 & 0.381 & 0.385 & 0.388 & 0.4 & 0.399 & 0.394 & 0.394 & 0.408 & 0.395 & 0.349 & 0.005 & 0.005 & 0.005 & 0.006 & 0.005 & 0.005 & 0.005 & 0.006 & 0.005 & 0.004 \\ 
   &  & p=196 & 0.381 & 0.385 & 0.602 & 0.835 & 0.399 & 0.394 & 0.394 & 1.024 & 0.395 & 0.559 & 0.005 & 0.005 & 0.029 & 1.343 & 0.005 & 0.005 & 0.005 & 2.346 & 0.005 & 0.894 \\ 
  \cmidrule{2-23} & \multirow{3}[2]{*}{$\rho=0.9$} & p=12 & 0.169 & 0.17 & 0.171 & 0.173 & 0.173 & 0.172 & 0.173 & 0.174 & 0.173 & 0.156 & 0.005 & 0.005 & 0.005 & 0.005 & 0.005 & 0.005 & 0.005 & 0.005 & 0.005 & 0.004 \\ 
   &  & p=100 & 0.17 & 0.172 & 0.173 & 0.179 & 0.179 & 0.176 & 0.176 & 0.183 & 0.176 & 0.156 & 0.005 & 0.005 & 0.005 & 0.006 & 0.005 & 0.005 & 0.005 & 0.006 & 0.005 & 0.004 \\ 
   &  & p=196 & 0.17 & 0.172 & 0.269 & 0.373 & 0.179 & 0.176 & 0.176 & 0.458 & 0.176 & 0.25 & 0.005 & 0.005 & 0.029 & 1.343 & 0.005 & 0.005 & 0.005 & 2.346 & 0.005 & 0.894 \\ 
  \midrule\multirow{9}[6]{*}{msnr} & \multirow{3}[2]{*}{$\rho=0$} & p=12 & 1.556 & 1.571 & 1.578 & 1.595 & 1.598 & 1.587 & 1.593 & 1.601 & 1.593 & 1.44 & 0.04 & 0.041 & 0.041 & 0.042 & 0.042 & 0.042 & 0.042 & 0.043 & 0.042 & 0.034 \\ 
   &  & p=100 & 1.571 & 1.589 & 1.598 & 1.648 & 1.647 & 1.623 & 1.624 & 1.684 & 1.627 & 1.44 & 0.041 & 0.043 & 0.043 & 0.048 & 0.046 & 0.045 & 0.045 & 0.051 & 0.045 & 0.034 \\ 
   &  & p=196 & 1.571 & 1.589 & 2.483 & 3.442 & 1.647 & 1.623 & 1.624 & 4.224 & 1.627 & 2.305 & 0.041 & 0.043 & 0.245 & 11.416 & 0.046 & 0.045 & 0.045 & 19.938 & 0.045 & 7.599 \\ 
  \cmidrule{2-23} & \multirow{3}[2]{*}{$\rho=0.5$} & p=12 & 1.1 & 1.111 & 1.116 & 1.128 & 1.13 & 1.122 & 1.127 & 1.132 & 1.127 & 1.018 & 0.04 & 0.041 & 0.041 & 0.042 & 0.042 & 0.042 & 0.042 & 0.043 & 0.042 & 0.034 \\ 
   &  & p=100 & 1.111 & 1.124 & 1.13 & 1.165 & 1.165 & 1.148 & 1.148 & 1.191 & 1.15 & 1.018 & 0.041 & 0.043 & 0.043 & 0.048 & 0.046 & 0.045 & 0.045 & 0.051 & 0.045 & 0.034 \\ 
   &  & p=196 & 1.111 & 1.124 & 1.755 & 2.434 & 1.165 & 1.148 & 1.148 & 2.987 & 1.15 & 1.63 & 0.041 & 0.043 & 0.245 & 11.416 & 0.046 & 0.045 & 0.045 & 19.938 & 0.045 & 7.599 \\ 
  \cmidrule{2-23} & \multirow{3}[2]{*}{$\rho=0.9$} & p=12 & 0.492 & 0.497 & 0.499 & 0.504 & 0.505 & 0.502 & 0.504 & 0.506 & 0.504 & 0.455 & 0.04 & 0.041 & 0.041 & 0.042 & 0.042 & 0.042 & 0.042 & 0.043 & 0.042 & 0.034 \\ 
   &  & p=100 & 0.497 & 0.503 & 0.505 & 0.521 & 0.521 & 0.513 & 0.514 & 0.533 & 0.514 & 0.455 & 0.041 & 0.043 & 0.043 & 0.048 & 0.046 & 0.045 & 0.045 & 0.051 & 0.045 & 0.034 \\ 
   &  & p=196 & 0.497 & 0.503 & 0.785 & 1.088 & 0.521 & 0.513 & 0.514 & 1.336 & 0.514 & 0.729 & 0.041 & 0.043 & 0.245 & 11.416 & 0.046 & 0.045 & 0.045 & 19.938 & 0.045 & 7.599 \\ 
  \midrule\multirow{9}[6]{*}{lsnr} & \multirow{3}[2]{*}{$\rho=0$} & p=12 & 3.496 & 3.527 & 3.542 & 3.579 & 3.596 & 3.562 & 3.573 & 3.591 & 3.573 & 4.217 & 0.203 & 0.206 & 0.208 & 0.213 & 0.214 & 0.211 & 0.212 & 0.214 & 0.212 & 0.33 \\ 
   &  & p=100 & 3.531 & 3.568 & 3.584 & 3.697 & 3.705 & 3.642 & 3.642 & 3.776 & 3.648 & 4.217 & 0.21 & 0.215 & 0.218 & 0.242 & 0.236 & 0.228 & 0.227 & 0.257 & 0.228 & 0.33 \\ 
   &  & p=196 & 3.531 & 3.568 & 5.64 & 7.781 & 3.705 & 3.642 & 3.642 & 9.453 & 3.648 & 6.039 & 0.21 & 0.215 & 1.242 & 57.096 & 0.236 & 0.228 & 0.227 & 99.689 & 0.228 & 38.138 \\ 
  \cmidrule{2-23} & \multirow{3}[2]{*}{$\rho=0.5$} & p=12 & 2.461 & 2.485 & 2.496 & 2.523 & 2.531 & 2.511 & 2.52 & 2.533 & 2.52 & 2.926 & 0.2 & 0.204 & 0.206 & 0.211 & 0.212 & 0.209 & 0.211 & 0.213 & 0.211 & 0.319 \\ 
   &  & p=100 & 2.485 & 2.514 & 2.527 & 2.607 & 2.608 & 2.567 & 2.569 & 2.664 & 2.573 & 2.926 & 0.207 & 0.213 & 0.216 & 0.24 & 0.233 & 0.226 & 0.226 & 0.255 & 0.227 & 0.319 \\ 
   &  & p=196 & 2.485 & 2.514 & 3.97 & 5.483 & 2.608 & 2.567 & 2.569 & 6.68 & 2.573 & 4.221 & 0.207 & 0.213 & 1.238 & 57.092 & 0.233 & 0.226 & 0.226 & 99.688 & 0.227 & 38.128 \\ 
  \cmidrule{2-23} & \multirow{3}[2]{*}{$\rho=0.9$} & p=12 & 1.1 & 1.111 & 1.116 & 1.128 & 1.131 & 1.122 & 1.127 & 1.132 & 1.127 & 1.305 & 0.2 & 0.204 & 0.206 & 0.211 & 0.211 & 0.208 & 0.21 & 0.213 & 0.21 & 0.317 \\ 
   &  & p=100 & 1.111 & 1.124 & 1.13 & 1.165 & 1.165 & 1.148 & 1.148 & 1.191 & 1.15 & 1.305 & 0.207 & 0.213 & 0.216 & 0.24 & 0.232 & 0.226 & 0.226 & 0.255 & 0.227 & 0.317 \\ 
   &  & p=196 & 1.111 & 1.124 & 1.775 & 2.452 & 1.165 & 1.148 & 1.148 & 2.987 & 1.15 & 1.883 & 0.207 & 0.213 & 1.238 & 57.092 & 0.232 & 0.226 & 0.226 & 99.688 & 0.227 & 38.125 \\ 
   \midrule 
 \multicolumn{1}{|c}{} & \multicolumn{1}{c}{} &       & \multicolumn{10}{c||}{Relative Test Error}                                    & \multicolumn{10}{c|}{Proportion of Variance Explained} \\
\midrule\multirow{9}[6]{*}{hsnr} & \multirow{3}[2]{*}{$\rho=0$} & p=12 & 1.04 & 1.041 & 1.041 & 1.042 & 1.042 & 1.042 & 1.042 & 1.042 & 1.042 & 1.034 & 0.89 & 0.89 & 0.89 & 0.89 & 0.89 & 0.89 & 0.89 & 0.89 & 0.89 & 0.891 \\ 
   &  & p=100 & 1.041 & 1.042 & 1.043 & 1.048 & 1.046 & 1.045 & 1.045 & 1.051 & 1.045 & 1.034 & 0.89 & 0.89 & 0.89 & 0.89 & 0.89 & 0.89 & 0.89 & 0.889 & 0.89 & 0.891 \\ 
   &  & p=196 & 1.041 & 1.042 & 1.244 & 12.282 & 1.046 & 1.045 & 1.045 & 20.791 & 1.045 & 8.534 & 0.89 & 0.89 & 0.869 & -0.304 & 0.89 & 0.89 & 0.89 & -1.2 & 0.89 & 0.096 \\ 
  \cmidrule{2-23} & \multirow{3}[2]{*}{$\rho=0.5$} & p=12 & 1.04 & 1.041 & 1.041 & 1.042 & 1.042 & 1.042 & 1.042 & 1.042 & 1.042 & 1.034 & 0.89 & 0.89 & 0.89 & 0.89 & 0.89 & 0.89 & 0.89 & 0.89 & 0.89 & 0.891 \\ 
   &  & p=100 & 1.041 & 1.042 & 1.043 & 1.048 & 1.046 & 1.045 & 1.045 & 1.051 & 1.045 & 1.034 & 0.89 & 0.89 & 0.89 & 0.89 & 0.89 & 0.89 & 0.89 & 0.889 & 0.89 & 0.891 \\ 
   &  & p=196 & 1.041 & 1.042 & 1.244 & 12.282 & 1.046 & 1.045 & 1.045 & 20.791 & 1.045 & 8.534 & 0.89 & 0.89 & 0.869 & -0.304 & 0.89 & 0.89 & 0.89 & -1.2 & 0.89 & 0.096 \\ 
  \cmidrule{2-23} & \multirow{3}[2]{*}{$\rho=0.9$} & p=12 & 1.04 & 1.041 & 1.041 & 1.042 & 1.042 & 1.042 & 1.042 & 1.042 & 1.042 & 1.034 & 0.89 & 0.89 & 0.89 & 0.89 & 0.89 & 0.89 & 0.89 & 0.89 & 0.89 & 0.891 \\ 
   &  & p=100 & 1.041 & 1.042 & 1.043 & 1.048 & 1.046 & 1.045 & 1.045 & 1.051 & 1.045 & 1.034 & 0.89 & 0.89 & 0.89 & 0.89 & 0.89 & 0.89 & 0.89 & 0.889 & 0.89 & 0.891 \\ 
   &  & p=196 & 1.041 & 1.042 & 1.244 & 12.282 & 1.046 & 1.045 & 1.045 & 20.791 & 1.045 & 8.534 & 0.89 & 0.89 & 0.869 & -0.304 & 0.89 & 0.89 & 0.89 & -1.2 & 0.89 & 0.096 \\ 
  \midrule\multirow{9}[6]{*}{msnr} & \multirow{3}[2]{*}{$\rho=0$} & p=12 & 1.04 & 1.041 & 1.041 & 1.042 & 1.042 & 1.042 & 1.042 & 1.042 & 1.042 & 1.034 & 0.481 & 0.48 & 0.48 & 0.48 & 0.48 & 0.48 & 0.48 & 0.479 & 0.48 & 0.484 \\ 
   &  & p=100 & 1.041 & 1.042 & 1.043 & 1.048 & 1.046 & 1.045 & 1.045 & 1.051 & 1.045 & 1.034 & 0.48 & 0.48 & 0.479 & 0.477 & 0.478 & 0.478 & 0.478 & 0.475 & 0.478 & 0.484 \\ 
   &  & p=196 & 1.041 & 1.042 & 1.244 & 12.282 & 1.046 & 1.045 & 1.045 & 20.791 & 1.045 & 8.534 & 0.48 & 0.48 & 0.379 & -5.155 & 0.478 & 0.478 & 0.478 & -9.402 & 0.478 & -3.271 \\ 
  \cmidrule{2-23} & \multirow{3}[2]{*}{$\rho=0.5$} & p=12 & 1.04 & 1.041 & 1.041 & 1.042 & 1.042 & 1.042 & 1.042 & 1.042 & 1.042 & 1.034 & 0.481 & 0.48 & 0.48 & 0.48 & 0.48 & 0.48 & 0.48 & 0.479 & 0.48 & 0.484 \\ 
   &  & p=100 & 1.041 & 1.042 & 1.043 & 1.048 & 1.046 & 1.045 & 1.045 & 1.051 & 1.045 & 1.034 & 0.48 & 0.48 & 0.479 & 0.477 & 0.478 & 0.478 & 0.478 & 0.475 & 0.478 & 0.484 \\ 
   &  & p=196 & 1.041 & 1.042 & 1.244 & 12.282 & 1.046 & 1.045 & 1.045 & 20.791 & 1.045 & 8.534 & 0.48 & 0.48 & 0.379 & -5.155 & 0.478 & 0.478 & 0.478 & -9.402 & 0.478 & -3.271 \\ 
  \cmidrule{2-23} & \multirow{3}[2]{*}{$\rho=0.9$} & p=12 & 1.04 & 1.041 & 1.041 & 1.042 & 1.042 & 1.042 & 1.042 & 1.042 & 1.042 & 1.034 & 0.481 & 0.48 & 0.48 & 0.48 & 0.48 & 0.48 & 0.48 & 0.479 & 0.48 & 0.484 \\ 
   &  & p=100 & 1.041 & 1.042 & 1.043 & 1.048 & 1.046 & 1.045 & 1.045 & 1.051 & 1.045 & 1.034 & 0.48 & 0.48 & 0.479 & 0.477 & 0.478 & 0.478 & 0.478 & 0.475 & 0.478 & 0.484 \\ 
   &  & p=196 & 1.041 & 1.042 & 1.244 & 12.282 & 1.046 & 1.045 & 1.045 & 20.791 & 1.045 & 8.534 & 0.48 & 0.48 & 0.379 & -5.155 & 0.478 & 0.478 & 0.478 & -9.402 & 0.478 & -3.271 \\ 
  \midrule\multirow{9}[6]{*}{lsnr} & \multirow{3}[2]{*}{$\rho=0$} & p=12 & 1.04 & 1.041 & 1.042 & 1.042 & 1.043 & 1.042 & 1.042 & 1.043 & 1.042 & 1.067 & 0.134 & 0.133 & 0.133 & 0.132 & 0.132 & 0.133 & 0.132 & 0.132 & 0.132 & 0.112 \\ 
   &  & p=100 & 1.042 & 1.043 & 1.043 & 1.048 & 1.047 & 1.045 & 1.045 & 1.051 & 1.046 & 1.067 & 0.133 & 0.132 & 0.131 & 0.127 & 0.128 & 0.13 & 0.13 & 0.125 & 0.13 & 0.112 \\ 
   &  & p=196 & 1.042 & 1.043 & 1.247 & 12.284 & 1.047 & 1.045 & 1.045 & 20.792 & 1.046 & 8.564 & 0.133 & 0.132 & -0.038 & -9.237 & 0.128 & 0.13 & 0.13 & -16.319 & 0.13 & -6.134 \\ 
  \cmidrule{2-23} & \multirow{3}[2]{*}{$\rho=0.5$} & p=12 & 1.04 & 1.041 & 1.041 & 1.042 & 1.042 & 1.042 & 1.042 & 1.043 & 1.042 & 1.064 & 0.134 & 0.134 & 0.133 & 0.132 & 0.132 & 0.133 & 0.133 & 0.132 & 0.133 & 0.114 \\ 
   &  & p=100 & 1.041 & 1.042 & 1.043 & 1.048 & 1.047 & 1.045 & 1.045 & 1.051 & 1.045 & 1.064 & 0.133 & 0.132 & 0.132 & 0.128 & 0.129 & 0.13 & 0.13 & 0.125 & 0.13 & 0.114 \\ 
   &  & p=196 & 1.041 & 1.042 & 1.246 & 12.284 & 1.047 & 1.045 & 1.045 & 20.791 & 1.045 & 8.562 & 0.133 & 0.132 & -0.038 & -9.237 & 0.129 & 0.13 & 0.13 & -16.318 & 0.13 & -6.132 \\ 
  \cmidrule{2-23} & \multirow{3}[2]{*}{$\rho=0.9$} & p=12 & 1.04 & 1.041 & 1.041 & 1.042 & 1.042 & 1.042 & 1.042 & 1.042 & 1.042 & 1.064 & 0.134 & 0.134 & 0.133 & 0.133 & 0.132 & 0.133 & 0.133 & 0.132 & 0.133 & 0.114 \\ 
   &  & p=100 & 1.041 & 1.042 & 1.043 & 1.048 & 1.046 & 1.045 & 1.045 & 1.051 & 1.045 & 1.064 & 0.133 & 0.132 & 0.132 & 0.128 & 0.129 & 0.13 & 0.13 & 0.125 & 0.13 & 0.114 \\ 
   &  & p=196 & 1.041 & 1.042 & 1.246 & 12.284 & 1.046 & 1.045 & 1.045 & 20.791 & 1.045 & 8.561 & 0.133 & 0.132 & -0.038 & -9.237 & 0.129 & 0.13 & 0.13 & -16.318 & 0.13 & -6.132 \\ 
   \midrule 
 \multicolumn{1}{|c}{} & \multicolumn{1}{c}{} &       & \multicolumn{10}{c||}{Kullback-Leibler Discrepancy}                                    & \multicolumn{10}{c|}{Number of Variables} \\
\midrule\multirow{9}[6]{*}{hsnr} & \multirow{3}[2]{*}{$\rho=0$} & p=12 & 3.154 & 3.155 & 3.156 & 3.157 & 3.157 & 3.156 & 3.157 & 3.157 & 3.157 & 3.147 & 6.434 & 6.501 & 6.54 & 6.628 & 6.72 & 6.592 & 6.619 & 6.662 & 6.619 & 6.036 \\ 
   &  & p=100 & 3.156 & 3.158 & 3.159 & 3.166 & 3.163 & 3.161 & 3.161 & 3.172 & 3.162 & 3.147 & 6.528 & 6.615 & 6.692 & 7.107 & 7.073 & 6.855 & 6.843 & 7.385 & 6.862 & 6.036 \\ 
   &  & p=196 & 3.156 & 3.158 & 3.809 & 2586.131 & 3.163 & 3.161 & 3.161 & 3618.445 & 3.162 & 2526.058 & 6.528 & 6.615 & 22.429 & 54.127 & 7.073 & 6.855 & 6.843 & 78.427 & 6.862 & 29.957 \\ 
  \cmidrule{2-23} & \multirow{3}[2]{*}{$\rho=0.5$} & p=12 & 2.461 & 2.462 & 2.463 & 2.464 & 2.464 & 2.463 & 2.464 & 2.464 & 2.464 & 2.454 & 6.434 & 6.501 & 6.54 & 6.628 & 6.72 & 6.592 & 6.619 & 6.662 & 6.619 & 6.036 \\ 
   &  & p=100 & 2.463 & 2.465 & 2.466 & 2.473 & 2.47 & 2.468 & 2.468 & 2.479 & 2.468 & 2.454 & 6.528 & 6.615 & 6.692 & 7.107 & 7.073 & 6.855 & 6.843 & 7.385 & 6.862 & 6.036 \\ 
   &  & p=196 & 2.463 & 2.465 & 3.116 & 2585.437 & 2.47 & 2.468 & 2.468 & 3617.752 & 2.468 & 2525.365 & 6.528 & 6.615 & 22.429 & 54.127 & 7.073 & 6.855 & 6.843 & 78.427 & 6.862 & 29.957 \\ 
  \cmidrule{2-23} & \multirow{3}[2]{*}{$\rho=0.9$} & p=12 & 0.852 & 0.853 & 0.853 & 0.854 & 0.854 & 0.854 & 0.854 & 0.855 & 0.854 & 0.844 & 6.434 & 6.501 & 6.54 & 6.628 & 6.72 & 6.592 & 6.619 & 6.662 & 6.619 & 6.036 \\ 
   &  & p=100 & 0.854 & 0.855 & 0.856 & 0.864 & 0.86 & 0.859 & 0.859 & 0.87 & 0.859 & 0.844 & 6.528 & 6.615 & 6.692 & 7.107 & 7.073 & 6.855 & 6.843 & 7.385 & 6.862 & 6.036 \\ 
   &  & p=196 & 0.854 & 0.855 & 1.507 & 2583.828 & 0.86 & 0.859 & 0.859 & 3616.143 & 0.859 & 2523.755 & 6.528 & 6.615 & 22.429 & 54.127 & 7.073 & 6.855 & 6.843 & 78.427 & 6.862 & 29.957 \\ 
  \midrule\multirow{9}[6]{*}{msnr} & \multirow{3}[2]{*}{$\rho=0$} & p=12 & 5.294 & 5.295 & 5.296 & 5.297 & 5.297 & 5.296 & 5.297 & 5.297 & 5.297 & 5.287 & 6.434 & 6.501 & 6.54 & 6.628 & 6.72 & 6.592 & 6.619 & 6.662 & 6.619 & 6.036 \\ 
   &  & p=100 & 5.297 & 5.298 & 5.299 & 5.306 & 5.303 & 5.302 & 5.301 & 5.312 & 5.302 & 5.287 & 6.528 & 6.615 & 6.692 & 7.107 & 7.073 & 6.855 & 6.843 & 7.385 & 6.862 & 6.036 \\ 
   &  & p=196 & 5.297 & 5.298 & 5.95 & 2588.271 & 5.303 & 5.302 & 5.301 & 3620.585 & 5.302 & 2528.198 & 6.528 & 6.615 & 22.429 & 54.127 & 7.073 & 6.855 & 6.843 & 78.427 & 6.862 & 29.957 \\ 
  \cmidrule{2-23} & \multirow{3}[2]{*}{$\rho=0.5$} & p=12 & 4.601 & 4.602 & 4.603 & 4.604 & 4.604 & 4.603 & 4.604 & 4.604 & 4.604 & 4.594 & 6.434 & 6.501 & 6.54 & 6.628 & 6.72 & 6.592 & 6.619 & 6.662 & 6.619 & 6.036 \\ 
   &  & p=100 & 4.603 & 4.605 & 4.606 & 4.613 & 4.61 & 4.608 & 4.608 & 4.619 & 4.608 & 4.594 & 6.528 & 6.615 & 6.692 & 7.107 & 7.073 & 6.855 & 6.843 & 7.385 & 6.862 & 6.036 \\ 
   &  & p=196 & 4.603 & 4.605 & 5.256 & 2587.578 & 4.61 & 4.608 & 4.608 & 3619.892 & 4.608 & 2527.505 & 6.528 & 6.615 & 22.429 & 54.127 & 7.073 & 6.855 & 6.843 & 78.427 & 6.862 & 29.957 \\ 
  \cmidrule{2-23} & \multirow{3}[2]{*}{$\rho=0.9$} & p=12 & 2.992 & 2.993 & 2.993 & 2.994 & 2.994 & 2.994 & 2.994 & 2.995 & 2.994 & 2.984 & 6.434 & 6.501 & 6.54 & 6.628 & 6.72 & 6.592 & 6.619 & 6.662 & 6.619 & 6.036 \\ 
   &  & p=100 & 2.994 & 2.995 & 2.996 & 3.004 & 3 & 2.999 & 2.999 & 3.01 & 2.999 & 2.984 & 6.528 & 6.615 & 6.692 & 7.107 & 7.073 & 6.855 & 6.843 & 7.385 & 6.862 & 6.036 \\ 
   &  & p=196 & 2.994 & 2.995 & 3.647 & 2585.968 & 3 & 2.999 & 2.999 & 3618.283 & 2.999 & 2525.895 & 6.528 & 6.615 & 22.429 & 54.127 & 7.073 & 6.855 & 6.843 & 78.427 & 6.862 & 29.957 \\ 
  \midrule\multirow{9}[6]{*}{lsnr} & \multirow{3}[2]{*}{$\rho=0$} & p=12 & 6.904 & 6.905 & 6.906 & 6.907 & 6.907 & 6.906 & 6.907 & 6.907 & 6.907 & 6.926 & 6.418 & 6.488 & 6.528 & 6.612 & 6.687 & 6.58 & 6.604 & 6.647 & 6.604 & 4.946 \\ 
   &  & p=100 & 6.906 & 6.908 & 6.909 & 6.916 & 6.913 & 6.911 & 6.911 & 6.922 & 6.911 & 6.926 & 6.512 & 6.602 & 6.681 & 7.096 & 7.04 & 6.843 & 6.828 & 7.37 & 6.847 & 4.946 \\ 
   &  & p=196 & 6.906 & 6.908 & 7.562 & 2589.883 & 6.913 & 6.911 & 6.911 & 3622.195 & 6.911 & 2529.834 & 6.512 & 6.602 & 22.333 & 54.037 & 7.04 & 6.843 & 6.828 & 78.421 & 6.847 & 28.987 \\ 
  \cmidrule{2-23} & \multirow{3}[2]{*}{$\rho=0.5$} & p=12 & 6.211 & 6.212 & 6.212 & 6.213 & 6.214 & 6.213 & 6.213 & 6.214 & 6.213 & 6.231 & 6.433 & 6.5 & 6.539 & 6.627 & 6.705 & 6.591 & 6.618 & 6.661 & 6.618 & 4.98 \\ 
   &  & p=100 & 6.213 & 6.214 & 6.215 & 6.223 & 6.219 & 6.218 & 6.218 & 6.229 & 6.218 & 6.231 & 6.527 & 6.614 & 6.691 & 7.106 & 7.058 & 6.854 & 6.842 & 7.384 & 6.861 & 4.98 \\ 
   &  & p=196 & 6.213 & 6.214 & 6.868 & 2589.189 & 6.219 & 6.218 & 6.218 & 3621.502 & 6.218 & 2529.139 & 6.527 & 6.614 & 22.354 & 54.058 & 7.058 & 6.854 & 6.842 & 78.426 & 6.861 & 29.016 \\ 
  \cmidrule{2-23} & \multirow{3}[2]{*}{$\rho=0.9$} & p=12 & 4.601 & 4.602 & 4.603 & 4.604 & 4.604 & 4.603 & 4.604 & 4.604 & 4.604 & 4.621 & 6.434 & 6.501 & 6.54 & 6.628 & 6.712 & 6.592 & 6.619 & 6.662 & 6.619 & 4.976 \\ 
   &  & p=100 & 4.603 & 4.605 & 4.606 & 4.613 & 4.61 & 4.608 & 4.608 & 4.619 & 4.608 & 4.621 & 6.528 & 6.615 & 6.692 & 7.107 & 7.065 & 6.855 & 6.843 & 7.385 & 6.862 & 4.976 \\ 
   &  & p=196 & 4.603 & 4.605 & 5.258 & 2587.579 & 4.61 & 4.608 & 4.608 & 3619.892 & 4.608 & 2527.529 & 6.528 & 6.615 & 22.355 & 54.059 & 7.065 & 6.855 & 6.843 & 78.427 & 6.862 & 29.017 \\ 
   \bottomrule 
\end{tabular}
}
\end{table}

%% latex table generated in R 4.0.2 by xtable 1.8-4 package
% Fri Jun 26 10:43:47 2020
\begin{table}[ht]
\centering
\caption{Fixed X, Sparse-Ex4, n=1000} 
\scalebox{0.52}{
\begin{tabular}{|c|c|c|cc|cc|cc|ccc|c||cc|cc|cc|ccc|c|}
  \toprule 
 \multicolumn{1}{|c}{} & \multicolumn{1}{c}{} &       & RAICc & AICc  & RC$_p$ & C$_p$ & 10F CV & LOO CV & S$_p$ & $\widehat{\text{C}}_p$ & GCV   & BIC   & RAICc & AICc  & RC$_p$ & C$_p$ & 10F CV & LOO CV & S$_p$ & $\widehat{\text{C}}_p$ & GCV   & BIC  \\
 \cmidrule{4-23}\multicolumn{1}{|c}{} & \multicolumn{1}{c}{} &       & \multicolumn{10}{c||}{RMSE}                                                   & \multicolumn{10}{c|}{Relative Risk}       \\
 \midrule\multirow{9}[6]{*}{hsnr} & \multirow{3}[2]{*}{$\rho=0$} & p=12 & 0.241 & 0.241 & 0.241 & 0.242 & 0.242 & 0.242 & 0.241 & 0.242 & 0.241 & 0.212 & 0.001 & 0.001 & 0.001 & 0.001 & 0.001 & 0.001 & 0.001 & 0.001 & 0.001 & 0.001 \\ 
   &  & p=500 & 0.248 & 0.249 & 0.249 & 0.251 & 0.251 & 0.251 & 0.251 & 0.253 & 0.251 & 0.212 & 0.001 & 0.001 & 0.001 & 0.001 & 0.001 & 0.001 & 0.001 & 0.001 & 0.001 & 0.001 \\ 
   &  & p=996 & 0.248 & 0.249 & 0.619 & 0.953 & 0.251 & 0.251 & 0.251 & 1.181 & 0.251 & 0.473 & 0.001 & 0.001 & 0.024 & 5.508 & 0.001 & 0.001 & 0.001 & 9.651 & 0.001 & 3.042 \\ 
  \cmidrule{2-23} & \multirow{3}[2]{*}{$\rho=0.5$} & p=12 & 0.17 & 0.17 & 0.17 & 0.171 & 0.171 & 0.171 & 0.171 & 0.171 & 0.171 & 0.15 & 0.001 & 0.001 & 0.001 & 0.001 & 0.001 & 0.001 & 0.001 & 0.001 & 0.001 & 0.001 \\ 
   &  & p=500 & 0.175 & 0.176 & 0.176 & 0.178 & 0.178 & 0.178 & 0.177 & 0.179 & 0.177 & 0.15 & 0.001 & 0.001 & 0.001 & 0.001 & 0.001 & 0.001 & 0.001 & 0.001 & 0.001 & 0.001 \\ 
   &  & p=996 & 0.175 & 0.176 & 0.437 & 0.674 & 0.178 & 0.178 & 0.177 & 0.835 & 0.177 & 0.335 & 0.001 & 0.001 & 0.024 & 5.508 & 0.001 & 0.001 & 0.001 & 9.651 & 0.001 & 3.042 \\ 
  \cmidrule{2-23} & \multirow{3}[2]{*}{$\rho=0.9$} & p=12 & 0.076 & 0.076 & 0.076 & 0.076 & 0.076 & 0.076 & 0.076 & 0.077 & 0.076 & 0.067 & 0.001 & 0.001 & 0.001 & 0.001 & 0.001 & 0.001 & 0.001 & 0.001 & 0.001 & 0.001 \\ 
   &  & p=500 & 0.078 & 0.079 & 0.079 & 0.079 & 0.079 & 0.079 & 0.079 & 0.08 & 0.079 & 0.067 & 0.001 & 0.001 & 0.001 & 0.001 & 0.001 & 0.001 & 0.001 & 0.001 & 0.001 & 0.001 \\ 
   &  & p=996 & 0.078 & 0.079 & 0.196 & 0.301 & 0.079 & 0.079 & 0.079 & 0.373 & 0.079 & 0.15 & 0.001 & 0.001 & 0.024 & 5.508 & 0.001 & 0.001 & 0.001 & 9.651 & 0.001 & 3.042 \\ 
  \midrule\multirow{9}[6]{*}{msnr} & \multirow{3}[2]{*}{$\rho=0$} & p=12 & 0.701 & 0.703 & 0.703 & 0.704 & 0.705 & 0.704 & 0.704 & 0.706 & 0.704 & 0.619 & 0.008 & 0.008 & 0.008 & 0.008 & 0.008 & 0.008 & 0.008 & 0.008 & 0.008 & 0.006 \\ 
   &  & p=500 & 0.723 & 0.727 & 0.727 & 0.732 & 0.732 & 0.732 & 0.732 & 0.737 & 0.732 & 0.619 & 0.009 & 0.009 & 0.009 & 0.009 & 0.009 & 0.009 & 0.009 & 0.009 & 0.009 & 0.006 \\ 
   &  & p=996 & 0.723 & 0.727 & 1.804 & 2.778 & 0.732 & 0.732 & 0.732 & 3.442 & 0.732 & 1.38 & 0.009 & 0.009 & 0.208 & 46.819 & 0.009 & 0.009 & 0.009 & 82.035 & 0.009 & 25.859 \\ 
  \cmidrule{2-23} & \multirow{3}[2]{*}{$\rho=0.5$} & p=12 & 0.496 & 0.497 & 0.497 & 0.498 & 0.498 & 0.498 & 0.498 & 0.499 & 0.498 & 0.438 & 0.008 & 0.008 & 0.008 & 0.008 & 0.008 & 0.008 & 0.008 & 0.008 & 0.008 & 0.006 \\ 
   &  & p=500 & 0.511 & 0.514 & 0.514 & 0.518 & 0.518 & 0.518 & 0.517 & 0.521 & 0.517 & 0.438 & 0.009 & 0.009 & 0.009 & 0.009 & 0.009 & 0.009 & 0.009 & 0.009 & 0.009 & 0.006 \\ 
   &  & p=996 & 0.511 & 0.514 & 1.275 & 1.964 & 0.518 & 0.518 & 0.517 & 2.434 & 0.517 & 0.975 & 0.009 & 0.009 & 0.208 & 46.819 & 0.009 & 0.009 & 0.009 & 82.035 & 0.009 & 25.859 \\ 
  \cmidrule{2-23} & \multirow{3}[2]{*}{$\rho=0.9$} & p=12 & 0.222 & 0.222 & 0.222 & 0.223 & 0.223 & 0.223 & 0.223 & 0.223 & 0.223 & 0.196 & 0.008 & 0.008 & 0.008 & 0.008 & 0.008 & 0.008 & 0.008 & 0.008 & 0.008 & 0.006 \\ 
   &  & p=500 & 0.229 & 0.23 & 0.23 & 0.231 & 0.232 & 0.232 & 0.231 & 0.233 & 0.231 & 0.196 & 0.009 & 0.009 & 0.009 & 0.009 & 0.009 & 0.009 & 0.009 & 0.009 & 0.009 & 0.006 \\ 
   &  & p=996 & 0.229 & 0.23 & 0.57 & 0.878 & 0.232 & 0.232 & 0.231 & 1.088 & 0.231 & 0.436 & 0.009 & 0.009 & 0.208 & 46.819 & 0.009 & 0.009 & 0.009 & 82.035 & 0.009 & 25.859 \\ 
  \midrule\multirow{9}[6]{*}{lsnr} & \multirow{3}[2]{*}{$\rho=0$} & p=12 & 1.568 & 1.571 & 1.572 & 1.575 & 1.575 & 1.575 & 1.574 & 1.579 & 1.574 & 1.385 & 0.04 & 0.04 & 0.04 & 0.04 & 0.04 & 0.04 & 0.04 & 0.04 & 0.04 & 0.031 \\ 
   &  & p=500 & 1.616 & 1.626 & 1.625 & 1.637 & 1.637 & 1.638 & 1.636 & 1.647 & 1.636 & 1.385 & 0.044 & 0.044 & 0.044 & 0.045 & 0.045 & 0.045 & 0.045 & 0.046 & 0.045 & 0.031 \\ 
   &  & p=996 & 1.616 & 1.626 & 4.033 & 6.211 & 1.637 & 1.638 & 1.636 & 7.697 & 1.636 & 3.085 & 0.044 & 0.044 & 1.04 & 234.093 & 0.045 & 0.045 & 0.045 & 410.176 & 0.045 & 129.296 \\ 
  \cmidrule{2-23} & \multirow{3}[2]{*}{$\rho=0.5$} & p=12 & 1.109 & 1.111 & 1.111 & 1.114 & 1.114 & 1.114 & 1.113 & 1.116 & 1.113 & 0.979 & 0.04 & 0.04 & 0.04 & 0.04 & 0.04 & 0.04 & 0.04 & 0.04 & 0.04 & 0.031 \\ 
   &  & p=500 & 1.143 & 1.149 & 1.149 & 1.157 & 1.158 & 1.158 & 1.157 & 1.165 & 1.157 & 0.979 & 0.044 & 0.044 & 0.044 & 0.045 & 0.045 & 0.045 & 0.045 & 0.046 & 0.045 & 0.031 \\ 
   &  & p=996 & 1.143 & 1.149 & 2.852 & 4.392 & 1.158 & 1.158 & 1.157 & 5.442 & 1.157 & 2.181 & 0.044 & 0.044 & 1.04 & 234.093 & 0.045 & 0.045 & 0.045 & 410.176 & 0.045 & 129.296 \\ 
  \cmidrule{2-23} & \multirow{3}[2]{*}{$\rho=0.9$} & p=12 & 0.496 & 0.497 & 0.497 & 0.498 & 0.498 & 0.498 & 0.498 & 0.499 & 0.498 & 0.438 & 0.04 & 0.04 & 0.04 & 0.04 & 0.04 & 0.04 & 0.04 & 0.04 & 0.04 & 0.031 \\ 
   &  & p=500 & 0.511 & 0.514 & 0.514 & 0.518 & 0.518 & 0.518 & 0.517 & 0.521 & 0.517 & 0.438 & 0.044 & 0.044 & 0.044 & 0.045 & 0.045 & 0.045 & 0.045 & 0.046 & 0.045 & 0.031 \\ 
   &  & p=996 & 0.511 & 0.514 & 1.275 & 1.964 & 0.518 & 0.518 & 0.517 & 2.434 & 0.517 & 0.975 & 0.044 & 0.044 & 1.04 & 234.093 & 0.045 & 0.045 & 0.045 & 410.176 & 0.045 & 129.296 \\ 
   \midrule 
 \multicolumn{1}{|c}{} & \multicolumn{1}{c}{} &       & \multicolumn{10}{c||}{Relative Test Error}                                    & \multicolumn{10}{c|}{Proportion of Variance Explained} \\
\midrule\multirow{9}[6]{*}{hsnr} & \multirow{3}[2]{*}{$\rho=0$} & p=12 & 1.008 & 1.008 & 1.008 & 1.008 & 1.008 & 1.008 & 1.008 & 1.008 & 1.008 & 1.006 & 0.895 & 0.895 & 0.895 & 0.895 & 0.895 & 0.895 & 0.895 & 0.895 & 0.895 & 0.896 \\ 
   &  & p=500 & 1.009 & 1.009 & 1.009 & 1.009 & 1.009 & 1.009 & 1.009 & 1.009 & 1.009 & 1.006 & 0.895 & 0.895 & 0.895 & 0.895 & 0.895 & 0.895 & 0.895 & 0.895 & 0.895 & 0.896 \\ 
   &  & p=996 & 1.009 & 1.009 & 1.212 & 48.611 & 1.009 & 1.009 & 1.009 & 84.423 & 1.009 & 27.297 & 0.895 & 0.895 & 0.874 & -4.041 & 0.895 & 0.895 & 0.895 & -7.754 & 0.895 & -1.83 \\ 
  \cmidrule{2-23} & \multirow{3}[2]{*}{$\rho=0.5$} & p=12 & 1.008 & 1.008 & 1.008 & 1.008 & 1.008 & 1.008 & 1.008 & 1.008 & 1.008 & 1.006 & 0.895 & 0.895 & 0.895 & 0.895 & 0.895 & 0.895 & 0.895 & 0.895 & 0.895 & 0.896 \\ 
   &  & p=500 & 1.009 & 1.009 & 1.009 & 1.009 & 1.009 & 1.009 & 1.009 & 1.009 & 1.009 & 1.006 & 0.895 & 0.895 & 0.895 & 0.895 & 0.895 & 0.895 & 0.895 & 0.895 & 0.895 & 0.896 \\ 
   &  & p=996 & 1.009 & 1.009 & 1.212 & 48.611 & 1.009 & 1.009 & 1.009 & 84.423 & 1.009 & 27.297 & 0.895 & 0.895 & 0.874 & -4.041 & 0.895 & 0.895 & 0.895 & -7.754 & 0.895 & -1.83 \\ 
  \cmidrule{2-23} & \multirow{3}[2]{*}{$\rho=0.9$} & p=12 & 1.008 & 1.008 & 1.008 & 1.008 & 1.008 & 1.008 & 1.008 & 1.008 & 1.008 & 1.006 & 0.895 & 0.895 & 0.895 & 0.895 & 0.895 & 0.895 & 0.895 & 0.895 & 0.895 & 0.896 \\ 
   &  & p=500 & 1.009 & 1.009 & 1.009 & 1.009 & 1.009 & 1.009 & 1.009 & 1.009 & 1.009 & 1.006 & 0.895 & 0.895 & 0.895 & 0.895 & 0.895 & 0.895 & 0.895 & 0.895 & 0.895 & 0.896 \\ 
   &  & p=996 & 1.009 & 1.009 & 1.212 & 48.611 & 1.009 & 1.009 & 1.009 & 84.423 & 1.009 & 27.297 & 0.895 & 0.895 & 0.874 & -4.041 & 0.895 & 0.895 & 0.895 & -7.754 & 0.895 & -1.83 \\ 
  \midrule\multirow{9}[6]{*}{msnr} & \multirow{3}[2]{*}{$\rho=0$} & p=12 & 1.008 & 1.008 & 1.008 & 1.008 & 1.008 & 1.008 & 1.008 & 1.008 & 1.008 & 1.006 & 0.5 & 0.5 & 0.5 & 0.5 & 0.5 & 0.5 & 0.5 & 0.5 & 0.5 & 0.501 \\ 
   &  & p=500 & 1.009 & 1.009 & 1.009 & 1.009 & 1.009 & 1.009 & 1.009 & 1.009 & 1.009 & 1.006 & 0.5 & 0.5 & 0.5 & 0.5 & 0.5 & 0.5 & 0.5 & 0.5 & 0.5 & 0.501 \\ 
   &  & p=996 & 1.009 & 1.009 & 1.212 & 48.611 & 1.009 & 1.009 & 1.009 & 84.423 & 1.009 & 27.297 & 0.5 & 0.5 & 0.399 & -23.102 & 0.5 & 0.5 & 0.5 & -40.858 & 0.5 & -12.534 \\ 
  \cmidrule{2-23} & \multirow{3}[2]{*}{$\rho=0.5$} & p=12 & 1.008 & 1.008 & 1.008 & 1.008 & 1.008 & 1.008 & 1.008 & 1.008 & 1.008 & 1.006 & 0.5 & 0.5 & 0.5 & 0.5 & 0.5 & 0.5 & 0.5 & 0.5 & 0.5 & 0.501 \\ 
   &  & p=500 & 1.009 & 1.009 & 1.009 & 1.009 & 1.009 & 1.009 & 1.009 & 1.009 & 1.009 & 1.006 & 0.5 & 0.5 & 0.5 & 0.5 & 0.5 & 0.5 & 0.5 & 0.5 & 0.5 & 0.501 \\ 
   &  & p=996 & 1.009 & 1.009 & 1.212 & 48.611 & 1.009 & 1.009 & 1.009 & 84.423 & 1.009 & 27.297 & 0.5 & 0.5 & 0.399 & -23.102 & 0.5 & 0.5 & 0.5 & -40.858 & 0.5 & -12.534 \\ 
  \cmidrule{2-23} & \multirow{3}[2]{*}{$\rho=0.9$} & p=12 & 1.008 & 1.008 & 1.008 & 1.008 & 1.008 & 1.008 & 1.008 & 1.008 & 1.008 & 1.006 & 0.5 & 0.5 & 0.5 & 0.5 & 0.5 & 0.5 & 0.5 & 0.5 & 0.5 & 0.501 \\ 
   &  & p=500 & 1.009 & 1.009 & 1.009 & 1.009 & 1.009 & 1.009 & 1.009 & 1.009 & 1.009 & 1.006 & 0.5 & 0.5 & 0.5 & 0.5 & 0.5 & 0.5 & 0.5 & 0.5 & 0.5 & 0.501 \\ 
   &  & p=996 & 1.009 & 1.009 & 1.212 & 48.611 & 1.009 & 1.009 & 1.009 & 84.423 & 1.009 & 27.297 & 0.5 & 0.5 & 0.399 & -23.102 & 0.5 & 0.5 & 0.5 & -40.858 & 0.5 & -12.534 \\ 
  \midrule\multirow{9}[6]{*}{lsnr} & \multirow{3}[2]{*}{$\rho=0$} & p=12 & 1.008 & 1.008 & 1.008 & 1.008 & 1.008 & 1.008 & 1.008 & 1.008 & 1.008 & 1.006 & 0.162 & 0.162 & 0.162 & 0.162 & 0.162 & 0.162 & 0.162 & 0.162 & 0.162 & 0.164 \\ 
   &  & p=500 & 1.009 & 1.009 & 1.009 & 1.009 & 1.009 & 1.009 & 1.009 & 1.009 & 1.009 & 1.006 & 0.162 & 0.162 & 0.162 & 0.161 & 0.161 & 0.161 & 0.161 & 0.161 & 0.161 & 0.164 \\ 
   &  & p=996 & 1.009 & 1.009 & 1.212 & 48.611 & 1.009 & 1.009 & 1.009 & 84.423 & 1.009 & 27.297 & 0.162 & 0.162 & -0.007 & -39.395 & 0.161 & 0.161 & 0.161 & -69.155 & 0.161 & -21.683 \\ 
  \cmidrule{2-23} & \multirow{3}[2]{*}{$\rho=0.5$} & p=12 & 1.008 & 1.008 & 1.008 & 1.008 & 1.008 & 1.008 & 1.008 & 1.008 & 1.008 & 1.006 & 0.162 & 0.162 & 0.162 & 0.162 & 0.162 & 0.162 & 0.162 & 0.162 & 0.162 & 0.164 \\ 
   &  & p=500 & 1.009 & 1.009 & 1.009 & 1.009 & 1.009 & 1.009 & 1.009 & 1.009 & 1.009 & 1.006 & 0.162 & 0.162 & 0.162 & 0.161 & 0.161 & 0.161 & 0.161 & 0.161 & 0.161 & 0.164 \\ 
   &  & p=996 & 1.009 & 1.009 & 1.212 & 48.611 & 1.009 & 1.009 & 1.009 & 84.423 & 1.009 & 27.297 & 0.162 & 0.162 & -0.007 & -39.395 & 0.161 & 0.161 & 0.161 & -69.155 & 0.161 & -21.683 \\ 
  \cmidrule{2-23} & \multirow{3}[2]{*}{$\rho=0.9$} & p=12 & 1.008 & 1.008 & 1.008 & 1.008 & 1.008 & 1.008 & 1.008 & 1.008 & 1.008 & 1.006 & 0.162 & 0.162 & 0.162 & 0.162 & 0.162 & 0.162 & 0.162 & 0.162 & 0.162 & 0.164 \\ 
   &  & p=500 & 1.009 & 1.009 & 1.009 & 1.009 & 1.009 & 1.009 & 1.009 & 1.009 & 1.009 & 1.006 & 0.162 & 0.162 & 0.162 & 0.161 & 0.161 & 0.161 & 0.161 & 0.161 & 0.161 & 0.164 \\ 
   &  & p=996 & 1.009 & 1.009 & 1.212 & 48.611 & 1.009 & 1.009 & 1.009 & 84.423 & 1.009 & 27.297 & 0.162 & 0.162 & -0.007 & -39.395 & 0.161 & 0.161 & 0.161 & -69.155 & 0.161 & -21.683 \\ 
   \midrule 
 \multicolumn{1}{|c}{} & \multicolumn{1}{c}{} &       & \multicolumn{10}{c||}{Kullback-Leibler Discrepancy}                                    & \multicolumn{10}{c|}{Number of Variables} \\
\midrule\multirow{9}[6]{*}{hsnr} & \multirow{3}[2]{*}{$\rho=0$} & p=12 & 3.101 & 3.101 & 3.101 & 3.101 & 3.101 & 3.101 & 3.101 & 3.101 & 3.101 & 3.099 & 6.643 & 6.658 & 6.659 & 6.677 & 6.72 & 6.68 & 6.672 & 6.698 & 6.673 & 6.01 \\ 
   &  & p=500 & 3.102 & 3.102 & 3.102 & 3.102 & 3.102 & 3.102 & 3.102 & 3.102 & 3.102 & 3.099 & 6.96 & 7.02 & 7.018 & 7.101 & 7.174 & 7.117 & 7.098 & 7.169 & 7.099 & 6.01 \\ 
   &  & p=996 & 3.102 & 3.102 & 3.379 & 390.736 & 3.102 & 3.102 & 3.102 & 589.682 & 3.102 & 371.017 & 6.96 & 7.02 & 95.117 & 262.275 & 7.174 & 7.117 & 7.098 & 360.193 & 7.099 & 104.015 \\ 
  \cmidrule{2-23} & \multirow{3}[2]{*}{$\rho=0.5$} & p=12 & 2.408 & 2.408 & 2.408 & 2.408 & 2.408 & 2.408 & 2.408 & 2.408 & 2.408 & 2.406 & 6.643 & 6.658 & 6.659 & 6.677 & 6.72 & 6.68 & 6.672 & 6.698 & 6.673 & 6.01 \\ 
   &  & p=500 & 2.408 & 2.409 & 2.409 & 2.409 & 2.409 & 2.409 & 2.409 & 2.409 & 2.409 & 2.406 & 6.96 & 7.02 & 7.018 & 7.101 & 7.174 & 7.117 & 7.098 & 7.169 & 7.099 & 6.01 \\ 
   &  & p=996 & 2.408 & 2.409 & 2.685 & 390.043 & 2.409 & 2.409 & 2.409 & 588.988 & 2.409 & 370.324 & 6.96 & 7.02 & 95.117 & 262.275 & 7.174 & 7.117 & 7.098 & 360.193 & 7.099 & 104.015 \\ 
  \cmidrule{2-23} & \multirow{3}[2]{*}{$\rho=0.9$} & p=12 & 0.798 & 0.798 & 0.798 & 0.798 & 0.798 & 0.798 & 0.798 & 0.798 & 0.798 & 0.796 & 6.643 & 6.658 & 6.659 & 6.677 & 6.72 & 6.68 & 6.672 & 6.698 & 6.673 & 6.01 \\ 
   &  & p=500 & 0.799 & 0.799 & 0.799 & 0.799 & 0.799 & 0.799 & 0.799 & 0.799 & 0.799 & 0.796 & 6.96 & 7.02 & 7.018 & 7.101 & 7.174 & 7.117 & 7.098 & 7.169 & 7.099 & 6.01 \\ 
   &  & p=996 & 0.799 & 0.799 & 1.076 & 388.434 & 0.799 & 0.799 & 0.799 & 587.379 & 0.799 & 368.715 & 6.96 & 7.02 & 95.117 & 262.275 & 7.174 & 7.117 & 7.098 & 360.193 & 7.099 & 104.015 \\ 
  \midrule\multirow{9}[6]{*}{msnr} & \multirow{3}[2]{*}{$\rho=0$} & p=12 & 5.241 & 5.241 & 5.241 & 5.241 & 5.241 & 5.241 & 5.241 & 5.241 & 5.241 & 5.239 & 6.643 & 6.658 & 6.659 & 6.677 & 6.72 & 6.68 & 6.672 & 6.698 & 6.673 & 6.01 \\ 
   &  & p=500 & 5.242 & 5.242 & 5.242 & 5.242 & 5.242 & 5.242 & 5.242 & 5.242 & 5.242 & 5.239 & 6.96 & 7.02 & 7.018 & 7.101 & 7.174 & 7.117 & 7.098 & 7.169 & 7.099 & 6.01 \\ 
   &  & p=996 & 5.242 & 5.242 & 5.519 & 392.876 & 5.242 & 5.242 & 5.242 & 591.822 & 5.242 & 373.157 & 6.96 & 7.02 & 95.117 & 262.275 & 7.174 & 7.117 & 7.098 & 360.193 & 7.099 & 104.015 \\ 
  \cmidrule{2-23} & \multirow{3}[2]{*}{$\rho=0.5$} & p=12 & 4.548 & 4.548 & 4.548 & 4.548 & 4.548 & 4.548 & 4.548 & 4.548 & 4.548 & 4.546 & 6.643 & 6.658 & 6.659 & 6.677 & 6.72 & 6.68 & 6.672 & 6.698 & 6.673 & 6.01 \\ 
   &  & p=500 & 4.548 & 4.549 & 4.549 & 4.549 & 4.549 & 4.549 & 4.549 & 4.549 & 4.549 & 4.546 & 6.96 & 7.02 & 7.018 & 7.101 & 7.174 & 7.117 & 7.098 & 7.169 & 7.099 & 6.01 \\ 
   &  & p=996 & 4.548 & 4.549 & 4.825 & 392.183 & 4.549 & 4.549 & 4.549 & 591.128 & 4.549 & 372.464 & 6.96 & 7.02 & 95.117 & 262.275 & 7.174 & 7.117 & 7.098 & 360.193 & 7.099 & 104.015 \\ 
  \cmidrule{2-23} & \multirow{3}[2]{*}{$\rho=0.9$} & p=12 & 2.938 & 2.938 & 2.938 & 2.938 & 2.938 & 2.938 & 2.938 & 2.938 & 2.938 & 2.936 & 6.643 & 6.658 & 6.659 & 6.677 & 6.72 & 6.68 & 6.672 & 6.698 & 6.673 & 6.01 \\ 
   &  & p=500 & 2.939 & 2.939 & 2.939 & 2.939 & 2.939 & 2.939 & 2.939 & 2.939 & 2.939 & 2.936 & 6.96 & 7.02 & 7.018 & 7.101 & 7.174 & 7.117 & 7.098 & 7.169 & 7.099 & 6.01 \\ 
   &  & p=996 & 2.939 & 2.939 & 3.216 & 390.574 & 2.939 & 2.939 & 2.939 & 589.519 & 2.939 & 370.855 & 6.96 & 7.02 & 95.117 & 262.275 & 7.174 & 7.117 & 7.098 & 360.193 & 7.099 & 104.015 \\ 
  \midrule\multirow{9}[6]{*}{lsnr} & \multirow{3}[2]{*}{$\rho=0$} & p=12 & 6.85 & 6.85 & 6.85 & 6.85 & 6.85 & 6.85 & 6.85 & 6.85 & 6.85 & 6.848 & 6.643 & 6.658 & 6.659 & 6.677 & 6.72 & 6.68 & 6.672 & 6.698 & 6.673 & 6.01 \\ 
   &  & p=500 & 6.851 & 6.851 & 6.851 & 6.851 & 6.851 & 6.851 & 6.851 & 6.852 & 6.851 & 6.848 & 6.96 & 7.02 & 7.018 & 7.101 & 7.174 & 7.117 & 7.098 & 7.169 & 7.099 & 6.01 \\ 
   &  & p=996 & 6.851 & 6.851 & 7.128 & 394.486 & 6.851 & 6.851 & 6.851 & 593.431 & 6.851 & 374.767 & 6.96 & 7.02 & 95.117 & 262.275 & 7.174 & 7.117 & 7.098 & 360.193 & 7.099 & 104.015 \\ 
  \cmidrule{2-23} & \multirow{3}[2]{*}{$\rho=0.5$} & p=12 & 6.157 & 6.157 & 6.157 & 6.157 & 6.157 & 6.157 & 6.157 & 6.157 & 6.157 & 6.155 & 6.643 & 6.658 & 6.659 & 6.677 & 6.72 & 6.68 & 6.672 & 6.698 & 6.673 & 6.01 \\ 
   &  & p=500 & 6.158 & 6.158 & 6.158 & 6.158 & 6.158 & 6.158 & 6.158 & 6.158 & 6.158 & 6.155 & 6.96 & 7.02 & 7.018 & 7.101 & 7.174 & 7.117 & 7.098 & 7.169 & 7.099 & 6.01 \\ 
   &  & p=996 & 6.158 & 6.158 & 6.435 & 393.793 & 6.158 & 6.158 & 6.158 & 592.738 & 6.158 & 374.073 & 6.96 & 7.02 & 95.117 & 262.275 & 7.174 & 7.117 & 7.098 & 360.193 & 7.099 & 104.015 \\ 
  \cmidrule{2-23} & \multirow{3}[2]{*}{$\rho=0.9$} & p=12 & 4.548 & 4.548 & 4.548 & 4.548 & 4.548 & 4.548 & 4.548 & 4.548 & 4.548 & 4.546 & 6.643 & 6.658 & 6.659 & 6.677 & 6.72 & 6.68 & 6.672 & 6.698 & 6.673 & 6.01 \\ 
   &  & p=500 & 4.548 & 4.549 & 4.549 & 4.549 & 4.549 & 4.549 & 4.549 & 4.549 & 4.549 & 4.546 & 6.96 & 7.02 & 7.018 & 7.101 & 7.174 & 7.117 & 7.098 & 7.169 & 7.099 & 6.01 \\ 
   &  & p=996 & 4.548 & 4.549 & 4.825 & 392.183 & 4.549 & 4.549 & 4.549 & 591.128 & 4.549 & 372.464 & 6.96 & 7.02 & 95.117 & 262.275 & 7.174 & 7.117 & 7.098 & 360.193 & 7.099 & 104.015 \\ 
   \bottomrule 
\end{tabular}
}
\end{table}

% latex table generated in R 4.0.2 by xtable 1.8-4 package
% Fri Jun 26 10:43:52 2020
\begin{table}[ht]
\centering
\caption{Random X, Dense-Ex1, n=40} 
\scalebox{0.52}{
\begin{tabular}{|c|c|c|cc|cc|cc|ccc|c||cc|cc|cc|ccc|c|}
  \toprule 
 \multicolumn{1}{|c}{} & \multicolumn{1}{c}{} &       & RAICc & AICc  & RC$_p$ & C$_p$ & 10F CV & LOO CV & S$_p$ & $\widehat{\text{C}}_p$ & GCV   & BIC   & RAICc & AICc  & RC$_p$ & C$_p$ & 10F CV & LOO CV & S$_p$ & $\widehat{\text{C}}_p$ & GCV   & BIC  \\
 \cmidrule{4-23}\multicolumn{1}{|c}{} & \multicolumn{1}{c}{} &       & \multicolumn{10}{c||}{RMSE}                                                   & \multicolumn{10}{c|}{Relative Risk}       \\
 \midrule\multirow{9}[6]{*}{hsnr} & \multirow{3}[2]{*}{$\rho=0$} & p=12 & 0.434 & 0.402 & 0.394 & 0.382 & 0.391 & 0.385 & 0.383 & 0.377 & 0.382 & 0.395 & 0.083 & 0.068 & 0.064 & 0.058 & 0.062 & 0.059 & 0.059 & 0.056 & 0.058 & 0.065 \\ 
   &  & p=20 & 0.685 & 0.595 & 0.555 & 0.523 & 0.536 & 0.53 & 0.528 & 0.512 & 0.527 & 0.544 & 0.21 & 0.165 & 0.146 & 0.133 & 0.138 & 0.137 & 0.136 & 0.129 & 0.135 & 0.145 \\ 
   &  & p=36 & 0.75 & 0.655 & 0.603 & 0.616 & 0.591 & 0.59 & 0.592 & 0.638 & 0.594 & 0.659 & 0.245 & 0.199 & 0.196 & 0.516 & 0.194 & 0.236 & 0.262 & 0.79 & 0.291 & 0.943 \\ 
  \cmidrule{2-23} & \multirow{3}[2]{*}{$\rho=0.5$} & p=12 & 0.263 & 0.243 & 0.24 & 0.234 & 0.239 & 0.236 & 0.235 & 0.232 & 0.234 & 0.24 & 0.079 & 0.064 & 0.061 & 0.056 & 0.06 & 0.058 & 0.057 & 0.054 & 0.057 & 0.061 \\ 
   &  & p=20 & 0.424 & 0.37 & 0.344 & 0.323 & 0.332 & 0.328 & 0.327 & 0.317 & 0.327 & 0.338 & 0.21 & 0.169 & 0.149 & 0.135 & 0.14 & 0.139 & 0.138 & 0.13 & 0.137 & 0.147 \\ 
   &  & p=36 & 0.454 & 0.398 & 0.368 & 0.377 & 0.36 & 0.36 & 0.362 & 0.391 & 0.363 & 0.404 & 0.236 & 0.193 & 0.19 & 0.505 & 0.192 & 0.234 & 0.256 & 0.781 & 0.277 & 0.938 \\ 
  \cmidrule{2-23} & \multirow{3}[2]{*}{$\rho=0.9$} & p=12 & 0.144 & 0.13 & 0.126 & 0.121 & 0.125 & 0.123 & 0.122 & 0.121 & 0.122 & 0.128 & 0.089 & 0.069 & 0.063 & 0.057 & 0.062 & 0.059 & 0.058 & 0.055 & 0.058 & 0.066 \\ 
   &  & p=20 & 0.209 & 0.189 & 0.176 & 0.168 & 0.173 & 0.171 & 0.17 & 0.164 & 0.169 & 0.18 & 0.172 & 0.149 & 0.135 & 0.129 & 0.134 & 0.133 & 0.132 & 0.126 & 0.131 & 0.144 \\ 
   &  & p=36 & 0.213 & 0.193 & 0.184 & 0.191 & 0.18 & 0.181 & 0.181 & 0.2 & 0.182 & 0.206 & 0.175 & 0.152 & 0.161 & 0.457 & 0.156 & 0.202 & 0.232 & 0.735 & 0.251 & 0.85 \\ 
  \midrule\multirow{9}[6]{*}{msnr} & \multirow{3}[2]{*}{$\rho=0$} & p=12 & 1.279 & 1.229 & 1.186 & 1.151 & 1.181 & 1.169 & 1.167 & 1.135 & 1.164 & 1.364 & 0.555 & 0.527 & 0.502 & 0.488 & 0.497 & 0.494 & 0.497 & 0.477 & 0.494 & 0.618 \\ 
   &  & p=20 & 1.438 & 1.38 & 1.335 & 1.353 & 1.357 & 1.35 & 1.347 & 1.372 & 1.346 & 1.535 & 0.64 & 0.621 & 0.618 & 0.694 & 0.651 & 0.657 & 0.656 & 0.77 & 0.66 & 0.729 \\ 
   &  & p=36 & 1.471 & 1.417 & 1.425 & 1.58 & 1.407 & 1.412 & 1.415 & 1.704 & 1.427 & 1.72 & 0.658 & 0.642 & 0.844 & 3.473 & 0.755 & 0.995 & 1.108 & 5.231 & 1.253 & 4.24 \\ 
  \cmidrule{2-23} & \multirow{3}[2]{*}{$\rho=0.5$} & p=12 & 0.803 & 0.77 & 0.743 & 0.722 & 0.74 & 0.733 & 0.733 & 0.714 & 0.729 & 0.845 & 0.561 & 0.536 & 0.509 & 0.494 & 0.504 & 0.502 & 0.504 & 0.489 & 0.5 & 0.613 \\ 
   &  & p=20 & 0.873 & 0.844 & 0.816 & 0.832 & 0.833 & 0.826 & 0.826 & 0.843 & 0.824 & 0.928 & 0.62 & 0.607 & 0.607 & 0.699 & 0.639 & 0.646 & 0.651 & 0.766 & 0.65 & 0.706 \\ 
   &  & p=36 & 0.887 & 0.863 & 0.869 & 0.965 & 0.856 & 0.86 & 0.861 & 1.044 & 0.868 & 1.04 & 0.634 & 0.623 & 0.825 & 3.416 & 0.728 & 1.002 & 1.076 & 5.338 & 1.251 & 4.165 \\ 
  \cmidrule{2-23} & \multirow{3}[2]{*}{$\rho=0.9$} & p=12 & 0.39 & 0.382 & 0.371 & 0.367 & 0.374 & 0.372 & 0.371 & 0.366 & 0.37 & 0.408 & 0.468 & 0.464 & 0.456 & 0.458 & 0.459 & 0.46 & 0.461 & 0.461 & 0.46 & 0.501 \\ 
   &  & p=20 & 0.404 & 0.4 & 0.395 & 0.409 & 0.402 & 0.403 & 0.403 & 0.416 & 0.403 & 0.425 & 0.465 & 0.47 & 0.495 & 0.598 & 0.528 & 0.541 & 0.54 & 0.655 & 0.544 & 0.508 \\ 
   &  & p=36 & 0.405 & 0.402 & 0.418 & 0.473 & 0.408 & 0.414 & 0.416 & 0.521 & 0.419 & 0.485 & 0.459 & 0.468 & 0.698 & 3.168 & 0.569 & 0.837 & 0.96 & 4.751 & 1.029 & 3.481 \\ 
  \midrule\multirow{9}[6]{*}{lsnr} & \multirow{3}[2]{*}{$\rho=0$} & p=12 & 2.032 & 2.042 & 2.065 & 2.112 & 2.081 & 2.094 & 2.09 & 2.136 & 2.094 & 2.049 & 1.185 & 1.22 & 1.321 & 1.431 & 1.34 & 1.37 & 1.373 & 1.473 & 1.381 & 1.14 \\ 
   &  & p=20 & 2.126 & 2.152 & 2.202 & 2.341 & 2.23 & 2.253 & 2.255 & 2.428 & 2.264 & 2.145 & 1.197 & 1.267 & 1.432 & 1.857 & 1.495 & 1.568 & 1.591 & 2.139 & 1.623 & 1.171 \\ 
   &  & p=36 & 2.15 & 2.175 & 2.424 & 2.862 & 2.279 & 2.343 & 2.348 & 3.294 & 2.389 & 2.642 & 1.201 & 1.268 & 2.599 & 14.951 & 1.653 & 2.835 & 3.014 & 21.828 & 3.639 & 12.954 \\ 
  \cmidrule{2-23} & \multirow{3}[2]{*}{$\rho=0.5$} & p=12 & 1.242 & 1.253 & 1.268 & 1.298 & 1.282 & 1.287 & 1.285 & 1.314 & 1.286 & 1.261 & 1.153 & 1.204 & 1.288 & 1.415 & 1.317 & 1.355 & 1.355 & 1.468 & 1.362 & 1.116 \\ 
   &  & p=20 & 1.292 & 1.303 & 1.342 & 1.422 & 1.354 & 1.371 & 1.364 & 1.482 & 1.376 & 1.307 & 1.171 & 1.223 & 1.418 & 1.829 & 1.459 & 1.554 & 1.536 & 2.141 & 1.597 & 1.14 \\ 
   &  & p=36 & 1.304 & 1.315 & 1.477 & 1.752 & 1.382 & 1.427 & 1.425 & 2.018 & 1.446 & 1.606 & 1.177 & 1.23 & 2.553 & 14.954 & 1.624 & 2.915 & 3.046 & 21.762 & 3.702 & 12.774 \\ 
  \cmidrule{2-23} & \multirow{3}[2]{*}{$\rho=0.9$} & p=12 & 0.577 & 0.586 & 0.599 & 0.616 & 0.605 & 0.607 & 0.604 & 0.626 & 0.607 & 0.592 & 0.941 & 1 & 1.096 & 1.19 & 1.118 & 1.141 & 1.126 & 1.257 & 1.138 & 0.936 \\ 
   &  & p=20 & 0.591 & 0.598 & 0.622 & 0.667 & 0.634 & 0.635 & 0.636 & 0.7 & 0.638 & 0.607 & 0.922 & 0.97 & 1.138 & 1.566 & 1.222 & 1.25 & 1.269 & 1.844 & 1.295 & 0.931 \\ 
   &  & p=36 & 0.592 & 0.601 & 0.707 & 0.85 & 0.641 & 0.659 & 0.66 & 0.991 & 0.672 & 0.771 & 0.913 & 0.96 & 2.352 & 14.113 & 1.324 & 2.457 & 2.668 & 20.66 & 3.342 & 12.332 \\ 
   \midrule 
 \multicolumn{1}{|c}{} & \multicolumn{1}{c}{} &       & \multicolumn{10}{c||}{Relative Test Error}                                    & \multicolumn{10}{c|}{Proportion of Variance Explained} \\
\midrule\multirow{9}[6]{*}{hsnr} & \multirow{3}[2]{*}{$\rho=0$} & p=12 & 1.703 & 1.574 & 1.543 & 1.495 & 1.526 & 1.503 & 1.5 & 1.475 & 1.497 & 1.548 & 0.821 & 0.835 & 0.838 & 0.844 & 0.84 & 0.843 & 0.843 & 0.846 & 0.843 & 0.838 \\ 
   &  & p=20 & 2.851 & 2.436 & 2.268 & 2.135 & 2.188 & 2.169 & 2.164 & 2.095 & 2.157 & 2.236 & 0.708 & 0.748 & 0.765 & 0.777 & 0.772 & 0.774 & 0.774 & 0.781 & 0.775 & 0.767 \\ 
   &  & p=36 & 3.161 & 2.75 & 2.714 & 5.384 & 2.687 & 3.064 & 3.303 & 7.681 & 3.558 & 9.11 & 0.677 & 0.717 & 0.721 & 0.436 & 0.722 & 0.684 & 0.661 & 0.192 & 0.635 & 0.055 \\ 
  \cmidrule{2-23} & \multirow{3}[2]{*}{$\rho=0.5$} & p=12 & 1.668 & 1.542 & 1.517 & 1.477 & 1.511 & 1.49 & 1.486 & 1.463 & 1.482 & 1.518 & 0.825 & 0.838 & 0.841 & 0.845 & 0.842 & 0.844 & 0.845 & 0.847 & 0.845 & 0.841 \\ 
   &  & p=20 & 2.862 & 2.469 & 2.294 & 2.151 & 2.206 & 2.187 & 2.18 & 2.113 & 2.177 & 2.261 & 0.708 & 0.745 & 0.762 & 0.776 & 0.77 & 0.772 & 0.773 & 0.779 & 0.773 & 0.765 \\ 
   &  & p=36 & 3.081 & 2.694 & 2.665 & 5.272 & 2.674 & 3.034 & 3.243 & 7.589 & 3.413 & 9.038 & 0.685 & 0.723 & 0.726 & 0.446 & 0.724 & 0.686 & 0.666 & 0.201 & 0.648 & 0.06 \\ 
  \cmidrule{2-23} & \multirow{3}[2]{*}{$\rho=0.9$} & p=12 & 1.748 & 1.586 & 1.537 & 1.483 & 1.527 & 1.503 & 1.494 & 1.473 & 1.492 & 1.561 & 0.817 & 0.834 & 0.839 & 0.845 & 0.84 & 0.843 & 0.844 & 0.846 & 0.844 & 0.837 \\ 
   &  & p=20 & 2.519 & 2.303 & 2.173 & 2.108 & 2.158 & 2.148 & 2.136 & 2.082 & 2.13 & 2.244 & 0.742 & 0.763 & 0.775 & 0.781 & 0.776 & 0.777 & 0.778 & 0.783 & 0.779 & 0.767 \\ 
   &  & p=36 & 2.544 & 2.338 & 2.41 & 4.916 & 2.356 & 2.748 & 3.004 & 7.278 & 3.166 & 8.318 & 0.739 & 0.76 & 0.752 & 0.488 & 0.757 & 0.715 & 0.689 & 0.241 & 0.672 & 0.137 \\ 
  \midrule\multirow{9}[6]{*}{msnr} & \multirow{3}[2]{*}{$\rho=0$} & p=12 & 1.572 & 1.539 & 1.513 & 1.496 & 1.508 & 1.503 & 1.507 & 1.483 & 1.504 & 1.638 & 0.229 & 0.244 & 0.256 & 0.264 & 0.259 & 0.261 & 0.259 & 0.27 & 0.26 & 0.197 \\ 
   &  & p=20 & 1.666 & 1.643 & 1.636 & 1.709 & 1.666 & 1.673 & 1.673 & 1.782 & 1.676 & 1.762 & 0.185 & 0.196 & 0.198 & 0.161 & 0.182 & 0.179 & 0.18 & 0.124 & 0.178 & 0.14 \\ 
   &  & p=36 & 1.685 & 1.664 & 1.868 & 4.45 & 1.778 & 2.026 & 2.137 & 6.256 & 2.297 & 5.27 & 0.176 & 0.185 & 0.084 & -1.202 & 0.128 & 0.007 & -0.048 & -2.081 & -0.123 & -1.594 \\ 
  \cmidrule{2-23} & \multirow{3}[2]{*}{$\rho=0.5$} & p=12 & 1.581 & 1.55 & 1.521 & 1.503 & 1.515 & 1.512 & 1.514 & 1.497 & 1.51 & 1.636 & 0.225 & 0.239 & 0.253 & 0.261 & 0.255 & 0.257 & 0.256 & 0.264 & 0.258 & 0.199 \\ 
   &  & p=20 & 1.646 & 1.629 & 1.625 & 1.713 & 1.656 & 1.664 & 1.669 & 1.776 & 1.668 & 1.738 & 0.195 & 0.202 & 0.203 & 0.159 & 0.188 & 0.184 & 0.182 & 0.127 & 0.182 & 0.151 \\ 
   &  & p=36 & 1.66 & 1.647 & 1.85 & 4.39 & 1.751 & 2.026 & 2.099 & 6.362 & 2.288 & 5.18 & 0.188 & 0.194 & 0.094 & -1.173 & 0.142 & 0.006 & -0.031 & -2.134 & -0.12 & -1.553 \\ 
  \cmidrule{2-23} & \multirow{3}[2]{*}{$\rho=0.9$} & p=12 & 1.486 & 1.479 & 1.468 & 1.468 & 1.471 & 1.471 & 1.472 & 1.469 & 1.47 & 1.523 & 0.271 & 0.274 & 0.279 & 0.278 & 0.277 & 0.277 & 0.276 & 0.277 & 0.277 & 0.254 \\ 
   &  & p=20 & 1.483 & 1.487 & 1.51 & 1.612 & 1.544 & 1.557 & 1.556 & 1.67 & 1.559 & 1.531 & 0.273 & 0.271 & 0.259 & 0.208 & 0.243 & 0.236 & 0.237 & 0.18 & 0.235 & 0.251 \\ 
   &  & p=36 & 1.478 & 1.485 & 1.719 & 4.19 & 1.591 & 1.855 & 1.982 & 5.803 & 2.055 & 4.51 & 0.276 & 0.272 & 0.158 & -1.065 & 0.221 & 0.089 & 0.026 & -1.853 & -0.009 & -1.221 \\ 
  \midrule\multirow{9}[6]{*}{lsnr} & \multirow{3}[2]{*}{$\rho=0$} & p=12 & 1.246 & 1.253 & 1.271 & 1.293 & 1.275 & 1.282 & 1.282 & 1.302 & 1.284 & 1.238 & -0.029 & -0.035 & -0.05 & -0.068 & -0.053 & -0.059 & -0.059 & -0.076 & -0.061 & -0.022 \\ 
   &  & p=20 & 1.249 & 1.263 & 1.296 & 1.383 & 1.309 & 1.323 & 1.327 & 1.439 & 1.333 & 1.245 & -0.031 & -0.043 & -0.07 & -0.142 & -0.081 & -0.093 & -0.096 & -0.188 & -0.101 & -0.028 \\ 
   &  & p=36 & 1.249 & 1.263 & 1.535 & 3.961 & 1.341 & 1.587 & 1.626 & 5.344 & 1.75 & 3.57 & -0.032 & -0.043 & -0.268 & -2.282 & -0.108 & -0.311 & -0.342 & -3.423 & -0.445 & -1.956 \\ 
  \cmidrule{2-23} & \multirow{3}[2]{*}{$\rho=0.5$} & p=12 & 1.239 & 1.249 & 1.266 & 1.291 & 1.271 & 1.279 & 1.279 & 1.301 & 1.28 & 1.233 & -0.023 & -0.032 & -0.045 & -0.066 & -0.05 & -0.056 & -0.056 & -0.075 & -0.058 & -0.018 \\ 
   &  & p=20 & 1.244 & 1.254 & 1.293 & 1.378 & 1.302 & 1.32 & 1.317 & 1.438 & 1.33 & 1.239 & -0.027 & -0.036 & -0.068 & -0.137 & -0.075 & -0.09 & -0.087 & -0.188 & -0.098 & -0.023 \\ 
   &  & p=36 & 1.245 & 1.255 & 1.525 & 3.951 & 1.339 & 1.602 & 1.629 & 5.335 & 1.751 & 3.535 & -0.028 & -0.037 & -0.259 & -2.274 & -0.105 & -0.323 & -0.345 & -3.414 & -0.447 & -1.927 \\ 
  \cmidrule{2-23} & \multirow{3}[2]{*}{$\rho=0.9$} & p=12 & 1.196 & 1.208 & 1.227 & 1.246 & 1.23 & 1.236 & 1.233 & 1.258 & 1.235 & 1.196 & 0.012 & 0.002 & -0.014 & -0.029 & -0.017 & -0.021 & -0.019 & -0.04 & -0.021 & 0.012 \\ 
   &  & p=20 & 1.191 & 1.201 & 1.235 & 1.322 & 1.251 & 1.257 & 1.261 & 1.379 & 1.267 & 1.195 & 0.016 & 0.008 & -0.02 & -0.092 & -0.033 & -0.039 & -0.042 & -0.139 & -0.047 & 0.013 \\ 
   &  & p=36 & 1.19 & 1.199 & 1.481 & 3.824 & 1.276 & 1.503 & 1.544 & 5.153 & 1.684 & 3.479 & 0.017 & 0.01 & -0.223 & -2.165 & -0.054 & -0.242 & -0.276 & -3.262 & -0.392 & -1.879 \\ 
   \midrule 
 \multicolumn{1}{|c}{} & \multicolumn{1}{c}{} &       & \multicolumn{10}{c||}{Kullback-Leibler Discrepancy}                                    & \multicolumn{10}{c|}{Number of Variables} \\
\midrule\multirow{9}[6]{*}{hsnr} & \multirow{3}[2]{*}{$\rho=0$} & p=12 & 1.263 & 1.176 & 1.153 & 1.114 & 1.134 & 1.118 & 1.118 & 1.096 & 1.116 & 1.153 & 10.934 & 11.396 & 11.514 & 11.723 & 11.583 & 11.674 & 11.7 & 11.801 & 11.714 & 11.498 \\ 
   &  & p=20 & 1.996 & 2.26 & 2.454 & 2.888 & 2.706 & 2.799 & 2.842 & 2.995 & 2.849 & 2.824 & 10.343 & 12.741 & 14.282 & 16.657 & 15.742 & 16.109 & 16.194 & 17.584 & 16.276 & 15.555 \\ 
   &  & p=36 & 2.055 & 2.36 & 4.976 & 145.005 & 6.382 & 61.391 & 76.425 & 188.496 & 88.007 & 215.179 & 9.89 & 12.267 & 15.468 & 21.926 & 16.684 & 17.99 & 18.421 & 27.985 & 19.068 & 27.807 \\ 
  \cmidrule{2-23} & \multirow{3}[2]{*}{$\rho=0.5$} & p=12 & 0.274 & 0.191 & 0.175 & 0.143 & 0.167 & 0.154 & 0.15 & 0.13 & 0.147 & 0.173 & 11.198 & 11.627 & 11.722 & 11.863 & 11.746 & 11.817 & 11.833 & 11.917 & 11.847 & 11.705 \\ 
   &  & p=20 & 1.004 & 1.257 & 1.503 & 1.959 & 1.748 & 1.868 & 1.907 & 2.064 & 1.918 & 1.873 & 10.148 & 12.471 & 14.173 & 16.744 & 15.713 & 16.154 & 16.243 & 17.682 & 16.333 & 15.49 \\ 
   &  & p=36 & 1.044 & 1.321 & 3.479 & 139.29 & 11.947 & 61.21 & 73.707 & 186.652 & 82.713 & 213.044 & 9.798 & 12.135 & 15.332 & 21.703 & 16.633 & 17.914 & 18.202 & 27.923 & 18.811 & 27.581 \\ 
  \cmidrule{2-23} & \multirow{3}[2]{*}{$\rho=0.9$} & p=12 & -1.014 & -1.104 & -1.135 & -1.174 & -1.145 & -1.159 & -1.166 & -1.182 & -1.168 & -1.122 & 10.516 & 11.274 & 11.507 & 11.765 & 11.567 & 11.668 & 11.711 & 11.826 & 11.724 & 11.395 \\ 
   &  & p=20 & -0.578 & -0.322 & -0.093 & 0.491 & 0.243 & 0.366 & 0.407 & 0.659 & 0.428 & 0.301 & 8.848 & 10.931 & 12.863 & 15.68 & 14.306 & 14.768 & 14.932 & 16.976 & 15.082 & 13.411 \\ 
   &  & p=36 & -0.582 & -0.361 & 2.354 & 134.139 & 4.381 & 54.799 & 72.646 & 181.084 & 79.769 & 204.867 & 8.668 & 10.657 & 13.699 & 20.161 & 14.682 & 15.975 & 16.337 & 26.771 & 16.873 & 25.155 \\ 
  \midrule\multirow{9}[6]{*}{msnr} & \multirow{3}[2]{*}{$\rho=0$} & p=12 & 3.056 & 3.09 & 3.113 & 3.154 & 3.125 & 3.134 & 3.148 & 3.168 & 3.148 & 3.12 & 5.222 & 6.19 & 7.132 & 8.143 & 7.552 & 7.684 & 7.773 & 8.714 & 7.838 & 4.827 \\ 
   &  & p=20 & 3.17 & 3.268 & 3.431 & 3.923 & 3.634 & 3.705 & 3.697 & 4.291 & 3.734 & 3.415 & 4.711 & 5.931 & 7.574 & 9.891 & 8.533 & 8.777 & 8.756 & 11.759 & 8.951 & 4.678 \\ 
   &  & p=36 & 3.192 & 3.29 & 5.82 & 138.976 & 5.646 & 51.213 & 61.388 & 177.08 & 70.874 & 171.079 & 4.597 & 5.821 & 9.076 & 15.335 & 8.79 & 9.724 & 10.015 & 22.182 & 10.526 & 14.332 \\ 
  \cmidrule{2-23} & \multirow{3}[2]{*}{$\rho=0.5$} & p=12 & 2.093 & 2.149 & 2.167 & 2.206 & 2.179 & 2.192 & 2.203 & 2.23 & 2.203 & 2.165 & 4.884 & 6.038 & 6.996 & 8.072 & 7.442 & 7.643 & 7.683 & 8.697 & 7.803 & 4.688 \\ 
   &  & p=20 & 2.175 & 2.255 & 2.437 & 2.965 & 2.653 & 2.711 & 2.741 & 3.314 & 2.761 & 2.423 & 4.554 & 5.613 & 7.431 & 9.741 & 8.268 & 8.507 & 8.581 & 11.58 & 8.732 & 4.512 \\ 
   &  & p=36 & 2.2 & 2.283 & 4.511 & 136.01 & 7.669 & 50.208 & 59.465 & 177.14 & 69.19 & 168.09 & 4.554 & 5.521 & 9.042 & 15.303 & 8.534 & 9.593 & 9.807 & 22.337 & 10.415 & 14.124 \\ 
  \cmidrule{2-23} & \multirow{3}[2]{*}{$\rho=0.9$} & p=12 & 0.672 & 0.728 & 0.766 & 0.825 & 0.784 & 0.805 & 0.805 & 0.854 & 0.809 & 0.707 & 3.899 & 4.837 & 5.903 & 7.029 & 6.19 & 6.467 & 6.424 & 7.578 & 6.557 & 3.405 \\ 
   &  & p=20 & 0.724 & 0.79 & 0.949 & 1.462 & 1.133 & 1.205 & 1.218 & 1.757 & 1.243 & 0.842 & 3.569 & 4.306 & 5.785 & 8.104 & 6.646 & 6.782 & 6.782 & 9.679 & 6.967 & 3.199 \\ 
   &  & p=36 & 0.731 & 0.806 & 3.464 & 130.41 & 4.121 & 45.218 & 58.858 & 167.676 & 63.665 & 156.712 & 3.567 & 4.295 & 7.756 & 13.747 & 6.85 & 7.447 & 7.881 & 20.308 & 8.264 & 11.647 \\ 
  \midrule\multirow{9}[6]{*}{lsnr} & \multirow{3}[2]{*}{$\rho=0$} & p=12 & 4.312 & 4.336 & 4.392 & 4.457 & 4.404 & 4.426 & 4.426 & 4.484 & 4.429 & 4.282 & 1.82 & 2.17 & 2.856 & 3.621 & 3.149 & 3.269 & 3.153 & 3.974 & 3.222 & 0.913 \\ 
   &  & p=20 & 4.385 & 4.427 & 4.541 & 4.894 & 4.601 & 4.67 & 4.678 & 5.15 & 4.698 & 4.379 & 1.658 & 2.058 & 2.982 & 4.51 & 3.326 & 3.537 & 3.412 & 5.593 & 3.555 & 0.905 \\ 
   &  & p=36 & 4.403 & 4.444 & 6.989 & 136.329 & 5.64 & 48.271 & 53.915 & 168.706 & 63.972 & 143.251 & 1.64 & 2.042 & 5.426 & 10.95 & 3.584 & 4.328 & 4.32 & 16.973 & 4.853 & 7.587 \\ 
  \cmidrule{2-23} & \multirow{3}[2]{*}{$\rho=0.5$} & p=12 & 3.348 & 3.377 & 3.425 & 3.501 & 3.445 & 3.466 & 3.467 & 3.531 & 3.475 & 3.322 & 1.742 & 2.112 & 2.748 & 3.555 & 3.062 & 3.152 & 3.079 & 3.939 & 3.165 & 0.921 \\ 
   &  & p=20 & 3.407 & 3.442 & 3.568 & 3.927 & 3.629 & 3.701 & 3.697 & 4.188 & 3.751 & 3.4 & 1.661 & 2.001 & 2.968 & 4.396 & 3.286 & 3.516 & 3.365 & 5.601 & 3.564 & 0.94 \\ 
   &  & p=36 & 3.423 & 3.457 & 5.714 & 134.473 & 4.809 & 47.325 & 51.953 & 168.604 & 62.722 & 141.299 & 1.66 & 2.006 & 5.332 & 11.039 & 3.549 & 4.336 & 4.321 & 17.134 & 4.84 & 7.506 \\ 
  \cmidrule{2-23} & \multirow{3}[2]{*}{$\rho=0.9$} & p=12 & 1.98 & 2.01 & 2.06 & 2.113 & 2.071 & 2.09 & 2.081 & 2.149 & 2.088 & 1.961 & 1.636 & 1.907 & 2.44 & 3.017 & 2.734 & 2.789 & 2.669 & 3.377 & 2.748 & 0.974 \\ 
   &  & p=20 & 2.046 & 2.075 & 2.18 & 2.551 & 2.255 & 2.29 & 2.302 & 2.774 & 2.327 & 2.044 & 1.577 & 1.845 & 2.603 & 3.84 & 2.941 & 3.01 & 2.939 & 4.889 & 3.06 & 1.009 \\ 
   &  & p=36 & 2.061 & 2.088 & 4.454 & 129.968 & 5.003 & 44.847 & 49.173 & 162.858 & 62.122 & 139.704 & 1.577 & 1.854 & 5.074 & 10.542 & 3.174 & 3.731 & 3.765 & 16.089 & 4.243 & 7.464 \\ 
   \bottomrule 
\end{tabular}
}
\end{table}

% latex table generated in R 4.0.2 by xtable 1.8-4 package
% Fri Jun 26 10:43:47 2020
\begin{table}[ht]
\centering
\caption{Fixed X, Dense-Ex1, n=200} 
\scalebox{0.52}{
\begin{tabular}{|c|c|c|cc|cc|cc|ccc|c||cc|cc|cc|ccc|c|}
  \toprule 
 \multicolumn{1}{|c}{} & \multicolumn{1}{c}{} &       & RAICc & AICc  & RC$_p$ & C$_p$ & 10F CV & LOO CV & S$_p$ & $\widehat{\text{C}}_p$ & GCV   & BIC   & RAICc & AICc  & RC$_p$ & C$_p$ & 10F CV & LOO CV & S$_p$ & $\widehat{\text{C}}_p$ & GCV   & BIC  \\
 \cmidrule{4-23}\multicolumn{1}{|c}{} & \multicolumn{1}{c}{} &       & \multicolumn{10}{c||}{RMSE}                                                   & \multicolumn{10}{c|}{Relative Risk}       \\
 \midrule\multirow{9}[6]{*}{hsnr} & \multirow{3}[2]{*}{$\rho=0$} & p=12 & 0.163 & 0.163 & 0.163 & 0.163 & 0.163 & 0.163 & 0.163 & 0.163 & 0.163 & 0.163 & 0.007 & 0.007 & 0.007 & 0.007 & 0.007 & 0.007 & 0.007 & 0.007 & 0.007 & 0.007 \\ 
   &  & p=100 & 0.302 & 0.3 & 0.299 & 0.301 & 0.301 & 0.3 & 0.3 & 0.31 & 0.3 & 0.338 & 0.026 & 0.025 & 0.025 & 0.026 & 0.026 & 0.025 & 0.025 & 0.029 & 0.025 & 0.032 \\ 
   &  & p=196 & 0.302 & 0.3 & 0.348 & 0.423 & 0.301 & 0.3 & 0.3 & 0.499 & 0.3 & 0.429 & 0.026 & 0.025 & 0.052 & 1.543 & 0.026 & 0.025 & 0.028 & 2.972 & 0.028 & 1.976 \\ 
  \cmidrule{2-23} & \multirow{3}[2]{*}{$\rho=0.5$} & p=12 & 0.101 & 0.101 & 0.101 & 0.101 & 0.101 & 0.101 & 0.101 & 0.101 & 0.101 & 0.101 & 0.007 & 0.007 & 0.007 & 0.007 & 0.007 & 0.007 & 0.007 & 0.007 & 0.007 & 0.007 \\ 
   &  & p=100 & 0.185 & 0.183 & 0.183 & 0.184 & 0.184 & 0.183 & 0.183 & 0.189 & 0.183 & 0.207 & 0.025 & 0.025 & 0.025 & 0.026 & 0.025 & 0.025 & 0.025 & 0.028 & 0.025 & 0.032 \\ 
   &  & p=196 & 0.185 & 0.183 & 0.214 & 0.259 & 0.184 & 0.183 & 0.184 & 0.305 & 0.184 & 0.263 & 0.025 & 0.025 & 0.052 & 1.566 & 0.025 & 0.025 & 0.028 & 2.969 & 0.028 & 1.984 \\ 
  \cmidrule{2-23} & \multirow{3}[2]{*}{$\rho=0.9$} & p=12 & 0.053 & 0.053 & 0.053 & 0.053 & 0.053 & 0.053 & 0.053 & 0.053 & 0.053 & 0.053 & 0.007 & 0.007 & 0.007 & 0.007 & 0.007 & 0.007 & 0.007 & 0.007 & 0.007 & 0.007 \\ 
   &  & p=100 & 0.094 & 0.094 & 0.093 & 0.094 & 0.094 & 0.094 & 0.094 & 0.096 & 0.094 & 0.107 & 0.024 & 0.024 & 0.024 & 0.025 & 0.024 & 0.024 & 0.024 & 0.027 & 0.024 & 0.032 \\ 
   &  & p=196 & 0.094 & 0.094 & 0.11 & 0.135 & 0.094 & 0.094 & 0.094 & 0.159 & 0.094 & 0.135 & 0.024 & 0.024 & 0.051 & 1.616 & 0.024 & 0.024 & 0.03 & 3.016 & 0.034 & 1.879 \\ 
  \midrule\multirow{9}[6]{*}{msnr} & \multirow{3}[2]{*}{$\rho=0$} & p=12 & 0.498 & 0.495 & 0.494 & 0.493 & 0.498 & 0.494 & 0.493 & 0.492 & 0.493 & 0.612 & 0.068 & 0.067 & 0.067 & 0.066 & 0.068 & 0.067 & 0.066 & 0.066 & 0.066 & 0.108 \\ 
   &  & p=100 & 0.734 & 0.733 & 0.733 & 0.743 & 0.74 & 0.734 & 0.736 & 0.756 & 0.735 & 0.868 & 0.146 & 0.145 & 0.145 & 0.152 & 0.149 & 0.147 & 0.147 & 0.161 & 0.147 & 0.216 \\ 
   &  & p=196 & 0.734 & 0.733 & 0.908 & 1.137 & 0.74 & 0.734 & 0.737 & 1.342 & 0.737 & 1.104 & 0.146 & 0.145 & 0.385 & 13.061 & 0.149 & 0.147 & 0.166 & 23.85 & 0.167 & 13.524 \\ 
  \cmidrule{2-23} & \multirow{3}[2]{*}{$\rho=0.5$} & p=12 & 0.303 & 0.302 & 0.302 & 0.301 & 0.304 & 0.301 & 0.301 & 0.3 & 0.301 & 0.379 & 0.066 & 0.065 & 0.065 & 0.064 & 0.066 & 0.065 & 0.064 & 0.064 & 0.064 & 0.113 \\ 
   &  & p=100 & 0.446 & 0.446 & 0.446 & 0.453 & 0.451 & 0.449 & 0.448 & 0.461 & 0.449 & 0.538 & 0.142 & 0.143 & 0.143 & 0.15 & 0.147 & 0.145 & 0.145 & 0.159 & 0.145 & 0.224 \\ 
   &  & p=196 & 0.446 & 0.446 & 0.555 & 0.698 & 0.451 & 0.449 & 0.449 & 0.818 & 0.45 & 0.681 & 0.142 & 0.143 & 0.381 & 13.256 & 0.147 & 0.145 & 0.164 & 23.429 & 0.165 & 13.374 \\ 
  \cmidrule{2-23} & \multirow{3}[2]{*}{$\rho=0.9$} & p=12 & 0.16 & 0.159 & 0.159 & 0.158 & 0.161 & 0.158 & 0.158 & 0.157 & 0.158 & 0.233 & 0.069 & 0.068 & 0.067 & 0.067 & 0.069 & 0.067 & 0.067 & 0.066 & 0.067 & 0.169 \\ 
   &  & p=100 & 0.223 & 0.223 & 0.223 & 0.227 & 0.225 & 0.224 & 0.224 & 0.231 & 0.224 & 0.283 & 0.13 & 0.13 & 0.131 & 0.137 & 0.134 & 0.133 & 0.133 & 0.146 & 0.133 & 0.239 \\ 
   &  & p=196 & 0.223 & 0.223 & 0.286 & 0.361 & 0.225 & 0.224 & 0.225 & 0.423 & 0.226 & 0.351 & 0.13 & 0.13 & 0.382 & 13.63 & 0.134 & 0.133 & 0.181 & 24.024 & 0.218 & 12.843 \\ 
  \midrule\multirow{9}[6]{*}{lsnr} & \multirow{3}[2]{*}{$\rho=0$} & p=12 & 1.2 & 1.193 & 1.187 & 1.178 & 1.191 & 1.181 & 1.183 & 1.175 & 1.183 & 1.448 & 0.411 & 0.406 & 0.4 & 0.393 & 0.406 & 0.396 & 0.397 & 0.39 & 0.397 & 0.657 \\ 
   &  & p=100 & 1.361 & 1.365 & 1.366 & 1.387 & 1.386 & 1.379 & 1.378 & 1.403 & 1.378 & 1.543 & 0.508 & 0.51 & 0.509 & 0.532 & 0.531 & 0.521 & 0.521 & 0.553 & 0.521 & 0.71 \\ 
   &  & p=196 & 1.361 & 1.365 & 1.802 & 2.344 & 1.386 & 1.379 & 1.378 & 2.763 & 1.382 & 2.011 & 0.508 & 0.51 & 1.735 & 64.807 & 0.531 & 0.521 & 0.521 & 113.551 & 0.625 & 53.57 \\ 
  \cmidrule{2-23} & \multirow{3}[2]{*}{$\rho=0.5$} & p=12 & 0.751 & 0.746 & 0.742 & 0.737 & 0.745 & 0.741 & 0.74 & 0.735 & 0.739 & 0.882 & 0.423 & 0.415 & 0.409 & 0.4 & 0.414 & 0.407 & 0.405 & 0.398 & 0.405 & 0.627 \\ 
   &  & p=100 & 0.827 & 0.83 & 0.831 & 0.844 & 0.838 & 0.838 & 0.839 & 0.852 & 0.839 & 0.924 & 0.499 & 0.501 & 0.501 & 0.527 & 0.514 & 0.512 & 0.513 & 0.542 & 0.513 & 0.665 \\ 
   &  & p=196 & 0.827 & 0.83 & 1.097 & 1.429 & 0.838 & 0.838 & 0.839 & 1.676 & 0.841 & 1.211 & 0.499 & 0.501 & 1.715 & 66.071 & 0.514 & 0.512 & 0.513 & 110.244 & 0.613 & 52.882 \\ 
  \cmidrule{2-23} & \multirow{3}[2]{*}{$\rho=0.9$} & p=12 & 0.375 & 0.375 & 0.374 & 0.374 & 0.375 & 0.373 & 0.374 & 0.373 & 0.374 & 0.418 & 0.397 & 0.394 & 0.392 & 0.39 & 0.394 & 0.39 & 0.391 & 0.388 & 0.391 & 0.491 \\ 
   &  & p=100 & 0.39 & 0.394 & 0.395 & 0.401 & 0.397 & 0.397 & 0.398 & 0.407 & 0.399 & 0.419 & 0.412 & 0.421 & 0.423 & 0.44 & 0.428 & 0.431 & 0.43 & 0.465 & 0.433 & 0.461 \\ 
   &  & p=196 & 0.39 & 0.394 & 0.55 & 0.726 & 0.397 & 0.397 & 0.398 & 0.851 & 0.401 & 0.576 & 0.412 & 0.421 & 1.684 & 67.917 & 0.428 & 0.431 & 0.43 & 114.267 & 0.676 & 52.475 \\ 
   \midrule 
 \multicolumn{1}{|c}{} & \multicolumn{1}{c}{} &       & \multicolumn{10}{c||}{Relative Test Error}                                    & \multicolumn{10}{c|}{Proportion of Variance Explained} \\
\midrule\multirow{9}[6]{*}{hsnr} & \multirow{3}[2]{*}{$\rho=0$} & p=12 & 1.064 & 1.064 & 1.064 & 1.064 & 1.064 & 1.064 & 1.064 & 1.064 & 1.064 & 1.064 & 0.891 & 0.891 & 0.891 & 0.891 & 0.891 & 0.891 & 0.891 & 0.891 & 0.891 & 0.891 \\ 
   &  & p=100 & 1.236 & 1.232 & 1.231 & 1.239 & 1.236 & 1.234 & 1.234 & 1.264 & 1.234 & 1.3 & 0.879 & 0.88 & 0.88 & 0.879 & 0.879 & 0.879 & 0.88 & 0.877 & 0.88 & 0.873 \\ 
   &  & p=196 & 1.236 & 1.232 & 1.479 & 15.263 & 1.236 & 1.234 & 1.256 & 28.469 & 1.257 & 19.259 & 0.879 & 0.88 & 0.856 & -0.49 & 0.879 & 0.879 & 0.877 & -1.78 & 0.877 & -0.88 \\ 
  \cmidrule{2-23} & \multirow{3}[2]{*}{$\rho=0.5$} & p=12 & 1.064 & 1.064 & 1.064 & 1.064 & 1.064 & 1.064 & 1.064 & 1.064 & 1.064 & 1.064 & 0.892 & 0.892 & 0.892 & 0.892 & 0.892 & 0.892 & 0.892 & 0.892 & 0.892 & 0.892 \\ 
   &  & p=100 & 1.234 & 1.231 & 1.23 & 1.238 & 1.235 & 1.233 & 1.233 & 1.261 & 1.233 & 1.298 & 0.88 & 0.88 & 0.88 & 0.879 & 0.88 & 0.88 & 0.88 & 0.877 & 0.88 & 0.874 \\ 
   &  & p=196 & 1.234 & 1.231 & 1.477 & 15.501 & 1.235 & 1.233 & 1.255 & 28.496 & 1.255 & 19.374 & 0.88 & 0.88 & 0.856 & -0.511 & 0.88 & 0.88 & 0.878 & -1.777 & 0.878 & -0.888 \\ 
  \cmidrule{2-23} & \multirow{3}[2]{*}{$\rho=0.9$} & p=12 & 1.064 & 1.064 & 1.064 & 1.064 & 1.064 & 1.064 & 1.064 & 1.064 & 1.064 & 1.064 & 0.89 & 0.89 & 0.89 & 0.89 & 0.89 & 0.89 & 0.89 & 0.89 & 0.89 & 0.89 \\ 
   &  & p=100 & 1.215 & 1.213 & 1.213 & 1.22 & 1.216 & 1.216 & 1.215 & 1.239 & 1.215 & 1.287 & 0.878 & 0.878 & 0.878 & 0.877 & 0.878 & 0.878 & 0.878 & 0.875 & 0.878 & 0.87 \\ 
   &  & p=196 & 1.215 & 1.213 & 1.457 & 15.44 & 1.216 & 1.216 & 1.266 & 27.954 & 1.304 & 17.789 & 0.878 & 0.878 & 0.853 & -0.554 & 0.878 & 0.878 & 0.873 & -1.813 & 0.869 & -0.79 \\ 
  \midrule\multirow{9}[6]{*}{msnr} & \multirow{3}[2]{*}{$\rho=0$} & p=12 & 1.07 & 1.069 & 1.069 & 1.069 & 1.07 & 1.069 & 1.069 & 1.068 & 1.069 & 1.111 & 0.473 & 0.474 & 0.474 & 0.474 & 0.473 & 0.474 & 0.474 & 0.474 & 0.474 & 0.453 \\ 
   &  & p=100 & 1.158 & 1.158 & 1.158 & 1.165 & 1.162 & 1.159 & 1.159 & 1.176 & 1.159 & 1.235 & 0.445 & 0.445 & 0.445 & 0.442 & 0.443 & 0.445 & 0.445 & 0.437 & 0.445 & 0.408 \\ 
   &  & p=196 & 1.158 & 1.158 & 1.418 & 15.201 & 1.162 & 1.159 & 1.181 & 26.933 & 1.182 & 15.705 & 0.445 & 0.445 & 0.321 & -6.283 & 0.443 & 0.445 & 0.434 & -11.903 & 0.434 & -6.524 \\ 
  \cmidrule{2-23} & \multirow{3}[2]{*}{$\rho=0.5$} & p=12 & 1.068 & 1.068 & 1.067 & 1.067 & 1.068 & 1.067 & 1.067 & 1.067 & 1.067 & 1.118 & 0.476 & 0.476 & 0.476 & 0.477 & 0.476 & 0.476 & 0.477 & 0.477 & 0.477 & 0.452 \\ 
   &  & p=100 & 1.155 & 1.155 & 1.156 & 1.163 & 1.16 & 1.158 & 1.158 & 1.173 & 1.158 & 1.244 & 0.447 & 0.447 & 0.447 & 0.443 & 0.445 & 0.446 & 0.446 & 0.438 & 0.446 & 0.405 \\ 
   &  & p=196 & 1.155 & 1.155 & 1.415 & 15.442 & 1.16 & 1.158 & 1.179 & 26.525 & 1.18 & 15.571 & 0.447 & 0.447 & 0.323 & -6.39 & 0.445 & 0.446 & 0.436 & -11.695 & 0.435 & -6.452 \\ 
  \cmidrule{2-23} & \multirow{3}[2]{*}{$\rho=0.9$} & p=12 & 1.07 & 1.069 & 1.069 & 1.068 & 1.071 & 1.069 & 1.068 & 1.068 & 1.068 & 1.173 & 0.471 & 0.472 & 0.472 & 0.472 & 0.471 & 0.472 & 0.472 & 0.473 & 0.472 & 0.421 \\ 
   &  & p=100 & 1.137 & 1.137 & 1.137 & 1.145 & 1.141 & 1.14 & 1.14 & 1.154 & 1.14 & 1.251 & 0.446 & 0.446 & 0.446 & 0.442 & 0.444 & 0.444 & 0.444 & 0.438 & 0.444 & 0.39 \\ 
   &  & p=196 & 1.137 & 1.137 & 1.402 & 15.33 & 1.141 & 1.14 & 1.191 & 26.257 & 1.229 & 14.502 & 0.446 & 0.446 & 0.317 & -6.473 & 0.444 & 0.444 & 0.42 & -11.8 & 0.401 & -6.07 \\ 
  \midrule\multirow{9}[6]{*}{lsnr} & \multirow{3}[2]{*}{$\rho=0$} & p=12 & 1.085 & 1.084 & 1.083 & 1.081 & 1.084 & 1.082 & 1.082 & 1.08 & 1.082 & 1.136 & 0.101 & 0.102 & 0.103 & 0.104 & 0.102 & 0.103 & 0.103 & 0.104 & 0.103 & 0.059 \\ 
   &  & p=100 & 1.111 & 1.111 & 1.111 & 1.116 & 1.115 & 1.113 & 1.113 & 1.12 & 1.113 & 1.154 & 0.088 & 0.088 & 0.088 & 0.084 & 0.084 & 0.085 & 0.086 & 0.08 & 0.086 & 0.052 \\ 
   &  & p=196 & 1.111 & 1.111 & 1.377 & 15.093 & 1.115 & 1.113 & 1.113 & 25.693 & 1.136 & 12.649 & 0.088 & 0.088 & -0.131 & -11.397 & 0.084 & 0.085 & 0.086 & -20.104 & 0.067 & -9.39 \\ 
  \cmidrule{2-23} & \multirow{3}[2]{*}{$\rho=0.5$} & p=12 & 1.088 & 1.086 & 1.085 & 1.083 & 1.086 & 1.085 & 1.084 & 1.083 & 1.084 & 1.13 & 0.099 & 0.101 & 0.102 & 0.103 & 0.101 & 0.102 & 0.102 & 0.103 & 0.102 & 0.064 \\ 
   &  & p=100 & 1.109 & 1.109 & 1.109 & 1.115 & 1.112 & 1.112 & 1.112 & 1.118 & 1.112 & 1.145 & 0.09 & 0.089 & 0.089 & 0.085 & 0.087 & 0.087 & 0.087 & 0.082 & 0.087 & 0.06 \\ 
   &  & p=196 & 1.109 & 1.109 & 1.374 & 15.396 & 1.112 & 1.112 & 1.112 & 25.021 & 1.134 & 12.523 & 0.09 & 0.089 & -0.128 & -11.642 & 0.087 & 0.087 & 0.087 & -19.545 & 0.069 & -9.282 \\ 
  \cmidrule{2-23} & \multirow{3}[2]{*}{$\rho=0.9$} & p=12 & 1.081 & 1.081 & 1.08 & 1.08 & 1.081 & 1.08 & 1.08 & 1.079 & 1.08 & 1.101 & 0.103 & 0.103 & 0.103 & 0.104 & 0.103 & 0.104 & 0.104 & 0.104 & 0.104 & 0.087 \\ 
   &  & p=100 & 1.087 & 1.088 & 1.089 & 1.092 & 1.09 & 1.091 & 1.09 & 1.098 & 1.091 & 1.097 & 0.102 & 0.101 & 0.1 & 0.097 & 0.099 & 0.099 & 0.099 & 0.093 & 0.099 & 0.094 \\ 
   &  & p=196 & 1.087 & 1.088 & 1.354 & 15.28 & 1.09 & 1.091 & 1.09 & 25.026 & 1.142 & 12.033 & 0.102 & 0.101 & -0.119 & -11.626 & 0.099 & 0.099 & 0.099 & -19.678 & 0.056 & -8.943 \\ 
   \midrule 
 \multicolumn{1}{|c}{} & \multicolumn{1}{c}{} &       & \multicolumn{10}{c||}{Kullback-Leibler Discrepancy}                                    & \multicolumn{10}{c|}{Number of Variables} \\
\midrule\multirow{9}[6]{*}{hsnr} & \multirow{3}[2]{*}{$\rho=0$} & p=12 & 0.314 & 0.314 & 0.314 & 0.314 & 0.314 & 0.314 & 0.314 & 0.314 & 0.314 & 0.314 & 12 & 12 & 12 & 12 & 12 & 12 & 12 & 12 & 12 & 12 \\ 
   &  & p=100 & 0.512 & 0.515 & 0.516 & 0.536 & 0.525 & 0.527 & 0.527 & 0.58 & 0.527 & 0.534 & 25.456 & 26.583 & 26.972 & 29.167 & 28.123 & 28.603 & 28.461 & 31.979 & 28.492 & 20.826 \\ 
   &  & p=196 & 0.512 & 0.515 & 0.928 & 104.955 & 0.525 & 0.527 & 1.307 & 151.701 & 1.772 & 132.908 & 25.456 & 26.583 & 41.353 & 74.405 & 28.123 & 28.603 & 28.625 & 111.014 & 28.658 & 65.115 \\ 
  \cmidrule{2-23} & \multirow{3}[2]{*}{$\rho=0.5$} & p=12 & -0.645 & -0.645 & -0.645 & -0.645 & -0.645 & -0.645 & -0.645 & -0.645 & -0.645 & -0.645 & 12 & 12 & 12 & 12 & 12 & 12 & 12 & 12 & 12 & 12 \\ 
   &  & p=100 & -0.467 & -0.465 & -0.464 & -0.443 & -0.454 & -0.452 & -0.451 & -0.402 & -0.451 & -0.446 & 25.109 & 26.232 & 26.639 & 28.858 & 27.82 & 28.27 & 28.251 & 31.622 & 28.346 & 20.527 \\ 
   &  & p=196 & -0.467 & -0.465 & -0.056 & 103.71 & -0.454 & -0.452 & 0.534 & 150.944 & 0.535 & 132.072 & 25.109 & 26.232 & 41.1 & 74.198 & 27.82 & 28.27 & 28.416 & 110.833 & 28.511 & 65.089 \\ 
  \cmidrule{2-23} & \multirow{3}[2]{*}{$\rho=0.9$} & p=12 & -1.953 & -1.953 & -1.953 & -1.953 & -1.953 & -1.953 & -1.953 & -1.953 & -1.953 & -1.953 & 12 & 12 & 12 & 12 & 12 & 12 & 12 & 12 & 12 & 12 \\ 
   &  & p=100 & -1.756 & -1.753 & -1.752 & -1.734 & -1.744 & -1.741 & -1.742 & -1.698 & -1.742 & -1.729 & 22.477 & 23.642 & 24.027 & 25.895 & 24.897 & 25.347 & 25.357 & 28.445 & 25.357 & 17.6 \\ 
   &  & p=196 & -1.756 & -1.753 & -1.365 & 101.575 & -1.744 & -1.741 & 2.006 & 147.281 & 4.07 & 126.559 & 22.477 & 23.642 & 39.012 & 72.64 & 24.897 & 25.347 & 25.692 & 107.566 & 25.865 & 59.17 \\ 
  \midrule\multirow{9}[6]{*}{msnr} & \multirow{3}[2]{*}{$\rho=0$} & p=12 & 2.46 & 2.459 & 2.459 & 2.458 & 2.46 & 2.459 & 2.458 & 2.458 & 2.458 & 2.493 & 11.639 & 11.676 & 11.681 & 11.699 & 11.644 & 11.689 & 11.698 & 11.714 & 11.698 & 10.297 \\ 
   &  & p=100 & 2.575 & 2.577 & 2.579 & 2.593 & 2.585 & 2.583 & 2.584 & 2.612 & 2.584 & 2.61 & 15.105 & 15.882 & 16.336 & 17.808 & 16.918 & 17.031 & 17.137 & 19.29 & 17.148 & 9.943 \\ 
   &  & p=196 & 2.575 & 2.577 & 3.004 & 107.026 & 2.585 & 2.583 & 3.364 & 151.008 & 3.829 & 123.07 & 15.105 & 15.882 & 33.264 & 66.853 & 16.918 & 17.031 & 17.304 & 99.484 & 17.323 & 47.07 \\ 
  \cmidrule{2-23} & \multirow{3}[2]{*}{$\rho=0.5$} & p=12 & 1.499 & 1.498 & 1.498 & 1.498 & 1.499 & 1.498 & 1.498 & 1.497 & 1.498 & 1.536 & 11.822 & 11.844 & 11.851 & 11.867 & 11.823 & 11.859 & 11.867 & 11.878 & 11.867 & 10.42 \\ 
   &  & p=100 & 1.594 & 1.598 & 1.599 & 1.613 & 1.605 & 1.605 & 1.604 & 1.631 & 1.605 & 1.637 & 15 & 15.688 & 16.163 & 17.573 & 16.675 & 16.818 & 16.792 & 19.044 & 16.858 & 9.532 \\ 
   &  & p=196 & 1.594 & 1.598 & 2.02 & 105.783 & 1.605 & 1.605 & 2.59 & 149.546 & 2.591 & 121.753 & 15 & 15.688 & 32.903 & 66.697 & 16.675 & 16.818 & 16.945 & 98.48 & 17.032 & 46.585 \\ 
  \cmidrule{2-23} & \multirow{3}[2]{*}{$\rho=0.9$} & p=12 & 0.193 & 0.192 & 0.191 & 0.191 & 0.193 & 0.191 & 0.191 & 0.19 & 0.191 & 0.259 & 11.622 & 11.675 & 11.698 & 11.729 & 11.61 & 11.714 & 11.719 & 11.755 & 11.721 & 8.005 \\ 
   &  & p=100 & 0.31 & 0.312 & 0.313 & 0.325 & 0.317 & 0.318 & 0.318 & 0.34 & 0.318 & 0.358 & 12.785 & 13.481 & 13.812 & 15.033 & 14.132 & 14.413 & 14.449 & 16.392 & 14.498 & 5.797 \\ 
   &  & p=196 & 0.31 & 0.312 & 0.716 & 103.458 & 0.317 & 0.318 & 4.066 & 146.045 & 6.13 & 115.525 & 12.785 & 13.481 & 30.947 & 64.487 & 14.132 & 14.413 & 14.81 & 95.741 & 15.034 & 40.007 \\ 
  \midrule\multirow{9}[6]{*}{lsnr} & \multirow{3}[2]{*}{$\rho=0$} & p=12 & 4.076 & 4.075 & 4.074 & 4.074 & 4.075 & 4.074 & 4.074 & 4.073 & 4.074 & 4.101 & 7.688 & 7.875 & 7.998 & 8.268 & 7.899 & 8.178 & 8.141 & 8.364 & 8.149 & 2.798 \\ 
   &  & p=100 & 4.135 & 4.138 & 4.139 & 4.147 & 4.144 & 4.143 & 4.143 & 4.155 & 4.143 & 4.151 & 7.798 & 8.374 & 8.689 & 9.799 & 8.993 & 9.288 & 9.322 & 10.5 & 9.353 & 2.511 \\ 
   &  & p=196 & 4.135 & 4.138 & 4.57 & 108.351 & 4.144 & 4.143 & 4.143 & 150.013 & 5.388 & 112.359 & 7.798 & 8.374 & 27.188 & 60.945 & 8.993 & 9.288 & 9.322 & 90.109 & 9.539 & 32.674 \\ 
  \cmidrule{2-23} & \multirow{3}[2]{*}{$\rho=0.5$} & p=12 & 3.118 & 3.117 & 3.117 & 3.116 & 3.117 & 3.117 & 3.116 & 3.116 & 3.116 & 3.139 & 7.411 & 7.651 & 7.805 & 8.1 & 7.698 & 7.934 & 7.97 & 8.189 & 7.987 & 2.809 \\ 
   &  & p=100 & 3.156 & 3.158 & 3.16 & 3.169 & 3.163 & 3.164 & 3.164 & 3.174 & 3.164 & 3.168 & 7.52 & 7.966 & 8.471 & 9.438 & 8.629 & 8.939 & 8.938 & 10.038 & 8.941 & 2.638 \\ 
   &  & p=196 & 3.156 & 3.158 & 3.586 & 107.338 & 3.163 & 3.164 & 3.164 & 148.218 & 4.149 & 110.681 & 7.52 & 7.966 & 26.972 & 60.775 & 8.629 & 8.939 & 8.938 & 88.293 & 9.126 & 32.401 \\ 
  \cmidrule{2-23} & \multirow{3}[2]{*}{$\rho=0.9$} & p=12 & 1.803 & 1.803 & 1.803 & 1.803 & 1.803 & 1.803 & 1.803 & 1.803 & 1.803 & 1.814 & 6.048 & 6.289 & 6.466 & 6.665 & 6.217 & 6.54 & 6.594 & 6.826 & 6.612 & 2.079 \\ 
   &  & p=100 & 1.868 & 1.871 & 1.873 & 1.878 & 1.874 & 1.875 & 1.875 & 1.886 & 1.875 & 1.872 & 5.55 & 5.981 & 6.31 & 6.933 & 6.324 & 6.497 & 6.51 & 7.472 & 6.581 & 2.004 \\ 
   &  & p=196 & 1.868 & 1.871 & 2.282 & 105.013 & 1.874 & 1.875 & 1.875 & 144.855 & 5.624 & 107.671 & 5.55 & 5.981 & 25.737 & 59.467 & 6.324 & 6.497 & 6.51 & 84.918 & 6.956 & 30.695 \\ 
   \bottomrule 
\end{tabular}
}
\end{table}

% latex table generated in R 4.0.2 by xtable 1.8-4 package
% Fri Jun 26 10:43:52 2020
\begin{table}[ht]
\centering
\caption{Random X, Dense-Ex1, n=1000} 
\scalebox{0.5}{
\begin{tabular}{|c|c|c|cc|cc|cc|ccc|c||cc|cc|cc|ccc|c|}
  \toprule 
 \multicolumn{1}{|c}{} & \multicolumn{1}{c}{} &       & RAICc & AICc  & RC$_p$ & C$_p$ & 10F CV & LOO CV & S$_p$ & $\widehat{\text{C}}_p$ & GCV   & BIC   & RAICc & AICc  & RC$_p$ & C$_p$ & 10F CV & LOO CV & S$_p$ & $\widehat{\text{C}}_p$ & GCV   & BIC  \\
 \cmidrule{4-23}\multicolumn{1}{|c}{} & \multicolumn{1}{c}{} &       & \multicolumn{10}{c||}{RMSE}                                                   & \multicolumn{10}{c|}{Relative Risk}       \\
 \midrule\multirow{9}[6]{*}{hsnr} & \multirow{3}[2]{*}{$\rho=0$} & p=12 & 0.074 & 0.074 & 0.074 & 0.074 & 0.074 & 0.074 & 0.074 & 0.074 & 0.074 & 0.074 & 0.001 & 0.001 & 0.001 & 0.001 & 0.001 & 0.001 & 0.001 & 0.001 & 0.001 & 0.001 \\ 
   &  & p=500 & 0.155 & 0.155 & 0.155 & 0.155 & 0.155 & 0.155 & 0.155 & 0.156 & 0.155 & 0.18 & 0.006 & 0.006 & 0.006 & 0.006 & 0.006 & 0.006 & 0.006 & 0.006 & 0.006 & 0.008 \\ 
   &  & p=996 & 0.155 & 0.155 & 0.229 & 0.32 & 0.155 & 0.155 & 0.155 & 0.379 & 0.155 & 0.25 & 0.006 & 0.006 & 0.03 & 6.89 & 0.006 & 0.006 & 0.006 & 12.846 & 0.006 & 4.978 \\ 
  \cmidrule{2-23} & \multirow{3}[2]{*}{$\rho=0.5$} & p=12 & 0.046 & 0.046 & 0.046 & 0.046 & 0.046 & 0.046 & 0.046 & 0.046 & 0.046 & 0.046 & 0.001 & 0.001 & 0.001 & 0.001 & 0.001 & 0.001 & 0.001 & 0.001 & 0.001 & 0.001 \\ 
   &  & p=500 & 0.095 & 0.095 & 0.095 & 0.095 & 0.095 & 0.095 & 0.095 & 0.095 & 0.095 & 0.11 & 0.006 & 0.006 & 0.006 & 0.006 & 0.006 & 0.006 & 0.006 & 0.006 & 0.006 & 0.008 \\ 
   &  & p=996 & 0.095 & 0.095 & 0.14 & 0.196 & 0.095 & 0.095 & 0.095 & 0.234 & 0.095 & 0.153 & 0.006 & 0.006 & 0.03 & 6.845 & 0.006 & 0.006 & 0.006 & 12.315 & 0.006 & 4.965 \\ 
  \cmidrule{2-23} & \multirow{3}[2]{*}{$\rho=0.9$} & p=12 & 0.024 & 0.024 & 0.024 & 0.024 & 0.024 & 0.024 & 0.024 & 0.024 & 0.024 & 0.024 & 0.001 & 0.001 & 0.001 & 0.001 & 0.001 & 0.001 & 0.001 & 0.001 & 0.001 & 0.001 \\ 
   &  & p=500 & 0.048 & 0.048 & 0.048 & 0.048 & 0.048 & 0.048 & 0.048 & 0.048 & 0.048 & 0.056 & 0.005 & 0.005 & 0.005 & 0.005 & 0.005 & 0.005 & 0.005 & 0.005 & 0.005 & 0.007 \\ 
   &  & p=996 & 0.048 & 0.048 & 0.072 & 0.101 & 0.048 & 0.048 & 0.048 & 0.12 & 0.048 & 0.078 & 0.005 & 0.005 & 0.03 & 6.678 & 0.005 & 0.005 & 0.005 & 12.272 & 0.005 & 4.708 \\ 
  \midrule\multirow{9}[6]{*}{msnr} & \multirow{3}[2]{*}{$\rho=0$} & p=12 & 0.216 & 0.216 & 0.216 & 0.216 & 0.216 & 0.216 & 0.216 & 0.216 & 0.216 & 0.218 & 0.012 & 0.012 & 0.012 & 0.012 & 0.012 & 0.012 & 0.012 & 0.012 & 0.012 & 0.012 \\ 
   &  & p=500 & 0.396 & 0.396 & 0.396 & 0.397 & 0.398 & 0.397 & 0.397 & 0.397 & 0.397 & 0.476 & 0.037 & 0.037 & 0.037 & 0.037 & 0.037 & 0.037 & 0.037 & 0.037 & 0.037 & 0.053 \\ 
   &  & p=996 & 0.396 & 0.396 & 0.628 & 0.893 & 0.398 & 0.397 & 0.397 & 1.066 & 0.397 & 0.657 & 0.037 & 0.037 & 0.246 & 58.553 & 0.037 & 0.037 & 0.037 & 108.057 & 0.037 & 36.984 \\ 
  \cmidrule{2-23} & \multirow{3}[2]{*}{$\rho=0.5$} & p=12 & 0.134 & 0.134 & 0.134 & 0.134 & 0.134 & 0.134 & 0.134 & 0.134 & 0.134 & 0.134 & 0.012 & 0.012 & 0.012 & 0.012 & 0.012 & 0.012 & 0.012 & 0.012 & 0.012 & 0.012 \\ 
   &  & p=500 & 0.243 & 0.242 & 0.242 & 0.243 & 0.243 & 0.243 & 0.242 & 0.243 & 0.242 & 0.29 & 0.036 & 0.036 & 0.036 & 0.037 & 0.036 & 0.036 & 0.036 & 0.037 & 0.036 & 0.052 \\ 
   &  & p=996 & 0.243 & 0.242 & 0.385 & 0.548 & 0.243 & 0.243 & 0.242 & 0.655 & 0.242 & 0.401 & 0.036 & 0.036 & 0.245 & 58.167 & 0.036 & 0.036 & 0.036 & 104.242 & 0.036 & 36.878 \\ 
  \cmidrule{2-23} & \multirow{3}[2]{*}{$\rho=0.9$} & p=12 & 0.069 & 0.069 & 0.069 & 0.069 & 0.069 & 0.069 & 0.069 & 0.069 & 0.069 & 0.069 & 0.012 & 0.012 & 0.012 & 0.012 & 0.012 & 0.012 & 0.012 & 0.012 & 0.012 & 0.012 \\ 
   &  & p=500 & 0.122 & 0.122 & 0.122 & 0.122 & 0.122 & 0.122 & 0.122 & 0.122 & 0.122 & 0.148 & 0.034 & 0.034 & 0.034 & 0.034 & 0.034 & 0.034 & 0.034 & 0.034 & 0.034 & 0.049 \\ 
   &  & p=996 & 0.122 & 0.122 & 0.196 & 0.282 & 0.122 & 0.122 & 0.122 & 0.336 & 0.122 & 0.204 & 0.034 & 0.034 & 0.243 & 56.752 & 0.034 & 0.034 & 0.034 & 103.682 & 0.034 & 35.435 \\ 
  \midrule\multirow{9}[6]{*}{lsnr} & \multirow{3}[2]{*}{$\rho=0$} & p=12 & 0.503 & 0.503 & 0.503 & 0.503 & 0.508 & 0.502 & 0.503 & 0.502 & 0.503 & 0.695 & 0.066 & 0.066 & 0.066 & 0.065 & 0.067 & 0.065 & 0.065 & 0.065 & 0.065 & 0.127 \\ 
   &  & p=500 & 0.776 & 0.778 & 0.777 & 0.781 & 0.781 & 0.78 & 0.78 & 0.782 & 0.78 & 0.987 & 0.139 & 0.14 & 0.14 & 0.142 & 0.142 & 0.141 & 0.141 & 0.142 & 0.141 & 0.225 \\ 
   &  & p=996 & 0.776 & 0.778 & 1.32 & 1.922 & 0.781 & 0.78 & 0.78 & 2.304 & 0.78 & 1.367 & 0.139 & 0.14 & 1.192 & 292.752 & 0.142 & 0.141 & 0.141 & 536.268 & 0.141 & 168.413 \\ 
  \cmidrule{2-23} & \multirow{3}[2]{*}{$\rho=0.5$} & p=12 & 0.307 & 0.306 & 0.306 & 0.306 & 0.31 & 0.306 & 0.306 & 0.306 & 0.306 & 0.443 & 0.063 & 0.063 & 0.063 & 0.063 & 0.065 & 0.063 & 0.063 & 0.063 & 0.063 & 0.136 \\ 
   &  & p=500 & 0.475 & 0.475 & 0.476 & 0.477 & 0.479 & 0.477 & 0.478 & 0.479 & 0.478 & 0.604 & 0.138 & 0.139 & 0.14 & 0.14 & 0.141 & 0.14 & 0.14 & 0.142 & 0.14 & 0.223 \\ 
   &  & p=996 & 0.475 & 0.475 & 0.809 & 1.179 & 0.479 & 0.477 & 0.478 & 1.412 & 0.478 & 0.837 & 0.138 & 0.139 & 1.19 & 290.823 & 0.141 & 0.14 & 0.14 & 515.434 & 0.14 & 168.017 \\ 
  \cmidrule{2-23} & \multirow{3}[2]{*}{$\rho=0.9$} & p=12 & 0.162 & 0.161 & 0.161 & 0.161 & 0.163 & 0.161 & 0.161 & 0.161 & 0.161 & 0.248 & 0.066 & 0.066 & 0.066 & 0.065 & 0.067 & 0.065 & 0.065 & 0.065 & 0.065 & 0.156 \\ 
   &  & p=500 & 0.235 & 0.235 & 0.236 & 0.237 & 0.238 & 0.236 & 0.236 & 0.237 & 0.236 & 0.302 & 0.125 & 0.125 & 0.126 & 0.127 & 0.128 & 0.126 & 0.126 & 0.127 & 0.126 & 0.205 \\ 
   &  & p=996 & 0.235 & 0.235 & 0.413 & 0.604 & 0.238 & 0.236 & 0.236 & 0.727 & 0.236 & 0.423 & 0.125 & 0.125 & 1.18 & 283.72 & 0.128 & 0.126 & 0.126 & 517.624 & 0.126 & 164.529 \\ 
   \midrule 
 \multicolumn{1}{|c}{} & \multicolumn{1}{c}{} &       & \multicolumn{10}{c||}{Relative Test Error}                                    & \multicolumn{10}{c|}{Proportion of Variance Explained} \\
\midrule\multirow{9}[6]{*}{hsnr} & \multirow{3}[2]{*}{$\rho=0$} & p=12 & 1.012 & 1.012 & 1.012 & 1.012 & 1.012 & 1.012 & 1.012 & 1.012 & 1.012 & 1.012 & 0.893 & 0.893 & 0.893 & 0.893 & 0.893 & 0.893 & 0.893 & 0.893 & 0.893 & 0.893 \\ 
   &  & p=500 & 1.048 & 1.048 & 1.048 & 1.048 & 1.048 & 1.048 & 1.048 & 1.048 & 1.048 & 1.064 & 0.89 & 0.89 & 0.89 & 0.89 & 0.89 & 0.89 & 0.89 & 0.89 & 0.89 & 0.888 \\ 
   &  & p=996 & 1.048 & 1.048 & 1.253 & 58.92 & 1.048 & 1.048 & 1.048 & 109.434 & 1.048 & 42.921 & 0.89 & 0.89 & 0.868 & -5.262 & 0.89 & 0.89 & 0.89 & -10.589 & 0.89 & -3.554 \\ 
  \cmidrule{2-23} & \multirow{3}[2]{*}{$\rho=0.5$} & p=12 & 1.012 & 1.012 & 1.012 & 1.012 & 1.012 & 1.012 & 1.012 & 1.012 & 1.012 & 1.012 & 0.893 & 0.893 & 0.893 & 0.893 & 0.893 & 0.893 & 0.893 & 0.893 & 0.893 & 0.893 \\ 
   &  & p=500 & 1.047 & 1.047 & 1.047 & 1.048 & 1.048 & 1.047 & 1.047 & 1.048 & 1.047 & 1.064 & 0.89 & 0.89 & 0.89 & 0.89 & 0.89 & 0.89 & 0.89 & 0.89 & 0.89 & 0.888 \\ 
   &  & p=996 & 1.047 & 1.047 & 1.252 & 58.601 & 1.048 & 1.047 & 1.047 & 105.127 & 1.047 & 42.901 & 0.89 & 0.89 & 0.868 & -5.222 & 0.89 & 0.89 & 0.89 & -10.116 & 0.89 & -3.543 \\ 
  \cmidrule{2-23} & \multirow{3}[2]{*}{$\rho=0.9$} & p=12 & 1.012 & 1.012 & 1.012 & 1.012 & 1.012 & 1.012 & 1.012 & 1.012 & 1.012 & 1.012 & 0.893 & 0.893 & 0.893 & 0.893 & 0.893 & 0.893 & 0.893 & 0.893 & 0.893 & 0.893 \\ 
   &  & p=500 & 1.045 & 1.045 & 1.045 & 1.045 & 1.045 & 1.045 & 1.045 & 1.045 & 1.045 & 1.061 & 0.89 & 0.89 & 0.89 & 0.89 & 0.89 & 0.89 & 0.89 & 0.89 & 0.89 & 0.888 \\ 
   &  & p=996 & 1.045 & 1.045 & 1.252 & 57.736 & 1.045 & 1.045 & 1.045 & 105.429 & 1.045 & 41.155 & 0.89 & 0.89 & 0.868 & -5.078 & 0.89 & 0.89 & 0.89 & -10.085 & 0.89 & -3.319 \\ 
  \midrule\multirow{9}[6]{*}{msnr} & \multirow{3}[2]{*}{$\rho=0$} & p=12 & 1.012 & 1.012 & 1.012 & 1.012 & 1.012 & 1.012 & 1.012 & 1.012 & 1.012 & 1.012 & 0.494 & 0.494 & 0.494 & 0.494 & 0.494 & 0.494 & 0.494 & 0.494 & 0.494 & 0.493 \\ 
   &  & p=500 & 1.037 & 1.037 & 1.037 & 1.037 & 1.037 & 1.037 & 1.037 & 1.037 & 1.037 & 1.053 & 0.482 & 0.482 & 0.482 & 0.481 & 0.481 & 0.482 & 0.481 & 0.481 & 0.481 & 0.474 \\ 
   &  & p=996 & 1.037 & 1.037 & 1.244 & 58.907 & 1.037 & 1.037 & 1.037 & 108.265 & 1.037 & 37.594 & 0.482 & 0.482 & 0.377 & -28.597 & 0.481 & 0.482 & 0.481 & -53.302 & 0.481 & -17.882 \\ 
  \cmidrule{2-23} & \multirow{3}[2]{*}{$\rho=0.5$} & p=12 & 1.012 & 1.012 & 1.012 & 1.012 & 1.012 & 1.012 & 1.012 & 1.012 & 1.012 & 1.012 & 0.494 & 0.494 & 0.494 & 0.494 & 0.494 & 0.494 & 0.494 & 0.494 & 0.494 & 0.494 \\ 
   &  & p=500 & 1.036 & 1.036 & 1.036 & 1.036 & 1.036 & 1.036 & 1.036 & 1.037 & 1.036 & 1.052 & 0.482 & 0.482 & 0.482 & 0.482 & 0.482 & 0.482 & 0.482 & 0.482 & 0.482 & 0.474 \\ 
   &  & p=996 & 1.036 & 1.036 & 1.244 & 58.589 & 1.036 & 1.036 & 1.036 & 104.685 & 1.036 & 37.574 & 0.482 & 0.482 & 0.378 & -28.421 & 0.482 & 0.482 & 0.482 & -51.454 & 0.482 & -17.85 \\ 
  \cmidrule{2-23} & \multirow{3}[2]{*}{$\rho=0.9$} & p=12 & 1.012 & 1.012 & 1.012 & 1.012 & 1.012 & 1.012 & 1.012 & 1.012 & 1.012 & 1.012 & 0.494 & 0.494 & 0.494 & 0.494 & 0.494 & 0.494 & 0.494 & 0.494 & 0.494 & 0.494 \\ 
   &  & p=500 & 1.034 & 1.034 & 1.034 & 1.034 & 1.034 & 1.034 & 1.034 & 1.034 & 1.034 & 1.049 & 0.483 & 0.483 & 0.483 & 0.483 & 0.483 & 0.483 & 0.483 & 0.483 & 0.483 & 0.476 \\ 
   &  & p=996 & 1.034 & 1.034 & 1.243 & 57.728 & 1.034 & 1.034 & 1.034 & 104.775 & 1.034 & 36.479 & 0.483 & 0.483 & 0.379 & -27.855 & 0.483 & 0.483 & 0.483 & -51.34 & 0.483 & -17.218 \\ 
  \midrule\multirow{9}[6]{*}{lsnr} & \multirow{3}[2]{*}{$\rho=0$} & p=12 & 1.013 & 1.013 & 1.013 & 1.013 & 1.013 & 1.013 & 1.013 & 1.013 & 1.013 & 1.025 & 0.156 & 0.156 & 0.156 & 0.156 & 0.155 & 0.156 & 0.156 & 0.156 & 0.156 & 0.145 \\ 
   &  & p=500 & 1.028 & 1.028 & 1.028 & 1.028 & 1.028 & 1.028 & 1.028 & 1.028 & 1.028 & 1.045 & 0.143 & 0.143 & 0.143 & 0.143 & 0.143 & 0.143 & 0.143 & 0.143 & 0.143 & 0.129 \\ 
   &  & p=996 & 1.028 & 1.028 & 1.237 & 58.905 & 1.028 & 1.028 & 1.028 & 107.494 & 1.028 & 34.284 & 0.143 & 0.143 & -0.031 & -48.161 & 0.143 & 0.143 & 0.143 & -88.657 & 0.143 & -27.614 \\ 
  \cmidrule{2-23} & \multirow{3}[2]{*}{$\rho=0.5$} & p=12 & 1.013 & 1.013 & 1.013 & 1.013 & 1.013 & 1.013 & 1.013 & 1.013 & 1.013 & 1.027 & 0.156 & 0.156 & 0.156 & 0.156 & 0.156 & 0.156 & 0.156 & 0.156 & 0.156 & 0.144 \\ 
   &  & p=500 & 1.028 & 1.028 & 1.028 & 1.028 & 1.028 & 1.028 & 1.028 & 1.028 & 1.028 & 1.045 & 0.144 & 0.144 & 0.143 & 0.143 & 0.143 & 0.143 & 0.143 & 0.143 & 0.143 & 0.13 \\ 
   &  & p=996 & 1.028 & 1.028 & 1.236 & 58.587 & 1.028 & 1.028 & 1.028 & 103.508 & 1.028 & 34.276 & 0.144 & 0.144 & -0.03 & -47.887 & 0.143 & 0.143 & 0.143 & -85.311 & 0.143 & -27.598 \\ 
  \cmidrule{2-23} & \multirow{3}[2]{*}{$\rho=0.9$} & p=12 & 1.013 & 1.013 & 1.013 & 1.013 & 1.013 & 1.013 & 1.013 & 1.013 & 1.013 & 1.031 & 0.156 & 0.156 & 0.156 & 0.156 & 0.156 & 0.156 & 0.156 & 0.156 & 0.156 & 0.141 \\ 
   &  & p=500 & 1.025 & 1.025 & 1.025 & 1.025 & 1.026 & 1.025 & 1.025 & 1.025 & 1.025 & 1.041 & 0.146 & 0.146 & 0.146 & 0.146 & 0.146 & 0.146 & 0.146 & 0.146 & 0.146 & 0.133 \\ 
   &  & p=996 & 1.025 & 1.025 & 1.237 & 57.72 & 1.026 & 1.025 & 1.025 & 104.609 & 1.025 & 33.894 & 0.146 & 0.146 & -0.03 & -47.089 & 0.146 & 0.146 & 0.146 & -86.14 & 0.146 & -27.238 \\ 
   \midrule 
 \multicolumn{1}{|c}{} & \multicolumn{1}{c}{} &       & \multicolumn{10}{c||}{Kullback-Leibler Discrepancy}                                    & \multicolumn{10}{c|}{Number of Variables} \\
\midrule\multirow{9}[6]{*}{hsnr} & \multirow{3}[2]{*}{$\rho=0$} & p=12 & 0.287 & 0.287 & 0.287 & 0.287 & 0.287 & 0.287 & 0.287 & 0.287 & 0.287 & 0.287 & 12 & 12 & 12 & 12 & 12 & 12 & 12 & 12 & 12 & 12 \\ 
   &  & p=500 & 0.42 & 0.42 & 0.42 & 0.42 & 0.42 & 0.42 & 0.42 & 0.421 & 0.42 & 0.434 & 35.97 & 36.344 & 36.441 & 36.976 & 36.492 & 36.873 & 36.843 & 37.417 & 36.847 & 28.028 \\ 
   &  & p=996 & 0.42 & 0.42 & 1.207 & 76903.101 & 0.42 & 0.42 & 0.42 & 108415.913 & 0.42 & 76781.638 & 35.97 & 36.344 & 118.541 & 296.26 & 36.492 & 36.873 & 36.843 & 410.064 & 36.847 & 152.871 \\ 
  \cmidrule{2-23} & \multirow{3}[2]{*}{$\rho=0.5$} & p=12 & -0.665 & -0.665 & -0.665 & -0.665 & -0.665 & -0.665 & -0.665 & -0.665 & -0.665 & -0.665 & 12 & 12 & 12 & 12 & 12 & 12 & 12 & 12 & 12 & 12 \\ 
   &  & p=500 & -0.557 & -0.556 & -0.556 & -0.556 & -0.556 & -0.556 & -0.556 & -0.556 & -0.556 & -0.542 & 35.752 & 36.096 & 36.245 & 36.572 & 36.012 & 36.49 & 36.501 & 37.051 & 36.501 & 27.723 \\ 
   &  & p=996 & -0.557 & -0.556 & 0.228 & 75597.583 & -0.556 & -0.556 & -0.556 & 105005.233 & -0.556 & 76757.66 & 35.752 & 36.096 & 117.97 & 295.91 & 36.012 & 36.49 & 36.501 & 413.719 & 36.501 & 152.629 \\ 
  \cmidrule{2-23} & \multirow{3}[2]{*}{$\rho=0.9$} & p=12 & -1.989 & -1.989 & -1.989 & -1.989 & -1.989 & -1.989 & -1.989 & -1.989 & -1.989 & -1.989 & 12 & 12 & 12 & 12 & 12 & 12 & 12 & 12 & 12 & 12 \\ 
   &  & p=500 & -1.865 & -1.865 & -1.865 & -1.865 & -1.865 & -1.865 & -1.865 & -1.864 & -1.865 & -1.851 & 33.178 & 33.561 & 33.711 & 34.19 & 33.652 & 34.007 & 34.038 & 34.583 & 34.038 & 25.179 \\ 
   &  & p=996 & -1.865 & -1.865 & -1.048 & 75576.902 & -1.865 & -1.865 & -1.865 & 106902.398 & -1.865 & 75342.334 & 33.178 & 33.561 & 115.955 & 294.011 & 33.652 & 34.007 & 34.038 & 407.529 & 34.038 & 144.612 \\ 
  \midrule\multirow{9}[6]{*}{msnr} & \multirow{3}[2]{*}{$\rho=0$} & p=12 & 2.427 & 2.427 & 2.427 & 2.427 & 2.427 & 2.427 & 2.427 & 2.427 & 2.427 & 2.428 & 12 & 12 & 12 & 12 & 12 & 12 & 12 & 12 & 12 & 11.982 \\ 
   &  & p=500 & 2.547 & 2.547 & 2.547 & 2.548 & 2.548 & 2.548 & 2.548 & 2.548 & 2.548 & 2.562 & 25.616 & 25.934 & 25.895 & 26.211 & 26.071 & 26.148 & 26.173 & 26.478 & 26.175 & 17.392 \\ 
   &  & p=996 & 2.547 & 2.547 & 3.338 & 76905.201 & 2.548 & 2.548 & 2.548 & 107911.181 & 2.548 & 72300.1 & 25.616 & 25.934 & 110.363 & 287.913 & 26.071 & 26.148 & 26.173 & 401.555 & 26.175 & 125.025 \\ 
  \cmidrule{2-23} & \multirow{3}[2]{*}{$\rho=0.5$} & p=12 & 1.475 & 1.475 & 1.475 & 1.475 & 1.475 & 1.475 & 1.475 & 1.475 & 1.475 & 1.475 & 12 & 12 & 12 & 12 & 12 & 12 & 12 & 12 & 12 & 12 \\ 
   &  & p=500 & 1.571 & 1.571 & 1.571 & 1.572 & 1.572 & 1.572 & 1.572 & 1.572 & 1.572 & 1.585 & 25.336 & 25.563 & 25.687 & 26.032 & 25.794 & 25.954 & 25.97 & 26.281 & 25.97 & 17.165 \\ 
   &  & p=996 & 1.571 & 1.571 & 2.359 & 75599.684 & 1.572 & 1.572 & 1.572 & 104930.642 & 1.572 & 72276.122 & 25.336 & 25.563 & 110.129 & 287.9 & 25.794 & 25.954 & 25.97 & 402.418 & 25.97 & 124.812 \\ 
  \cmidrule{2-23} & \multirow{3}[2]{*}{$\rho=0.9$} & p=12 & 0.151 & 0.151 & 0.151 & 0.151 & 0.151 & 0.151 & 0.151 & 0.151 & 0.151 & 0.151 & 12 & 12 & 12 & 12 & 12 & 12 & 12 & 12 & 12 & 11.998 \\ 
   &  & p=500 & 0.263 & 0.263 & 0.263 & 0.263 & 0.263 & 0.263 & 0.263 & 0.263 & 0.263 & 0.277 & 22.994 & 23.242 & 23.353 & 23.616 & 23.385 & 23.559 & 23.524 & 23.805 & 23.534 & 14.494 \\ 
   &  & p=996 & 0.263 & 0.263 & 1.082 & 75579.034 & 0.263 & 0.263 & 0.263 & 106722.711 & 0.263 & 71332.246 & 22.994 & 23.242 & 107.748 & 286.465 & 23.385 & 23.559 & 23.524 & 397.024 & 23.534 & 119.516 \\ 
  \midrule\multirow{9}[6]{*}{lsnr} & \multirow{3}[2]{*}{$\rho=0$} & p=12 & 4.038 & 4.038 & 4.038 & 4.038 & 4.038 & 4.038 & 4.038 & 4.038 & 4.038 & 4.05 & 11.618 & 11.622 & 11.622 & 11.628 & 11.573 & 11.636 & 11.628 & 11.64 & 11.628 & 9.407 \\ 
   &  & p=500 & 4.147 & 4.148 & 4.148 & 4.148 & 4.148 & 4.148 & 4.148 & 4.148 & 4.148 & 4.163 & 17.999 & 18.283 & 18.256 & 18.568 & 18.37 & 18.528 & 18.573 & 18.833 & 18.573 & 9.184 \\ 
   &  & p=996 & 4.147 & 4.148 & 4.939 & 76906.834 & 4.148 & 4.148 & 4.148 & 107778.903 & 4.148 & 69388.889 & 17.999 & 18.283 & 103.999 & 282.923 & 18.37 & 18.528 & 18.573 & 392.937 & 18.573 & 108.847 \\ 
  \cmidrule{2-23} & \multirow{3}[2]{*}{$\rho=0.5$} & p=12 & 3.085 & 3.085 & 3.085 & 3.085 & 3.085 & 3.085 & 3.085 & 3.085 & 3.085 & 3.099 & 11.844 & 11.848 & 11.849 & 11.852 & 11.782 & 11.846 & 11.852 & 11.854 & 11.852 & 9.435 \\ 
   &  & p=500 & 3.172 & 3.172 & 3.172 & 3.172 & 3.172 & 3.172 & 3.172 & 3.172 & 3.172 & 3.187 & 17.664 & 17.876 & 18.047 & 18.303 & 18.103 & 18.182 & 18.211 & 18.55 & 18.212 & 8.899 \\ 
   &  & p=996 & 3.172 & 3.172 & 3.96 & 75601.317 & 3.172 & 3.172 & 3.172 & 104560.814 & 3.172 & 69376.666 & 17.664 & 17.876 & 103.758 & 282.767 & 18.103 & 18.182 & 18.211 & 391.794 & 18.212 & 108.592 \\ 
  \cmidrule{2-23} & \multirow{3}[2]{*}{$\rho=0.9$} & p=12 & 1.762 & 1.762 & 1.762 & 1.762 & 1.762 & 1.762 & 1.762 & 1.762 & 1.762 & 1.78 & 11.601 & 11.622 & 11.622 & 11.628 & 11.543 & 11.632 & 11.628 & 11.636 & 11.628 & 7.766 \\ 
   &  & p=500 & 1.863 & 1.863 & 1.863 & 1.864 & 1.864 & 1.863 & 1.863 & 1.864 & 1.863 & 1.878 & 15.296 & 15.505 & 15.632 & 15.889 & 15.601 & 15.809 & 15.774 & 16.02 & 15.779 & 6.376 \\ 
   &  & p=996 & 1.863 & 1.863 & 2.685 & 75580.631 & 1.864 & 1.863 & 1.863 & 106684.517 & 1.863 & 69022.699 & 15.296 & 15.505 & 101.677 & 279.857 & 15.601 & 15.809 & 15.774 & 391.233 & 15.779 & 105.349 \\ 
   \bottomrule 
\end{tabular}
}
\end{table}

%% latex table generated in R 4.0.2 by xtable 1.8-4 package
% Fri Jun 26 10:43:47 2020
\begin{table}[ht]
\centering
\caption{Fixed X, Dense-Ex2, n=40} 
\scalebox{0.52}{
\begin{tabular}{|c|c|c|cc|cc|cc|ccc|c||cc|cc|cc|ccc|c|}
  \toprule 
 \multicolumn{1}{|c}{} & \multicolumn{1}{c}{} &       & RAICc & AICc  & RC$_p$ & C$_p$ & 10F CV & LOO CV & S$_p$ & $\widehat{\text{C}}_p$ & GCV   & BIC   & RAICc & AICc  & RC$_p$ & C$_p$ & 10F CV & LOO CV & S$_p$ & $\widehat{\text{C}}_p$ & GCV   & BIC  \\
 \cmidrule{4-23}\multicolumn{1}{|c}{} & \multicolumn{1}{c}{} &       & \multicolumn{10}{c||}{RMSE}                                                   & \multicolumn{10}{c|}{Relative Risk}       \\
 \midrule\multirow{9}[6]{*}{hsnr} & \multirow{3}[2]{*}{$\rho=0$} & p=12 & 0.675 & 0.657 & 0.654 & 0.646 & 0.656 & 0.649 & 0.646 & 0.645 & 0.646 & 0.651 & 0.083 & 0.075 & 0.074 & 0.072 & 0.075 & 0.073 & 0.072 & 0.071 & 0.072 & 0.073 \\ 
   &  & p=20 & 2.141 & 1.003 & 0.845 & 0.805 & 0.816 & 0.81 & 0.807 & 0.801 & 0.806 & 0.805 & 0.65 & 0.191 & 0.132 & 0.111 & 0.117 & 0.114 & 0.112 & 0.108 & 0.112 & 0.111 \\ 
   &  & p=36 & 2.51 & 2.113 & 1.624 & 1.437 & 1.694 & 1.576 & 1.574 & 1.377 & 1.545 & 1.392 & 0.894 & 0.785 & 0.666 & 0.824 & 0.686 & 0.721 & 0.73 & 0.898 & 0.748 & 0.936 \\ 
  \cmidrule{2-23} & \multirow{3}[2]{*}{$\rho=0.5$} & p=12 & 0.403 & 0.397 & 0.397 & 0.396 & 0.398 & 0.397 & 0.396 & 0.396 & 0.396 & 0.397 & 0.077 & 0.072 & 0.072 & 0.072 & 0.073 & 0.072 & 0.072 & 0.072 & 0.072 & 0.072 \\ 
   &  & p=20 & 1.336 & 0.609 & 0.48 & 0.472 & 0.476 & 0.472 & 0.472 & 0.472 & 0.472 & 0.472 & 0.709 & 0.197 & 0.11 & 0.104 & 0.107 & 0.104 & 0.104 & 0.104 & 0.104 & 0.104 \\ 
   &  & p=36 & 1.44 & 1.253 & 0.963 & 0.844 & 1.008 & 0.93 & 0.935 & 0.811 & 0.916 & 0.821 & 0.864 & 0.787 & 0.663 & 0.856 & 0.691 & 0.731 & 0.744 & 0.922 & 0.76 & 0.95 \\ 
  \cmidrule{2-23} & \multirow{3}[2]{*}{$\rho=0.9$} & p=12 & 0.185 & 0.182 & 0.182 & 0.181 & 0.182 & 0.182 & 0.181 & 0.181 & 0.181 & 0.182 & 0.073 & 0.068 & 0.067 & 0.067 & 0.069 & 0.067 & 0.067 & 0.067 & 0.067 & 0.068 \\ 
   &  & p=20 & 0.568 & 0.255 & 0.194 & 0.193 & 0.194 & 0.193 & 0.193 & 0.193 & 0.193 & 0.194 & 0.703 & 0.183 & 0.083 & 0.082 & 0.083 & 0.082 & 0.082 & 0.082 & 0.082 & 0.083 \\ 
   &  & p=36 & 0.682 & 0.57 & 0.415 & 0.349 & 0.446 & 0.4 & 0.403 & 0.333 & 0.39 & 0.341 & 0.753 & 0.682 & 0.588 & 0.802 & 0.625 & 0.676 & 0.69 & 0.854 & 0.706 & 0.844 \\ 
  \midrule\multirow{9}[6]{*}{msnr} & \multirow{3}[2]{*}{$\rho=0$} & p=12 & 2.446 & 2.304 & 2.169 & 2.073 & 2.148 & 2.127 & 2.116 & 2.029 & 2.107 & 2.524 & 0.916 & 0.848 & 0.79 & 0.737 & 0.769 & 0.76 & 0.756 & 0.705 & 0.752 & 0.914 \\ 
   &  & p=20 & 2.865 & 2.803 & 2.619 & 2.524 & 2.654 & 2.628 & 2.634 & 2.491 & 2.623 & 2.98 & 0.951 & 0.936 & 0.906 & 0.931 & 0.923 & 0.93 & 0.938 & 0.939 & 0.94 & 0.978 \\ 
   &  & p=36 & 3.673 & 3.579 & 3.39 & 3.574 & 3.395 & 3.407 & 3.409 & 3.683 & 3.415 & 3.938 & 1.126 & 1.152 & 1.405 & 3.343 & 1.313 & 1.664 & 1.667 & 4.775 & 1.798 & 3.834 \\ 
  \cmidrule{2-23} & \multirow{3}[2]{*}{$\rho=0.5$} & p=12 & 1.506 & 1.434 & 1.352 & 1.292 & 1.342 & 1.325 & 1.319 & 1.263 & 1.314 & 1.563 & 0.948 & 0.9 & 0.842 & 0.778 & 0.82 & 0.808 & 0.8 & 0.743 & 0.795 & 0.942 \\ 
   &  & p=20 & 1.681 & 1.653 & 1.564 & 1.506 & 1.581 & 1.567 & 1.571 & 1.486 & 1.567 & 1.763 & 0.942 & 0.93 & 0.918 & 0.939 & 0.931 & 0.937 & 0.937 & 0.941 & 0.938 & 0.97 \\ 
   &  & p=36 & 2.144 & 2.084 & 1.988 & 2.108 & 1.996 & 2.003 & 2.007 & 2.173 & 2.005 & 2.313 & 1.109 & 1.138 & 1.383 & 3.319 & 1.293 & 1.632 & 1.618 & 4.833 & 1.771 & 3.809 \\ 
  \cmidrule{2-23} & \multirow{3}[2]{*}{$\rho=0.9$} & p=12 & 0.669 & 0.642 & 0.612 & 0.588 & 0.609 & 0.602 & 0.599 & 0.577 & 0.597 & 0.698 & 0.892 & 0.847 & 0.801 & 0.736 & 0.778 & 0.764 & 0.761 & 0.704 & 0.756 & 0.889 \\ 
   &  & p=20 & 0.667 & 0.664 & 0.646 & 0.62 & 0.649 & 0.641 & 0.642 & 0.611 & 0.64 & 0.704 & 0.814 & 0.812 & 0.805 & 0.792 & 0.803 & 0.799 & 0.804 & 0.78 & 0.803 & 0.858 \\ 
   &  & p=36 & 0.858 & 0.846 & 0.823 & 0.875 & 0.827 & 0.832 & 0.83 & 0.903 & 0.832 & 0.937 & 0.915 & 0.939 & 1.157 & 2.839 & 1.068 & 1.363 & 1.337 & 4.161 & 1.462 & 3.271 \\ 
  \midrule\multirow{9}[6]{*}{lsnr} & \multirow{3}[2]{*}{$\rho=0$} & p=12 & 3.508 & 3.545 & 3.623 & 3.727 & 3.657 & 3.66 & 3.665 & 3.774 & 3.674 & 3.555 & 1.472 & 1.564 & 1.786 & 1.981 & 1.842 & 1.857 & 1.859 & 2.073 & 1.881 & 1.311 \\ 
   &  & p=20 & 3.482 & 3.518 & 3.683 & 3.959 & 3.723 & 3.745 & 3.73 & 4.07 & 3.764 & 3.454 & 1.222 & 1.268 & 1.509 & 2.038 & 1.578 & 1.617 & 1.573 & 2.245 & 1.629 & 1.132 \\ 
   &  & p=36 & 4.436 & 4.5 & 5.118 & 5.94 & 4.754 & 4.872 & 4.873 & 6.709 & 4.972 & 5.404 & 1.344 & 1.434 & 2.935 & 11.622 & 1.926 & 2.661 & 2.755 & 17.317 & 3.532 & 10.462 \\ 
  \cmidrule{2-23} & \multirow{3}[2]{*}{$\rho=0.5$} & p=12 & 2.157 & 2.185 & 2.239 & 2.307 & 2.249 & 2.256 & 2.269 & 2.328 & 2.271 & 2.197 & 1.472 & 1.584 & 1.837 & 2.051 & 1.856 & 1.888 & 1.921 & 2.123 & 1.934 & 1.323 \\ 
   &  & p=20 & 2.062 & 2.082 & 2.175 & 2.342 & 2.187 & 2.202 & 2.206 & 2.408 & 2.222 & 2.04 & 1.21 & 1.254 & 1.484 & 1.999 & 1.529 & 1.569 & 1.563 & 2.228 & 1.616 & 1.123 \\ 
   &  & p=36 & 2.617 & 2.648 & 3.016 & 3.501 & 2.809 & 2.873 & 2.873 & 3.96 & 2.918 & 3.168 & 1.336 & 1.415 & 2.895 & 11.419 & 1.929 & 2.773 & 2.802 & 17.419 & 3.334 & 10.205 \\ 
  \cmidrule{2-23} & \multirow{3}[2]{*}{$\rho=0.9$} & p=12 & 0.997 & 1.007 & 1.031 & 1.059 & 1.032 & 1.037 & 1.046 & 1.071 & 1.047 & 1.006 & 1.402 & 1.498 & 1.732 & 1.929 & 1.729 & 1.77 & 1.82 & 2.006 & 1.837 & 1.246 \\ 
   &  & p=20 & 0.83 & 0.84 & 0.878 & 0.944 & 0.88 & 0.882 & 0.883 & 0.973 & 0.888 & 0.828 & 1.049 & 1.086 & 1.256 & 1.632 & 1.272 & 1.294 & 1.283 & 1.801 & 1.305 & 1.03 \\ 
   &  & p=36 & 1.061 & 1.074 & 1.235 & 1.441 & 1.131 & 1.152 & 1.156 & 1.608 & 1.178 & 1.297 & 1.165 & 1.223 & 2.439 & 9.906 & 1.598 & 2.118 & 2.281 & 14.538 & 2.797 & 8.727 \\ 
   \midrule 
 \multicolumn{1}{|c}{} & \multicolumn{1}{c}{} &       & \multicolumn{10}{c||}{Relative Test Error}                                    & \multicolumn{10}{c|}{Proportion of Variance Explained} \\
\midrule\multirow{9}[6]{*}{hsnr} & \multirow{3}[2]{*}{$\rho=0$} & p=12 & 1.546 & 1.499 & 1.491 & 1.474 & 1.497 & 1.48 & 1.474 & 1.472 & 1.474 & 1.485 & 0.797 & 0.803 & 0.804 & 0.806 & 0.803 & 0.806 & 0.806 & 0.807 & 0.806 & 0.805 \\ 
   &  & p=20 & 7.756 & 2.988 & 2.369 & 2.156 & 2.219 & 2.181 & 2.168 & 2.126 & 2.159 & 2.152 & 0.32 & 0.738 & 0.792 & 0.811 & 0.805 & 0.809 & 0.81 & 0.813 & 0.811 & 0.811 \\ 
   &  & p=36 & 8.967 & 7.999 & 6.941 & 8.344 & 7.119 & 7.428 & 7.512 & 9.01 & 7.67 & 9.344 & 0.096 & 0.193 & 0.3 & 0.159 & 0.282 & 0.251 & 0.242 & 0.091 & 0.226 & 0.058 \\ 
  \cmidrule{2-23} & \multirow{3}[2]{*}{$\rho=0.5$} & p=12 & 1.497 & 1.467 & 1.467 & 1.465 & 1.472 & 1.467 & 1.465 & 1.463 & 1.465 & 1.467 & 0.799 & 0.803 & 0.803 & 0.803 & 0.802 & 0.803 & 0.803 & 0.804 & 0.803 & 0.803 \\ 
   &  & p=20 & 8.495 & 3.078 & 2.167 & 2.101 & 2.132 & 2.105 & 2.101 & 2.101 & 2.101 & 2.101 & 0.266 & 0.734 & 0.813 & 0.818 & 0.816 & 0.818 & 0.818 & 0.818 & 0.818 & 0.818 \\ 
   &  & p=36 & 8.722 & 8.034 & 6.922 & 8.654 & 7.172 & 7.531 & 7.653 & 9.238 & 7.787 & 9.485 & 0.122 & 0.191 & 0.303 & 0.129 & 0.278 & 0.242 & 0.23 & 0.07 & 0.216 & 0.045 \\ 
  \cmidrule{2-23} & \multirow{3}[2]{*}{$\rho=0.9$} & p=12 & 1.502 & 1.47 & 1.466 & 1.464 & 1.474 & 1.466 & 1.464 & 1.463 & 1.464 & 1.467 & 0.81 & 0.814 & 0.815 & 0.815 & 0.814 & 0.815 & 0.815 & 0.815 & 0.815 & 0.815 \\ 
   &  & p=20 & 10.464 & 3.461 & 2.12 & 2.101 & 2.116 & 2.101 & 2.101 & 2.101 & 2.101 & 2.118 & 0.277 & 0.761 & 0.853 & 0.855 & 0.854 & 0.855 & 0.855 & 0.855 & 0.855 & 0.854 \\ 
   &  & p=36 & 8.783 & 8.045 & 7.074 & 9.288 & 7.459 & 7.983 & 8.126 & 9.818 & 8.298 & 9.721 & 0.225 & 0.29 & 0.376 & 0.18 & 0.342 & 0.295 & 0.283 & 0.133 & 0.268 & 0.142 \\ 
  \midrule\multirow{9}[6]{*}{msnr} & \multirow{3}[2]{*}{$\rho=0$} & p=12 & 1.713 & 1.66 & 1.615 & 1.574 & 1.599 & 1.592 & 1.589 & 1.549 & 1.585 & 1.712 & 0.037 & 0.066 & 0.092 & 0.115 & 0.101 & 0.105 & 0.107 & 0.129 & 0.109 & 0.038 \\ 
   &  & p=20 & 2.164 & 2.145 & 2.108 & 2.138 & 2.129 & 2.138 & 2.148 & 2.149 & 2.149 & 2.197 & 0.027 & 0.035 & 0.052 & 0.038 & 0.043 & 0.038 & 0.034 & 0.034 & 0.033 & 0.012 \\ 
   &  & p=36 & 2.181 & 2.208 & 2.474 & 4.506 & 2.377 & 2.746 & 2.748 & 6.009 & 2.885 & 5.022 & -0.065 & -0.078 & -0.207 & -1.199 & -0.16 & -0.34 & -0.341 & -1.933 & -0.408 & -1.451 \\ 
  \cmidrule{2-23} & \multirow{3}[2]{*}{$\rho=0.5$} & p=12 & 1.72 & 1.683 & 1.639 & 1.59 & 1.622 & 1.613 & 1.607 & 1.564 & 1.604 & 1.715 & 0.022 & 0.043 & 0.068 & 0.096 & 0.078 & 0.083 & 0.086 & 0.111 & 0.088 & 0.025 \\ 
   &  & p=20 & 2.171 & 2.156 & 2.142 & 2.168 & 2.157 & 2.165 & 2.165 & 2.17 & 2.166 & 2.206 & 0.032 & 0.039 & 0.045 & 0.034 & 0.038 & 0.035 & 0.035 & 0.033 & 0.035 & 0.017 \\ 
   &  & p=36 & 2.166 & 2.196 & 2.454 & 4.489 & 2.36 & 2.716 & 2.701 & 6.081 & 2.862 & 5.004 & -0.056 & -0.071 & -0.196 & -1.188 & -0.15 & -0.324 & -0.317 & -1.964 & -0.395 & -1.44 \\ 
  \cmidrule{2-23} & \multirow{3}[2]{*}{$\rho=0.9$} & p=12 & 1.725 & 1.688 & 1.651 & 1.598 & 1.632 & 1.621 & 1.619 & 1.572 & 1.615 & 1.722 & 0.048 & 0.069 & 0.089 & 0.118 & 0.099 & 0.106 & 0.107 & 0.133 & 0.109 & 0.05 \\ 
   &  & p=20 & 2.29 & 2.287 & 2.276 & 2.255 & 2.272 & 2.266 & 2.274 & 2.236 & 2.272 & 2.359 & 0.114 & 0.115 & 0.119 & 0.128 & 0.121 & 0.123 & 0.12 & 0.135 & 0.121 & 0.087 \\ 
   &  & p=36 & 2.112 & 2.141 & 2.407 & 4.45 & 2.298 & 2.657 & 2.625 & 6.056 & 2.777 & 4.975 & 0.047 & 0.034 & -0.086 & -1.009 & -0.037 & -0.199 & -0.185 & -1.734 & -0.254 & -1.246 \\ 
  \midrule\multirow{9}[6]{*}{lsnr} & \multirow{3}[2]{*}{$\rho=0$} & p=12 & 1.229 & 1.244 & 1.278 & 1.308 & 1.287 & 1.289 & 1.289 & 1.323 & 1.293 & 1.204 & -0.064 & -0.076 & -0.106 & -0.132 & -0.113 & -0.115 & -0.116 & -0.145 & -0.119 & -0.042 \\ 
   &  & p=20 & 1.299 & 1.31 & 1.369 & 1.499 & 1.386 & 1.396 & 1.385 & 1.549 & 1.398 & 1.277 & -0.044 & -0.053 & -0.1 & -0.204 & -0.114 & -0.121 & -0.113 & -0.245 & -0.124 & -0.026 \\ 
   &  & p=36 & 1.282 & 1.301 & 1.616 & 3.438 & 1.404 & 1.558 & 1.578 & 4.633 & 1.741 & 3.195 & -0.06 & -0.075 & -0.336 & -1.842 & -0.161 & -0.288 & -0.304 & -2.829 & -0.439 & -1.641 \\ 
  \cmidrule{2-23} & \multirow{3}[2]{*}{$\rho=0.5$} & p=12 & 1.223 & 1.24 & 1.279 & 1.311 & 1.282 & 1.287 & 1.292 & 1.322 & 1.294 & 1.201 & -0.062 & -0.077 & -0.11 & -0.139 & -0.113 & -0.117 & -0.121 & -0.148 & -0.123 & -0.043 \\ 
   &  & p=20 & 1.301 & 1.312 & 1.369 & 1.497 & 1.38 & 1.39 & 1.389 & 1.554 & 1.402 & 1.279 & -0.042 & -0.051 & -0.096 & -0.199 & -0.105 & -0.113 & -0.112 & -0.245 & -0.123 & -0.024 \\ 
   &  & p=36 & 1.281 & 1.297 & 1.609 & 3.401 & 1.406 & 1.583 & 1.589 & 4.663 & 1.701 & 3.146 & -0.058 & -0.072 & -0.329 & -1.81 & -0.161 & -0.308 & -0.313 & -2.852 & -0.406 & -1.599 \\ 
  \cmidrule{2-23} & \multirow{3}[2]{*}{$\rho=0.9$} & p=12 & 1.228 & 1.244 & 1.282 & 1.314 & 1.281 & 1.288 & 1.296 & 1.326 & 1.299 & 1.202 & -0.056 & -0.07 & -0.102 & -0.13 & -0.102 & -0.108 & -0.115 & -0.141 & -0.117 & -0.034 \\ 
   &  & p=20 & 1.332 & 1.344 & 1.398 & 1.517 & 1.403 & 1.41 & 1.406 & 1.571 & 1.414 & 1.327 & -0.012 & -0.021 & -0.062 & -0.152 & -0.065 & -0.071 & -0.068 & -0.193 & -0.073 & -0.007 \\ 
   &  & p=36 & 1.283 & 1.297 & 1.593 & 3.408 & 1.388 & 1.515 & 1.554 & 4.534 & 1.68 & 3.121 & -0.032 & -0.044 & -0.281 & -1.741 & -0.117 & -0.219 & -0.251 & -2.647 & -0.351 & -1.511 \\ 
   \midrule 
 \multicolumn{1}{|c}{} & \multicolumn{1}{c}{} &       & \multicolumn{10}{c||}{Kullback-Leibler Discrepancy}                                    & \multicolumn{10}{c|}{Number of Variables} \\
\midrule\multirow{9}[6]{*}{hsnr} & \multirow{3}[2]{*}{$\rho=0$} & p=12 & 1.993 & 1.977 & 1.974 & 1.967 & 1.975 & 1.969 & 1.967 & 1.966 & 1.967 & 1.972 & 11.825 & 11.914 & 11.93 & 11.967 & 11.927 & 11.958 & 11.967 & 11.973 & 11.967 & 11.942 \\ 
   &  & p=20 & 2.878 & 2.967 & 2.89 & 2.886 & 2.881 & 2.886 & 2.887 & 2.882 & 2.887 & 2.886 & 7.247 & 18.078 & 19.56 & 19.886 & 19.807 & 19.847 & 19.867 & 19.931 & 19.879 & 19.889 \\ 
   &  & p=36 & 3.171 & 3.009 & 5.146 & 28.073 & 5.63 & 18.576 & 19.911 & 32.846 & 22.496 & 35.236 & 8.313 & 11.56 & 18.996 & 28.897 & 18.487 & 22.452 & 22.2 & 32.863 & 23.496 & 33.963 \\ 
  \cmidrule{2-23} & \multirow{3}[2]{*}{$\rho=0.5$} & p=12 & 1.007 & 0.998 & 0.998 & 0.997 & 1 & 0.998 & 0.997 & 0.997 & 0.997 & 0.998 & 11.95 & 11.995 & 11.995 & 11.998 & 11.988 & 11.995 & 11.998 & 12 & 11.998 & 11.995 \\ 
   &  & p=20 & 1.875 & 1.954 & 1.839 & 1.829 & 1.833 & 1.83 & 1.829 & 1.829 & 1.829 & 1.829 & 5.917 & 17.857 & 19.915 & 19.999 & 19.958 & 19.995 & 19.999 & 20 & 19.999 & 19.999 \\ 
   &  & p=36 & 2.071 & 1.937 & 3.02 & 27.329 & 4.862 & 17.3 & 19.134 & 31.888 & 21.574 & 34.153 & 8.563 & 11.071 & 19.058 & 29.147 & 18.49 & 22.611 & 22.385 & 32.879 & 23.702 & 33.981 \\ 
  \cmidrule{2-23} & \multirow{3}[2]{*}{$\rho=0.9$} & p=12 & -0.555 & -0.563 & -0.565 & -0.565 & -0.562 & -0.565 & -0.565 & -0.565 & -0.565 & -0.564 & 11.938 & 11.988 & 11.996 & 11.998 & 11.982 & 11.996 & 11.998 & 12 & 11.998 & 11.994 \\ 
   &  & p=20 & 0.133 & 0.193 & 0.042 & 0.04 & 0.041 & 0.04 & 0.04 & 0.04 & 0.04 & 0.042 & 4.054 & 17.435 & 19.974 & 20 & 19.977 & 20 & 20 & 20 & 20 & 19.968 \\ 
   &  & p=36 & 0.499 & 0.263 & 1.824 & 25.732 & 3.111 & 16.989 & 18.051 & 30.228 & 20.598 & 32.546 & 6.005 & 9.75 & 19.442 & 29.901 & 18.046 & 22.819 & 22.603 & 33.416 & 24.232 & 34.16 \\ 
  \midrule\multirow{9}[6]{*}{msnr} & \multirow{3}[2]{*}{$\rho=0$} & p=12 & 4.061 & 4.095 & 4.095 & 4.109 & 4.1 & 4.106 & 4.109 & 4.113 & 4.109 & 4.127 & 4.939 & 6.499 & 7.854 & 9.056 & 8.221 & 8.421 & 8.559 & 9.585 & 8.664 & 4.995 \\ 
   &  & p=20 & 4.041 & 4.093 & 4.318 & 4.724 & 4.473 & 4.563 & 4.582 & 4.854 & 4.604 & 4.297 & 3.049 & 4.265 & 8.426 & 12.73 & 8.731 & 9.55 & 9.554 & 14.344 & 9.893 & 3.471 \\ 
   &  & p=36 & 4.54 & 4.574 & 6.767 & 24.49 & 5.182 & 13.139 & 13.617 & 30.955 & 15.096 & 30.291 & 3.187 & 4.32 & 10.024 & 16.956 & 8.612 & 10.287 & 10.073 & 24.51 & 10.948 & 14.936 \\ 
  \cmidrule{2-23} & \multirow{3}[2]{*}{$\rho=0.5$} & p=12 & 3.082 & 3.125 & 3.132 & 3.151 & 3.137 & 3.147 & 3.149 & 3.153 & 3.151 & 3.152 & 4.796 & 6.294 & 7.739 & 9.059 & 8.021 & 8.345 & 8.49 & 9.603 & 8.605 & 4.839 \\ 
   &  & p=20 & 2.981 & 3.024 & 3.276 & 3.685 & 3.403 & 3.499 & 3.528 & 3.807 & 3.543 & 3.242 & 3.048 & 4.061 & 8.072 & 12.632 & 8.297 & 9.147 & 9.165 & 14.234 & 9.414 & 3.352 \\ 
   &  & p=36 & 3.474 & 3.504 & 4.217 & 23.166 & 4.039 & 11.595 & 12.166 & 30.051 & 14.128 & 29.175 & 3.344 & 4.499 & 9.999 & 16.837 & 8.375 & 9.981 & 9.835 & 24.582 & 10.775 & 15.039 \\ 
  \cmidrule{2-23} & \multirow{3}[2]{*}{$\rho=0.9$} & p=12 & 1.499 & 1.545 & 1.557 & 1.583 & 1.566 & 1.575 & 1.578 & 1.59 & 1.58 & 1.574 & 4.931 & 6.339 & 7.609 & 8.946 & 7.923 & 8.223 & 8.384 & 9.488 & 8.469 & 4.989 \\ 
   &  & p=20 & 1.146 & 1.181 & 1.514 & 1.952 & 1.592 & 1.705 & 1.737 & 2.041 & 1.757 & 1.393 & 2.799 & 3.301 & 7.531 & 12.685 & 7.393 & 8.411 & 8.497 & 14.035 & 8.738 & 2.967 \\ 
   &  & p=36 & 1.662 & 1.7 & 2.591 & 21.705 & 2.217 & 10.04 & 10.417 & 28.333 & 11.985 & 27.446 & 2.984 & 3.941 & 9.577 & 16.567 & 7.306 & 8.908 & 8.831 & 24.101 & 9.724 & 14.897 \\ 
  \midrule\multirow{9}[6]{*}{lsnr} & \multirow{3}[2]{*}{$\rho=0$} & p=12 & 5.364 & 5.385 & 5.43 & 5.488 & 5.45 & 5.457 & 5.458 & 5.514 & 5.465 & 5.35 & 1.595 & 1.975 & 2.849 & 3.713 & 3.13 & 3.186 & 3.13 & 4.134 & 3.256 & 0.82 \\ 
   &  & p=20 & 5.277 & 5.296 & 5.415 & 5.707 & 5.467 & 5.523 & 5.5 & 5.829 & 5.528 & 5.278 & 1.214 & 1.459 & 2.765 & 4.84 & 3.057 & 3.228 & 3.007 & 5.776 & 3.25 & 0.573 \\ 
   &  & p=36 & 5.758 & 5.783 & 7.948 & 24.157 & 6.018 & 11.034 & 11.894 & 30.132 & 14.028 & 26.762 & 1.134 & 1.467 & 5.655 & 11.336 & 3.084 & 3.845 & 3.772 & 16.844 & 4.49 & 8.131 \\ 
  \cmidrule{2-23} & \multirow{3}[2]{*}{$\rho=0.5$} & p=12 & 4.392 & 4.418 & 4.465 & 4.528 & 4.48 & 4.492 & 4.498 & 4.548 & 4.502 & 4.387 & 1.579 & 1.956 & 2.849 & 3.765 & 3.097 & 3.172 & 3.213 & 4.133 & 3.308 & 0.838 \\ 
   &  & p=20 & 4.231 & 4.252 & 4.366 & 4.659 & 4.421 & 4.459 & 4.447 & 4.791 & 4.477 & 4.224 & 1.226 & 1.492 & 2.742 & 4.78 & 2.93 & 3.022 & 3.022 & 5.758 & 3.22 & 0.581 \\ 
   &  & p=36 & 4.703 & 4.726 & 6.824 & 22.86 & 4.988 & 10.301 & 10.907 & 29.176 & 12.606 & 25.491 & 1.176 & 1.484 & 5.633 & 11.354 & 3.202 & 3.932 & 3.836 & 16.996 & 4.375 & 7.954 \\ 
  \cmidrule{2-23} & \multirow{3}[2]{*}{$\rho=0.9$} & p=12 & 2.832 & 2.859 & 2.909 & 2.967 & 2.919 & 2.931 & 2.941 & 2.992 & 2.945 & 2.825 & 1.642 & 2.009 & 2.958 & 3.881 & 3.182 & 3.225 & 3.342 & 4.321 & 3.428 & 0.831 \\ 
   &  & p=20 & 2.429 & 2.453 & 2.56 & 2.856 & 2.609 & 2.628 & 2.621 & 2.992 & 2.632 & 2.432 & 1.199 & 1.496 & 2.597 & 4.627 & 2.739 & 2.754 & 2.668 & 5.552 & 2.796 & 0.593 \\ 
   &  & p=36 & 2.921 & 2.941 & 3.682 & 21.217 & 3.173 & 8.346 & 8.933 & 27.112 & 10.72 & 23.433 & 1.2 & 1.471 & 5.554 & 11.465 & 2.949 & 3.433 & 3.494 & 16.569 & 4.091 & 7.774 \\ 
   \bottomrule 
\end{tabular}
}
\end{table}

%% latex table generated in R 4.0.2 by xtable 1.8-4 package
% Fri Jun 26 10:43:53 2020
\begin{table}[ht]
\centering
\caption{Random X, Dense-Ex2, n=200} 
\scalebox{0.51}{
\begin{tabular}{|c|c|c|cc|cc|cc|ccc|c||cc|cc|cc|ccc|c|}
  \toprule 
 \multicolumn{1}{|c}{} & \multicolumn{1}{c}{} &       & RAICc & AICc  & RC$_p$ & C$_p$ & 10F CV & LOO CV & S$_p$ & $\widehat{\text{C}}_p$ & GCV   & BIC   & RAICc & AICc  & RC$_p$ & C$_p$ & 10F CV & LOO CV & S$_p$ & $\widehat{\text{C}}_p$ & GCV   & BIC  \\
 \cmidrule{4-23}\multicolumn{1}{|c}{} & \multicolumn{1}{c}{} &       & \multicolumn{10}{c||}{RMSE}                                                   & \multicolumn{10}{c|}{Relative Risk}       \\
 \midrule\multirow{9}[6]{*}{hsnr} & \multirow{3}[2]{*}{$\rho=0$} & p=12 & 0.252 & 0.252 & 0.252 & 0.252 & 0.252 & 0.252 & 0.252 & 0.252 & 0.252 & 0.252 & 0.008 & 0.008 & 0.008 & 0.008 & 0.008 & 0.008 & 0.008 & 0.008 & 0.008 & 0.008 \\ 
   &  & p=100 & 1.527 & 1.332 & 1.27 & 1.202 & 1.221 & 1.21 & 1.209 & 1.191 & 1.209 & 1.487 & 0.179 & 0.145 & 0.133 & 0.122 & 0.125 & 0.124 & 0.124 & 0.12 & 0.124 & 0.176 \\ 
   &  & p=196 & 1.671 & 1.471 & 1.394 & 1.444 & 1.343 & 1.325 & 1.325 & 1.592 & 1.324 & 1.695 & 0.21 & 0.177 & 0.176 & 1.557 & 0.161 & 0.169 & 0.179 & 3.994 & 0.179 & 6.364 \\ 
  \cmidrule{2-23} & \multirow{3}[2]{*}{$\rho=0.5$} & p=12 & 0.154 & 0.154 & 0.154 & 0.154 & 0.154 & 0.154 & 0.154 & 0.154 & 0.154 & 0.154 & 0.008 & 0.008 & 0.008 & 0.008 & 0.008 & 0.008 & 0.008 & 0.008 & 0.008 & 0.008 \\ 
   &  & p=100 & 0.894 & 0.781 & 0.744 & 0.705 & 0.717 & 0.71 & 0.71 & 0.698 & 0.709 & 0.879 & 0.179 & 0.145 & 0.134 & 0.123 & 0.127 & 0.125 & 0.125 & 0.121 & 0.125 & 0.179 \\ 
   &  & p=196 & 0.976 & 0.859 & 0.815 & 0.844 & 0.784 & 0.774 & 0.774 & 0.93 & 0.774 & 0.991 & 0.21 & 0.176 & 0.175 & 1.588 & 0.16 & 0.168 & 0.178 & 3.958 & 0.187 & 6.354 \\ 
  \cmidrule{2-23} & \multirow{3}[2]{*}{$\rho=0.9$} & p=12 & 0.073 & 0.073 & 0.073 & 0.073 & 0.073 & 0.073 & 0.073 & 0.073 & 0.073 & 0.073 & 0.008 & 0.008 & 0.008 & 0.008 & 0.008 & 0.008 & 0.008 & 0.008 & 0.008 & 0.008 \\ 
   &  & p=100 & 0.38 & 0.338 & 0.32 & 0.303 & 0.31 & 0.306 & 0.306 & 0.3 & 0.306 & 0.392 & 0.172 & 0.144 & 0.132 & 0.124 & 0.127 & 0.126 & 0.126 & 0.122 & 0.126 & 0.185 \\ 
   &  & p=196 & 0.405 & 0.361 & 0.344 & 0.358 & 0.332 & 0.328 & 0.328 & 0.396 & 0.328 & 0.425 & 0.193 & 0.165 & 0.166 & 1.541 & 0.151 & 0.159 & 0.177 & 3.939 & 0.177 & 6.212 \\ 
  \midrule\multirow{9}[6]{*}{msnr} & \multirow{3}[2]{*}{$\rho=0$} & p=12 & 0.74 & 0.739 & 0.739 & 0.739 & 0.744 & 0.739 & 0.739 & 0.738 & 0.739 & 0.768 & 0.066 & 0.066 & 0.066 & 0.066 & 0.067 & 0.066 & 0.066 & 0.066 & 0.066 & 0.073 \\ 
   &  & p=100 & 3.205 & 3.09 & 2.994 & 2.988 & 3.015 & 2.997 & 3 & 3.077 & 2.999 & 4.445 & 0.57 & 0.548 & 0.531 & 0.566 & 0.544 & 0.548 & 0.549 & 0.653 & 0.549 & 0.899 \\ 
   &  & p=196 & 3.293 & 3.171 & 3.262 & 3.664 & 3.1 & 3.078 & 3.081 & 4.231 & 3.083 & 4.629 & 0.589 & 0.567 & 0.713 & 11.964 & 0.565 & 0.568 & 0.569 & 29.288 & 0.629 & 17.897 \\ 
  \cmidrule{2-23} & \multirow{3}[2]{*}{$\rho=0.5$} & p=12 & 0.448 & 0.448 & 0.448 & 0.448 & 0.448 & 0.448 & 0.448 & 0.448 & 0.448 & 0.449 & 0.065 & 0.065 & 0.065 & 0.065 & 0.065 & 0.065 & 0.065 & 0.065 & 0.065 & 0.066 \\ 
   &  & p=100 & 1.872 & 1.807 & 1.751 & 1.747 & 1.766 & 1.753 & 1.755 & 1.798 & 1.755 & 2.586 & 0.567 & 0.545 & 0.529 & 0.563 & 0.544 & 0.547 & 0.548 & 0.649 & 0.548 & 0.889 \\ 
   &  & p=196 & 1.919 & 1.851 & 1.905 & 2.142 & 1.81 & 1.799 & 1.8 & 2.472 & 1.801 & 2.697 & 0.585 & 0.562 & 0.712 & 12.108 & 0.562 & 0.567 & 0.566 & 29.57 & 0.625 & 17.92 \\ 
  \cmidrule{2-23} & \multirow{3}[2]{*}{$\rho=0.9$} & p=12 & 0.213 & 0.213 & 0.213 & 0.213 & 0.213 & 0.213 & 0.213 & 0.213 & 0.213 & 0.215 & 0.065 & 0.065 & 0.065 & 0.065 & 0.065 & 0.065 & 0.065 & 0.065 & 0.065 & 0.068 \\ 
   &  & p=100 & 0.784 & 0.758 & 0.737 & 0.737 & 0.745 & 0.74 & 0.74 & 0.76 & 0.74 & 1.047 & 0.529 & 0.511 & 0.5 & 0.534 & 0.512 & 0.517 & 0.517 & 0.617 & 0.517 & 0.78 \\ 
   &  & p=196 & 0.798 & 0.773 & 0.799 & 0.905 & 0.758 & 0.753 & 0.753 & 1.046 & 0.753 & 1.103 & 0.542 & 0.525 & 0.672 & 12.205 & 0.525 & 0.528 & 0.528 & 28.767 & 0.582 & 16.895 \\ 
  \midrule\multirow{9}[6]{*}{lsnr} & \multirow{3}[2]{*}{$\rho=0$} & p=12 & 1.844 & 1.82 & 1.809 & 1.792 & 1.834 & 1.805 & 1.798 & 1.783 & 1.796 & 2.773 & 0.413 & 0.402 & 0.398 & 0.39 & 0.408 & 0.395 & 0.393 & 0.386 & 0.392 & 0.87 \\ 
   &  & p=100 & 4.826 & 4.845 & 4.847 & 4.925 & 4.862 & 4.876 & 4.881 & 5.039 & 4.882 & 4.919 & 1.065 & 1.088 & 1.119 & 1.193 & 1.115 & 1.133 & 1.135 & 1.298 & 1.136 & 1.015 \\ 
   &  & p=196 & 4.884 & 4.901 & 5.501 & 6.687 & 4.918 & 4.935 & 4.938 & 7.837 & 4.939 & 5.831 & 1.069 & 1.093 & 2.021 & 57.941 & 1.121 & 1.139 & 1.139 & 116.827 & 1.141 & 44.203 \\ 
  \cmidrule{2-23} & \multirow{3}[2]{*}{$\rho=0.5$} & p=12 & 1.126 & 1.109 & 1.101 & 1.089 & 1.116 & 1.095 & 1.094 & 1.081 & 1.093 & 1.698 & 0.418 & 0.406 & 0.399 & 0.39 & 0.409 & 0.394 & 0.394 & 0.385 & 0.394 & 0.876 \\ 
   &  & p=100 & 2.811 & 2.82 & 2.827 & 2.871 & 2.837 & 2.847 & 2.844 & 2.941 & 2.845 & 2.873 & 1.055 & 1.074 & 1.11 & 1.181 & 1.111 & 1.127 & 1.124 & 1.293 & 1.125 & 1.013 \\ 
   &  & p=196 & 2.844 & 2.85 & 3.209 & 3.893 & 2.869 & 2.878 & 2.878 & 4.572 & 2.878 & 3.406 & 1.059 & 1.079 & 2.011 & 58.924 & 1.116 & 1.129 & 1.13 & 115.687 & 1.131 & 44.261 \\ 
  \cmidrule{2-23} & \multirow{3}[2]{*}{$\rho=0.9$} & p=12 & 0.539 & 0.533 & 0.527 & 0.521 & 0.535 & 0.525 & 0.523 & 0.518 & 0.523 & 0.78 & 0.423 & 0.414 & 0.405 & 0.395 & 0.418 & 0.402 & 0.4 & 0.391 & 0.4 & 0.818 \\ 
   &  & p=100 & 1.143 & 1.149 & 1.158 & 1.176 & 1.156 & 1.161 & 1.161 & 1.198 & 1.162 & 1.171 & 0.943 & 0.966 & 1.004 & 1.067 & 0.995 & 1.009 & 1.008 & 1.148 & 1.013 & 0.925 \\ 
   &  & p=196 & 1.153 & 1.159 & 1.326 & 1.623 & 1.167 & 1.171 & 1.172 & 1.909 & 1.173 & 1.4 & 0.946 & 0.969 & 1.902 & 58.963 & 1.001 & 1.013 & 1.015 & 114.298 & 1.018 & 43.301 \\ 
   \midrule 
 \multicolumn{1}{|c}{} & \multicolumn{1}{c}{} &       & \multicolumn{10}{c||}{Relative Test Error}                                    & \multicolumn{10}{c|}{Proportion of Variance Explained} \\
\midrule\multirow{9}[6]{*}{hsnr} & \multirow{3}[2]{*}{$\rho=0$} & p=12 & 1.065 & 1.065 & 1.065 & 1.065 & 1.065 & 1.065 & 1.065 & 1.065 & 1.065 & 1.065 & 0.888 & 0.888 & 0.888 & 0.888 & 0.888 & 0.888 & 0.888 & 0.888 & 0.888 & 0.888 \\ 
   &  & p=100 & 2.524 & 2.225 & 2.129 & 2.034 & 2.062 & 2.048 & 2.046 & 2.015 & 2.046 & 2.487 & 0.734 & 0.765 & 0.775 & 0.785 & 0.782 & 0.784 & 0.784 & 0.787 & 0.784 & 0.737 \\ 
   &  & p=196 & 2.788 & 2.503 & 2.495 & 14.184 & 2.366 & 2.435 & 2.517 & 34.59 & 2.517 & 54.937 & 0.706 & 0.736 & 0.737 & -0.497 & 0.75 & 0.743 & 0.734 & -2.672 & 0.734 & -4.793 \\ 
  \cmidrule{2-23} & \multirow{3}[2]{*}{$\rho=0.5$} & p=12 & 1.065 & 1.065 & 1.065 & 1.065 & 1.065 & 1.065 & 1.065 & 1.065 & 1.065 & 1.065 & 0.888 & 0.888 & 0.888 & 0.888 & 0.888 & 0.888 & 0.888 & 0.888 & 0.888 & 0.888 \\ 
   &  & p=100 & 2.524 & 2.231 & 2.134 & 2.043 & 2.072 & 2.057 & 2.057 & 2.022 & 2.056 & 2.51 & 0.734 & 0.764 & 0.775 & 0.784 & 0.781 & 0.783 & 0.783 & 0.786 & 0.783 & 0.735 \\ 
   &  & p=196 & 2.782 & 2.494 & 2.487 & 14.436 & 2.357 & 2.43 & 2.51 & 34.395 & 2.585 & 54.863 & 0.707 & 0.737 & 0.738 & -0.525 & 0.751 & 0.744 & 0.735 & -2.641 & 0.727 & -4.784 \\ 
  \cmidrule{2-23} & \multirow{3}[2]{*}{$\rho=0.9$} & p=12 & 1.065 & 1.065 & 1.065 & 1.065 & 1.065 & 1.065 & 1.065 & 1.065 & 1.065 & 1.065 & 0.888 & 0.888 & 0.888 & 0.888 & 0.888 & 0.888 & 0.888 & 0.888 & 0.888 & 0.888 \\ 
   &  & p=100 & 2.464 & 2.222 & 2.123 & 2.05 & 2.077 & 2.064 & 2.065 & 2.03 & 2.065 & 2.566 & 0.74 & 0.766 & 0.776 & 0.784 & 0.781 & 0.782 & 0.782 & 0.786 & 0.782 & 0.729 \\ 
   &  & p=196 & 2.64 & 2.403 & 2.41 & 14.053 & 2.284 & 2.36 & 2.509 & 34.294 & 2.512 & 53.645 & 0.722 & 0.747 & 0.746 & -0.483 & 0.759 & 0.752 & 0.736 & -2.624 & 0.736 & -4.656 \\ 
  \midrule\multirow{9}[6]{*}{msnr} & \multirow{3}[2]{*}{$\rho=0$} & p=12 & 1.066 & 1.066 & 1.066 & 1.066 & 1.067 & 1.066 & 1.066 & 1.066 & 1.066 & 1.072 & 0.469 & 0.469 & 0.469 & 0.469 & 0.468 & 0.469 & 0.469 & 0.469 & 0.469 & 0.465 \\ 
   &  & p=100 & 1.569 & 1.547 & 1.53 & 1.565 & 1.543 & 1.547 & 1.548 & 1.651 & 1.548 & 1.903 & 0.216 & 0.227 & 0.235 & 0.218 & 0.228 & 0.226 & 0.226 & 0.174 & 0.226 & 0.05 \\ 
   &  & p=196 & 1.589 & 1.566 & 1.712 & 12.927 & 1.563 & 1.567 & 1.568 & 30.003 & 1.633 & 18.819 & 0.206 & 0.217 & 0.144 & -5.46 & 0.218 & 0.216 & 0.216 & -14.037 & 0.185 & -8.41 \\ 
  \cmidrule{2-23} & \multirow{3}[2]{*}{$\rho=0.5$} & p=12 & 1.065 & 1.065 & 1.065 & 1.065 & 1.065 & 1.065 & 1.065 & 1.065 & 1.065 & 1.066 & 0.469 & 0.469 & 0.469 & 0.469 & 0.469 & 0.469 & 0.469 & 0.469 & 0.469 & 0.469 \\ 
   &  & p=100 & 1.567 & 1.544 & 1.528 & 1.562 & 1.543 & 1.546 & 1.547 & 1.648 & 1.548 & 1.894 & 0.217 & 0.228 & 0.236 & 0.219 & 0.229 & 0.227 & 0.227 & 0.176 & 0.226 & 0.055 \\ 
   &  & p=196 & 1.584 & 1.562 & 1.711 & 13.061 & 1.561 & 1.566 & 1.565 & 30.292 & 1.63 & 18.844 & 0.208 & 0.219 & 0.145 & -5.529 & 0.22 & 0.217 & 0.217 & -14.178 & 0.187 & -8.422 \\ 
  \cmidrule{2-23} & \multirow{3}[2]{*}{$\rho=0.9$} & p=12 & 1.065 & 1.065 & 1.065 & 1.065 & 1.065 & 1.065 & 1.065 & 1.065 & 1.065 & 1.068 & 0.469 & 0.469 & 0.469 & 0.469 & 0.469 & 0.469 & 0.469 & 0.469 & 0.469 & 0.467 \\ 
   &  & p=100 & 1.529 & 1.511 & 1.5 & 1.534 & 1.511 & 1.517 & 1.516 & 1.617 & 1.516 & 1.785 & 0.236 & 0.245 & 0.251 & 0.234 & 0.245 & 0.242 & 0.242 & 0.192 & 0.242 & 0.11 \\ 
   &  & p=196 & 1.542 & 1.525 & 1.673 & 13.152 & 1.524 & 1.528 & 1.528 & 29.619 & 1.592 & 17.835 & 0.23 & 0.238 & 0.164 & -5.576 & 0.238 & 0.237 & 0.236 & -13.808 & 0.207 & -7.914 \\ 
  \midrule\multirow{9}[6]{*}{lsnr} & \multirow{3}[2]{*}{$\rho=0$} & p=12 & 1.083 & 1.08 & 1.079 & 1.078 & 1.082 & 1.079 & 1.078 & 1.077 & 1.078 & 1.176 & 0.099 & 0.101 & 0.102 & 0.103 & 0.1 & 0.102 & 0.103 & 0.104 & 0.103 & 0.022 \\ 
   &  & p=100 & 1.213 & 1.217 & 1.223 & 1.238 & 1.222 & 1.226 & 1.226 & 1.259 & 1.227 & 1.204 & -0.01 & -0.014 & -0.019 & -0.031 & -0.018 & -0.021 & -0.022 & -0.049 & -0.022 & -0.003 \\ 
   &  & p=196 & 1.214 & 1.218 & 1.403 & 12.549 & 1.224 & 1.227 & 1.227 & 24.281 & 1.228 & 9.769 & -0.011 & -0.015 & -0.169 & -9.451 & -0.019 & -0.022 & -0.022 & -19.218 & -0.023 & -7.141 \\ 
  \cmidrule{2-23} & \multirow{3}[2]{*}{$\rho=0.5$} & p=12 & 1.084 & 1.081 & 1.08 & 1.078 & 1.082 & 1.079 & 1.079 & 1.077 & 1.079 & 1.177 & 0.098 & 0.101 & 0.102 & 0.103 & 0.1 & 0.102 & 0.102 & 0.104 & 0.103 & 0.021 \\ 
   &  & p=100 & 1.211 & 1.215 & 1.222 & 1.236 & 1.222 & 1.225 & 1.224 & 1.258 & 1.225 & 1.203 & -0.009 & -0.012 & -0.018 & -0.029 & -0.018 & -0.02 & -0.02 & -0.048 & -0.02 & -0.002 \\ 
   &  & p=196 & 1.212 & 1.216 & 1.401 & 12.73 & 1.223 & 1.225 & 1.226 & 24.023 & 1.226 & 9.781 & -0.009 & -0.012 & -0.167 & -9.604 & -0.018 & -0.021 & -0.021 & -19.008 & -0.021 & -7.151 \\ 
  \cmidrule{2-23} & \multirow{3}[2]{*}{$\rho=0.9$} & p=12 & 1.084 & 1.083 & 1.081 & 1.079 & 1.083 & 1.08 & 1.08 & 1.078 & 1.08 & 1.165 & 0.098 & 0.099 & 0.101 & 0.102 & 0.098 & 0.101 & 0.101 & 0.103 & 0.101 & 0.031 \\ 
   &  & p=100 & 1.189 & 1.193 & 1.2 & 1.213 & 1.199 & 1.202 & 1.201 & 1.229 & 1.202 & 1.186 & 0.01 & 0.006 & 0 & -0.01 & 0.002 & -0.001 & 0 & -0.024 & -0.001 & 0.013 \\ 
   &  & p=196 & 1.189 & 1.194 & 1.381 & 12.729 & 1.2 & 1.202 & 1.203 & 23.838 & 1.203 & 9.618 & 0.01 & 0.006 & -0.15 & -9.604 & 0.001 & -0.001 & -0.002 & -18.84 & -0.002 & -7.012 \\ 
   \midrule 
 \multicolumn{1}{|c}{} & \multicolumn{1}{c}{} &       & \multicolumn{10}{c||}{Kullback-Leibler Discrepancy}                                    & \multicolumn{10}{c|}{Number of Variables} \\
\midrule\multirow{9}[6]{*}{hsnr} & \multirow{3}[2]{*}{$\rho=0$} & p=12 & 1.167 & 1.167 & 1.167 & 1.167 & 1.167 & 1.167 & 1.167 & 1.167 & 1.167 & 1.167 & 12 & 12 & 12 & 12 & 12 & 12 & 12 & 12 & 12 & 12 \\ 
   &  & p=100 & 3.358 & 3.496 & 3.564 & 4.103 & 3.917 & 4.04 & 4.046 & 4.301 & 4.047 & 3.506 & 55.826 & 67.582 & 72.802 & 88.015 & 83.049 & 85.879 & 86.006 & 94.321 & 86.044 & 59.991 \\ 
   &  & p=196 & 3.442 & 3.575 & 4.49 & 2652.866 & 4.14 & 116.711 & 225.922 & 4380.182 & 225.933 & 5790.051 & 53.883 & 65.085 & 80.745 & 117.638 & 81.695 & 86.345 & 86.387 & 169.849 & 86.585 & 169.976 \\ 
  \cmidrule{2-23} & \multirow{3}[2]{*}{$\rho=0.5$} & p=12 & 0.175 & 0.175 & 0.175 & 0.175 & 0.175 & 0.175 & 0.175 & 0.175 & 0.175 & 0.175 & 12 & 12 & 12 & 12 & 12 & 12 & 12 & 12 & 12 & 12 \\ 
   &  & p=100 & 2.284 & 2.423 & 2.491 & 3.054 & 2.844 & 2.977 & 2.983 & 3.243 & 2.988 & 2.422 & 55.66 & 67.417 & 72.672 & 88.34 & 82.784 & 85.833 & 85.932 & 94.551 & 86.108 & 59.11 \\ 
   &  & p=196 & 2.365 & 2.494 & 3.388 & 2723.87 & 3.057 & 115.643 & 224.829 & 4396.314 & 277.914 & 5786.601 & 53.649 & 64.901 & 80.478 & 117.462 & 81.525 & 86.18 & 86.141 & 169.289 & 86.453 & 169.333 \\ 
  \cmidrule{2-23} & \multirow{3}[2]{*}{$\rho=0.9$} & p=12 & -1.317 & -1.317 & -1.317 & -1.317 & -1.317 & -1.317 & -1.317 & -1.317 & -1.317 & -1.317 & 12 & 12 & 12 & 12 & 12 & 12 & 12 & 12 & 12 & 12 \\ 
   &  & p=100 & 0.529 & 0.645 & 0.733 & 1.348 & 1.094 & 1.24 & 1.252 & 1.559 & 1.257 & 0.639 & 53.618 & 64.265 & 70.539 & 87.61 & 80.67 & 84.158 & 84.354 & 94.462 & 84.465 & 53.586 \\ 
   &  & p=196 & 0.59 & 0.71 & 1.578 & 2631.15 & 1.177 & 113.727 & 284.973 & 4360.501 & 285.012 & 5738.959 & 52.281 & 62.89 & 78.332 & 115.356 & 78.433 & 83.125 & 83.312 & 167.957 & 83.617 & 164.742 \\ 
  \midrule\multirow{9}[6]{*}{msnr} & \multirow{3}[2]{*}{$\rho=0$} & p=12 & 3.308 & 3.307 & 3.307 & 3.307 & 3.308 & 3.307 & 3.307 & 3.307 & 3.307 & 3.314 & 11.974 & 11.977 & 11.977 & 11.979 & 11.958 & 11.977 & 11.979 & 11.98 & 11.979 & 11.853 \\ 
   &  & p=100 & 4.723 & 4.748 & 4.786 & 4.973 & 4.834 & 4.876 & 4.877 & 5.316 & 4.878 & 4.837 & 27.204 & 32.404 & 38.352 & 48.658 & 40.212 & 43.12 & 43.027 & 59.917 & 43.179 & 4.723 \\ 
   &  & p=196 & 4.752 & 4.777 & 5.462 & 2603.495 & 4.856 & 4.896 & 4.897 & 4175.849 & 117.165 & 3700.415 & 26.635 & 31.957 & 48.554 & 84.64 & 39.34 & 42.117 & 42.006 & 140.212 & 42.347 & 57.391 \\ 
  \cmidrule{2-23} & \multirow{3}[2]{*}{$\rho=0.5$} & p=12 & 2.315 & 2.315 & 2.315 & 2.315 & 2.315 & 2.315 & 2.315 & 2.315 & 2.315 & 2.316 & 11.999 & 11.999 & 11.999 & 11.999 & 11.997 & 11.999 & 11.999 & 11.999 & 11.999 & 11.989 \\ 
   &  & p=100 & 3.649 & 3.672 & 3.711 & 3.893 & 3.761 & 3.799 & 3.803 & 4.237 & 3.806 & 3.761 & 27.025 & 32.107 & 38.149 & 48.392 & 39.984 & 42.584 & 42.646 & 59.56 & 42.881 & 4.732 \\ 
   &  & p=196 & 3.676 & 3.699 & 4.394 & 2679.53 & 3.78 & 3.826 & 3.821 & 4221.818 & 116.087 & 3702.617 & 26.595 & 31.74 & 48.634 & 84.323 & 39.128 & 41.951 & 41.659 & 139.794 & 41.903 & 57.606 \\ 
  \cmidrule{2-23} & \multirow{3}[2]{*}{$\rho=0.9$} & p=12 & 0.823 & 0.823 & 0.823 & 0.823 & 0.823 & 0.823 & 0.823 & 0.823 & 0.823 & 0.825 & 11.996 & 11.996 & 11.996 & 11.998 & 11.996 & 11.996 & 11.998 & 11.998 & 11.998 & 11.952 \\ 
   &  & p=100 & 1.92 & 1.943 & 1.976 & 2.143 & 2.008 & 2.05 & 2.05 & 2.456 & 2.05 & 2.005 & 24.918 & 30.079 & 35.452 & 45.565 & 36.438 & 39.344 & 39.199 & 55.872 & 39.242 & 4.114 \\ 
   &  & p=196 & 1.941 & 1.962 & 2.637 & 2623.849 & 2.025 & 2.064 & 2.063 & 4122.364 & 114.329 & 3605.77 & 24.731 & 29.51 & 46.436 & 82.212 & 35.892 & 38.593 & 38.704 & 137.113 & 38.995 & 53.653 \\ 
  \midrule\multirow{9}[6]{*}{lsnr} & \multirow{3}[2]{*}{$\rho=0$} & p=12 & 4.933 & 4.931 & 4.93 & 4.928 & 4.932 & 4.929 & 4.929 & 4.928 & 4.929 & 5.01 & 10.138 & 10.32 & 10.408 & 10.545 & 10.214 & 10.443 & 10.486 & 10.608 & 10.499 & 2.528 \\ 
   &  & p=100 & 6.003 & 6.013 & 6.027 & 6.064 & 6.025 & 6.035 & 6.034 & 6.121 & 6.034 & 5.982 & 6.766 & 8.183 & 11.193 & 14.521 & 10.066 & 10.796 & 10.821 & 16.359 & 10.868 & 0.323 \\ 
   &  & p=196 & 6.021 & 6.031 & 6.698 & 2600.217 & 6.046 & 6.054 & 6.053 & 3847.648 & 6.054 & 2701.179 & 6.536 & 7.878 & 27.802 & 61.646 & 9.888 & 10.564 & 10.65 & 98.208 & 10.719 & 28.266 \\ 
  \cmidrule{2-23} & \multirow{3}[2]{*}{$\rho=0.5$} & p=12 & 3.942 & 3.94 & 3.938 & 3.937 & 3.94 & 3.938 & 3.937 & 3.936 & 3.937 & 4.019 & 10.304 & 10.514 & 10.636 & 10.789 & 10.482 & 10.711 & 10.719 & 10.88 & 10.727 & 2.382 \\ 
   &  & p=100 & 4.929 & 4.937 & 4.954 & 4.988 & 4.953 & 4.962 & 4.96 & 5.05 & 4.96 & 4.91 & 6.716 & 7.945 & 11.052 & 14.268 & 10.05 & 10.832 & 10.731 & 16.353 & 10.786 & 0.371 \\ 
   &  & p=196 & 4.947 & 4.955 & 5.621 & 2686.135 & 4.971 & 4.978 & 4.979 & 3869.607 & 4.979 & 2701.919 & 6.522 & 7.867 & 27.658 & 61.116 & 9.903 & 10.414 & 10.527 & 97.598 & 10.56 & 28.285 \\ 
  \cmidrule{2-23} & \multirow{3}[2]{*}{$\rho=0.9$} & p=12 & 2.45 & 2.449 & 2.447 & 2.445 & 2.449 & 2.446 & 2.446 & 2.444 & 2.446 & 2.515 & 10.042 & 10.213 & 10.387 & 10.59 & 10.183 & 10.457 & 10.489 & 10.662 & 10.49 & 2.137 \\ 
   &  & p=100 & 3.214 & 3.222 & 3.237 & 3.265 & 3.234 & 3.24 & 3.239 & 3.314 & 3.241 & 3.201 & 6.143 & 7.236 & 9.672 & 12.358 & 8.747 & 9.489 & 9.304 & 13.891 & 9.484 & 0.806 \\ 
   &  & p=196 & 3.227 & 3.235 & 3.898 & 2615.265 & 3.248 & 3.253 & 3.253 & 3799.924 & 3.255 & 2673.846 & 6.005 & 7.139 & 26.722 & 60.066 & 8.7 & 9.35 & 9.31 & 95.353 & 9.475 & 27.9 \\ 
   \bottomrule 
\end{tabular}
}
\end{table}

%% latex table generated in R 4.0.2 by xtable 1.8-4 package
% Fri Jun 26 10:43:48 2020
\begin{table}[ht]
\centering
\caption{Fixed X, Dense-Ex2, n=1000} 
\scalebox{0.52}{
\begin{tabular}{|c|c|c|cc|cc|cc|ccc|c||cc|cc|cc|ccc|c|}
  \toprule 
 \multicolumn{1}{|c}{} & \multicolumn{1}{c}{} &       & RAICc & AICc  & RC$_p$ & C$_p$ & 10F CV & LOO CV & S$_p$ & $\widehat{\text{C}}_p$ & GCV   & BIC   & RAICc & AICc  & RC$_p$ & C$_p$ & 10F CV & LOO CV & S$_p$ & $\widehat{\text{C}}_p$ & GCV   & BIC  \\
 \cmidrule{4-23}\multicolumn{1}{|c}{} & \multicolumn{1}{c}{} &       & \multicolumn{10}{c||}{RMSE}                                                   & \multicolumn{10}{c|}{Relative Risk}       \\
 \midrule\multirow{9}[6]{*}{hsnr} & \multirow{3}[2]{*}{$\rho=0$} & p=12 & 0.112 & 0.112 & 0.112 & 0.112 & 0.112 & 0.112 & 0.112 & 0.112 & 0.112 & 0.112 & 0.001 & 0.001 & 0.001 & 0.001 & 0.001 & 0.001 & 0.001 & 0.001 & 0.001 & 0.001 \\ 
   &  & p=500 & 0.711 & 0.704 & 0.703 & 0.7 & 0.702 & 0.7 & 0.7 & 0.704 & 0.7 & 0.861 & 0.025 & 0.024 & 0.024 & 0.024 & 0.024 & 0.024 & 0.024 & 0.025 & 0.024 & 0.035 \\ 
   &  & p=996 & 0.711 & 0.704 & 0.799 & 0.981 & 0.702 & 0.7 & 0.7 & 1.151 & 0.7 & 1.098 & 0.025 & 0.024 & 0.044 & 5.721 & 0.024 & 0.024 & 0.024 & 12.06 & 0.024 & 8.646 \\ 
  \cmidrule{2-23} & \multirow{3}[2]{*}{$\rho=0.5$} & p=12 & 0.068 & 0.068 & 0.068 & 0.068 & 0.068 & 0.068 & 0.068 & 0.068 & 0.068 & 0.068 & 0.001 & 0.001 & 0.001 & 0.001 & 0.001 & 0.001 & 0.001 & 0.001 & 0.001 & 0.001 \\ 
   &  & p=500 & 0.416 & 0.412 & 0.411 & 0.41 & 0.411 & 0.41 & 0.41 & 0.412 & 0.41 & 0.506 & 0.025 & 0.024 & 0.024 & 0.024 & 0.024 & 0.024 & 0.024 & 0.025 & 0.024 & 0.036 \\ 
   &  & p=996 & 0.416 & 0.412 & 0.467 & 0.574 & 0.411 & 0.41 & 0.41 & 0.679 & 0.41 & 0.644 & 0.025 & 0.024 & 0.044 & 5.73 & 0.024 & 0.024 & 0.024 & 12.4 & 0.024 & 8.669 \\ 
  \cmidrule{2-23} & \multirow{3}[2]{*}{$\rho=0.9$} & p=12 & 0.032 & 0.032 & 0.032 & 0.032 & 0.032 & 0.032 & 0.032 & 0.032 & 0.032 & 0.032 & 0.001 & 0.001 & 0.001 & 0.001 & 0.001 & 0.001 & 0.001 & 0.001 & 0.001 & 0.001 \\ 
   &  & p=500 & 0.176 & 0.175 & 0.175 & 0.174 & 0.175 & 0.174 & 0.174 & 0.175 & 0.174 & 0.218 & 0.024 & 0.024 & 0.024 & 0.024 & 0.024 & 0.024 & 0.024 & 0.024 & 0.024 & 0.036 \\ 
   &  & p=996 & 0.176 & 0.175 & 0.199 & 0.245 & 0.175 & 0.174 & 0.174 & 0.286 & 0.174 & 0.274 & 0.024 & 0.024 & 0.044 & 5.883 & 0.024 & 0.024 & 0.024 & 12.096 & 0.024 & 8.454 \\ 
  \midrule\multirow{9}[6]{*}{msnr} & \multirow{3}[2]{*}{$\rho=0$} & p=12 & 0.327 & 0.327 & 0.327 & 0.327 & 0.327 & 0.327 & 0.327 & 0.327 & 0.327 & 0.327 & 0.012 & 0.012 & 0.012 & 0.012 & 0.012 & 0.012 & 0.012 & 0.012 & 0.012 & 0.012 \\ 
   &  & p=500 & 1.733 & 1.726 & 1.724 & 1.724 & 1.724 & 1.724 & 1.726 & 1.731 & 1.725 & 2.334 & 0.137 & 0.136 & 0.136 & 0.137 & 0.136 & 0.136 & 0.137 & 0.138 & 0.137 & 0.237 \\ 
   &  & p=996 & 1.733 & 1.726 & 2.084 & 2.624 & 1.724 & 1.724 & 1.726 & 3.094 & 1.725 & 2.798 & 0.137 & 0.136 & 0.317 & 48.012 & 0.136 & 0.136 & 0.137 & 98.667 & 0.137 & 48.066 \\ 
  \cmidrule{2-23} & \multirow{3}[2]{*}{$\rho=0.5$} & p=12 & 0.199 & 0.199 & 0.199 & 0.199 & 0.199 & 0.199 & 0.199 & 0.199 & 0.199 & 0.199 & 0.012 & 0.012 & 0.012 & 0.012 & 0.012 & 0.012 & 0.012 & 0.012 & 0.012 & 0.012 \\ 
   &  & p=500 & 1.012 & 1.009 & 1.007 & 1.009 & 1.008 & 1.008 & 1.009 & 1.012 & 1.009 & 1.365 & 0.137 & 0.136 & 0.136 & 0.137 & 0.136 & 0.136 & 0.137 & 0.138 & 0.137 & 0.237 \\ 
   &  & p=996 & 1.012 & 1.009 & 1.22 & 1.535 & 1.008 & 1.008 & 1.009 & 1.807 & 1.009 & 1.638 & 0.137 & 0.136 & 0.317 & 48.032 & 0.136 & 0.136 & 0.137 & 98.408 & 0.137 & 48.176 \\ 
  \cmidrule{2-23} & \multirow{3}[2]{*}{$\rho=0.9$} & p=12 & 0.094 & 0.094 & 0.094 & 0.094 & 0.094 & 0.094 & 0.094 & 0.094 & 0.094 & 0.094 & 0.012 & 0.012 & 0.012 & 0.012 & 0.012 & 0.012 & 0.012 & 0.012 & 0.012 & 0.012 \\ 
   &  & p=500 & 0.426 & 0.425 & 0.425 & 0.426 & 0.425 & 0.425 & 0.425 & 0.428 & 0.426 & 0.582 & 0.132 & 0.131 & 0.131 & 0.133 & 0.132 & 0.132 & 0.132 & 0.134 & 0.132 & 0.235 \\ 
   &  & p=996 & 0.426 & 0.425 & 0.518 & 0.654 & 0.425 & 0.425 & 0.425 & 0.77 & 0.426 & 0.693 & 0.132 & 0.131 & 0.315 & 49.284 & 0.132 & 0.132 & 0.132 & 97.977 & 0.132 & 45.934 \\ 
  \midrule\multirow{9}[6]{*}{lsnr} & \multirow{3}[2]{*}{$\rho=0$} & p=12 & 0.733 & 0.733 & 0.733 & 0.733 & 0.735 & 0.733 & 0.733 & 0.733 & 0.733 & 0.791 & 0.061 & 0.061 & 0.061 & 0.061 & 0.061 & 0.061 & 0.061 & 0.061 & 0.061 & 0.073 \\ 
   &  & p=500 & 3.182 & 3.174 & 3.17 & 3.173 & 3.186 & 3.175 & 3.176 & 3.182 & 3.176 & 4.543 & 0.438 & 0.437 & 0.437 & 0.44 & 0.44 & 0.439 & 0.439 & 0.443 & 0.439 & 0.863 \\ 
   &  & p=996 & 3.182 & 3.174 & 4.158 & 5.411 & 3.186 & 3.175 & 3.176 & 6.334 & 3.176 & 5.31 & 0.438 & 0.437 & 1.392 & 239.878 & 0.44 & 0.439 & 0.439 & 469.897 & 0.439 & 165.154 \\ 
  \cmidrule{2-23} & \multirow{3}[2]{*}{$\rho=0.5$} & p=12 & 0.445 & 0.445 & 0.445 & 0.445 & 0.445 & 0.445 & 0.445 & 0.445 & 0.445 & 0.45 & 0.061 & 0.061 & 0.061 & 0.061 & 0.06 & 0.061 & 0.061 & 0.061 & 0.061 & 0.062 \\ 
   &  & p=500 & 1.859 & 1.857 & 1.853 & 1.854 & 1.863 & 1.857 & 1.857 & 1.861 & 1.857 & 2.639 & 0.437 & 0.437 & 0.436 & 0.439 & 0.44 & 0.439 & 0.439 & 0.443 & 0.439 & 0.851 \\ 
   &  & p=996 & 1.859 & 1.857 & 2.429 & 3.168 & 1.863 & 1.857 & 1.857 & 3.713 & 1.857 & 3.09 & 0.437 & 0.437 & 1.393 & 239.99 & 0.44 & 0.439 & 0.439 & 475.091 & 0.439 & 165.499 \\ 
  \cmidrule{2-23} & \multirow{3}[2]{*}{$\rho=0.9$} & p=12 & 0.211 & 0.211 & 0.211 & 0.211 & 0.211 & 0.211 & 0.211 & 0.211 & 0.211 & 0.215 & 0.061 & 0.061 & 0.061 & 0.061 & 0.061 & 0.061 & 0.061 & 0.061 & 0.061 & 0.064 \\ 
   &  & p=500 & 0.781 & 0.78 & 0.779 & 0.78 & 0.782 & 0.78 & 0.78 & 0.782 & 0.78 & 1.071 & 0.42 & 0.42 & 0.419 & 0.422 & 0.423 & 0.421 & 0.421 & 0.425 & 0.421 & 0.765 \\ 
   &  & p=996 & 0.781 & 0.78 & 1.028 & 1.345 & 0.782 & 0.78 & 0.78 & 1.573 & 0.78 & 1.268 & 0.42 & 0.42 & 1.385 & 246.247 & 0.423 & 0.421 & 0.421 & 466.027 & 0.421 & 159.91 \\ 
   \midrule 
 \multicolumn{1}{|c}{} & \multicolumn{1}{c}{} &       & \multicolumn{10}{c||}{Relative Test Error}                                    & \multicolumn{10}{c|}{Proportion of Variance Explained} \\
\midrule\multirow{9}[6]{*}{hsnr} & \multirow{3}[2]{*}{$\rho=0$} & p=12 & 1.012 & 1.012 & 1.012 & 1.012 & 1.012 & 1.012 & 1.012 & 1.012 & 1.012 & 1.012 & 0.894 & 0.894 & 0.894 & 0.894 & 0.894 & 0.894 & 0.894 & 0.894 & 0.894 & 0.894 \\ 
   &  & p=500 & 1.213 & 1.209 & 1.208 & 1.207 & 1.208 & 1.207 & 1.207 & 1.211 & 1.207 & 1.302 & 0.873 & 0.873 & 0.873 & 0.874 & 0.873 & 0.874 & 0.874 & 0.873 & 0.874 & 0.864 \\ 
   &  & p=996 & 1.213 & 1.209 & 1.375 & 49.91 & 1.208 & 1.207 & 1.207 & 104.096 & 1.207 & 74.91 & 0.873 & 0.873 & 0.856 & -4.227 & 0.873 & 0.874 & 0.874 & -9.902 & 0.874 & -6.845 \\ 
  \cmidrule{2-23} & \multirow{3}[2]{*}{$\rho=0.5$} & p=12 & 1.012 & 1.012 & 1.012 & 1.012 & 1.012 & 1.012 & 1.012 & 1.012 & 1.012 & 1.012 & 0.894 & 0.894 & 0.894 & 0.894 & 0.894 & 0.894 & 0.894 & 0.894 & 0.894 & 0.894 \\ 
   &  & p=500 & 1.212 & 1.208 & 1.208 & 1.207 & 1.208 & 1.207 & 1.207 & 1.21 & 1.207 & 1.303 & 0.873 & 0.873 & 0.873 & 0.873 & 0.873 & 0.873 & 0.873 & 0.873 & 0.873 & 0.863 \\ 
   &  & p=996 & 1.212 & 1.208 & 1.375 & 49.868 & 1.208 & 1.207 & 1.207 & 106.747 & 1.207 & 74.932 & 0.873 & 0.873 & 0.856 & -4.234 & 0.873 & 0.873 & 0.873 & -10.203 & 0.873 & -6.864 \\ 
  \cmidrule{2-23} & \multirow{3}[2]{*}{$\rho=0.9$} & p=12 & 1.012 & 1.012 & 1.012 & 1.012 & 1.012 & 1.012 & 1.012 & 1.012 & 1.012 & 1.012 & 0.894 & 0.894 & 0.894 & 0.894 & 0.894 & 0.894 & 0.894 & 0.894 & 0.894 & 0.894 \\ 
   &  & p=500 & 1.206 & 1.203 & 1.202 & 1.202 & 1.203 & 1.202 & 1.202 & 1.205 & 1.202 & 1.303 & 0.873 & 0.873 & 0.873 & 0.873 & 0.873 & 0.873 & 0.873 & 0.873 & 0.873 & 0.863 \\ 
   &  & p=996 & 1.206 & 1.203 & 1.371 & 50.868 & 1.203 & 1.202 & 1.202 & 103.534 & 1.202 & 72.665 & 0.873 & 0.873 & 0.855 & -4.368 & 0.873 & 0.873 & 0.873 & -9.925 & 0.873 & -6.668 \\ 
  \midrule\multirow{9}[6]{*}{msnr} & \multirow{3}[2]{*}{$\rho=0$} & p=12 & 1.012 & 1.012 & 1.012 & 1.012 & 1.012 & 1.012 & 1.012 & 1.012 & 1.012 & 1.012 & 0.496 & 0.496 & 0.496 & 0.496 & 0.496 & 0.496 & 0.496 & 0.496 & 0.496 & 0.496 \\ 
   &  & p=500 & 1.138 & 1.137 & 1.137 & 1.137 & 1.137 & 1.137 & 1.137 & 1.139 & 1.137 & 1.238 & 0.433 & 0.433 & 0.433 & 0.433 & 0.433 & 0.433 & 0.433 & 0.432 & 0.433 & 0.383 \\ 
   &  & p=996 & 1.138 & 1.137 & 1.318 & 49.285 & 1.137 & 1.137 & 1.137 & 100.229 & 1.137 & 49.339 & 0.433 & 0.433 & 0.343 & -23.573 & 0.433 & 0.433 & 0.433 & -48.972 & 0.433 & -23.6 \\ 
  \cmidrule{2-23} & \multirow{3}[2]{*}{$\rho=0.5$} & p=12 & 1.012 & 1.012 & 1.012 & 1.012 & 1.012 & 1.012 & 1.012 & 1.012 & 1.012 & 1.012 & 0.496 & 0.496 & 0.496 & 0.496 & 0.496 & 0.496 & 0.496 & 0.496 & 0.496 & 0.496 \\ 
   &  & p=500 & 1.137 & 1.137 & 1.136 & 1.137 & 1.136 & 1.137 & 1.137 & 1.139 & 1.137 & 1.237 & 0.432 & 0.433 & 0.433 & 0.432 & 0.433 & 0.433 & 0.432 & 0.432 & 0.432 & 0.382 \\ 
   &  & p=996 & 1.137 & 1.137 & 1.318 & 49.192 & 1.136 & 1.137 & 1.137 & 99.735 & 1.137 & 49.336 & 0.432 & 0.433 & 0.342 & -23.555 & 0.433 & 0.433 & 0.432 & -48.785 & 0.432 & -23.627 \\ 
  \cmidrule{2-23} & \multirow{3}[2]{*}{$\rho=0.9$} & p=12 & 1.012 & 1.012 & 1.012 & 1.012 & 1.012 & 1.012 & 1.012 & 1.012 & 1.012 & 1.012 & 0.496 & 0.496 & 0.496 & 0.496 & 0.496 & 0.496 & 0.496 & 0.496 & 0.496 & 0.496 \\ 
   &  & p=500 & 1.132 & 1.131 & 1.131 & 1.132 & 1.131 & 1.132 & 1.132 & 1.134 & 1.132 & 1.234 & 0.433 & 0.434 & 0.434 & 0.433 & 0.433 & 0.433 & 0.433 & 0.432 & 0.433 & 0.382 \\ 
   &  & p=996 & 1.132 & 1.131 & 1.315 & 50.15 & 1.131 & 1.132 & 1.132 & 98.71 & 1.132 & 46.808 & 0.433 & 0.434 & 0.342 & -24.109 & 0.433 & 0.433 & 0.433 & -48.423 & 0.433 & -22.436 \\ 
  \midrule\multirow{9}[6]{*}{lsnr} & \multirow{3}[2]{*}{$\rho=0$} & p=12 & 1.012 & 1.012 & 1.012 & 1.012 & 1.012 & 1.012 & 1.012 & 1.012 & 1.012 & 1.015 & 0.158 & 0.158 & 0.158 & 0.158 & 0.158 & 0.158 & 0.158 & 0.158 & 0.158 & 0.156 \\ 
   &  & p=500 & 1.088 & 1.088 & 1.088 & 1.088 & 1.089 & 1.088 & 1.088 & 1.089 & 1.088 & 1.174 & 0.094 & 0.094 & 0.094 & 0.094 & 0.094 & 0.094 & 0.094 & 0.093 & 0.094 & 0.023 \\ 
   &  & p=996 & 1.088 & 1.088 & 1.28 & 49.249 & 1.089 & 1.088 & 1.088 & 95.515 & 1.088 & 34.219 & 0.094 & 0.094 & -0.066 & -40.002 & 0.094 & 0.094 & 0.094 & -78.52 & 0.094 & -27.489 \\ 
  \cmidrule{2-23} & \multirow{3}[2]{*}{$\rho=0.5$} & p=12 & 1.012 & 1.012 & 1.012 & 1.012 & 1.012 & 1.012 & 1.012 & 1.012 & 1.012 & 1.013 & 0.158 & 0.158 & 0.158 & 0.158 & 0.158 & 0.158 & 0.158 & 0.158 & 0.158 & 0.157 \\ 
   &  & p=500 & 1.088 & 1.088 & 1.087 & 1.088 & 1.088 & 1.088 & 1.088 & 1.089 & 1.088 & 1.171 & 0.094 & 0.094 & 0.094 & 0.094 & 0.094 & 0.094 & 0.094 & 0.093 & 0.094 & 0.025 \\ 
   &  & p=996 & 1.088 & 1.088 & 1.28 & 49.158 & 1.088 & 1.088 & 1.088 & 96.334 & 1.088 & 34.21 & 0.094 & 0.094 & -0.066 & -39.942 & 0.094 & 0.094 & 0.094 & -79.234 & 0.094 & -27.492 \\ 
  \cmidrule{2-23} & \multirow{3}[2]{*}{$\rho=0.9$} & p=12 & 1.012 & 1.012 & 1.012 & 1.012 & 1.012 & 1.012 & 1.012 & 1.012 & 1.012 & 1.013 & 0.158 & 0.158 & 0.158 & 0.158 & 0.158 & 0.158 & 0.158 & 0.158 & 0.158 & 0.157 \\ 
   &  & p=500 & 1.084 & 1.084 & 1.084 & 1.084 & 1.084 & 1.084 & 1.084 & 1.085 & 1.084 & 1.153 & 0.097 & 0.096 & 0.097 & 0.096 & 0.096 & 0.096 & 0.096 & 0.096 & 0.096 & 0.039 \\ 
   &  & p=996 & 1.084 & 1.084 & 1.276 & 50.115 & 1.084 & 1.084 & 1.084 & 93.952 & 1.084 & 32.895 & 0.097 & 0.096 & -0.064 & -40.782 & 0.096 & 0.096 & 0.096 & -77.329 & 0.096 & -26.425 \\ 
   \midrule 
 \multicolumn{1}{|c}{} & \multicolumn{1}{c}{} &       & \multicolumn{10}{c||}{Kullback-Leibler Discrepancy}                                    & \multicolumn{10}{c|}{Number of Variables} \\
\midrule\multirow{9}[6]{*}{hsnr} & \multirow{3}[2]{*}{$\rho=0$} & p=12 & 1.098 & 1.098 & 1.098 & 1.098 & 1.098 & 1.098 & 1.098 & 1.098 & 1.098 & 1.098 & 12 & 12 & 12 & 12 & 12 & 12 & 12 & 12 & 12 & 12 \\ 
   &  & p=500 & 2.255 & 2.256 & 2.256 & 2.26 & 2.258 & 2.26 & 2.26 & 2.269 & 2.26 & 2.305 & 127.946 & 133.182 & 134.083 & 141.219 & 137.597 & 140.079 & 140.128 & 147.832 & 140.136 & 98.817 \\ 
   &  & p=996 & 2.255 & 2.256 & 2.497 & 391.623 & 2.258 & 2.26 & 2.26 & 637.167 & 2.26 & 603.846 & 127.946 & 133.182 & 193.613 & 363.155 & 137.593 & 140.071 & 140.125 & 530.445 & 140.136 & 352.479 \\ 
  \cmidrule{2-23} & \multirow{3}[2]{*}{$\rho=0.5$} & p=12 & 0.108 & 0.108 & 0.108 & 0.108 & 0.108 & 0.108 & 0.108 & 0.108 & 0.108 & 0.108 & 12 & 12 & 12 & 12 & 12 & 12 & 12 & 12 & 12 & 12 \\ 
   &  & p=500 & 1.184 & 1.185 & 1.186 & 1.19 & 1.188 & 1.189 & 1.189 & 1.198 & 1.189 & 1.236 & 127.639 & 132.845 & 133.758 & 140.938 & 137.397 & 139.66 & 139.727 & 147.45 & 139.734 & 98.096 \\ 
   &  & p=996 & 1.184 & 1.185 & 1.428 & 392.995 & 1.188 & 1.189 & 1.189 & 646.182 & 1.189 & 602.795 & 127.636 & 132.845 & 193.51 & 363.437 & 137.397 & 139.652 & 139.726 & 537.863 & 139.734 & 351.954 \\ 
  \cmidrule{2-23} & \multirow{3}[2]{*}{$\rho=0.9$} & p=12 & -1.385 & -1.385 & -1.385 & -1.385 & -1.385 & -1.385 & -1.385 & -1.385 & -1.385 & -1.385 & 12 & 12 & 12 & 12 & 12 & 12 & 12 & 12 & 12 & 12 \\ 
   &  & p=500 & -0.514 & -0.512 & -0.512 & -0.507 & -0.51 & -0.508 & -0.508 & -0.5 & -0.508 & -0.458 & 124.594 & 129.55 & 130.466 & 137.301 & 133.667 & 136.188 & 136.238 & 144.124 & 136.238 & 93.348 \\ 
   &  & p=996 & -0.514 & -0.512 & -0.269 & 397.997 & -0.51 & -0.508 & -0.508 & 633.828 & -0.508 & 591.61 & 124.594 & 129.55 & 190.882 & 360.624 & 133.667 & 136.184 & 136.238 & 521.693 & 136.238 & 339.768 \\ 
  \midrule\multirow{9}[6]{*}{msnr} & \multirow{3}[2]{*}{$\rho=0$} & p=12 & 3.238 & 3.238 & 3.238 & 3.238 & 3.238 & 3.238 & 3.238 & 3.238 & 3.238 & 3.238 & 12 & 12 & 12 & 12 & 12 & 12 & 12 & 12 & 12 & 12 \\ 
   &  & p=500 & 4.326 & 4.327 & 4.328 & 4.33 & 4.328 & 4.329 & 4.33 & 4.333 & 4.33 & 4.402 & 79.588 & 83.39 & 84.807 & 90.056 & 85.699 & 88.253 & 88.344 & 94.161 & 88.391 & 43.529 \\ 
   &  & p=996 & 4.326 & 4.327 & 4.584 & 392.892 & 4.328 & 4.329 & 4.33 & 628.527 & 4.33 & 492.85 & 79.588 & 83.39 & 156.249 & 323.679 & 85.699 & 88.253 & 88.344 & 476.579 & 88.391 & 214.034 \\ 
  \cmidrule{2-23} & \multirow{3}[2]{*}{$\rho=0.5$} & p=12 & 2.248 & 2.248 & 2.248 & 2.248 & 2.248 & 2.248 & 2.248 & 2.248 & 2.248 & 2.248 & 12 & 12 & 12 & 12 & 12 & 12 & 12 & 12 & 12 & 12 \\ 
   &  & p=500 & 3.256 & 3.257 & 3.257 & 3.26 & 3.257 & 3.259 & 3.259 & 3.263 & 3.259 & 3.331 & 79.447 & 83.023 & 84.602 & 89.743 & 85.164 & 87.865 & 88.055 & 93.597 & 88.06 & 43.16 \\ 
   &  & p=996 & 3.256 & 3.257 & 3.515 & 394.142 & 3.257 & 3.259 & 3.259 & 629.931 & 3.259 & 491.771 & 79.447 & 83.023 & 155.806 & 323.398 & 85.164 & 87.865 & 88.055 & 474.511 & 88.06 & 213.658 \\ 
  \cmidrule{2-23} & \multirow{3}[2]{*}{$\rho=0.9$} & p=12 & 0.755 & 0.755 & 0.755 & 0.755 & 0.755 & 0.755 & 0.755 & 0.755 & 0.755 & 0.755 & 12 & 12 & 12 & 12 & 12 & 12 & 12 & 12 & 12 & 12 \\ 
   &  & p=500 & 1.558 & 1.559 & 1.559 & 1.562 & 1.56 & 1.561 & 1.561 & 1.565 & 1.561 & 1.635 & 76.627 & 79.933 & 80.976 & 85.507 & 81.499 & 83.906 & 83.907 & 89.221 & 83.943 & 39.047 \\ 
   &  & p=996 & 1.558 & 1.559 & 1.818 & 399.137 & 1.56 & 1.561 & 1.561 & 624.227 & 1.561 & 475.861 & 76.627 & 79.933 & 153.247 & 320.689 & 81.499 & 83.906 & 83.907 & 470.493 & 83.943 & 199.693 \\ 
  \midrule\multirow{9}[6]{*}{lsnr} & \multirow{3}[2]{*}{$\rho=0$} & p=12 & 4.847 & 4.847 & 4.847 & 4.847 & 4.847 & 4.847 & 4.847 & 4.847 & 4.847 & 4.85 & 11.979 & 11.98 & 11.98 & 11.981 & 11.974 & 11.98 & 11.981 & 11.981 & 11.981 & 11.772 \\ 
   &  & p=500 & 5.89 & 5.89 & 5.89 & 5.891 & 5.891 & 5.891 & 5.891 & 5.892 & 5.891 & 5.961 & 45.12 & 47.327 & 48.841 & 51.993 & 47.917 & 50.47 & 50.506 & 53.685 & 50.518 & 5.059 \\ 
   &  & p=996 & 5.89 & 5.89 & 6.156 & 394.464 & 5.891 & 5.891 & 5.891 & 617.667 & 5.891 & 409.146 & 45.12 & 47.327 & 126.417 & 295.061 & 47.917 & 50.47 & 50.506 & 431.501 & 50.518 & 123.031 \\ 
  \cmidrule{2-23} & \multirow{3}[2]{*}{$\rho=0.5$} & p=12 & 3.857 & 3.857 & 3.857 & 3.857 & 3.857 & 3.857 & 3.857 & 3.857 & 3.857 & 3.858 & 11.999 & 11.999 & 11.999 & 11.999 & 12 & 11.999 & 11.999 & 11.999 & 11.999 & 11.97 \\ 
   &  & p=500 & 4.819 & 4.82 & 4.82 & 4.821 & 4.82 & 4.821 & 4.821 & 4.822 & 4.821 & 4.889 & 44.555 & 46.877 & 48.349 & 51.646 & 47.355 & 50.037 & 50.016 & 53.263 & 50.027 & 5.123 \\ 
   &  & p=996 & 4.819 & 4.82 & 5.087 & 395.717 & 4.82 & 4.821 & 4.821 & 621.622 & 4.821 & 408.064 & 44.555 & 46.877 & 126.215 & 295.48 & 47.355 & 50.037 & 50.016 & 432.04 & 50.027 & 123.114 \\ 
  \cmidrule{2-23} & \multirow{3}[2]{*}{$\rho=0.9$} & p=12 & 2.364 & 2.364 & 2.364 & 2.364 & 2.364 & 2.364 & 2.364 & 2.364 & 2.364 & 2.365 & 11.998 & 11.998 & 11.998 & 11.998 & 11.998 & 11.998 & 11.998 & 11.998 & 11.998 & 11.947 \\ 
   &  & p=500 & 3.123 & 3.123 & 3.123 & 3.124 & 3.124 & 3.124 & 3.124 & 3.125 & 3.124 & 3.181 & 40.942 & 43.413 & 44.561 & 47.079 & 43.446 & 45.892 & 45.862 & 48.67 & 45.935 & 4.012 \\ 
   &  & p=996 & 3.123 & 3.123 & 3.391 & 400.712 & 3.124 & 3.124 & 3.124 & 612.539 & 3.124 & 401.388 & 40.942 & 43.413 & 123.22 & 292.373 & 43.446 & 45.892 & 45.862 & 424.268 & 45.935 & 119.105 \\ 
   \bottomrule 
\end{tabular}
}
\end{table}


\clearpage
% latex table generated in R 4.0.2 by xtable 1.8-4 package
% Fri Jun 26 10:43:48 2020
\begin{table}[ht]
\centering
\caption{Random X, Sparse-Ex1, n=40} 
\scalebox{0.52}{
\begin{tabular}{|c|c|c|cc|cc|cc|ccc|c||cc|cc|cc|ccc|c|}
  \toprule 
 \multicolumn{1}{|c}{} & \multicolumn{1}{c}{} &       & RAICc & AICc  & RC$_p$ & C$_p$ & 10F CV & LOO CV & S$_p$ & $\widehat{\text{C}}_p$ & GCV   & BIC   & RAICc & AICc  & RC$_p$ & C$_p$ & 10F CV & LOO CV & S$_p$ & $\widehat{\text{C}}_p$ & GCV   & BIC  \\
 \cmidrule{4-23}\multicolumn{1}{|c}{} & \multicolumn{1}{c}{} &       & \multicolumn{10}{c||}{RMSE}                                                   & \multicolumn{10}{c|}{Relative Risk}       \\
 \midrule\multirow{9}[6]{*}{hsnr} & \multirow{3}[2]{*}{$\rho=0$} & p=12 & 0.82 & 0.826 & 0.835 & 0.863 & 0.869 & 0.857 & 0.857 & 0.885 & 0.859 & 0.832 & 0.027 & 0.028 & 0.029 & 0.032 & 0.032 & 0.031 & 0.031 & 0.034 & 0.031 & 0.028 \\ 
   &  & p=20 & 0.82 & 0.828 & 0.848 & 0.916 & 0.9 & 0.887 & 0.882 & 0.987 & 0.885 & 0.84 & 0.027 & 0.028 & 0.031 & 0.041 & 0.037 & 0.036 & 0.035 & 0.052 & 0.035 & 0.029 \\ 
   &  & p=36 & 0.82 & 0.828 & 0.95 & 1.154 & 0.912 & 0.923 & 0.925 & 1.38 & 0.937 & 1.132 & 0.027 & 0.028 & 0.061 & 0.328 & 0.04 & 0.07 & 0.075 & 0.515 & 0.088 & 0.395 \\ 
  \cmidrule{2-23} & \multirow{3}[2]{*}{$\rho=0.5$} & p=12 & 1.27 & 1.279 & 1.288 & 1.318 & 1.319 & 1.309 & 1.304 & 1.339 & 1.306 & 1.296 & 0.028 & 0.029 & 0.03 & 0.033 & 0.032 & 0.032 & 0.032 & 0.034 & 0.032 & 0.03 \\ 
   &  & p=20 & 1.27 & 1.281 & 1.306 & 1.395 & 1.363 & 1.349 & 1.348 & 1.485 & 1.352 & 1.306 & 0.028 & 0.03 & 0.032 & 0.043 & 0.037 & 0.037 & 0.037 & 0.052 & 0.037 & 0.031 \\ 
   &  & p=36 & 1.27 & 1.281 & 1.456 & 1.744 & 1.375 & 1.399 & 1.4 & 2.058 & 1.422 & 1.721 & 0.028 & 0.03 & 0.061 & 0.324 & 0.04 & 0.072 & 0.074 & 0.511 & 0.087 & 0.382 \\ 
  \cmidrule{2-23} & \multirow{3}[2]{*}{$\rho=0.9$} & p=12 & 1.73 & 1.729 & 1.746 & 1.782 & 1.788 & 1.771 & 1.759 & 1.817 & 1.763 & 1.789 & 0.026 & 0.027 & 0.028 & 0.03 & 0.029 & 0.029 & 0.029 & 0.031 & 0.029 & 0.028 \\ 
   &  & p=20 & 1.73 & 1.726 & 1.778 & 1.88 & 1.841 & 1.833 & 1.817 & 1.996 & 1.828 & 1.799 & 0.026 & 0.026 & 0.03 & 0.038 & 0.034 & 0.034 & 0.033 & 0.047 & 0.034 & 0.029 \\ 
   &  & p=36 & 1.727 & 1.726 & 2.005 & 2.41 & 1.855 & 1.892 & 1.9 & 2.8 & 1.929 & 2.344 & 0.026 & 0.027 & 0.059 & 0.323 & 0.037 & 0.068 & 0.077 & 0.504 & 0.091 & 0.35 \\ 
  \midrule\multirow{9}[6]{*}{msnr} & \multirow{3}[2]{*}{$\rho=0$} & p=12 & 2.596 & 2.564 & 2.554 & 2.583 & 2.601 & 2.583 & 2.569 & 2.631 & 2.571 & 2.783 & 0.256 & 0.253 & 0.253 & 0.27 & 0.268 & 0.266 & 0.263 & 0.285 & 0.264 & 0.296 \\ 
   &  & p=20 & 2.596 & 2.566 & 2.596 & 2.717 & 2.669 & 2.648 & 2.625 & 2.888 & 2.637 & 2.797 & 0.256 & 0.253 & 0.271 & 0.336 & 0.299 & 0.297 & 0.29 & 0.426 & 0.296 & 0.302 \\ 
   &  & p=36 & 2.596 & 2.566 & 2.917 & 3.472 & 2.696 & 2.738 & 2.734 & 3.973 & 2.78 & 3.467 & 0.256 & 0.253 & 0.532 & 2.791 & 0.325 & 0.572 & 0.608 & 4.217 & 0.726 & 2.889 \\ 
  \cmidrule{2-23} & \multirow{3}[2]{*}{$\rho=0.5$} & p=12 & 3.688 & 3.672 & 3.658 & 3.736 & 3.739 & 3.725 & 3.696 & 3.787 & 3.692 & 3.962 & 0.227 & 0.23 & 0.232 & 0.253 & 0.248 & 0.25 & 0.243 & 0.266 & 0.243 & 0.263 \\ 
   &  & p=20 & 3.687 & 3.672 & 3.728 & 3.944 & 3.839 & 3.815 & 3.782 & 4.173 & 3.795 & 3.976 & 0.226 & 0.231 & 0.249 & 0.323 & 0.277 & 0.281 & 0.271 & 0.401 & 0.275 & 0.268 \\ 
   &  & p=36 & 3.687 & 3.672 & 4.226 & 5.029 & 3.874 & 3.966 & 3.926 & 5.808 & 3.991 & 4.934 & 0.226 & 0.231 & 0.502 & 2.711 & 0.301 & 0.58 & 0.576 & 4.155 & 0.682 & 2.715 \\ 
  \cmidrule{2-23} & \multirow{3}[2]{*}{$\rho=0.9$} & p=12 & 4.56 & 4.584 & 4.655 & 4.774 & 4.73 & 4.727 & 4.707 & 4.863 & 4.71 & 4.786 & 0.167 & 0.173 & 0.185 & 0.203 & 0.194 & 0.197 & 0.195 & 0.215 & 0.195 & 0.178 \\ 
   &  & p=20 & 4.561 & 4.592 & 4.744 & 5.097 & 4.842 & 4.888 & 4.841 & 5.34 & 4.864 & 4.78 & 0.168 & 0.174 & 0.2 & 0.274 & 0.219 & 0.229 & 0.224 & 0.331 & 0.229 & 0.178 \\ 
   &  & p=36 & 4.558 & 4.589 & 5.449 & 6.647 & 4.893 & 5.055 & 5.045 & 7.736 & 5.136 & 6.154 & 0.167 & 0.173 & 0.453 & 2.642 & 0.241 & 0.505 & 0.537 & 3.994 & 0.686 & 2.521 \\ 
  \midrule\multirow{9}[6]{*}{lsnr} & \multirow{3}[2]{*}{$\rho=0$} & p=12 & 5.119 & 5.14 & 5.168 & 5.263 & 5.24 & 5.207 & 5.178 & 5.326 & 5.196 & 5.471 & 0.901 & 0.938 & 0.963 & 1.045 & 1.007 & 1.009 & 0.988 & 1.097 & 0.997 & 0.975 \\ 
   &  & p=20 & 5.119 & 5.14 & 5.275 & 5.546 & 5.341 & 5.333 & 5.268 & 5.815 & 5.303 & 5.471 & 0.901 & 0.938 & 1.044 & 1.352 & 1.117 & 1.141 & 1.071 & 1.65 & 1.102 & 0.975 \\ 
   &  & p=36 & 5.119 & 5.14 & 5.988 & 7.279 & 5.394 & 5.514 & 5.483 & 8.329 & 5.593 & 6.851 & 0.901 & 0.938 & 2.343 & 13.386 & 1.224 & 2.454 & 2.498 & 19.618 & 2.99 & 11.93 \\ 
  \cmidrule{2-23} & \multirow{3}[2]{*}{$\rho=0.5$} & p=12 & 7.24 & 7.316 & 7.382 & 7.579 & 7.47 & 7.452 & 7.423 & 7.659 & 7.426 & 7.8 & 0.812 & 0.851 & 0.895 & 0.983 & 0.933 & 0.931 & 0.92 & 1.039 & 0.924 & 0.895 \\ 
   &  & p=20 & 7.24 & 7.316 & 7.518 & 8.033 & 7.632 & 7.651 & 7.607 & 8.372 & 7.623 & 7.793 & 0.812 & 0.852 & 0.958 & 1.307 & 1.039 & 1.069 & 1.036 & 1.57 & 1.05 & 0.891 \\ 
   &  & p=36 & 7.24 & 7.316 & 8.663 & 10.514 & 7.735 & 7.955 & 7.925 & 12.27 & 8.067 & 9.874 & 0.812 & 0.852 & 2.24 & 12.931 & 1.176 & 2.453 & 2.498 & 19.762 & 3.011 & 11.733 \\ 
  \cmidrule{2-23} & \multirow{3}[2]{*}{$\rho=0.9$} & p=12 & 8.023 & 8.144 & 8.462 & 8.79 & 8.797 & 8.788 & 8.582 & 9.065 & 8.607 & 8.418 & 0.545 & 0.57 & 0.646 & 0.742 & 0.719 & 0.723 & 0.686 & 0.809 & 0.691 & 0.575 \\ 
   &  & p=20 & 8.079 & 8.232 & 8.59 & 9.556 & 9.002 & 9.059 & 8.849 & 10.136 & 8.899 & 8.455 & 0.553 & 0.588 & 0.705 & 1.084 & 0.817 & 0.845 & 0.8 & 1.325 & 0.821 & 0.583 \\ 
   &  & p=36 & 8.055 & 8.231 & 10.605 & 13.436 & 9.153 & 9.427 & 9.302 & 15.9 & 9.621 & 11.652 & 0.55 & 0.586 & 2.035 & 12.898 & 0.941 & 2.172 & 2.235 & 19.216 & 3.029 & 11.282 \\ 
   \midrule 
 \multicolumn{1}{|c}{} & \multicolumn{1}{c}{} &       & \multicolumn{10}{c||}{Relative Test Error}                                    & \multicolumn{10}{c|}{Proportion of Variance Explained} \\
\midrule\multirow{9}[6]{*}{hsnr} & \multirow{3}[2]{*}{$\rho=0$} & p=12 & 1.233 & 1.239 & 1.248 & 1.273 & 1.273 & 1.264 & 1.267 & 1.29 & 1.268 & 1.243 & 0.871 & 0.871 & 0.87 & 0.868 & 0.867 & 0.868 & 0.868 & 0.866 & 0.868 & 0.87 \\ 
   &  & p=20 & 1.233 & 1.241 & 1.262 & 1.351 & 1.312 & 1.304 & 1.301 & 1.444 & 1.304 & 1.253 & 0.871 & 0.871 & 0.868 & 0.859 & 0.863 & 0.864 & 0.864 & 0.849 & 0.864 & 0.869 \\ 
   &  & p=36 & 1.233 & 1.241 & 1.514 & 3.866 & 1.344 & 1.587 & 1.636 & 5.49 & 1.734 & 4.429 & 0.871 & 0.871 & 0.842 & 0.602 & 0.86 & 0.833 & 0.829 & 0.435 & 0.818 & 0.543 \\ 
  \cmidrule{2-23} & \multirow{3}[2]{*}{$\rho=0.5$} & p=12 & 1.245 & 1.255 & 1.261 & 1.282 & 1.277 & 1.276 & 1.273 & 1.297 & 1.274 & 1.26 & 0.87 & 0.869 & 0.868 & 0.866 & 0.867 & 0.867 & 0.867 & 0.865 & 0.867 & 0.869 \\ 
   &  & p=20 & 1.245 & 1.256 & 1.275 & 1.363 & 1.315 & 1.315 & 1.313 & 1.445 & 1.316 & 1.269 & 0.87 & 0.869 & 0.867 & 0.858 & 0.863 & 0.863 & 0.863 & 0.849 & 0.863 & 0.868 \\ 
   &  & p=36 & 1.245 & 1.256 & 1.519 & 3.822 & 1.344 & 1.601 & 1.629 & 5.44 & 1.739 & 4.309 & 0.87 & 0.869 & 0.841 & 0.606 & 0.86 & 0.832 & 0.829 & 0.439 & 0.818 & 0.554 \\ 
  \cmidrule{2-23} & \multirow{3}[2]{*}{$\rho=0.9$} & p=12 & 1.226 & 1.23 & 1.239 & 1.257 & 1.252 & 1.25 & 1.247 & 1.27 & 1.249 & 1.244 & 0.872 & 0.871 & 0.871 & 0.869 & 0.869 & 0.869 & 0.87 & 0.867 & 0.87 & 0.87 \\ 
   &  & p=20 & 1.227 & 1.229 & 1.255 & 1.327 & 1.287 & 1.288 & 1.286 & 1.4 & 1.292 & 1.25 & 0.872 & 0.872 & 0.869 & 0.861 & 0.865 & 0.865 & 0.866 & 0.853 & 0.865 & 0.87 \\ 
   &  & p=36 & 1.225 & 1.229 & 1.498 & 3.76 & 1.317 & 1.559 & 1.631 & 5.327 & 1.747 & 3.985 & 0.872 & 0.872 & 0.843 & 0.607 & 0.863 & 0.835 & 0.827 & 0.446 & 0.815 & 0.583 \\ 
  \midrule\multirow{9}[6]{*}{msnr} & \multirow{3}[2]{*}{$\rho=0$} & p=12 & 1.262 & 1.258 & 1.258 & 1.276 & 1.273 & 1.271 & 1.269 & 1.292 & 1.27 & 1.303 & 0.379 & 0.381 & 0.381 & 0.372 & 0.374 & 0.375 & 0.376 & 0.365 & 0.375 & 0.359 \\ 
   &  & p=20 & 1.262 & 1.259 & 1.277 & 1.34 & 1.302 & 1.302 & 1.296 & 1.432 & 1.301 & 1.309 & 0.379 & 0.381 & 0.372 & 0.34 & 0.359 & 0.359 & 0.362 & 0.295 & 0.36 & 0.356 \\ 
   &  & p=36 & 1.262 & 1.259 & 1.533 & 3.872 & 1.33 & 1.566 & 1.61 & 5.326 & 1.719 & 3.961 & 0.379 & 0.381 & 0.244 & -0.894 & 0.345 & 0.226 & 0.206 & -1.606 & 0.15 & -0.941 \\ 
  \cmidrule{2-23} & \multirow{3}[2]{*}{$\rho=0.5$} & p=12 & 1.231 & 1.235 & 1.237 & 1.258 & 1.252 & 1.255 & 1.248 & 1.272 & 1.247 & 1.269 & 0.394 & 0.392 & 0.391 & 0.38 & 0.383 & 0.382 & 0.386 & 0.374 & 0.386 & 0.375 \\ 
   &  & p=20 & 1.231 & 1.235 & 1.255 & 1.326 & 1.28 & 1.284 & 1.275 & 1.405 & 1.279 & 1.274 & 0.394 & 0.392 & 0.382 & 0.346 & 0.369 & 0.367 & 0.372 & 0.307 & 0.37 & 0.373 \\ 
   &  & p=36 & 1.231 & 1.235 & 1.506 & 3.771 & 1.308 & 1.571 & 1.576 & 5.248 & 1.683 & 3.776 & 0.394 & 0.392 & 0.258 & -0.851 & 0.356 & 0.222 & 0.222 & -1.573 & 0.169 & -0.853 \\ 
  \cmidrule{2-23} & \multirow{3}[2]{*}{$\rho=0.9$} & p=12 & 1.17 & 1.177 & 1.189 & 1.206 & 1.197 & 1.2 & 1.198 & 1.219 & 1.198 & 1.183 & 0.423 & 0.42 & 0.414 & 0.405 & 0.409 & 0.408 & 0.409 & 0.399 & 0.409 & 0.417 \\ 
   &  & p=20 & 1.172 & 1.178 & 1.205 & 1.277 & 1.222 & 1.232 & 1.228 & 1.334 & 1.233 & 1.183 & 0.423 & 0.419 & 0.406 & 0.37 & 0.397 & 0.392 & 0.395 & 0.342 & 0.392 & 0.417 \\ 
   &  & p=36 & 1.171 & 1.178 & 1.453 & 3.66 & 1.249 & 1.492 & 1.529 & 5.022 & 1.663 & 3.526 & 0.423 & 0.419 & 0.282 & -0.808 & 0.385 & 0.259 & 0.242 & -1.478 & 0.172 & -0.745 \\ 
  \midrule\multirow{9}[6]{*}{lsnr} & \multirow{3}[2]{*}{$\rho=0$} & p=12 & 1.187 & 1.195 & 1.2 & 1.216 & 1.209 & 1.21 & 1.205 & 1.227 & 1.207 & 1.204 & 0.019 & 0.013 & 0.009 & -0.005 & 0.001 & 0.001 & 0.005 & -0.014 & 0.003 & 0.005 \\ 
   &  & p=20 & 1.187 & 1.195 & 1.217 & 1.277 & 1.23 & 1.235 & 1.222 & 1.336 & 1.228 & 1.204 & 0.019 & 0.013 & -0.005 & -0.055 & -0.016 & -0.021 & -0.01 & -0.104 & -0.014 & 0.005 \\ 
   &  & p=36 & 1.187 & 1.195 & 1.471 & 3.754 & 1.254 & 1.488 & 1.505 & 5.009 & 1.6 & 3.451 & 0.019 & 0.013 & -0.216 & -2.098 & -0.036 & -0.231 & -0.244 & -3.136 & -0.323 & -1.849 \\ 
  \cmidrule{2-23} & \multirow{3}[2]{*}{$\rho=0.5$} & p=12 & 1.168 & 1.176 & 1.185 & 1.202 & 1.192 & 1.193 & 1.19 & 1.213 & 1.19 & 1.187 & 0.035 & 0.028 & 0.021 & 0.007 & 0.014 & 0.014 & 0.017 & -0.003 & 0.016 & 0.019 \\ 
   &  & p=20 & 1.168 & 1.176 & 1.198 & 1.265 & 1.213 & 1.22 & 1.213 & 1.32 & 1.216 & 1.186 & 0.035 & 0.028 & 0.01 & -0.046 & -0.003 & -0.008 & -0.003 & -0.091 & -0.005 & 0.02 \\ 
   &  & p=36 & 1.168 & 1.176 & 1.452 & 3.644 & 1.244 & 1.487 & 1.501 & 5.041 & 1.604 & 3.404 & 0.035 & 0.028 & -0.201 & -2.01 & -0.028 & -0.231 & -0.242 & -3.163 & -0.327 & -1.812 \\ 
  \cmidrule{2-23} & \multirow{3}[2]{*}{$\rho=0.9$} & p=12 & 1.112 & 1.117 & 1.133 & 1.151 & 1.147 & 1.148 & 1.14 & 1.164 & 1.141 & 1.119 & 0.081 & 0.076 & 0.063 & 0.048 & 0.051 & 0.051 & 0.057 & 0.037 & 0.056 & 0.075 \\ 
   &  & p=20 & 1.114 & 1.121 & 1.144 & 1.219 & 1.166 & 1.172 & 1.163 & 1.266 & 1.168 & 1.121 & 0.079 & 0.073 & 0.054 & -0.008 & 0.036 & 0.031 & 0.038 & -0.048 & 0.034 & 0.073 \\ 
   &  & p=36 & 1.113 & 1.121 & 1.406 & 3.594 & 1.195 & 1.423 & 1.439 & 4.861 & 1.584 & 3.248 & 0.08 & 0.073 & -0.164 & -1.977 & 0.012 & -0.18 & -0.192 & -3.025 & -0.315 & -1.693 \\ 
   \midrule 
 \multicolumn{1}{|c}{} & \multicolumn{1}{c}{} &       & \multicolumn{10}{c||}{Kullback-Leibler Discrepancy}                                    & \multicolumn{10}{c|}{Number of Variables} \\
\midrule\multirow{9}[6]{*}{hsnr} & \multirow{3}[2]{*}{$\rho=0$} & p=12 & 2.741 & 2.763 & 2.788 & 2.845 & 2.835 & 2.828 & 2.833 & 2.891 & 2.836 & 2.759 & 6.007 & 6.144 & 6.271 & 6.614 & 6.625 & 6.555 & 6.545 & 6.954 & 6.564 & 6.041 \\ 
   &  & p=20 & 2.741 & 2.77 & 2.845 & 3.193 & 2.992 & 3.007 & 2.996 & 3.553 & 3.01 & 2.811 & 6.007 & 6.16 & 6.41 & 7.435 & 7.046 & 6.972 & 6.89 & 8.661 & 6.932 & 6.136 \\ 
   &  & p=36 & 2.741 & 2.77 & 5.3 & 132.746 & 4.028 & 47.066 & 54.433 & 168.88 & 62.386 & 161.072 & 6.007 & 6.16 & 8.308 & 13.578 & 7.297 & 7.83 & 7.904 & 19.221 & 8.198 & 14.005 \\ 
  \cmidrule{2-23} & \multirow{3}[2]{*}{$\rho=0.5$} & p=12 & 3.545 & 3.579 & 3.6 & 3.652 & 3.635 & 3.641 & 3.637 & 3.692 & 3.639 & 3.574 & 5.546 & 5.75 & 5.891 & 6.29 & 6.262 & 6.256 & 6.21 & 6.627 & 6.233 & 5.597 \\ 
   &  & p=20 & 3.545 & 3.586 & 3.655 & 4.014 & 3.795 & 3.827 & 3.821 & 4.332 & 3.838 & 3.63 & 5.546 & 5.762 & 6.031 & 7.113 & 6.671 & 6.653 & 6.611 & 8.294 & 6.676 & 5.687 \\ 
   &  & p=36 & 3.545 & 3.586 & 5.84 & 132.123 & 4.898 & 48.611 & 53.322 & 169.602 & 61.467 & 159.191 & 5.546 & 5.762 & 7.99 & 13.14 & 6.901 & 7.477 & 7.533 & 18.844 & 7.869 & 13.441 \\ 
  \cmidrule{2-23} & \multirow{3}[2]{*}{$\rho=0.9$} & p=12 & 4.194 & 4.216 & 4.244 & 4.292 & 4.277 & 4.274 & 4.271 & 4.328 & 4.275 & 4.211 & 4.479 & 4.722 & 4.922 & 5.336 & 5.281 & 5.244 & 5.216 & 5.682 & 5.251 & 4.334 \\ 
   &  & p=20 & 4.196 & 4.215 & 4.308 & 4.624 & 4.435 & 4.44 & 4.461 & 4.905 & 4.482 & 4.242 & 4.482 & 4.692 & 5.116 & 6.08 & 5.667 & 5.626 & 5.617 & 7.196 & 5.692 & 4.38 \\ 
   &  & p=36 & 4.195 & 4.217 & 6.548 & 129.469 & 7.318 & 48.141 & 54.996 & 166.869 & 64.994 & 152.924 & 4.488 & 4.699 & 7.079 & 12.372 & 5.883 & 6.352 & 6.577 & 17.845 & 6.933 & 11.699 \\ 
  \midrule\multirow{9}[6]{*}{msnr} & \multirow{3}[2]{*}{$\rho=0$} & p=12 & 4.878 & 4.894 & 4.912 & 4.966 & 4.943 & 4.95 & 4.949 & 5.011 & 4.952 & 4.913 & 4.293 & 4.577 & 4.829 & 5.326 & 5.153 & 5.221 & 5.176 & 5.739 & 5.216 & 3.971 \\ 
   &  & p=20 & 4.878 & 4.898 & 4.978 & 5.27 & 5.07 & 5.099 & 5.092 & 5.623 & 5.11 & 4.953 & 4.293 & 4.584 & 5.054 & 6.03 & 5.523 & 5.587 & 5.496 & 7.404 & 5.582 & 4.044 \\ 
   &  & p=36 & 4.878 & 4.898 & 7.426 & 134.786 & 6.061 & 48.783 & 56.268 & 169.214 & 63.366 & 152.886 & 4.293 & 4.584 & 6.981 & 12.379 & 5.732 & 6.364 & 6.455 & 17.998 & 6.842 & 11.374 \\ 
  \cmidrule{2-23} & \multirow{3}[2]{*}{$\rho=0.5$} & p=12 & 5.641 & 5.667 & 5.686 & 5.741 & 5.717 & 5.728 & 5.719 & 5.778 & 5.721 & 5.672 & 3.827 & 4.11 & 4.363 & 4.849 & 4.713 & 4.746 & 4.669 & 5.252 & 4.714 & 3.492 \\ 
   &  & p=20 & 5.641 & 5.668 & 5.748 & 6.049 & 5.846 & 5.877 & 5.87 & 6.363 & 5.892 & 5.702 & 3.833 & 4.11 & 4.563 & 5.591 & 5.075 & 5.078 & 4.999 & 6.857 & 5.099 & 3.546 \\ 
   &  & p=36 & 5.641 & 5.668 & 7.935 & 133.866 & 6.924 & 50.644 & 54.84 & 169.426 & 63.046 & 149.514 & 3.833 & 4.11 & 6.605 & 11.851 & 5.273 & 5.955 & 5.861 & 17.539 & 6.277 & 10.399 \\ 
  \cmidrule{2-23} & \multirow{3}[2]{*}{$\rho=0.9$} & p=12 & 6.245 & 6.273 & 6.306 & 6.351 & 6.323 & 6.331 & 6.329 & 6.382 & 6.331 & 6.252 & 2.527 & 2.774 & 3.071 & 3.467 & 3.362 & 3.345 & 3.304 & 3.805 & 3.325 & 2.115 \\ 
   &  & p=20 & 6.248 & 6.276 & 6.363 & 6.668 & 6.444 & 6.469 & 6.476 & 6.892 & 6.499 & 6.252 & 2.549 & 2.792 & 3.261 & 4.301 & 3.679 & 3.748 & 3.661 & 5.223 & 3.759 & 2.095 \\ 
   &  & p=36 & 6.248 & 6.276 & 8.617 & 130.727 & 9.279 & 50.23 & 55.323 & 165.097 & 66.126 & 145.663 & 2.539 & 2.796 & 5.51 & 10.741 & 3.898 & 4.465 & 4.556 & 15.98 & 4.918 & 8.557 \\ 
  \midrule\multirow{9}[6]{*}{lsnr} & \multirow{3}[2]{*}{$\rho=0$} & p=12 & 6.385 & 6.41 & 6.433 & 6.477 & 6.452 & 6.455 & 6.45 & 6.505 & 6.456 & 6.386 & 1.995 & 2.287 & 2.626 & 3.111 & 2.936 & 2.902 & 2.851 & 3.426 & 2.918 & 1.174 \\ 
   &  & p=20 & 6.385 & 6.41 & 6.494 & 6.753 & 6.542 & 6.578 & 6.552 & 6.999 & 6.57 & 6.386 & 1.995 & 2.287 & 2.914 & 3.869 & 3.222 & 3.236 & 3.082 & 4.856 & 3.186 & 1.174 \\ 
   &  & p=36 & 6.385 & 6.41 & 8.949 & 135.519 & 7.481 & 49.669 & 53.946 & 166.282 & 63.703 & 142.936 & 1.995 & 2.287 & 5.171 & 10.509 & 3.425 & 3.966 & 3.96 & 15.751 & 4.381 & 7.596 \\ 
  \cmidrule{2-23} & \multirow{3}[2]{*}{$\rho=0.5$} & p=12 & 7.165 & 7.186 & 7.218 & 7.262 & 7.235 & 7.24 & 7.234 & 7.293 & 7.236 & 7.167 & 1.927 & 2.154 & 2.51 & 2.932 & 2.744 & 2.716 & 2.684 & 3.237 & 2.731 & 1.16 \\ 
   &  & p=20 & 7.165 & 7.187 & 7.269 & 7.549 & 7.333 & 7.367 & 7.35 & 7.769 & 7.371 & 7.163 & 1.927 & 2.155 & 2.726 & 3.72 & 3.035 & 3.033 & 2.974 & 4.599 & 3.068 & 1.15 \\ 
   &  & p=36 & 7.165 & 7.187 & 9.466 & 134.173 & 8.434 & 50.408 & 54.602 & 168.59 & 63.814 & 142.66 & 1.927 & 2.155 & 5.029 & 10.23 & 3.307 & 3.828 & 3.845 & 15.698 & 4.269 & 7.434 \\ 
  \cmidrule{2-23} & \multirow{3}[2]{*}{$\rho=0.9$} & p=12 & 7.783 & 7.8 & 7.841 & 7.884 & 7.866 & 7.874 & 7.859 & 7.914 & 7.861 & 7.776 & 1.521 & 1.658 & 1.966 & 2.306 & 2.237 & 2.239 & 2.11 & 2.568 & 2.145 & 1.052 \\ 
   &  & p=20 & 7.786 & 7.808 & 7.891 & 8.181 & 7.957 & 7.983 & 7.977 & 8.362 & 7.994 & 7.779 & 1.524 & 1.664 & 2.099 & 3.108 & 2.473 & 2.517 & 2.369 & 3.836 & 2.452 & 1.067 \\ 
   &  & p=36 & 7.785 & 7.807 & 10.168 & 132.026 & 10.798 & 51.4 & 55.507 & 164.981 & 67.322 & 141.038 & 1.509 & 1.663 & 4.676 & 9.972 & 2.733 & 3.203 & 3.215 & 14.728 & 3.682 & 7.034 \\ 
   \bottomrule 
\end{tabular}
}
\end{table}

% latex table generated in R 4.0.2 by xtable 1.8-4 package
% Fri Jun 26 10:43:44 2020
\begin{table}[ht]
\centering
\caption{Fixed X, Sparse-Ex1, n=200} 
\scalebox{0.52}{
\begin{tabular}{|c|c|c|cc|cc|cc|ccc|c||cc|cc|cc|ccc|c|}
  \toprule 
 \multicolumn{1}{|c}{} & \multicolumn{1}{c}{} &       & RAICc & AICc  & RC$_p$ & C$_p$ & 10F CV & LOO CV & S$_p$ & $\widehat{\text{C}}_p$ & GCV   & BIC   & RAICc & AICc  & RC$_p$ & C$_p$ & 10F CV & LOO CV & S$_p$ & $\widehat{\text{C}}_p$ & GCV   & BIC  \\
 \cmidrule{4-23}\multicolumn{1}{|c}{} & \multicolumn{1}{c}{} &       & \multicolumn{10}{c||}{RMSE}                                                   & \multicolumn{10}{c|}{Relative Risk}       \\
 \midrule\multirow{9}[6]{*}{hsnr} & \multirow{3}[2]{*}{$\rho=0$} & p=12 & 0.382 & 0.385 & 0.386 & 0.389 & 0.388 & 0.389 & 0.388 & 0.392 & 0.388 & 0.348 & 0.005 & 0.005 & 0.005 & 0.005 & 0.005 & 0.005 & 0.005 & 0.005 & 0.005 & 0.004 \\ 
   &  & p=100 & 0.387 & 0.392 & 0.395 & 0.405 & 0.402 & 0.403 & 0.401 & 0.412 & 0.401 & 0.348 & 0.005 & 0.005 & 0.006 & 0.006 & 0.006 & 0.006 & 0.006 & 0.006 & 0.006 & 0.004 \\ 
   &  & p=196 & 0.387 & 0.392 & 0.646 & 0.901 & 0.402 & 0.403 & 0.401 & 1.095 & 0.403 & 0.613 & 0.005 & 0.005 & 0.04 & 1.733 & 0.006 & 0.006 & 0.006 & 2.918 & 0.009 & 1.388 \\ 
  \cmidrule{2-23} & \multirow{3}[2]{*}{$\rho=0.5$} & p=12 & 0.566 & 0.571 & 0.573 & 0.576 & 0.576 & 0.575 & 0.576 & 0.58 & 0.575 & 0.533 & 0.005 & 0.005 & 0.005 & 0.005 & 0.005 & 0.005 & 0.005 & 0.005 & 0.005 & 0.005 \\ 
   &  & p=100 & 0.575 & 0.581 & 0.587 & 0.603 & 0.599 & 0.596 & 0.592 & 0.613 & 0.592 & 0.533 & 0.005 & 0.005 & 0.006 & 0.006 & 0.006 & 0.006 & 0.006 & 0.006 & 0.006 & 0.005 \\ 
   &  & p=196 & 0.575 & 0.581 & 0.962 & 1.339 & 0.599 & 0.596 & 0.592 & 1.616 & 0.595 & 0.924 & 0.005 & 0.005 & 0.039 & 1.742 & 0.006 & 0.006 & 0.006 & 2.853 & 0.008 & 1.373 \\ 
  \cmidrule{2-23} & \multirow{3}[2]{*}{$\rho=0.9$} & p=12 & 0.803 & 0.805 & 0.807 & 0.809 & 0.81 & 0.811 & 0.809 & 0.813 & 0.809 & 0.82 & 0.005 & 0.005 & 0.005 & 0.005 & 0.005 & 0.005 & 0.005 & 0.005 & 0.005 & 0.005 \\ 
   &  & p=100 & 0.811 & 0.823 & 0.828 & 0.85 & 0.838 & 0.842 & 0.836 & 0.859 & 0.837 & 0.826 & 0.005 & 0.005 & 0.005 & 0.006 & 0.006 & 0.006 & 0.006 & 0.006 & 0.006 & 0.005 \\ 
   &  & p=196 & 0.811 & 0.823 & 1.354 & 1.873 & 0.838 & 0.841 & 0.836 & 2.222 & 0.844 & 1.348 & 0.005 & 0.005 & 0.038 & 1.681 & 0.006 & 0.006 & 0.006 & 2.765 & 0.012 & 1.313 \\ 
  \midrule\multirow{9}[6]{*}{msnr} & \multirow{3}[2]{*}{$\rho=0$} & p=12 & 1.148 & 1.155 & 1.159 & 1.166 & 1.168 & 1.166 & 1.164 & 1.173 & 1.164 & 1.2 & 0.046 & 0.046 & 0.047 & 0.047 & 0.047 & 0.047 & 0.047 & 0.048 & 0.047 & 0.049 \\ 
   &  & p=100 & 1.161 & 1.176 & 1.185 & 1.211 & 1.206 & 1.205 & 1.199 & 1.228 & 1.199 & 1.2 & 0.047 & 0.049 & 0.05 & 0.053 & 0.052 & 0.052 & 0.052 & 0.055 & 0.052 & 0.049 \\ 
   &  & p=196 & 1.161 & 1.176 & 1.927 & 2.67 & 1.206 & 1.205 & 1.199 & 3.21 & 1.205 & 1.939 & 0.047 & 0.049 & 0.341 & 14.731 & 0.052 & 0.052 & 0.052 & 24.807 & 0.076 & 11.72 \\ 
  \cmidrule{2-23} & \multirow{3}[2]{*}{$\rho=0.5$} & p=12 & 1.703 & 1.715 & 1.717 & 1.726 & 1.722 & 1.73 & 1.724 & 1.736 & 1.724 & 1.823 & 0.045 & 0.045 & 0.046 & 0.046 & 0.046 & 0.046 & 0.046 & 0.047 & 0.046 & 0.051 \\ 
   &  & p=100 & 1.729 & 1.746 & 1.757 & 1.794 & 1.779 & 1.786 & 1.769 & 1.821 & 1.77 & 1.823 & 0.047 & 0.048 & 0.049 & 0.052 & 0.051 & 0.051 & 0.05 & 0.055 & 0.05 & 0.051 \\ 
   &  & p=196 & 1.729 & 1.746 & 2.857 & 3.944 & 1.779 & 1.786 & 1.769 & 4.696 & 1.778 & 2.903 & 0.047 & 0.048 & 0.336 & 14.807 & 0.051 & 0.051 & 0.05 & 24.05 & 0.073 & 11.467 \\ 
  \cmidrule{2-23} & \multirow{3}[2]{*}{$\rho=0.9$} & p=12 & 2.269 & 2.281 & 2.288 & 2.3 & 2.298 & 2.294 & 2.292 & 2.303 & 2.292 & 2.465 & 0.04 & 0.041 & 0.041 & 0.041 & 0.041 & 0.041 & 0.041 & 0.041 & 0.041 & 0.046 \\ 
   &  & p=100 & 2.295 & 2.324 & 2.341 & 2.372 & 2.366 & 2.368 & 2.353 & 2.394 & 2.353 & 2.488 & 0.041 & 0.043 & 0.044 & 0.046 & 0.045 & 0.045 & 0.044 & 0.047 & 0.044 & 0.047 \\ 
   &  & p=196 & 2.295 & 2.324 & 3.919 & 5.413 & 2.366 & 2.368 & 2.353 & 6.354 & 2.373 & 3.964 & 0.041 & 0.043 & 0.322 & 14.286 & 0.045 & 0.045 & 0.044 & 23.324 & 0.096 & 11.043 \\ 
  \midrule\multirow{9}[6]{*}{lsnr} & \multirow{3}[2]{*}{$\rho=0$} & p=12 & 2.638 & 2.655 & 2.662 & 2.672 & 2.678 & 2.667 & 2.666 & 2.678 & 2.666 & 3.192 & 0.233 & 0.237 & 0.238 & 0.241 & 0.24 & 0.239 & 0.239 & 0.242 & 0.24 & 0.321 \\ 
   &  & p=100 & 2.656 & 2.691 & 2.721 & 2.765 & 2.736 & 2.73 & 2.735 & 2.782 & 2.736 & 3.192 & 0.239 & 0.247 & 0.255 & 0.27 & 0.257 & 0.258 & 0.26 & 0.275 & 0.261 & 0.321 \\ 
   &  & p=196 & 2.656 & 2.691 & 4.375 & 6.002 & 2.736 & 2.73 & 2.735 & 7.126 & 2.749 & 4.724 & 0.239 & 0.247 & 1.709 & 73.369 & 0.257 & 0.258 & 0.26 & 122.666 & 0.379 & 56.446 \\ 
  \cmidrule{2-23} & \multirow{3}[2]{*}{$\rho=0.5$} & p=12 & 3.731 & 3.747 & 3.759 & 3.769 & 3.789 & 3.782 & 3.764 & 3.793 & 3.767 & 4.426 & 0.211 & 0.213 & 0.214 & 0.216 & 0.217 & 0.217 & 0.215 & 0.219 & 0.216 & 0.281 \\ 
   &  & p=100 & 3.771 & 3.801 & 3.843 & 3.901 & 3.88 & 3.871 & 3.845 & 3.938 & 3.853 & 4.426 & 0.217 & 0.223 & 0.23 & 0.244 & 0.235 & 0.235 & 0.231 & 0.251 & 0.234 & 0.281 \\ 
   &  & p=196 & 3.771 & 3.801 & 6.361 & 8.781 & 3.88 & 3.871 & 3.845 & 10.354 & 3.871 & 6.73 & 0.217 & 0.223 & 1.67 & 74.005 & 0.235 & 0.235 & 0.231 & 119.624 & 0.347 & 55.333 \\ 
  \cmidrule{2-23} & \multirow{3}[2]{*}{$\rho=0.9$} & p=12 & 4.699 & 4.737 & 4.759 & 4.773 & 4.8 & 4.746 & 4.762 & 4.789 & 4.765 & 5.11 & 0.167 & 0.17 & 0.172 & 0.173 & 0.174 & 0.171 & 0.172 & 0.175 & 0.172 & 0.178 \\ 
   &  & p=100 & 4.761 & 4.809 & 4.897 & 4.957 & 4.953 & 4.881 & 4.89 & 5.026 & 4.89 & 5.133 & 0.173 & 0.177 & 0.187 & 0.195 & 0.193 & 0.185 & 0.186 & 0.205 & 0.186 & 0.178 \\ 
   &  & p=196 & 4.761 & 4.809 & 8.476 & 11.826 & 4.953 & 4.881 & 4.89 & 13.933 & 4.936 & 8.301 & 0.173 & 0.177 & 1.58 & 71.402 & 0.193 & 0.185 & 0.186 & 116.06 & 0.444 & 51.014 \\ 
   \midrule 
 \multicolumn{1}{|c}{} & \multicolumn{1}{c}{} &       & \multicolumn{10}{c||}{Relative Test Error}                                    & \multicolumn{10}{c|}{Proportion of Variance Explained} \\
\midrule\multirow{9}[6]{*}{hsnr} & \multirow{3}[2]{*}{$\rho=0$} & p=12 & 1.041 & 1.042 & 1.042 & 1.043 & 1.043 & 1.043 & 1.043 & 1.044 & 1.043 & 1.034 & 0.886 & 0.886 & 0.886 & 0.886 & 0.886 & 0.886 & 0.886 & 0.886 & 0.886 & 0.887 \\ 
   &  & p=100 & 1.043 & 1.044 & 1.046 & 1.049 & 1.048 & 1.048 & 1.048 & 1.051 & 1.048 & 1.034 & 0.886 & 0.886 & 0.885 & 0.885 & 0.885 & 0.885 & 0.885 & 0.885 & 0.885 & 0.887 \\ 
   &  & p=196 & 1.043 & 1.044 & 1.323 & 15.091 & 1.048 & 1.048 & 1.048 & 24.733 & 1.07 & 12.289 & 0.886 & 0.886 & 0.855 & -0.653 & 0.885 & 0.885 & 0.885 & -1.708 & 0.883 & -0.346 \\ 
  \cmidrule{2-23} & \multirow{3}[2]{*}{$\rho=0.5$} & p=12 & 1.041 & 1.042 & 1.042 & 1.043 & 1.043 & 1.043 & 1.043 & 1.044 & 1.043 & 1.037 & 0.887 & 0.887 & 0.887 & 0.887 & 0.887 & 0.887 & 0.887 & 0.887 & 0.887 & 0.888 \\ 
   &  & p=100 & 1.043 & 1.044 & 1.046 & 1.05 & 1.048 & 1.048 & 1.047 & 1.052 & 1.047 & 1.037 & 0.887 & 0.887 & 0.887 & 0.886 & 0.886 & 0.887 & 0.887 & 0.886 & 0.887 & 0.888 \\ 
   &  & p=196 & 1.043 & 1.044 & 1.323 & 15.346 & 1.048 & 1.048 & 1.047 & 24.5 & 1.069 & 12.307 & 0.887 & 0.887 & 0.857 & -0.662 & 0.886 & 0.887 & 0.887 & -1.653 & 0.884 & -0.333 \\ 
  \cmidrule{2-23} & \multirow{3}[2]{*}{$\rho=0.9$} & p=12 & 1.043 & 1.043 & 1.044 & 1.044 & 1.044 & 1.044 & 1.044 & 1.044 & 1.044 & 1.045 & 0.891 & 0.891 & 0.891 & 0.891 & 0.891 & 0.891 & 0.891 & 0.891 & 0.891 & 0.891 \\ 
   &  & p=100 & 1.044 & 1.046 & 1.046 & 1.05 & 1.048 & 1.049 & 1.048 & 1.052 & 1.048 & 1.046 & 0.89 & 0.889 & 0.889 & 0.889 & 0.889 & 0.889 & 0.889 & 0.889 & 0.889 & 0.889 \\ 
   &  & p=196 & 1.044 & 1.046 & 1.322 & 15.214 & 1.048 & 1.048 & 1.048 & 24.374 & 1.099 & 12.103 & 0.89 & 0.889 & 0.86 & -0.609 & 0.889 & 0.889 & 0.889 & -1.578 & 0.884 & -0.28 \\ 
  \midrule\multirow{9}[6]{*}{msnr} & \multirow{3}[2]{*}{$\rho=0$} & p=12 & 1.044 & 1.044 & 1.045 & 1.045 & 1.045 & 1.045 & 1.045 & 1.046 & 1.045 & 1.047 & 0.467 & 0.466 & 0.466 & 0.466 & 0.466 & 0.466 & 0.466 & 0.466 & 0.466 & 0.465 \\ 
   &  & p=100 & 1.045 & 1.047 & 1.048 & 1.051 & 1.05 & 1.05 & 1.05 & 1.053 & 1.05 & 1.047 & 0.466 & 0.465 & 0.464 & 0.463 & 0.464 & 0.463 & 0.464 & 0.462 & 0.464 & 0.465 \\ 
   &  & p=196 & 1.045 & 1.047 & 1.326 & 15.094 & 1.05 & 1.05 & 1.05 & 24.734 & 1.072 & 12.213 & 0.466 & 0.465 & 0.322 & -6.714 & 0.464 & 0.463 & 0.464 & -11.64 & 0.452 & -5.242 \\ 
  \cmidrule{2-23} & \multirow{3}[2]{*}{$\rho=0.5$} & p=12 & 1.043 & 1.044 & 1.044 & 1.045 & 1.044 & 1.045 & 1.045 & 1.045 & 1.045 & 1.049 & 0.47 & 0.47 & 0.47 & 0.469 & 0.47 & 0.469 & 0.469 & 0.469 & 0.469 & 0.467 \\ 
   &  & p=100 & 1.045 & 1.046 & 1.047 & 1.051 & 1.049 & 1.05 & 1.049 & 1.053 & 1.049 & 1.049 & 0.469 & 0.468 & 0.468 & 0.466 & 0.467 & 0.467 & 0.467 & 0.465 & 0.467 & 0.467 \\ 
   &  & p=196 & 1.045 & 1.046 & 1.325 & 15.347 & 1.049 & 1.05 & 1.049 & 24.302 & 1.07 & 12.111 & 0.469 & 0.468 & 0.327 & -6.795 & 0.467 & 0.467 & 0.467 & -11.343 & 0.456 & -5.151 \\ 
  \cmidrule{2-23} & \multirow{3}[2]{*}{$\rho=0.9$} & p=12 & 1.04 & 1.041 & 1.041 & 1.042 & 1.041 & 1.041 & 1.041 & 1.042 & 1.041 & 1.046 & 0.482 & 0.482 & 0.482 & 0.482 & 0.482 & 0.482 & 0.482 & 0.482 & 0.482 & 0.479 \\ 
   &  & p=100 & 1.041 & 1.042 & 1.043 & 1.046 & 1.045 & 1.045 & 1.044 & 1.047 & 1.044 & 1.047 & 0.478 & 0.477 & 0.477 & 0.476 & 0.476 & 0.476 & 0.477 & 0.475 & 0.477 & 0.475 \\ 
   &  & p=196 & 1.041 & 1.042 & 1.32 & 15.208 & 1.045 & 1.045 & 1.044 & 24.196 & 1.095 & 11.983 & 0.478 & 0.477 & 0.338 & -6.625 & 0.476 & 0.476 & 0.477 & -11.131 & 0.451 & -5.008 \\ 
  \midrule\multirow{9}[6]{*}{lsnr} & \multirow{3}[2]{*}{$\rho=0$} & p=12 & 1.045 & 1.045 & 1.046 & 1.046 & 1.046 & 1.046 & 1.046 & 1.046 & 1.046 & 1.061 & 0.123 & 0.123 & 0.122 & 0.122 & 0.122 & 0.122 & 0.122 & 0.122 & 0.122 & 0.109 \\ 
   &  & p=100 & 1.046 & 1.047 & 1.049 & 1.052 & 1.049 & 1.049 & 1.05 & 1.053 & 1.05 & 1.061 & 0.122 & 0.121 & 0.12 & 0.117 & 0.119 & 0.119 & 0.119 & 0.116 & 0.119 & 0.109 \\ 
   &  & p=196 & 1.046 & 1.047 & 1.327 & 15.039 & 1.049 & 1.049 & 1.05 & 24.472 & 1.073 & 11.801 & 0.122 & 0.121 & -0.114 & -11.624 & 0.119 & 0.119 & 0.119 & -19.541 & 0.1 & -8.905 \\ 
  \cmidrule{2-23} & \multirow{3}[2]{*}{$\rho=0.5$} & p=12 & 1.041 & 1.041 & 1.042 & 1.042 & 1.042 & 1.042 & 1.042 & 1.042 & 1.042 & 1.054 & 0.128 & 0.128 & 0.128 & 0.127 & 0.127 & 0.127 & 0.127 & 0.127 & 0.127 & 0.117 \\ 
   &  & p=100 & 1.042 & 1.043 & 1.045 & 1.047 & 1.046 & 1.046 & 1.045 & 1.049 & 1.045 & 1.054 & 0.127 & 0.126 & 0.125 & 0.123 & 0.124 & 0.124 & 0.125 & 0.122 & 0.124 & 0.117 \\ 
   &  & p=196 & 1.042 & 1.043 & 1.324 & 15.341 & 1.046 & 1.046 & 1.045 & 24.181 & 1.067 & 11.723 & 0.127 & 0.126 & -0.109 & -11.851 & 0.124 & 0.124 & 0.125 & -19.256 & 0.106 & -8.82 \\ 
  \cmidrule{2-23} & \multirow{3}[2]{*}{$\rho=0.9$} & p=12 & 1.034 & 1.034 & 1.035 & 1.035 & 1.035 & 1.035 & 1.035 & 1.035 & 1.035 & 1.036 & 0.14 & 0.139 & 0.139 & 0.139 & 0.139 & 0.139 & 0.139 & 0.138 & 0.139 & 0.138 \\ 
   &  & p=100 & 1.034 & 1.035 & 1.037 & 1.039 & 1.038 & 1.037 & 1.037 & 1.041 & 1.037 & 1.035 & 0.137 & 0.137 & 0.135 & 0.134 & 0.134 & 0.135 & 0.135 & 0.132 & 0.135 & 0.136 \\ 
   &  & p=196 & 1.034 & 1.035 & 1.314 & 15.202 & 1.038 & 1.037 & 1.037 & 24.084 & 1.088 & 11.147 & 0.137 & 0.137 & -0.096 & -11.68 & 0.134 & 0.135 & 0.135 & -19.089 & 0.092 & -8.298 \\ 
   \midrule 
 \multicolumn{1}{|c}{} & \multicolumn{1}{c}{} &       & \multicolumn{10}{c||}{Kullback-Leibler Discrepancy}                                    & \multicolumn{10}{c|}{Number of Variables} \\
\midrule\multirow{9}[6]{*}{hsnr} & \multirow{3}[2]{*}{$\rho=0$} & p=12 & 2.499 & 2.5 & 2.5 & 2.501 & 2.5 & 2.501 & 2.501 & 2.502 & 2.501 & 2.491 & 6.508 & 6.562 & 6.595 & 6.657 & 6.7 & 6.66 & 6.639 & 6.726 & 6.64 & 6.032 \\ 
   &  & p=100 & 2.501 & 2.502 & 2.504 & 2.508 & 2.506 & 2.507 & 2.506 & 2.511 & 2.506 & 2.491 & 6.628 & 6.728 & 6.867 & 7.144 & 7.101 & 7.068 & 6.993 & 7.323 & 6.994 & 6.032 \\ 
   &  & p=196 & 2.501 & 2.502 & 2.947 & 106.918 & 2.506 & 2.507 & 2.506 & 146.314 & 3.751 & 109.825 & 6.628 & 6.728 & 25.908 & 59.763 & 7.101 & 7.068 & 6.993 & 84.379 & 7.18 & 35.067 \\ 
  \cmidrule{2-23} & \multirow{3}[2]{*}{$\rho=0.5$} & p=12 & 3.285 & 3.286 & 3.286 & 3.287 & 3.287 & 3.287 & 3.287 & 3.288 & 3.287 & 3.28 & 6.454 & 6.521 & 6.554 & 6.604 & 6.612 & 6.602 & 6.601 & 6.679 & 6.603 & 5.969 \\ 
   &  & p=100 & 3.287 & 3.289 & 3.291 & 3.295 & 3.294 & 3.293 & 3.292 & 3.298 & 3.292 & 3.28 & 6.588 & 6.68 & 6.825 & 7.144 & 7.104 & 7 & 6.906 & 7.36 & 6.907 & 5.969 \\ 
   &  & p=196 & 3.287 & 3.289 & 3.729 & 107.479 & 3.294 & 3.293 & 3.292 & 147.138 & 4.278 & 110.52 & 6.588 & 6.68 & 25.903 & 59.568 & 7.104 & 7 & 6.906 & 83.866 & 7.092 & 35.006 \\ 
  \cmidrule{2-23} & \multirow{3}[2]{*}{$\rho=0.9$} & p=12 & 3.928 & 3.928 & 3.929 & 3.929 & 3.929 & 3.929 & 3.929 & 3.93 & 3.929 & 3.928 & 5.928 & 5.975 & 6.001 & 6.066 & 6.073 & 6.055 & 6.059 & 6.126 & 6.059 & 5.073 \\ 
   &  & p=100 & 3.944 & 3.945 & 3.946 & 3.951 & 3.949 & 3.95 & 3.948 & 3.954 & 3.948 & 3.943 & 5.969 & 6.121 & 6.208 & 6.541 & 6.431 & 6.434 & 6.343 & 6.735 & 6.352 & 5.067 \\ 
   &  & p=196 & 3.944 & 3.945 & 4.364 & 107.035 & 3.949 & 3.949 & 3.948 & 145.521 & 7.697 & 110.69 & 5.969 & 6.121 & 25.512 & 59.042 & 6.431 & 6.42 & 6.343 & 82.188 & 6.723 & 33.861 \\ 
  \midrule\multirow{9}[6]{*}{msnr} & \multirow{3}[2]{*}{$\rho=0$} & p=12 & 4.641 & 4.642 & 4.642 & 4.643 & 4.642 & 4.643 & 4.643 & 4.644 & 4.643 & 4.642 & 6.316 & 6.378 & 6.419 & 6.491 & 6.486 & 6.489 & 6.47 & 6.563 & 6.472 & 5.506 \\ 
   &  & p=100 & 4.643 & 4.645 & 4.646 & 4.65 & 4.648 & 4.649 & 4.648 & 4.653 & 4.648 & 4.642 & 6.422 & 6.556 & 6.701 & 6.983 & 6.885 & 6.891 & 6.832 & 7.16 & 6.834 & 5.506 \\ 
   &  & p=196 & 4.643 & 4.645 & 5.089 & 109.06 & 4.648 & 4.649 & 4.648 & 148.456 & 5.893 & 111.673 & 6.422 & 6.556 & 25.733 & 59.545 & 6.885 & 6.891 & 6.832 & 84.273 & 7.02 & 34.421 \\ 
  \cmidrule{2-23} & \multirow{3}[2]{*}{$\rho=0.5$} & p=12 & 5.427 & 5.427 & 5.428 & 5.428 & 5.428 & 5.428 & 5.428 & 5.429 & 5.428 & 5.43 & 5.837 & 5.925 & 5.946 & 6.015 & 6.02 & 6.024 & 5.997 & 6.104 & 6 & 4.82 \\ 
   &  & p=100 & 5.429 & 5.43 & 5.432 & 5.435 & 5.433 & 5.435 & 5.433 & 5.438 & 5.433 & 5.43 & 5.979 & 6.091 & 6.177 & 6.489 & 6.426 & 6.405 & 6.283 & 6.722 & 6.291 & 4.82 \\ 
   &  & p=196 & 5.429 & 5.43 & 5.871 & 109.619 & 5.433 & 5.435 & 5.433 & 148.719 & 6.419 & 112.06 & 5.979 & 6.091 & 25.38 & 58.999 & 6.426 & 6.405 & 6.283 & 82.563 & 6.476 & 33.641 \\ 
  \cmidrule{2-23} & \multirow{3}[2]{*}{$\rho=0.9$} & p=12 & 6.065 & 6.066 & 6.066 & 6.067 & 6.066 & 6.066 & 6.066 & 6.067 & 6.066 & 6.069 & 4.679 & 4.756 & 4.792 & 4.884 & 4.855 & 4.864 & 4.837 & 4.95 & 4.846 & 3.37 \\ 
   &  & p=100 & 6.081 & 6.082 & 6.083 & 6.086 & 6.085 & 6.085 & 6.084 & 6.088 & 6.084 & 6.084 & 4.699 & 4.859 & 4.953 & 5.194 & 5.154 & 5.129 & 5.042 & 5.333 & 5.042 & 3.344 \\ 
   &  & p=196 & 6.081 & 6.082 & 6.502 & 109.168 & 6.085 & 6.085 & 6.084 & 147.222 & 9.832 & 112.179 & 4.699 & 4.859 & 24.552 & 57.875 & 5.154 & 5.129 & 5.042 & 80.429 & 5.419 & 32.029 \\ 
  \midrule\multirow{9}[6]{*}{lsnr} & \multirow{3}[2]{*}{$\rho=0$} & p=12 & 6.251 & 6.252 & 6.253 & 6.253 & 6.253 & 6.253 & 6.253 & 6.254 & 6.253 & 6.267 & 5.185 & 5.283 & 5.353 & 5.434 & 5.35 & 5.384 & 5.407 & 5.49 & 5.416 & 3.315 \\ 
   &  & p=100 & 6.253 & 6.255 & 6.257 & 6.26 & 6.257 & 6.258 & 6.258 & 6.262 & 6.258 & 6.267 & 5.266 & 5.45 & 5.622 & 5.902 & 5.645 & 5.688 & 5.734 & 6.035 & 5.743 & 3.315 \\ 
   &  & p=196 & 6.253 & 6.255 & 6.7 & 110.476 & 6.257 & 6.258 & 6.258 & 149.554 & 7.503 & 112.013 & 5.266 & 5.45 & 24.881 & 58.363 & 5.645 & 5.688 & 5.734 & 82.556 & 5.929 & 31.797 \\ 
  \cmidrule{2-23} & \multirow{3}[2]{*}{$\rho=0.5$} & p=12 & 7.034 & 7.034 & 7.035 & 7.035 & 7.035 & 7.035 & 7.035 & 7.036 & 7.035 & 7.046 & 4.591 & 4.692 & 4.728 & 4.794 & 4.839 & 4.815 & 4.767 & 4.886 & 4.781 & 2.911 \\ 
   &  & p=100 & 7.035 & 7.037 & 7.038 & 7.042 & 7.04 & 7.04 & 7.038 & 7.043 & 7.039 & 7.046 & 4.697 & 4.851 & 5.028 & 5.301 & 5.154 & 5.114 & 5.028 & 5.43 & 5.085 & 2.911 \\ 
   &  & p=196 & 7.035 & 7.037 & 7.479 & 111.222 & 7.04 & 7.04 & 7.038 & 150.027 & 8.025 & 112.353 & 4.697 & 4.851 & 24.445 & 57.919 & 5.154 & 5.114 & 5.028 & 81.218 & 5.272 & 31.252 \\ 
  \cmidrule{2-23} & \multirow{3}[2]{*}{$\rho=0.9$} & p=12 & 7.669 & 7.669 & 7.67 & 7.67 & 7.67 & 7.67 & 7.67 & 7.671 & 7.67 & 7.671 & 3.125 & 3.208 & 3.28 & 3.353 & 3.352 & 3.296 & 3.302 & 3.417 & 3.317 & 1.585 \\ 
   &  & p=100 & 7.684 & 7.685 & 7.687 & 7.689 & 7.689 & 7.687 & 7.688 & 7.692 & 7.688 & 7.685 & 3.158 & 3.258 & 3.454 & 3.62 & 3.643 & 3.48 & 3.473 & 3.866 & 3.475 & 1.605 \\ 
   &  & p=196 & 7.684 & 7.685 & 8.107 & 110.773 & 7.689 & 7.687 & 7.688 & 148.518 & 11.436 & 110.896 & 3.158 & 3.258 & 23.412 & 56.787 & 3.643 & 3.48 & 3.473 & 79.01 & 3.856 & 28.793 \\ 
   \bottomrule 
\end{tabular}
}
\end{table}

% latex table generated in R 4.0.2 by xtable 1.8-4 package
% Fri Jun 26 10:43:44 2020
\begin{table}[ht]
\centering
\caption{Fixed X, Sparse-Ex1, n=1000} 
\scalebox{0.52}{
\begin{tabular}{|c|c|c|cc|cc|cc|ccc|c||cc|cc|cc|ccc|c|}
  \toprule 
 \multicolumn{1}{|c}{} & \multicolumn{1}{c}{} &       & RAICc & AICc  & RC$_p$ & C$_p$ & 10F CV & LOO CV & S$_p$ & $\widehat{\text{C}}_p$ & GCV   & BIC   & RAICc & AICc  & RC$_p$ & C$_p$ & 10F CV & LOO CV & S$_p$ & $\widehat{\text{C}}_p$ & GCV   & BIC  \\
 \cmidrule{4-23}\multicolumn{1}{|c}{} & \multicolumn{1}{c}{} &       & \multicolumn{10}{c||}{RMSE}                                                   & \multicolumn{10}{c|}{Relative Risk}       \\
 \midrule\multirow{9}[6]{*}{hsnr} & \multirow{3}[2]{*}{$\rho=0$} & p=12 & 0.165 & 0.166 & 0.166 & 0.166 & 0.166 & 0.166 & 0.166 & 0.166 & 0.166 & 0.146 & 0.001 & 0.001 & 0.001 & 0.001 & 0.001 & 0.001 & 0.001 & 0.001 & 0.001 & 0.001 \\ 
   &  & p=500 & 0.17 & 0.171 & 0.171 & 0.172 & 0.173 & 0.173 & 0.172 & 0.174 & 0.172 & 0.146 & 0.001 & 0.001 & 0.001 & 0.001 & 0.001 & 0.001 & 0.001 & 0.001 & 0.001 & 0.001 \\ 
   &  & p=996 & 0.17 & 0.171 & 0.425 & 0.654 & 0.173 & 0.173 & 0.172 & 0.811 & 0.172 & 0.325 & 0.001 & 0.001 & 0.023 & 5.236 & 0.001 & 0.001 & 0.001 & 9.175 & 0.001 & 2.892 \\ 
  \cmidrule{2-23} & \multirow{3}[2]{*}{$\rho=0.5$} & p=12 & 0.247 & 0.247 & 0.247 & 0.248 & 0.248 & 0.248 & 0.248 & 0.248 & 0.248 & 0.218 & 0.001 & 0.001 & 0.001 & 0.001 & 0.001 & 0.001 & 0.001 & 0.001 & 0.001 & 0.001 \\ 
   &  & p=500 & 0.254 & 0.256 & 0.256 & 0.257 & 0.256 & 0.256 & 0.256 & 0.257 & 0.256 & 0.218 & 0.001 & 0.001 & 0.001 & 0.001 & 0.001 & 0.001 & 0.001 & 0.001 & 0.001 & 0.001 \\ 
   &  & p=996 & 0.254 & 0.256 & 0.635 & 0.98 & 0.256 & 0.256 & 0.256 & 1.217 & 0.256 & 0.486 & 0.001 & 0.001 & 0.023 & 5.253 & 0.001 & 0.001 & 0.001 & 9.477 & 0.001 & 2.906 \\ 
  \cmidrule{2-23} & \multirow{3}[2]{*}{$\rho=0.9$} & p=12 & 0.352 & 0.353 & 0.353 & 0.354 & 0.356 & 0.354 & 0.354 & 0.355 & 0.354 & 0.353 & 0.001 & 0.001 & 0.001 & 0.001 & 0.001 & 0.001 & 0.001 & 0.001 & 0.001 & 0.001 \\ 
   &  & p=500 & 0.365 & 0.367 & 0.367 & 0.368 & 0.37 & 0.367 & 0.368 & 0.369 & 0.368 & 0.354 & 0.001 & 0.001 & 0.001 & 0.001 & 0.001 & 0.001 & 0.001 & 0.001 & 0.001 & 0.001 \\ 
   &  & p=996 & 0.365 & 0.367 & 0.903 & 1.392 & 0.37 & 0.367 & 0.368 & 1.731 & 0.368 & 0.73 & 0.001 & 0.001 & 0.024 & 5.462 & 0.001 & 0.001 & 0.001 & 9.574 & 0.001 & 2.964 \\ 
  \midrule\multirow{9}[6]{*}{msnr} & \multirow{3}[2]{*}{$\rho=0$} & p=12 & 0.482 & 0.483 & 0.483 & 0.484 & 0.484 & 0.484 & 0.484 & 0.485 & 0.484 & 0.426 & 0.008 & 0.008 & 0.008 & 0.008 & 0.008 & 0.008 & 0.008 & 0.008 & 0.008 & 0.006 \\ 
   &  & p=500 & 0.497 & 0.499 & 0.499 & 0.503 & 0.503 & 0.503 & 0.503 & 0.506 & 0.503 & 0.426 & 0.008 & 0.008 & 0.008 & 0.009 & 0.009 & 0.009 & 0.009 & 0.009 & 0.009 & 0.006 \\ 
   &  & p=996 & 0.497 & 0.499 & 1.239 & 1.908 & 0.503 & 0.503 & 0.503 & 2.365 & 0.503 & 0.948 & 0.008 & 0.008 & 0.198 & 44.507 & 0.009 & 0.009 & 0.009 & 77.985 & 0.009 & 24.582 \\ 
  \cmidrule{2-23} & \multirow{3}[2]{*}{$\rho=0.5$} & p=12 & 0.726 & 0.727 & 0.727 & 0.729 & 0.735 & 0.729 & 0.728 & 0.73 & 0.728 & 0.729 & 0.008 & 0.008 & 0.008 & 0.008 & 0.008 & 0.008 & 0.008 & 0.008 & 0.008 & 0.008 \\ 
   &  & p=500 & 0.748 & 0.751 & 0.753 & 0.756 & 0.758 & 0.754 & 0.754 & 0.757 & 0.754 & 0.73 & 0.008 & 0.009 & 0.009 & 0.009 & 0.009 & 0.009 & 0.009 & 0.009 & 0.009 & 0.008 \\ 
   &  & p=996 & 0.748 & 0.751 & 1.863 & 2.867 & 0.758 & 0.754 & 0.754 & 3.554 & 0.754 & 1.501 & 0.008 & 0.009 & 0.199 & 44.651 & 0.009 & 0.009 & 0.009 & 80.555 & 0.009 & 24.704 \\ 
  \cmidrule{2-23} & \multirow{3}[2]{*}{$\rho=0.9$} & p=12 & 1.037 & 1.039 & 1.041 & 1.042 & 1.047 & 1.043 & 1.042 & 1.043 & 1.042 & 1.137 & 0.008 & 0.008 & 0.008 & 0.008 & 0.008 & 0.008 & 0.008 & 0.008 & 0.008 & 0.009 \\ 
   &  & p=500 & 1.07 & 1.075 & 1.075 & 1.081 & 1.078 & 1.08 & 1.081 & 1.083 & 1.082 & 1.139 & 0.009 & 0.009 & 0.009 & 0.009 & 0.009 & 0.009 & 0.009 & 0.009 & 0.009 & 0.009 \\ 
   &  & p=996 & 1.07 & 1.075 & 2.647 & 4.073 & 1.078 & 1.08 & 1.081 & 5.05 & 1.082 & 2.225 & 0.009 & 0.009 & 0.203 & 46.429 & 0.009 & 0.009 & 0.009 & 81.378 & 0.009 & 25.193 \\ 
  \midrule\multirow{9}[6]{*}{lsnr} & \multirow{3}[2]{*}{$\rho=0$} & p=12 & 1.103 & 1.105 & 1.105 & 1.107 & 1.114 & 1.107 & 1.107 & 1.109 & 1.107 & 1.222 & 0.04 & 0.04 & 0.04 & 0.04 & 0.04 & 0.04 & 0.04 & 0.04 & 0.04 & 0.048 \\ 
   &  & p=500 & 1.135 & 1.141 & 1.141 & 1.149 & 1.153 & 1.149 & 1.148 & 1.154 & 1.148 & 1.222 & 0.043 & 0.044 & 0.044 & 0.044 & 0.045 & 0.045 & 0.044 & 0.045 & 0.044 & 0.048 \\ 
   &  & p=996 & 1.135 & 1.141 & 2.805 & 4.299 & 1.153 & 1.149 & 1.148 & 5.292 & 1.148 & 2.353 & 0.043 & 0.044 & 0.991 & 222.537 & 0.045 & 0.045 & 0.044 & 389.172 & 0.044 & 122.439 \\ 
  \cmidrule{2-23} & \multirow{3}[2]{*}{$\rho=0.5$} & p=12 & 1.65 & 1.651 & 1.652 & 1.654 & 1.67 & 1.653 & 1.653 & 1.655 & 1.654 & 1.887 & 0.039 & 0.039 & 0.039 & 0.039 & 0.04 & 0.039 & 0.039 & 0.04 & 0.039 & 0.052 \\ 
   &  & p=500 & 1.694 & 1.701 & 1.707 & 1.713 & 1.72 & 1.703 & 1.702 & 1.708 & 1.702 & 1.886 & 0.043 & 0.043 & 0.043 & 0.044 & 0.044 & 0.043 & 0.043 & 0.043 & 0.043 & 0.052 \\ 
   &  & p=996 & 1.694 & 1.701 & 4.199 & 6.446 & 1.72 & 1.703 & 1.702 & 7.955 & 1.702 & 3.573 & 0.043 & 0.043 & 0.996 & 223.258 & 0.044 & 0.043 & 0.043 & 402.772 & 0.043 & 123.04 \\ 
  \cmidrule{2-23} & \multirow{3}[2]{*}{$\rho=0.9$} & p=12 & 2.288 & 2.29 & 2.291 & 2.293 & 2.291 & 2.291 & 2.292 & 2.293 & 2.292 & 2.761 & 0.039 & 0.039 & 0.039 & 0.039 & 0.039 & 0.039 & 0.039 & 0.039 & 0.039 & 0.056 \\ 
   &  & p=500 & 2.326 & 2.34 & 2.339 & 2.349 & 2.35 & 2.355 & 2.347 & 2.362 & 2.347 & 2.762 & 0.041 & 0.042 & 0.042 & 0.042 & 0.042 & 0.042 & 0.042 & 0.043 & 0.042 & 0.057 \\ 
   &  & p=996 & 2.326 & 2.34 & 5.907 & 9.086 & 2.35 & 2.355 & 2.347 & 11.219 & 2.347 & 5.107 & 0.041 & 0.042 & 1.014 & 232.144 & 0.042 & 0.042 & 0.042 & 406.246 & 0.042 & 124.468 \\ 
   \midrule 
 \multicolumn{1}{|c}{} & \multicolumn{1}{c}{} &       & \multicolumn{10}{c||}{Relative Test Error}                                    & \multicolumn{10}{c|}{Proportion of Variance Explained} \\
\midrule\multirow{9}[6]{*}{hsnr} & \multirow{3}[2]{*}{$\rho=0$} & p=12 & 1.008 & 1.008 & 1.008 & 1.008 & 1.008 & 1.008 & 1.008 & 1.008 & 1.008 & 1.006 & 0.9 & 0.9 & 0.9 & 0.9 & 0.9 & 0.9 & 0.9 & 0.9 & 0.9 & 0.9 \\ 
   &  & p=500 & 1.009 & 1.009 & 1.009 & 1.009 & 1.009 & 1.009 & 1.009 & 1.009 & 1.009 & 1.006 & 0.9 & 0.9 & 0.9 & 0.9 & 0.9 & 0.9 & 0.9 & 0.9 & 0.9 & 0.9 \\ 
   &  & p=996 & 1.009 & 1.009 & 1.212 & 48.611 & 1.009 & 1.009 & 1.009 & 84.423 & 1.009 & 27.297 & 0.9 & 0.9 & 0.88 & -3.816 & 0.9 & 0.9 & 0.9 & -7.365 & 0.9 & -1.705 \\ 
  \cmidrule{2-23} & \multirow{3}[2]{*}{$\rho=0.5$} & p=12 & 1.008 & 1.008 & 1.008 & 1.008 & 1.008 & 1.008 & 1.008 & 1.008 & 1.008 & 1.006 & 0.9 & 0.9 & 0.9 & 0.9 & 0.9 & 0.9 & 0.9 & 0.9 & 0.9 & 0.9 \\ 
   &  & p=500 & 1.009 & 1.009 & 1.009 & 1.009 & 1.009 & 1.009 & 1.009 & 1.009 & 1.009 & 1.006 & 0.9 & 0.9 & 0.9 & 0.9 & 0.9 & 0.9 & 0.9 & 0.9 & 0.9 & 0.9 \\ 
   &  & p=996 & 1.009 & 1.009 & 1.212 & 48.52 & 1.009 & 1.009 & 1.009 & 86.73 & 1.009 & 27.29 & 0.9 & 0.9 & 0.879 & -3.83 & 0.9 & 0.9 & 0.9 & -7.633 & 0.9 & -1.716 \\ 
  \cmidrule{2-23} & \multirow{3}[2]{*}{$\rho=0.9$} & p=12 & 1.008 & 1.008 & 1.008 & 1.008 & 1.008 & 1.008 & 1.008 & 1.008 & 1.008 & 1.008 & 0.898 & 0.898 & 0.898 & 0.898 & 0.898 & 0.898 & 0.898 & 0.898 & 0.898 & 0.898 \\ 
   &  & p=500 & 1.009 & 1.009 & 1.009 & 1.009 & 1.009 & 1.009 & 1.009 & 1.009 & 1.009 & 1.008 & 0.898 & 0.898 & 0.898 & 0.898 & 0.898 & 0.898 & 0.898 & 0.898 & 0.898 & 0.898 \\ 
   &  & p=996 & 1.009 & 1.009 & 1.212 & 49.481 & 1.009 & 1.009 & 1.009 & 85.975 & 1.009 & 27.306 & 0.898 & 0.898 & 0.877 & -4.01 & 0.898 & 0.898 & 0.898 & -7.706 & 0.898 & -1.765 \\ 
  \midrule\multirow{9}[6]{*}{msnr} & \multirow{3}[2]{*}{$\rho=0$} & p=12 & 1.008 & 1.008 & 1.008 & 1.008 & 1.008 & 1.008 & 1.008 & 1.008 & 1.008 & 1.006 & 0.513 & 0.513 & 0.513 & 0.513 & 0.513 & 0.513 & 0.513 & 0.513 & 0.513 & 0.514 \\ 
   &  & p=500 & 1.009 & 1.009 & 1.009 & 1.009 & 1.009 & 1.009 & 1.009 & 1.009 & 1.009 & 1.006 & 0.513 & 0.512 & 0.512 & 0.512 & 0.512 & 0.512 & 0.512 & 0.512 & 0.512 & 0.514 \\ 
   &  & p=996 & 1.009 & 1.009 & 1.212 & 48.611 & 1.009 & 1.009 & 1.009 & 84.423 & 1.009 & 27.297 & 0.513 & 0.512 & 0.415 & -22.486 & 0.512 & 0.512 & 0.512 & -39.789 & 0.512 & -12.189 \\ 
  \cmidrule{2-23} & \multirow{3}[2]{*}{$\rho=0.5$} & p=12 & 1.008 & 1.008 & 1.008 & 1.008 & 1.008 & 1.008 & 1.008 & 1.008 & 1.008 & 1.008 & 0.512 & 0.512 & 0.512 & 0.512 & 0.512 & 0.512 & 0.512 & 0.512 & 0.512 & 0.512 \\ 
   &  & p=500 & 1.009 & 1.009 & 1.009 & 1.009 & 1.009 & 1.009 & 1.009 & 1.009 & 1.009 & 1.008 & 0.511 & 0.511 & 0.511 & 0.511 & 0.511 & 0.511 & 0.511 & 0.511 & 0.511 & 0.512 \\ 
   &  & p=996 & 1.009 & 1.009 & 1.212 & 48.52 & 1.009 & 1.009 & 1.009 & 86.73 & 1.009 & 27.292 & 0.511 & 0.511 & 0.413 & -22.505 & 0.511 & 0.511 & 0.511 & -41.015 & 0.511 & -12.221 \\ 
  \cmidrule{2-23} & \multirow{3}[2]{*}{$\rho=0.9$} & p=12 & 1.008 & 1.008 & 1.008 & 1.008 & 1.008 & 1.008 & 1.008 & 1.008 & 1.008 & 1.01 & 0.507 & 0.507 & 0.507 & 0.507 & 0.507 & 0.507 & 0.507 & 0.507 & 0.507 & 0.506 \\ 
   &  & p=500 & 1.009 & 1.009 & 1.009 & 1.009 & 1.009 & 1.009 & 1.009 & 1.009 & 1.009 & 1.01 & 0.506 & 0.506 & 0.506 & 0.506 & 0.506 & 0.506 & 0.506 & 0.506 & 0.506 & 0.506 \\ 
   &  & p=996 & 1.009 & 1.009 & 1.212 & 49.481 & 1.009 & 1.009 & 1.009 & 85.975 & 1.009 & 27.307 & 0.506 & 0.506 & 0.407 & -23.206 & 0.506 & 0.506 & 0.506 & -41.058 & 0.506 & -12.358 \\ 
  \midrule\multirow{9}[6]{*}{lsnr} & \multirow{3}[2]{*}{$\rho=0$} & p=12 & 1.008 & 1.008 & 1.008 & 1.009 & 1.009 & 1.009 & 1.009 & 1.009 & 1.009 & 1.01 & 0.169 & 0.169 & 0.169 & 0.169 & 0.169 & 0.169 & 0.169 & 0.169 & 0.169 & 0.168 \\ 
   &  & p=500 & 1.009 & 1.009 & 1.009 & 1.01 & 1.01 & 1.01 & 1.01 & 1.01 & 1.01 & 1.01 & 0.169 & 0.169 & 0.169 & 0.168 & 0.168 & 0.168 & 0.168 & 0.168 & 0.168 & 0.168 \\ 
   &  & p=996 & 1.009 & 1.009 & 1.212 & 48.611 & 1.01 & 1.01 & 1.01 & 84.263 & 1.01 & 27.196 & 0.169 & 0.169 & 0.002 & -39.044 & 0.168 & 0.168 & 0.168 & -68.412 & 0.168 & -21.403 \\ 
  \cmidrule{2-23} & \multirow{3}[2]{*}{$\rho=0.5$} & p=12 & 1.008 & 1.008 & 1.008 & 1.008 & 1.009 & 1.008 & 1.008 & 1.008 & 1.008 & 1.011 & 0.169 & 0.169 & 0.169 & 0.169 & 0.168 & 0.169 & 0.169 & 0.169 & 0.169 & 0.166 \\ 
   &  & p=500 & 1.009 & 1.009 & 1.009 & 1.009 & 1.009 & 1.009 & 1.009 & 1.009 & 1.009 & 1.011 & 0.168 & 0.168 & 0.168 & 0.168 & 0.168 & 0.168 & 0.168 & 0.168 & 0.168 & 0.166 \\ 
   &  & p=996 & 1.009 & 1.009 & 1.212 & 48.521 & 1.009 & 1.009 & 1.009 & 86.73 & 1.009 & 27.189 & 0.168 & 0.168 & 0.001 & -39.005 & 0.168 & 0.168 & 0.168 & -70.51 & 0.168 & -21.418 \\ 
  \cmidrule{2-23} & \multirow{3}[2]{*}{$\rho=0.9$} & p=12 & 1.008 & 1.008 & 1.008 & 1.008 & 1.008 & 1.008 & 1.008 & 1.008 & 1.008 & 1.012 & 0.166 & 0.166 & 0.166 & 0.166 & 0.166 & 0.166 & 0.166 & 0.166 & 0.166 & 0.163 \\ 
   &  & p=500 & 1.009 & 1.009 & 1.009 & 1.009 & 1.009 & 1.009 & 1.009 & 1.009 & 1.009 & 1.012 & 0.166 & 0.166 & 0.166 & 0.165 & 0.165 & 0.165 & 0.166 & 0.165 & 0.166 & 0.163 \\ 
   &  & p=996 & 1.009 & 1.009 & 1.212 & 49.481 & 1.009 & 1.009 & 1.009 & 85.841 & 1.009 & 26.994 & 0.166 & 0.166 & -0.002 & -39.933 & 0.165 & 0.165 & 0.166 & -70.011 & 0.166 & -21.33 \\ 
   \midrule 
 \multicolumn{1}{|c}{} & \multicolumn{1}{c}{} &       & \multicolumn{10}{c||}{Kullback-Leibler Discrepancy}                                    & \multicolumn{10}{c|}{Number of Variables} \\
\midrule\multirow{9}[6]{*}{hsnr} & \multirow{3}[2]{*}{$\rho=0$} & p=12 & 2.35 & 2.35 & 2.35 & 2.35 & 2.35 & 2.35 & 2.35 & 2.35 & 2.35 & 2.348 & 6.643 & 6.658 & 6.659 & 6.677 & 6.72 & 6.68 & 6.672 & 6.698 & 6.673 & 6.01 \\ 
   &  & p=500 & 2.351 & 2.351 & 2.351 & 2.351 & 2.351 & 2.351 & 2.351 & 2.351 & 2.351 & 2.348 & 6.96 & 7.02 & 7.018 & 7.101 & 7.174 & 7.117 & 7.098 & 7.169 & 7.099 & 6.01 \\ 
   &  & p=996 & 2.351 & 2.351 & 2.628 & 389.985 & 2.351 & 2.351 & 2.351 & 588.931 & 2.351 & 370.266 & 6.96 & 7.02 & 95.117 & 262.275 & 7.174 & 7.117 & 7.098 & 360.193 & 7.099 & 104.015 \\ 
  \cmidrule{2-23} & \multirow{3}[2]{*}{$\rho=0.5$} & p=12 & 3.154 & 3.154 & 3.154 & 3.154 & 3.154 & 3.154 & 3.154 & 3.154 & 3.154 & 3.152 & 6.617 & 6.631 & 6.637 & 6.653 & 6.708 & 6.654 & 6.647 & 6.663 & 6.647 & 6.005 \\ 
   &  & p=500 & 3.155 & 3.155 & 3.155 & 3.155 & 3.155 & 3.155 & 3.155 & 3.155 & 3.155 & 3.152 & 6.943 & 6.992 & 7.01 & 7.053 & 7.104 & 7.024 & 7.023 & 7.057 & 7.023 & 6.005 \\ 
   &  & p=996 & 3.155 & 3.155 & 3.432 & 393.113 & 3.155 & 3.155 & 3.155 & 598.289 & 3.155 & 371.06 & 6.943 & 6.992 & 95.072 & 263.028 & 7.104 & 7.024 & 7.023 & 362.184 & 7.023 & 104.007 \\ 
  \cmidrule{2-23} & \multirow{3}[2]{*}{$\rho=0.9$} & p=12 & 3.856 & 3.856 & 3.856 & 3.856 & 3.856 & 3.856 & 3.856 & 3.856 & 3.856 & 3.856 & 6.555 & 6.569 & 6.592 & 6.609 & 6.649 & 6.6 & 6.609 & 6.621 & 6.609 & 5.789 \\ 
   &  & p=500 & 3.856 & 3.856 & 3.856 & 3.856 & 3.856 & 3.856 & 3.856 & 3.856 & 3.856 & 3.855 & 6.941 & 6.99 & 6.999 & 7.022 & 7.091 & 7.011 & 7.025 & 7.051 & 7.025 & 5.782 \\ 
   &  & p=996 & 3.856 & 3.856 & 4.134 & 400.505 & 3.856 & 3.856 & 3.856 & 593.378 & 3.856 & 371.717 & 6.941 & 6.99 & 94.879 & 262.796 & 7.091 & 7.011 & 7.025 & 362.135 & 7.025 & 103.799 \\ 
  \midrule\multirow{9}[6]{*}{msnr} & \multirow{3}[2]{*}{$\rho=0$} & p=12 & 4.49 & 4.49 & 4.49 & 4.49 & 4.49 & 4.49 & 4.49 & 4.49 & 4.49 & 4.488 & 6.643 & 6.658 & 6.659 & 6.677 & 6.72 & 6.68 & 6.672 & 6.698 & 6.673 & 6.009 \\ 
   &  & p=500 & 4.491 & 4.491 & 4.491 & 4.491 & 4.491 & 4.491 & 4.491 & 4.491 & 4.491 & 4.488 & 6.96 & 7.02 & 7.018 & 7.101 & 7.174 & 7.117 & 7.098 & 7.169 & 7.099 & 6.009 \\ 
   &  & p=996 & 4.491 & 4.491 & 4.768 & 392.126 & 4.491 & 4.491 & 4.491 & 591.071 & 4.491 & 372.406 & 6.96 & 7.02 & 95.117 & 262.275 & 7.174 & 7.117 & 7.098 & 360.193 & 7.099 & 104.014 \\ 
  \cmidrule{2-23} & \multirow{3}[2]{*}{$\rho=0.5$} & p=12 & 5.294 & 5.294 & 5.294 & 5.294 & 5.294 & 5.294 & 5.294 & 5.294 & 5.294 & 5.294 & 6.568 & 6.585 & 6.591 & 6.607 & 6.657 & 6.604 & 6.601 & 6.617 & 6.601 & 5.756 \\ 
   &  & p=500 & 5.295 & 5.295 & 5.295 & 5.295 & 5.295 & 5.295 & 5.295 & 5.295 & 5.295 & 5.294 & 6.894 & 6.943 & 6.969 & 7.012 & 7.056 & 6.98 & 6.974 & 7.008 & 6.974 & 5.755 \\ 
   &  & p=996 & 5.295 & 5.295 & 5.573 & 395.253 & 5.295 & 5.295 & 5.295 & 600.429 & 5.295 & 373.202 & 6.894 & 6.943 & 95.029 & 262.985 & 7.056 & 6.98 & 6.974 & 362.151 & 6.974 & 103.78 \\ 
  \cmidrule{2-23} & \multirow{3}[2]{*}{$\rho=0.9$} & p=12 & 5.996 & 5.996 & 5.996 & 5.996 & 5.996 & 5.996 & 5.996 & 5.996 & 5.996 & 5.997 & 5.889 & 5.912 & 5.929 & 5.941 & 5.927 & 5.948 & 5.94 & 5.949 & 5.941 & 4.607 \\ 
   &  & p=500 & 5.996 & 5.996 & 5.996 & 5.996 & 5.996 & 5.996 & 5.996 & 5.996 & 5.996 & 5.997 & 6.245 & 6.299 & 6.288 & 6.377 & 6.341 & 6.367 & 6.379 & 6.408 & 6.384 & 4.592 \\ 
   &  & p=996 & 5.996 & 5.996 & 6.274 & 402.645 & 5.996 & 5.996 & 5.996 & 595.518 & 5.996 & 373.858 & 6.245 & 6.299 & 94.332 & 262.293 & 6.341 & 6.367 & 6.379 & 361.715 & 6.384 & 102.727 \\ 
  \midrule\multirow{9}[6]{*}{lsnr} & \multirow{3}[2]{*}{$\rho=0$} & p=12 & 6.1 & 6.1 & 6.1 & 6.1 & 6.1 & 6.1 & 6.1 & 6.1 & 6.1 & 6.101 & 6.488 & 6.499 & 6.506 & 6.529 & 6.565 & 6.527 & 6.524 & 6.552 & 6.526 & 5.342 \\ 
   &  & p=500 & 6.1 & 6.101 & 6.101 & 6.101 & 6.101 & 6.101 & 6.101 & 6.101 & 6.101 & 6.101 & 6.796 & 6.857 & 6.857 & 6.945 & 6.999 & 6.953 & 6.939 & 7.002 & 6.941 & 5.342 \\ 
   &  & p=996 & 6.1 & 6.101 & 6.378 & 393.735 & 6.101 & 6.101 & 6.101 & 591.88 & 6.101 & 372.177 & 6.796 & 6.857 & 94.952 & 262.1 & 6.999 & 6.953 & 6.939 & 359.11 & 6.941 & 102.416 \\ 
  \cmidrule{2-23} & \multirow{3}[2]{*}{$\rho=0.5$} & p=12 & 6.904 & 6.904 & 6.904 & 6.904 & 6.904 & 6.904 & 6.904 & 6.904 & 6.904 & 6.906 & 6.083 & 6.098 & 6.113 & 6.125 & 6.141 & 6.122 & 6.12 & 6.143 & 6.125 & 4.631 \\ 
   &  & p=500 & 6.904 & 6.904 & 6.904 & 6.905 & 6.905 & 6.904 & 6.904 & 6.905 & 6.904 & 6.906 & 6.391 & 6.448 & 6.489 & 6.532 & 6.52 & 6.465 & 6.456 & 6.499 & 6.456 & 4.632 \\ 
   &  & p=996 & 6.904 & 6.904 & 7.182 & 396.863 & 6.905 & 6.904 & 6.904 & 602.039 & 6.904 & 372.972 & 6.391 & 6.448 & 94.55 & 262.534 & 6.52 & 6.465 & 6.456 & 361.817 & 6.456 & 101.768 \\ 
  \cmidrule{2-23} & \multirow{3}[2]{*}{$\rho=0.9$} & p=12 & 7.605 & 7.605 & 7.605 & 7.605 & 7.605 & 7.605 & 7.605 & 7.605 & 7.605 & 7.608 & 4.874 & 4.898 & 4.908 & 4.925 & 4.909 & 4.92 & 4.92 & 4.938 & 4.92 & 2.948 \\ 
   &  & p=500 & 7.605 & 7.605 & 7.605 & 7.605 & 7.605 & 7.605 & 7.605 & 7.605 & 7.605 & 7.608 & 5.147 & 5.224 & 5.205 & 5.266 & 5.236 & 5.288 & 5.264 & 5.347 & 5.264 & 2.945 \\ 
   &  & p=996 & 7.605 & 7.605 & 7.883 & 404.255 & 7.605 & 7.605 & 7.605 & 596.673 & 7.605 & 371.682 & 5.147 & 5.224 & 93.577 & 261.488 & 5.236 & 5.288 & 5.264 & 360.038 & 5.264 & 99.272 \\ 
   \bottomrule 
\end{tabular}
}
\end{table}

% latex table generated in R 4.0.2 by xtable 1.8-4 package
% Fri Jun 26 10:43:45 2020
\begin{table}[ht]
\centering
\caption{Fixed X, Sparse-Ex2, n=40} 
\scalebox{0.52}{
\begin{tabular}{|c|c|c|cc|cc|cc|ccc|c||cc|cc|cc|ccc|c|}
  \toprule 
 \multicolumn{1}{|c}{} & \multicolumn{1}{c}{} &       & RAICc & AICc  & RC$_p$ & C$_p$ & 10F CV & LOO CV & S$_p$ & $\widehat{\text{C}}_p$ & GCV   & BIC   & RAICc & AICc  & RC$_p$ & C$_p$ & 10F CV & LOO CV & S$_p$ & $\widehat{\text{C}}_p$ & GCV   & BIC  \\
 \cmidrule{4-23}\multicolumn{1}{|c}{} & \multicolumn{1}{c}{} &       & \multicolumn{10}{c||}{RMSE}                                                   & \multicolumn{10}{c|}{Relative Risk}       \\
 \midrule\multirow{9}[6]{*}{hsnr} & \multirow{3}[2]{*}{$\rho=0$} & p=12 & 0.985 & 1.016 & 1.033 & 1.088 & 1.082 & 1.076 & 1.081 & 1.124 & 1.084 & 1.005 & 0.021 & 0.023 & 0.023 & 0.026 & 0.026 & 0.026 & 0.026 & 0.028 & 0.026 & 0.022 \\ 
   &  & p=20 & 0.985 & 1.019 & 1.055 & 1.157 & 1.121 & 1.123 & 1.127 & 1.251 & 1.132 & 1.017 & 0.021 & 0.023 & 0.025 & 0.035 & 0.03 & 0.031 & 0.031 & 0.044 & 0.032 & 0.023 \\ 
   &  & p=36 & 0.985 & 1.019 & 1.185 & 1.442 & 1.136 & 1.164 & 1.162 & 1.721 & 1.191 & 1.4 & 0.021 & 0.023 & 0.046 & 0.207 & 0.034 & 0.048 & 0.048 & 0.33 & 0.061 & 0.245 \\ 
  \cmidrule{2-23} & \multirow{3}[2]{*}{$\rho=0.5$} & p=12 & 1.421 & 1.461 & 1.486 & 1.556 & 1.553 & 1.551 & 1.541 & 1.62 & 1.547 & 1.449 & 0.02 & 0.022 & 0.023 & 0.025 & 0.025 & 0.025 & 0.025 & 0.028 & 0.025 & 0.021 \\ 
   &  & p=20 & 1.423 & 1.464 & 1.511 & 1.66 & 1.609 & 1.616 & 1.608 & 1.804 & 1.619 & 1.46 & 0.02 & 0.022 & 0.024 & 0.034 & 0.029 & 0.03 & 0.03 & 0.043 & 0.031 & 0.022 \\ 
   &  & p=36 & 1.423 & 1.464 & 1.701 & 2.071 & 1.631 & 1.669 & 1.659 & 2.476 & 1.702 & 2.008 & 0.02 & 0.022 & 0.044 & 0.198 & 0.033 & 0.047 & 0.047 & 0.315 & 0.06 & 0.237 \\ 
  \cmidrule{2-23} & \multirow{3}[2]{*}{$\rho=0.9$} & p=12 & 2.056 & 2.082 & 2.098 & 2.164 & 2.171 & 2.147 & 2.14 & 2.227 & 2.146 & 2.097 & 0.023 & 0.024 & 0.025 & 0.026 & 0.026 & 0.026 & 0.026 & 0.028 & 0.026 & 0.025 \\ 
   &  & p=20 & 2.055 & 2.088 & 2.137 & 2.312 & 2.249 & 2.244 & 2.232 & 2.455 & 2.241 & 2.111 & 0.023 & 0.024 & 0.026 & 0.036 & 0.031 & 0.031 & 0.031 & 0.043 & 0.032 & 0.025 \\ 
   &  & p=36 & 2.054 & 2.09 & 2.403 & 2.882 & 2.276 & 2.322 & 2.302 & 3.355 & 2.364 & 2.816 & 0.023 & 0.024 & 0.045 & 0.197 & 0.034 & 0.048 & 0.047 & 0.306 & 0.06 & 0.229 \\ 
  \midrule\multirow{9}[6]{*}{msnr} & \multirow{3}[2]{*}{$\rho=0$} & p=12 & 2.94 & 2.995 & 3.032 & 3.182 & 3.188 & 3.155 & 3.165 & 3.284 & 3.171 & 3.245 & 0.191 & 0.198 & 0.203 & 0.226 & 0.229 & 0.222 & 0.224 & 0.243 & 0.224 & 0.245 \\ 
   &  & p=20 & 2.94 & 3.005 & 3.1 & 3.384 & 3.3 & 3.292 & 3.3 & 3.653 & 3.311 & 3.281 & 0.191 & 0.201 & 0.22 & 0.303 & 0.266 & 0.267 & 0.269 & 0.377 & 0.271 & 0.257 \\ 
   &  & p=36 & 2.94 & 3.005 & 3.619 & 4.337 & 3.345 & 3.412 & 3.403 & 5.021 & 3.483 & 4.322 & 0.191 & 0.201 & 0.424 & 1.783 & 0.295 & 0.413 & 0.412 & 2.804 & 0.523 & 2.101 \\ 
  \cmidrule{2-23} & \multirow{3}[2]{*}{$\rho=0.5$} & p=12 & 4.514 & 4.512 & 4.538 & 4.66 & 4.68 & 4.678 & 4.634 & 4.795 & 4.638 & 4.859 & 0.211 & 0.211 & 0.214 & 0.228 & 0.229 & 0.228 & 0.224 & 0.243 & 0.224 & 0.255 \\ 
   &  & p=20 & 4.519 & 4.52 & 4.618 & 4.976 & 4.832 & 4.86 & 4.824 & 5.311 & 4.841 & 4.886 & 0.212 & 0.213 & 0.228 & 0.305 & 0.263 & 0.27 & 0.269 & 0.37 & 0.272 & 0.262 \\ 
   &  & p=36 & 4.519 & 4.52 & 5.39 & 6.35 & 4.895 & 5.01 & 4.969 & 7.239 & 5.082 & 6.299 & 0.212 & 0.213 & 0.424 & 1.721 & 0.29 & 0.41 & 0.413 & 2.665 & 0.52 & 1.98 \\ 
  \cmidrule{2-23} & \multirow{3}[2]{*}{$\rho=0.9$} & p=12 & 6.274 & 6.216 & 6.209 & 6.312 & 6.373 & 6.297 & 6.26 & 6.456 & 6.264 & 6.834 & 0.199 & 0.2 & 0.202 & 0.215 & 0.215 & 0.213 & 0.209 & 0.229 & 0.21 & 0.232 \\ 
   &  & p=20 & 6.289 & 6.214 & 6.31 & 6.736 & 6.545 & 6.529 & 6.482 & 7.075 & 6.505 & 6.858 & 0.2 & 0.201 & 0.216 & 0.292 & 0.245 & 0.254 & 0.246 & 0.342 & 0.249 & 0.236 \\ 
   &  & p=36 & 6.29 & 6.229 & 7.309 & 8.637 & 6.623 & 6.733 & 6.671 & 9.692 & 6.835 & 8.598 & 0.2 & 0.201 & 0.389 & 1.667 & 0.263 & 0.384 & 0.374 & 2.554 & 0.473 & 1.828 \\ 
  \midrule\multirow{9}[6]{*}{lsnr} & \multirow{3}[2]{*}{$\rho=0$} & p=12 & 7.677 & 7.649 & 7.652 & 7.758 & 7.755 & 7.717 & 7.732 & 7.837 & 7.74 & 7.608 & 1.13 & 1.146 & 1.174 & 1.229 & 1.215 & 1.207 & 1.209 & 1.272 & 1.213 & 1.099 \\ 
   &  & p=20 & 7.677 & 7.652 & 7.761 & 8.139 & 7.925 & 7.926 & 7.911 & 8.494 & 7.927 & 7.621 & 1.13 & 1.149 & 1.24 & 1.568 & 1.35 & 1.374 & 1.345 & 1.852 & 1.352 & 1.108 \\ 
   &  & p=36 & 7.677 & 7.652 & 8.464 & 9.997 & 7.979 & 8.155 & 8.098 & 11.332 & 8.203 & 9.347 & 1.13 & 1.149 & 2.103 & 8.878 & 1.447 & 2.069 & 1.981 & 13.479 & 2.374 & 8.18 \\ 
  \cmidrule{2-23} & \multirow{3}[2]{*}{$\rho=0.5$} & p=12 & 10.871 & 10.864 & 10.855 & 10.958 & 10.938 & 10.933 & 10.926 & 11.069 & 10.926 & 10.929 & 1.076 & 1.094 & 1.113 & 1.156 & 1.137 & 1.143 & 1.144 & 1.202 & 1.146 & 1.08 \\ 
   &  & p=20 & 10.871 & 10.877 & 11 & 11.479 & 11.17 & 11.224 & 11.183 & 11.995 & 11.223 & 10.943 & 1.076 & 1.102 & 1.178 & 1.485 & 1.269 & 1.3 & 1.271 & 1.734 & 1.297 & 1.089 \\ 
   &  & p=36 & 10.871 & 10.877 & 11.966 & 14.21 & 11.244 & 11.498 & 11.428 & 16.121 & 11.642 & 13.342 & 1.076 & 1.102 & 1.972 & 8.465 & 1.339 & 1.951 & 1.865 & 12.946 & 2.38 & 7.689 \\ 
  \cmidrule{2-23} & \multirow{3}[2]{*}{$\rho=0.9$} & p=12 & 12.638 & 12.752 & 12.942 & 13.304 & 12.992 & 13.168 & 13.131 & 13.541 & 13.154 & 13.306 & 0.726 & 0.764 & 0.815 & 0.887 & 0.826 & 0.864 & 0.854 & 0.945 & 0.86 & 0.799 \\ 
   &  & p=20 & 12.611 & 12.77 & 13.23 & 14.076 & 13.319 & 13.5 & 13.43 & 14.802 & 13.545 & 13.299 & 0.721 & 0.769 & 0.89 & 1.218 & 0.95 & 0.997 & 0.969 & 1.442 & 1.007 & 0.802 \\ 
   &  & p=36 & 12.613 & 12.77 & 14.976 & 18.072 & 13.363 & 13.902 & 13.831 & 20.805 & 14.012 & 16.771 & 0.721 & 0.768 & 1.683 & 8.083 & 0.982 & 1.509 & 1.58 & 12.15 & 1.767 & 7.158 \\ 
   \midrule 
 \multicolumn{1}{|c}{} & \multicolumn{1}{c}{} &       & \multicolumn{10}{c||}{Relative Test Error}                                    & \multicolumn{10}{c|}{Proportion of Variance Explained} \\
\midrule\multirow{9}[6]{*}{hsnr} & \multirow{3}[2]{*}{$\rho=0$} & p=12 & 1.229 & 1.247 & 1.256 & 1.288 & 1.287 & 1.281 & 1.284 & 1.311 & 1.286 & 1.241 & 0.897 & 0.896 & 0.895 & 0.892 & 0.892 & 0.893 & 0.892 & 0.89 & 0.892 & 0.896 \\ 
   &  & p=20 & 1.229 & 1.251 & 1.278 & 1.387 & 1.333 & 1.34 & 1.342 & 1.484 & 1.346 & 1.255 & 0.897 & 0.895 & 0.893 & 0.884 & 0.888 & 0.888 & 0.888 & 0.876 & 0.887 & 0.895 \\ 
   &  & p=36 & 1.229 & 1.251 & 1.507 & 3.263 & 1.372 & 1.527 & 1.526 & 4.606 & 1.67 & 3.682 & 0.897 & 0.895 & 0.874 & 0.727 & 0.885 & 0.872 & 0.872 & 0.614 & 0.86 & 0.692 \\ 
  \cmidrule{2-23} & \multirow{3}[2]{*}{$\rho=0.5$} & p=12 & 1.227 & 1.244 & 1.255 & 1.284 & 1.283 & 1.281 & 1.278 & 1.311 & 1.28 & 1.24 & 0.899 & 0.898 & 0.897 & 0.895 & 0.895 & 0.895 & 0.895 & 0.893 & 0.895 & 0.898 \\ 
   &  & p=20 & 1.228 & 1.247 & 1.273 & 1.386 & 1.33 & 1.338 & 1.338 & 1.485 & 1.344 & 1.251 & 0.899 & 0.898 & 0.896 & 0.886 & 0.891 & 0.89 & 0.89 & 0.878 & 0.89 & 0.898 \\ 
   &  & p=36 & 1.228 & 1.247 & 1.496 & 3.221 & 1.366 & 1.524 & 1.529 & 4.527 & 1.67 & 3.657 & 0.899 & 0.898 & 0.877 & 0.736 & 0.888 & 0.875 & 0.875 & 0.629 & 0.863 & 0.7 \\ 
  \cmidrule{2-23} & \multirow{3}[2]{*}{$\rho=0.9$} & p=12 & 1.268 & 1.275 & 1.282 & 1.301 & 1.303 & 1.295 & 1.293 & 1.321 & 1.295 & 1.281 & 0.898 & 0.897 & 0.897 & 0.895 & 0.895 & 0.896 & 0.896 & 0.894 & 0.896 & 0.897 \\ 
   &  & p=20 & 1.268 & 1.28 & 1.302 & 1.412 & 1.355 & 1.36 & 1.355 & 1.488 & 1.361 & 1.291 & 0.898 & 0.897 & 0.895 & 0.886 & 0.891 & 0.891 & 0.891 & 0.88 & 0.891 & 0.896 \\ 
   &  & p=36 & 1.267 & 1.28 & 1.518 & 3.253 & 1.384 & 1.545 & 1.539 & 4.495 & 1.691 & 3.615 & 0.898 & 0.897 & 0.878 & 0.738 & 0.889 & 0.876 & 0.876 & 0.638 & 0.864 & 0.709 \\ 
  \midrule\multirow{9}[6]{*}{msnr} & \multirow{3}[2]{*}{$\rho=0$} & p=12 & 1.245 & 1.255 & 1.261 & 1.291 & 1.295 & 1.286 & 1.288 & 1.313 & 1.289 & 1.316 & 0.455 & 0.451 & 0.448 & 0.436 & 0.434 & 0.438 & 0.437 & 0.426 & 0.436 & 0.425 \\ 
   &  & p=20 & 1.245 & 1.259 & 1.283 & 1.389 & 1.342 & 1.344 & 1.346 & 1.485 & 1.349 & 1.33 & 0.455 & 0.449 & 0.439 & 0.392 & 0.413 & 0.412 & 0.411 & 0.35 & 0.41 & 0.418 \\ 
   &  & p=36 & 1.245 & 1.259 & 1.546 & 3.294 & 1.38 & 1.532 & 1.53 & 4.607 & 1.673 & 3.704 & 0.455 & 0.449 & 0.324 & -0.441 & 0.397 & 0.33 & 0.331 & -1.015 & 0.268 & -0.62 \\ 
  \cmidrule{2-23} & \multirow{3}[2]{*}{$\rho=0.5$} & p=12 & 1.278 & 1.278 & 1.283 & 1.3 & 1.302 & 1.301 & 1.295 & 1.32 & 1.296 & 1.336 & 0.448 & 0.449 & 0.447 & 0.439 & 0.438 & 0.439 & 0.441 & 0.431 & 0.441 & 0.423 \\ 
   &  & p=20 & 1.279 & 1.281 & 1.301 & 1.401 & 1.346 & 1.355 & 1.354 & 1.488 & 1.358 & 1.346 & 0.448 & 0.447 & 0.439 & 0.395 & 0.419 & 0.415 & 0.416 & 0.358 & 0.414 & 0.419 \\ 
   &  & p=36 & 1.279 & 1.281 & 1.558 & 3.268 & 1.382 & 1.54 & 1.545 & 4.513 & 1.685 & 3.609 & 0.448 & 0.447 & 0.328 & -0.41 & 0.404 & 0.335 & 0.334 & -0.947 & 0.273 & -0.557 \\ 
  \cmidrule{2-23} & \multirow{3}[2]{*}{$\rho=0.9$} & p=12 & 1.268 & 1.269 & 1.272 & 1.289 & 1.29 & 1.286 & 1.282 & 1.309 & 1.282 & 1.312 & 0.459 & 0.459 & 0.457 & 0.45 & 0.45 & 0.452 & 0.453 & 0.442 & 0.453 & 0.441 \\ 
   &  & p=20 & 1.27 & 1.271 & 1.291 & 1.393 & 1.329 & 1.341 & 1.331 & 1.46 & 1.335 & 1.317 & 0.459 & 0.458 & 0.45 & 0.406 & 0.433 & 0.428 & 0.432 & 0.377 & 0.431 & 0.438 \\ 
   &  & p=36 & 1.269 & 1.271 & 1.524 & 3.243 & 1.354 & 1.516 & 1.503 & 4.436 & 1.637 & 3.459 & 0.459 & 0.458 & 0.35 & -0.383 & 0.423 & 0.354 & 0.359 & -0.891 & 0.302 & -0.475 \\ 
  \midrule\multirow{9}[6]{*}{lsnr} & \multirow{3}[2]{*}{$\rho=0$} & p=12 & 1.291 & 1.295 & 1.302 & 1.316 & 1.313 & 1.311 & 1.311 & 1.327 & 1.312 & 1.283 & -0.027 & -0.03 & -0.036 & -0.047 & -0.044 & -0.042 & -0.043 & -0.056 & -0.044 & -0.02 \\ 
   &  & p=20 & 1.291 & 1.296 & 1.319 & 1.404 & 1.347 & 1.354 & 1.346 & 1.477 & 1.348 & 1.285 & -0.027 & -0.03 & -0.049 & -0.116 & -0.072 & -0.077 & -0.071 & -0.174 & -0.072 & -0.022 \\ 
   &  & p=36 & 1.291 & 1.296 & 1.541 & 3.284 & 1.372 & 1.532 & 1.51 & 4.468 & 1.611 & 3.105 & -0.027 & -0.03 & -0.226 & -1.612 & -0.091 & -0.219 & -0.201 & -2.554 & -0.281 & -1.469 \\ 
  \cmidrule{2-23} & \multirow{3}[2]{*}{$\rho=0.5$} & p=12 & 1.284 & 1.288 & 1.293 & 1.305 & 1.3 & 1.301 & 1.301 & 1.317 & 1.302 & 1.285 & -0.016 & -0.02 & -0.024 & -0.033 & -0.029 & -0.03 & -0.03 & -0.042 & -0.031 & -0.017 \\ 
   &  & p=20 & 1.284 & 1.29 & 1.311 & 1.391 & 1.334 & 1.343 & 1.335 & 1.457 & 1.342 & 1.287 & -0.016 & -0.021 & -0.037 & -0.101 & -0.056 & -0.063 & -0.057 & -0.153 & -0.062 & -0.018 \\ 
   &  & p=36 & 1.284 & 1.29 & 1.52 & 3.231 & 1.353 & 1.514 & 1.492 & 4.412 & 1.627 & 3.027 & -0.016 & -0.021 & -0.203 & -1.557 & -0.071 & -0.198 & -0.18 & -2.492 & -0.288 & -1.395 \\ 
  \cmidrule{2-23} & \multirow{3}[2]{*}{$\rho=0.9$} & p=12 & 1.195 & 1.206 & 1.219 & 1.239 & 1.222 & 1.233 & 1.23 & 1.254 & 1.231 & 1.215 & 0.058 & 0.05 & 0.039 & 0.024 & 0.037 & 0.029 & 0.031 & 0.012 & 0.03 & 0.043 \\ 
   &  & p=20 & 1.194 & 1.207 & 1.239 & 1.328 & 1.256 & 1.268 & 1.261 & 1.388 & 1.271 & 1.216 & 0.059 & 0.049 & 0.023 & -0.046 & 0.011 & 0.001 & 0.007 & -0.094 & -0.001 & 0.042 \\ 
   &  & p=36 & 1.194 & 1.207 & 1.453 & 3.175 & 1.264 & 1.406 & 1.425 & 4.269 & 1.475 & 2.926 & 0.059 & 0.049 & -0.145 & -1.502 & 0.004 & -0.108 & -0.123 & -2.364 & -0.163 & -1.306 \\ 
   \midrule 
 \multicolumn{1}{|c}{} & \multicolumn{1}{c}{} &       & \multicolumn{10}{c||}{Kullback-Leibler Discrepancy}                                    & \multicolumn{10}{c|}{Number of Variables} \\
\midrule\multirow{9}[6]{*}{hsnr} & \multirow{3}[2]{*}{$\rho=0$} & p=12 & 3.144 & 3.178 & 3.192 & 3.242 & 3.233 & 3.229 & 3.236 & 3.276 & 3.238 & 3.173 & 6.148 & 6.319 & 6.418 & 6.815 & 6.868 & 6.758 & 6.758 & 7.138 & 6.778 & 6.274 \\ 
   &  & p=20 & 3.144 & 3.186 & 3.244 & 3.459 & 3.346 & 3.378 & 3.376 & 3.647 & 3.385 & 3.218 & 6.148 & 6.343 & 6.624 & 7.65 & 7.301 & 7.289 & 7.241 & 8.775 & 7.29 & 6.391 \\ 
   &  & p=36 & 3.144 & 3.186 & 5.239 & 21.118 & 3.474 & 8.63 & 9.569 & 27.251 & 11.117 & 26.12 & 6.148 & 6.343 & 8.522 & 13.668 & 7.526 & 8.022 & 7.939 & 18.913 & 8.414 & 14.278 \\ 
  \cmidrule{2-23} & \multirow{3}[2]{*}{$\rho=0.5$} & p=12 & 3.87 & 3.902 & 3.918 & 3.964 & 3.958 & 3.957 & 3.955 & 4.004 & 3.958 & 3.897 & 6.154 & 6.31 & 6.417 & 6.78 & 6.866 & 6.774 & 6.694 & 7.16 & 6.721 & 6.271 \\ 
   &  & p=20 & 3.873 & 3.907 & 3.966 & 4.188 & 4.068 & 4.1 & 4.104 & 4.376 & 4.112 & 3.932 & 6.162 & 6.322 & 6.591 & 7.657 & 7.299 & 7.296 & 7.203 & 8.808 & 7.261 & 6.349 \\ 
   &  & p=36 & 3.873 & 3.907 & 5.92 & 21.579 & 4.203 & 9.376 & 10.344 & 27.875 & 11.658 & 26.724 & 6.162 & 6.322 & 8.502 & 13.627 & 7.524 & 7.969 & 7.91 & 18.97 & 8.362 & 14.214 \\ 
  \cmidrule{2-23} & \multirow{3}[2]{*}{$\rho=0.9$} & p=12 & 4.492 & 4.52 & 4.532 & 4.575 & 4.57 & 4.564 & 4.563 & 4.61 & 4.566 & 4.515 & 5.879 & 6.088 & 6.191 & 6.614 & 6.643 & 6.528 & 6.498 & 7.017 & 6.539 & 5.95 \\ 
   &  & p=20 & 4.49 & 4.524 & 4.578 & 4.807 & 4.683 & 4.706 & 4.704 & 4.963 & 4.714 & 4.541 & 5.88 & 6.104 & 6.369 & 7.504 & 7.111 & 7.087 & 7.004 & 8.528 & 7.061 & 6.018 \\ 
   &  & p=36 & 4.49 & 4.526 & 6.451 & 22.223 & 4.88 & 9.837 & 10.885 & 28.535 & 12.412 & 27.206 & 5.878 & 6.115 & 8.235 & 13.484 & 7.318 & 7.851 & 7.735 & 18.884 & 8.29 & 13.799 \\ 
  \midrule\multirow{9}[6]{*}{msnr} & \multirow{3}[2]{*}{$\rho=0$} & p=12 & 5.293 & 5.323 & 5.335 & 5.384 & 5.378 & 5.372 & 5.379 & 5.417 & 5.38 & 5.351 & 6.059 & 6.272 & 6.392 & 6.803 & 6.823 & 6.734 & 6.74 & 7.133 & 6.766 & 5.857 \\ 
   &  & p=20 & 5.293 & 5.331 & 5.388 & 5.601 & 5.49 & 5.521 & 5.519 & 5.788 & 5.527 & 5.397 & 6.059 & 6.296 & 6.596 & 7.638 & 7.256 & 7.265 & 7.223 & 8.77 & 7.278 & 5.974 \\ 
   &  & p=36 & 5.293 & 5.331 & 7.399 & 23.274 & 5.618 & 10.773 & 11.711 & 29.391 & 13.259 & 28.219 & 6.059 & 6.296 & 8.313 & 13.501 & 7.481 & 7.998 & 7.921 & 18.91 & 8.402 & 13.893 \\ 
  \cmidrule{2-23} & \multirow{3}[2]{*}{$\rho=0.5$} & p=12 & 6.037 & 6.06 & 6.072 & 6.111 & 6.104 & 6.105 & 6.103 & 6.146 & 6.105 & 6.083 & 5.784 & 6.036 & 6.178 & 6.574 & 6.619 & 6.523 & 6.474 & 6.985 & 6.504 & 5.621 \\ 
   &  & p=20 & 6.04 & 6.065 & 6.122 & 6.336 & 6.213 & 6.246 & 6.251 & 6.512 & 6.258 & 6.117 & 5.792 & 6.049 & 6.359 & 7.449 & 7.058 & 7.035 & 6.985 & 8.609 & 7.047 & 5.698 \\ 
   &  & p=36 & 6.04 & 6.065 & 8.091 & 23.743 & 6.348 & 11.521 & 12.491 & 29.976 & 13.805 & 28.566 & 5.792 & 6.049 & 8.086 & 13.329 & 7.286 & 7.71 & 7.695 & 18.776 & 8.158 & 13.413 \\ 
  \cmidrule{2-23} & \multirow{3}[2]{*}{$\rho=0.9$} & p=12 & 6.627 & 6.644 & 6.657 & 6.695 & 6.687 & 6.687 & 6.683 & 6.729 & 6.685 & 6.663 & 4.474 & 4.784 & 5.037 & 5.499 & 5.406 & 5.434 & 5.359 & 5.956 & 5.386 & 4.1 \\ 
   &  & p=20 & 6.629 & 6.646 & 6.701 & 6.918 & 6.783 & 6.818 & 6.803 & 7.065 & 6.811 & 6.686 & 4.494 & 4.807 & 5.2 & 6.414 & 5.774 & 5.95 & 5.825 & 7.509 & 5.903 & 4.172 \\ 
   &  & p=36 & 6.63 & 6.648 & 8.592 & 24.31 & 6.969 & 11.87 & 12.962 & 30.559 & 14.062 & 28.56 & 4.494 & 4.821 & 7.116 & 12.436 & 5.973 & 6.663 & 6.532 & 18.064 & 7.037 & 11.738 \\ 
  \midrule\multirow{9}[6]{*}{lsnr} & \multirow{3}[2]{*}{$\rho=0$} & p=12 & 6.905 & 6.934 & 6.96 & 6.993 & 6.975 & 6.979 & 6.982 & 7.02 & 6.984 & 6.877 & 2 & 2.687 & 3.516 & 4.149 & 3.765 & 3.877 & 3.843 & 4.606 & 3.917 & 0.868 \\ 
   &  & p=20 & 6.905 & 6.938 & 7.006 & 7.193 & 7.069 & 7.107 & 7.099 & 7.359 & 7.104 & 6.896 & 2 & 2.692 & 3.612 & 4.995 & 4.16 & 4.364 & 4.233 & 6.274 & 4.322 & 0.907 \\ 
   &  & p=36 & 6.905 & 6.938 & 8.993 & 24.862 & 7.167 & 12.337 & 13.146 & 30.773 & 14.252 & 27.66 & 2 & 2.692 & 5.803 & 11.218 & 4.339 & 5.131 & 4.93 & 16.947 & 5.32 & 8.123 \\ 
  \cmidrule{2-23} & \multirow{3}[2]{*}{$\rho=0.5$} & p=12 & 7.625 & 7.648 & 7.673 & 7.703 & 7.682 & 7.693 & 7.693 & 7.733 & 7.696 & 7.602 & 2.225 & 2.756 & 3.436 & 3.965 & 3.649 & 3.799 & 3.737 & 4.453 & 3.827 & 1.018 \\ 
   &  & p=20 & 7.625 & 7.654 & 7.719 & 7.905 & 7.769 & 7.816 & 7.808 & 8.055 & 7.823 & 7.619 & 2.225 & 2.779 & 3.6 & 4.908 & 4.039 & 4.264 & 4.14 & 6.022 & 4.275 & 1.047 \\ 
   &  & p=36 & 7.625 & 7.654 & 9.662 & 25.296 & 7.868 & 13.031 & 13.793 & 31.367 & 15.022 & 27.948 & 2.225 & 2.779 & 5.72 & 11.083 & 4.202 & 4.939 & 4.76 & 16.765 & 5.298 & 7.924 \\ 
  \cmidrule{2-23} & \multirow{3}[2]{*}{$\rho=0.9$} & p=12 & 8.165 & 8.182 & 8.211 & 8.249 & 8.218 & 8.236 & 8.231 & 8.276 & 8.236 & 8.163 & 2.014 & 2.305 & 2.747 & 3.242 & 2.96 & 3.06 & 3.052 & 3.649 & 3.116 & 1.176 \\ 
   &  & p=20 & 8.163 & 8.185 & 8.26 & 8.446 & 8.304 & 8.335 & 8.328 & 8.58 & 8.346 & 8.173 & 1.993 & 2.318 & 2.982 & 4.107 & 3.35 & 3.459 & 3.411 & 5.144 & 3.564 & 1.172 \\ 
   &  & p=36 & 8.163 & 8.185 & 10.132 & 25.847 & 8.44 & 13.124 & 14.328 & 31.793 & 14.938 & 28.262 & 1.995 & 2.322 & 5.308 & 10.713 & 3.406 & 4.162 & 4.098 & 16.011 & 4.403 & 7.76 \\ 
   \bottomrule 
\end{tabular}
}
\end{table}

% latex table generated in R 4.0.2 by xtable 1.8-4 package
% Fri Jun 26 10:43:45 2020
\begin{table}[ht]
\centering
\caption{Fixed X, Sparse-Ex2, n=200} 
\scalebox{0.52}{
\begin{tabular}{|c|c|c|cc|cc|cc|ccc|c||cc|cc|cc|ccc|c|}
  \toprule 
 \multicolumn{1}{|c}{} & \multicolumn{1}{c}{} &       & RAICc & AICc  & RC$_p$ & C$_p$ & 10F CV & LOO CV & S$_p$ & $\widehat{\text{C}}_p$ & GCV   & BIC   & RAICc & AICc  & RC$_p$ & C$_p$ & 10F CV & LOO CV & S$_p$ & $\widehat{\text{C}}_p$ & GCV   & BIC  \\
 \cmidrule{4-23}\multicolumn{1}{|c}{} & \multicolumn{1}{c}{} &       & \multicolumn{10}{c||}{RMSE}                                                   & \multicolumn{10}{c|}{Relative Risk}       \\
 \midrule\multirow{9}[6]{*}{hsnr} & \multirow{3}[2]{*}{$\rho=0$} & p=12 & 0.554 & 0.558 & 0.561 & 0.565 & 0.564 & 0.565 & 0.564 & 0.57 & 0.564 & 0.505 & 0.005 & 0.005 & 0.005 & 0.006 & 0.005 & 0.006 & 0.006 & 0.006 & 0.006 & 0.004 \\ 
   &  & p=100 & 0.562 & 0.568 & 0.574 & 0.588 & 0.584 & 0.585 & 0.582 & 0.598 & 0.582 & 0.505 & 0.006 & 0.006 & 0.006 & 0.006 & 0.006 & 0.006 & 0.006 & 0.007 & 0.006 & 0.004 \\ 
   &  & p=196 & 0.562 & 0.568 & 0.937 & 1.309 & 0.584 & 0.585 & 0.582 & 1.589 & 0.585 & 0.89 & 0.006 & 0.006 & 0.042 & 1.813 & 0.006 & 0.006 & 0.006 & 3.053 & 0.009 & 1.452 \\ 
  \cmidrule{2-23} & \multirow{3}[2]{*}{$\rho=0.5$} & p=12 & 0.807 & 0.811 & 0.816 & 0.821 & 0.822 & 0.819 & 0.821 & 0.828 & 0.821 & 0.738 & 0.005 & 0.005 & 0.005 & 0.005 & 0.005 & 0.005 & 0.005 & 0.006 & 0.005 & 0.004 \\ 
   &  & p=100 & 0.819 & 0.828 & 0.837 & 0.858 & 0.853 & 0.851 & 0.844 & 0.875 & 0.846 & 0.738 & 0.006 & 0.006 & 0.006 & 0.006 & 0.006 & 0.006 & 0.006 & 0.007 & 0.006 & 0.004 \\ 
   &  & p=196 & 0.819 & 0.828 & 1.371 & 1.913 & 0.853 & 0.851 & 0.844 & 2.313 & 0.85 & 1.303 & 0.006 & 0.006 & 0.042 & 1.855 & 0.006 & 0.006 & 0.006 & 3.039 & 0.009 & 1.466 \\ 
  \cmidrule{2-23} & \multirow{3}[2]{*}{$\rho=0.9$} & p=12 & 1.097 & 1.104 & 1.109 & 1.116 & 1.109 & 1.114 & 1.115 & 1.122 & 1.115 & 1.006 & 0.005 & 0.005 & 0.005 & 0.005 & 0.005 & 0.005 & 0.005 & 0.005 & 0.005 & 0.004 \\ 
   &  & p=100 & 1.106 & 1.125 & 1.13 & 1.167 & 1.151 & 1.157 & 1.153 & 1.178 & 1.155 & 1.006 & 0.005 & 0.006 & 0.006 & 0.006 & 0.006 & 0.006 & 0.006 & 0.006 & 0.006 & 0.004 \\ 
   &  & p=196 & 1.106 & 1.125 & 1.869 & 2.595 & 1.151 & 1.157 & 1.153 & 3.108 & 1.165 & 1.768 & 0.005 & 0.006 & 0.041 & 1.82 & 0.006 & 0.006 & 0.006 & 3 & 0.013 & 1.446 \\ 
  \midrule\multirow{9}[6]{*}{msnr} & \multirow{3}[2]{*}{$\rho=0$} & p=12 & 1.616 & 1.628 & 1.634 & 1.647 & 1.644 & 1.646 & 1.644 & 1.66 & 1.644 & 1.473 & 0.045 & 0.046 & 0.046 & 0.047 & 0.046 & 0.047 & 0.047 & 0.048 & 0.047 & 0.037 \\ 
   &  & p=100 & 1.638 & 1.657 & 1.674 & 1.715 & 1.701 & 1.706 & 1.696 & 1.742 & 1.696 & 1.473 & 0.047 & 0.049 & 0.05 & 0.054 & 0.052 & 0.053 & 0.052 & 0.056 & 0.052 & 0.037 \\ 
   &  & p=196 & 1.638 & 1.657 & 2.732 & 3.815 & 1.701 & 1.706 & 1.696 & 4.633 & 1.704 & 2.594 & 0.047 & 0.049 & 0.353 & 15.407 & 0.052 & 0.053 & 0.052 & 25.948 & 0.077 & 12.343 \\ 
  \cmidrule{2-23} & \multirow{3}[2]{*}{$\rho=0.5$} & p=12 & 2.356 & 2.366 & 2.379 & 2.396 & 2.399 & 2.389 & 2.393 & 2.413 & 2.395 & 2.174 & 0.045 & 0.046 & 0.046 & 0.047 & 0.047 & 0.046 & 0.047 & 0.047 & 0.047 & 0.039 \\ 
   &  & p=100 & 2.389 & 2.416 & 2.44 & 2.501 & 2.491 & 2.482 & 2.462 & 2.552 & 2.467 & 2.174 & 0.047 & 0.048 & 0.05 & 0.054 & 0.053 & 0.053 & 0.051 & 0.057 & 0.052 & 0.039 \\ 
   &  & p=196 & 2.389 & 2.416 & 3.999 & 5.58 & 2.491 & 2.482 & 2.462 & 6.743 & 2.478 & 3.817 & 0.047 & 0.048 & 0.354 & 15.763 & 0.053 & 0.053 & 0.051 & 25.834 & 0.076 & 12.46 \\ 
  \cmidrule{2-23} & \multirow{3}[2]{*}{$\rho=0.9$} & p=12 & 3.317 & 3.328 & 3.341 & 3.354 & 3.347 & 3.361 & 3.349 & 3.366 & 3.349 & 3.453 & 0.047 & 0.048 & 0.048 & 0.049 & 0.048 & 0.049 & 0.048 & 0.049 & 0.048 & 0.051 \\ 
   &  & p=100 & 3.338 & 3.392 & 3.405 & 3.497 & 3.458 & 3.483 & 3.457 & 3.526 & 3.464 & 3.451 & 0.049 & 0.051 & 0.051 & 0.056 & 0.053 & 0.055 & 0.053 & 0.057 & 0.054 & 0.051 \\ 
   &  & p=196 & 3.338 & 3.392 & 5.546 & 7.643 & 3.458 & 3.483 & 3.457 & 9.08 & 3.492 & 5.561 & 0.049 & 0.051 & 0.35 & 15.467 & 0.053 & 0.055 & 0.053 & 25.456 & 0.11 & 12.042 \\ 
  \midrule\multirow{9}[6]{*}{lsnr} & \multirow{3}[2]{*}{$\rho=0$} & p=12 & 3.639 & 3.664 & 3.676 & 3.705 & 3.704 & 3.702 & 3.695 & 3.733 & 3.696 & 4.327 & 0.23 & 0.233 & 0.235 & 0.238 & 0.237 & 0.238 & 0.237 & 0.241 & 0.237 & 0.341 \\ 
   &  & p=100 & 3.683 & 3.727 & 3.762 & 3.854 & 3.83 & 3.834 & 3.81 & 3.913 & 3.811 & 4.327 & 0.239 & 0.246 & 0.254 & 0.271 & 0.264 & 0.266 & 0.263 & 0.282 & 0.263 & 0.341 \\ 
   &  & p=196 & 3.683 & 3.727 & 6.25 & 8.655 & 3.83 & 3.834 & 3.81 & 10.373 & 3.829 & 6.695 & 0.239 & 0.246 & 1.787 & 77.054 & 0.264 & 0.266 & 0.263 & 129.744 & 0.387 & 61.378 \\ 
  \cmidrule{2-23} & \multirow{3}[2]{*}{$\rho=0.5$} & p=12 & 5.471 & 5.481 & 5.513 & 5.543 & 5.545 & 5.528 & 5.535 & 5.576 & 5.54 & 6.658 & 0.243 & 0.244 & 0.246 & 0.25 & 0.248 & 0.248 & 0.249 & 0.252 & 0.249 & 0.363 \\ 
   &  & p=100 & 5.541 & 5.597 & 5.644 & 5.752 & 5.724 & 5.725 & 5.683 & 5.863 & 5.692 & 6.658 & 0.252 & 0.258 & 0.266 & 0.281 & 0.274 & 0.277 & 0.271 & 0.297 & 0.272 & 0.363 \\ 
   &  & p=196 & 5.541 & 5.597 & 9.256 & 12.744 & 5.724 & 5.725 & 5.683 & 15.081 & 5.719 & 10.038 & 0.252 & 0.258 & 1.8 & 78.839 & 0.274 & 0.277 & 0.271 & 128.539 & 0.393 & 61.966 \\ 
  \cmidrule{2-23} & \multirow{3}[2]{*}{$\rho=0.9$} & p=12 & 7.545 & 7.557 & 7.575 & 7.583 & 7.588 & 7.548 & 7.586 & 7.592 & 7.585 & 9.362 & 0.238 & 0.239 & 0.241 & 0.242 & 0.241 & 0.24 & 0.242 & 0.243 & 0.242 & 0.329 \\ 
   &  & p=100 & 7.562 & 7.629 & 7.684 & 7.795 & 7.769 & 7.728 & 7.712 & 7.868 & 7.725 & 9.362 & 0.241 & 0.248 & 0.254 & 0.269 & 0.26 & 0.261 & 0.256 & 0.276 & 0.259 & 0.329 \\ 
   &  & p=196 & 7.562 & 7.629 & 12.643 & 17.313 & 7.769 & 7.728 & 7.712 & 20.168 & 7.787 & 13.645 & 0.241 & 0.248 & 1.755 & 77.057 & 0.26 & 0.261 & 0.256 & 126.394 & 0.537 & 58.637 \\ 
   \midrule 
 \multicolumn{1}{|c}{} & \multicolumn{1}{c}{} &       & \multicolumn{10}{c||}{Relative Test Error}                                    & \multicolumn{10}{c|}{Proportion of Variance Explained} \\
\midrule\multirow{9}[6]{*}{hsnr} & \multirow{3}[2]{*}{$\rho=0$} & p=12 & 1.041 & 1.042 & 1.042 & 1.043 & 1.043 & 1.043 & 1.043 & 1.044 & 1.043 & 1.034 & 0.881 & 0.881 & 0.881 & 0.881 & 0.881 & 0.881 & 0.881 & 0.881 & 0.881 & 0.882 \\ 
   &  & p=100 & 1.043 & 1.044 & 1.046 & 1.049 & 1.048 & 1.048 & 1.048 & 1.051 & 1.048 & 1.034 & 0.881 & 0.881 & 0.881 & 0.88 & 0.881 & 0.881 & 0.881 & 0.88 & 0.881 & 0.882 \\ 
   &  & p=196 & 1.043 & 1.044 & 1.323 & 15.091 & 1.048 & 1.048 & 1.048 & 24.733 & 1.07 & 12.289 & 0.881 & 0.881 & 0.849 & -0.72 & 0.881 & 0.881 & 0.881 & -1.819 & 0.878 & -0.401 \\ 
  \cmidrule{2-23} & \multirow{3}[2]{*}{$\rho=0.5$} & p=12 & 1.041 & 1.041 & 1.042 & 1.043 & 1.042 & 1.042 & 1.042 & 1.043 & 1.043 & 1.034 & 0.881 & 0.881 & 0.881 & 0.881 & 0.881 & 0.881 & 0.881 & 0.881 & 0.881 & 0.882 \\ 
   &  & p=100 & 1.043 & 1.044 & 1.046 & 1.049 & 1.048 & 1.048 & 1.047 & 1.052 & 1.047 & 1.034 & 0.881 & 0.88 & 0.88 & 0.88 & 0.88 & 0.88 & 0.88 & 0.88 & 0.88 & 0.882 \\ 
   &  & p=196 & 1.043 & 1.044 & 1.322 & 15.346 & 1.048 & 1.048 & 1.047 & 24.511 & 1.069 & 12.338 & 0.881 & 0.88 & 0.849 & -0.757 & 0.88 & 0.88 & 0.88 & -1.806 & 0.878 & -0.412 \\ 
  \cmidrule{2-23} & \multirow{3}[2]{*}{$\rho=0.9$} & p=12 & 1.041 & 1.042 & 1.042 & 1.043 & 1.042 & 1.042 & 1.042 & 1.043 & 1.042 & 1.035 & 0.882 & 0.882 & 0.882 & 0.882 & 0.882 & 0.882 & 0.882 & 0.882 & 0.882 & 0.883 \\ 
   &  & p=100 & 1.042 & 1.044 & 1.045 & 1.049 & 1.047 & 1.048 & 1.047 & 1.051 & 1.048 & 1.035 & 0.882 & 0.882 & 0.882 & 0.881 & 0.882 & 0.881 & 0.882 & 0.881 & 0.881 & 0.883 \\ 
   &  & p=196 & 1.042 & 1.044 & 1.32 & 15.263 & 1.047 & 1.048 & 1.047 & 24.518 & 1.099 & 12.334 & 0.882 & 0.882 & 0.851 & -0.727 & 0.882 & 0.881 & 0.882 & -1.774 & 0.876 & -0.395 \\ 
  \midrule\multirow{9}[6]{*}{msnr} & \multirow{3}[2]{*}{$\rho=0$} & p=12 & 1.041 & 1.042 & 1.042 & 1.043 & 1.043 & 1.043 & 1.043 & 1.044 & 1.043 & 1.034 & 0.456 & 0.456 & 0.456 & 0.455 & 0.455 & 0.455 & 0.455 & 0.455 & 0.455 & 0.46 \\ 
   &  & p=100 & 1.043 & 1.044 & 1.046 & 1.049 & 1.048 & 1.048 & 1.048 & 1.051 & 1.048 & 1.034 & 0.455 & 0.455 & 0.454 & 0.452 & 0.453 & 0.453 & 0.453 & 0.451 & 0.453 & 0.46 \\ 
   &  & p=196 & 1.043 & 1.044 & 1.323 & 15.091 & 1.048 & 1.048 & 1.048 & 24.733 & 1.07 & 12.289 & 0.455 & 0.455 & 0.309 & -6.882 & 0.453 & 0.453 & 0.453 & -11.918 & 0.441 & -5.419 \\ 
  \cmidrule{2-23} & \multirow{3}[2]{*}{$\rho=0.5$} & p=12 & 1.041 & 1.041 & 1.042 & 1.043 & 1.042 & 1.042 & 1.042 & 1.043 & 1.043 & 1.035 & 0.455 & 0.455 & 0.455 & 0.454 & 0.454 & 0.454 & 0.454 & 0.454 & 0.454 & 0.458 \\ 
   &  & p=100 & 1.043 & 1.044 & 1.046 & 1.049 & 1.048 & 1.048 & 1.047 & 1.052 & 1.047 & 1.035 & 0.454 & 0.453 & 0.453 & 0.451 & 0.451 & 0.451 & 0.452 & 0.449 & 0.452 & 0.458 \\ 
   &  & p=196 & 1.043 & 1.044 & 1.323 & 15.346 & 1.048 & 1.048 & 1.047 & 24.511 & 1.069 & 12.339 & 0.454 & 0.453 & 0.308 & -7.034 & 0.451 & 0.451 & 0.452 & -11.833 & 0.44 & -5.46 \\ 
  \cmidrule{2-23} & \multirow{3}[2]{*}{$\rho=0.9$} & p=12 & 1.044 & 1.044 & 1.044 & 1.045 & 1.044 & 1.045 & 1.045 & 1.045 & 1.045 & 1.047 & 0.457 & 0.457 & 0.457 & 0.456 & 0.457 & 0.456 & 0.457 & 0.456 & 0.457 & 0.455 \\ 
   &  & p=100 & 1.045 & 1.047 & 1.047 & 1.051 & 1.049 & 1.051 & 1.049 & 1.053 & 1.05 & 1.047 & 0.456 & 0.455 & 0.455 & 0.453 & 0.454 & 0.453 & 0.454 & 0.452 & 0.454 & 0.455 \\ 
   &  & p=196 & 1.045 & 1.047 & 1.322 & 15.264 & 1.049 & 1.051 & 1.049 & 24.475 & 1.101 & 12.105 & 0.456 & 0.455 & 0.312 & -6.941 & 0.454 & 0.453 & 0.454 & -11.733 & 0.427 & -5.297 \\ 
  \midrule\multirow{9}[6]{*}{lsnr} & \multirow{3}[2]{*}{$\rho=0$} & p=12 & 1.042 & 1.043 & 1.043 & 1.044 & 1.043 & 1.043 & 1.043 & 1.044 & 1.043 & 1.062 & 0.119 & 0.119 & 0.118 & 0.118 & 0.118 & 0.118 & 0.118 & 0.117 & 0.118 & 0.102 \\ 
   &  & p=100 & 1.044 & 1.045 & 1.046 & 1.05 & 1.048 & 1.049 & 1.048 & 1.052 & 1.048 & 1.062 & 0.118 & 0.117 & 0.115 & 0.113 & 0.114 & 0.113 & 0.114 & 0.111 & 0.114 & 0.102 \\ 
   &  & p=196 & 1.044 & 1.045 & 1.327 & 15.095 & 1.048 & 1.049 & 1.048 & 24.733 & 1.071 & 12.227 & 0.118 & 0.117 & -0.122 & -11.761 & 0.114 & 0.113 & 0.114 & -19.908 & 0.095 & -9.337 \\ 
  \cmidrule{2-23} & \multirow{3}[2]{*}{$\rho=0.5$} & p=12 & 1.044 & 1.044 & 1.045 & 1.045 & 1.045 & 1.045 & 1.045 & 1.046 & 1.045 & 1.066 & 0.117 & 0.116 & 0.116 & 0.116 & 0.116 & 0.116 & 0.116 & 0.115 & 0.116 & 0.098 \\ 
   &  & p=100 & 1.046 & 1.047 & 1.048 & 1.051 & 1.05 & 1.05 & 1.049 & 1.054 & 1.05 & 1.066 & 0.115 & 0.114 & 0.113 & 0.111 & 0.112 & 0.111 & 0.112 & 0.108 & 0.112 & 0.098 \\ 
   &  & p=196 & 1.046 & 1.047 & 1.328 & 15.35 & 1.05 & 1.05 & 1.049 & 24.396 & 1.071 & 12.279 & 0.115 & 0.114 & -0.123 & -11.986 & 0.112 & 0.111 & 0.112 & -19.639 & 0.094 & -9.388 \\ 
  \cmidrule{2-23} & \multirow{3}[2]{*}{$\rho=0.9$} & p=12 & 1.044 & 1.044 & 1.044 & 1.045 & 1.044 & 1.044 & 1.045 & 1.045 & 1.045 & 1.061 & 0.119 & 0.119 & 0.118 & 0.118 & 0.118 & 0.118 & 0.118 & 0.118 & 0.118 & 0.105 \\ 
   &  & p=100 & 1.044 & 1.046 & 1.047 & 1.05 & 1.048 & 1.048 & 1.047 & 1.051 & 1.048 & 1.061 & 0.118 & 0.117 & 0.116 & 0.114 & 0.115 & 0.115 & 0.116 & 0.113 & 0.115 & 0.105 \\ 
   &  & p=196 & 1.044 & 1.046 & 1.324 & 15.212 & 1.048 & 1.048 & 1.047 & 24.312 & 1.099 & 11.815 & 0.118 & 0.117 & -0.118 & -11.843 & 0.115 & 0.115 & 0.116 & -19.526 & 0.072 & -8.975 \\ 
   \midrule 
 \multicolumn{1}{|c}{} & \multicolumn{1}{c}{} &       & \multicolumn{10}{c||}{Kullback-Leibler Discrepancy}                                    & \multicolumn{10}{c|}{Number of Variables} \\
\midrule\multirow{9}[6]{*}{hsnr} & \multirow{3}[2]{*}{$\rho=0$} & p=12 & 3.244 & 3.245 & 3.245 & 3.246 & 3.246 & 3.246 & 3.246 & 3.247 & 3.246 & 3.236 & 6.508 & 6.562 & 6.595 & 6.657 & 6.7 & 6.66 & 6.639 & 6.726 & 6.64 & 6.032 \\ 
   &  & p=100 & 3.246 & 3.248 & 3.25 & 3.254 & 3.252 & 3.253 & 3.252 & 3.256 & 3.252 & 3.236 & 6.628 & 6.728 & 6.867 & 7.144 & 7.101 & 7.068 & 6.993 & 7.323 & 6.994 & 6.032 \\ 
   &  & p=196 & 3.246 & 3.248 & 3.692 & 107.663 & 3.252 & 3.253 & 3.252 & 147.06 & 4.497 & 110.571 & 6.628 & 6.728 & 25.908 & 59.763 & 7.101 & 7.068 & 6.993 & 84.379 & 7.18 & 35.067 \\ 
  \cmidrule{2-23} & \multirow{3}[2]{*}{$\rho=0.5$} & p=12 & 4 & 4 & 4.001 & 4.002 & 4.001 & 4.001 & 4.002 & 4.002 & 4.002 & 3.992 & 6.491 & 6.521 & 6.565 & 6.625 & 6.675 & 6.607 & 6.616 & 6.687 & 6.623 & 6.033 \\ 
   &  & p=100 & 4.002 & 4.003 & 4.005 & 4.009 & 4.008 & 4.008 & 4.007 & 4.013 & 4.007 & 3.992 & 6.612 & 6.708 & 6.843 & 7.134 & 7.119 & 7.034 & 6.924 & 7.369 & 6.943 & 6.033 \\ 
   &  & p=196 & 4.002 & 4.003 & 4.443 & 108.194 & 4.008 & 4.008 & 4.007 & 147.889 & 4.993 & 111.537 & 6.612 & 6.708 & 25.921 & 59.65 & 7.119 & 7.034 & 6.924 & 84.047 & 7.128 & 35.255 \\ 
  \cmidrule{2-23} & \multirow{3}[2]{*}{$\rho=0.9$} & p=12 & 4.609 & 4.61 & 4.61 & 4.611 & 4.61 & 4.611 & 4.611 & 4.611 & 4.611 & 4.602 & 6.497 & 6.544 & 6.581 & 6.629 & 6.641 & 6.628 & 6.624 & 6.671 & 6.624 & 6.02 \\ 
   &  & p=100 & 4.611 & 4.613 & 4.613 & 4.619 & 4.616 & 4.618 & 4.616 & 4.621 & 4.617 & 4.602 & 6.579 & 6.736 & 6.789 & 7.176 & 7.073 & 7.078 & 6.983 & 7.273 & 7.022 & 6.018 \\ 
   &  & p=196 & 4.611 & 4.613 & 5.031 & 107.89 & 4.616 & 4.618 & 4.616 & 146.532 & 8.365 & 112.263 & 6.579 & 6.736 & 26.084 & 59.637 & 7.073 & 7.078 & 6.983 & 83.068 & 7.393 & 35.245 \\ 
  \midrule\multirow{9}[6]{*}{msnr} & \multirow{3}[2]{*}{$\rho=0$} & p=12 & 5.384 & 5.385 & 5.385 & 5.386 & 5.386 & 5.386 & 5.386 & 5.387 & 5.386 & 5.376 & 6.508 & 6.562 & 6.595 & 6.657 & 6.7 & 6.66 & 6.639 & 6.726 & 6.64 & 6.032 \\ 
   &  & p=100 & 5.386 & 5.388 & 5.39 & 5.394 & 5.392 & 5.393 & 5.392 & 5.396 & 5.392 & 5.376 & 6.628 & 6.728 & 6.867 & 7.144 & 7.101 & 7.068 & 6.993 & 7.323 & 6.994 & 6.032 \\ 
   &  & p=196 & 5.386 & 5.388 & 5.832 & 109.803 & 5.392 & 5.393 & 5.392 & 149.2 & 6.637 & 112.711 & 6.628 & 6.728 & 25.908 & 59.763 & 7.101 & 7.068 & 6.993 & 84.379 & 7.18 & 35.067 \\ 
  \cmidrule{2-23} & \multirow{3}[2]{*}{$\rho=0.5$} & p=12 & 6.14 & 6.14 & 6.141 & 6.142 & 6.142 & 6.142 & 6.142 & 6.143 & 6.142 & 6.133 & 6.489 & 6.52 & 6.564 & 6.624 & 6.672 & 6.606 & 6.615 & 6.686 & 6.622 & 6.016 \\ 
   &  & p=100 & 6.142 & 6.143 & 6.145 & 6.149 & 6.148 & 6.148 & 6.147 & 6.153 & 6.147 & 6.133 & 6.61 & 6.707 & 6.842 & 7.133 & 7.116 & 7.033 & 6.923 & 7.368 & 6.942 & 6.016 \\ 
   &  & p=196 & 6.142 & 6.143 & 6.584 & 110.334 & 6.148 & 6.148 & 6.147 & 150.029 & 7.133 & 113.677 & 6.61 & 6.707 & 25.919 & 59.648 & 7.116 & 7.033 & 6.923 & 84.047 & 7.127 & 35.24 \\ 
  \cmidrule{2-23} & \multirow{3}[2]{*}{$\rho=0.9$} & p=12 & 6.751 & 6.752 & 6.752 & 6.753 & 6.752 & 6.753 & 6.752 & 6.753 & 6.752 & 6.752 & 6.114 & 6.164 & 6.2 & 6.274 & 6.233 & 6.277 & 6.262 & 6.324 & 6.262 & 5.178 \\ 
   &  & p=100 & 6.752 & 6.755 & 6.755 & 6.76 & 6.758 & 6.76 & 6.758 & 6.762 & 6.759 & 6.752 & 6.166 & 6.345 & 6.387 & 6.766 & 6.642 & 6.708 & 6.604 & 6.907 & 6.643 & 5.179 \\ 
   &  & p=196 & 6.752 & 6.755 & 7.173 & 110.03 & 6.758 & 6.76 & 6.758 & 148.473 & 10.507 & 113.5 & 6.166 & 6.345 & 25.669 & 59.149 & 6.642 & 6.708 & 6.604 & 82.493 & 7.014 & 33.953 \\ 
  \midrule\multirow{9}[6]{*}{lsnr} & \multirow{3}[2]{*}{$\rho=0$} & p=12 & 6.994 & 6.995 & 6.995 & 6.996 & 6.996 & 6.996 & 6.996 & 6.997 & 6.996 & 7.014 & 6.478 & 6.54 & 6.575 & 6.637 & 6.675 & 6.642 & 6.62 & 6.707 & 6.621 & 4.955 \\ 
   &  & p=100 & 6.996 & 6.998 & 7 & 7.004 & 7.002 & 7.003 & 7.002 & 7.006 & 7.002 & 7.014 & 6.602 & 6.708 & 6.848 & 7.126 & 7.079 & 7.051 & 6.975 & 7.305 & 6.976 & 4.955 \\ 
   &  & p=196 & 6.996 & 6.998 & 7.445 & 111.416 & 7.002 & 7.003 & 7.002 & 150.81 & 8.247 & 114.043 & 6.602 & 6.708 & 25.764 & 59.634 & 7.079 & 7.051 & 6.975 & 84.367 & 7.162 & 33.92 \\ 
  \cmidrule{2-23} & \multirow{3}[2]{*}{$\rho=0.5$} & p=12 & 7.752 & 7.752 & 7.753 & 7.754 & 7.753 & 7.753 & 7.753 & 7.754 & 7.754 & 7.772 & 6.224 & 6.262 & 6.311 & 6.375 & 6.382 & 6.352 & 6.368 & 6.452 & 6.375 & 4.694 \\ 
   &  & p=100 & 7.754 & 7.755 & 7.757 & 7.761 & 7.759 & 7.76 & 7.758 & 7.764 & 7.759 & 7.772 & 6.338 & 6.459 & 6.569 & 6.856 & 6.78 & 6.769 & 6.671 & 7.1 & 6.69 & 4.694 \\ 
   &  & p=196 & 7.754 & 7.755 & 8.198 & 111.947 & 7.759 & 7.76 & 7.758 & 151.213 & 8.745 & 115.009 & 6.338 & 6.459 & 25.586 & 59.252 & 6.78 & 6.769 & 6.671 & 83.144 & 6.875 & 33.914 \\ 
  \cmidrule{2-23} & \multirow{3}[2]{*}{$\rho=0.9$} & p=12 & 8.361 & 8.361 & 8.362 & 8.362 & 8.362 & 8.362 & 8.362 & 8.362 & 8.362 & 8.378 & 4.854 & 4.941 & 4.997 & 5.086 & 5.028 & 5.058 & 5.081 & 5.138 & 5.083 & 2.592 \\ 
   &  & p=100 & 8.362 & 8.363 & 8.365 & 8.368 & 8.366 & 8.366 & 8.365 & 8.37 & 8.366 & 8.378 & 4.895 & 5.056 & 5.199 & 5.495 & 5.374 & 5.372 & 5.29 & 5.667 & 5.327 & 2.592 \\ 
   &  & p=196 & 8.362 & 8.363 & 8.784 & 111.451 & 8.366 & 8.366 & 8.365 & 149.768 & 12.114 & 113.836 & 4.895 & 5.056 & 24.607 & 57.972 & 5.374 & 5.372 & 5.29 & 81.065 & 5.698 & 31.021 \\ 
   \bottomrule 
\end{tabular}
}
\end{table}

% latex table generated in R 4.0.2 by xtable 1.8-4 package
% Fri Jun 26 10:43:50 2020
\begin{table}[ht]
\centering
\caption{Random X, Sparse-Ex2, n=1000} 
\scalebox{0.52}{
\begin{tabular}{|c|c|c|cc|cc|cc|ccc|c||cc|cc|cc|ccc|c|}
  \toprule 
 \multicolumn{1}{|c}{} & \multicolumn{1}{c}{} &       & RAICc & AICc  & RC$_p$ & C$_p$ & 10F CV & LOO CV & S$_p$ & $\widehat{\text{C}}_p$ & GCV   & BIC   & RAICc & AICc  & RC$_p$ & C$_p$ & 10F CV & LOO CV & S$_p$ & $\widehat{\text{C}}_p$ & GCV   & BIC  \\
 \cmidrule{4-23}\multicolumn{1}{|c}{} & \multicolumn{1}{c}{} &       & \multicolumn{10}{c||}{RMSE}                                                   & \multicolumn{10}{c|}{Relative Risk}       \\
 \midrule\multirow{9}[6]{*}{hsnr} & \multirow{3}[2]{*}{$\rho=0$} & p=12 & 0.239 & 0.239 & 0.239 & 0.24 & 0.239 & 0.24 & 0.24 & 0.24 & 0.24 & 0.211 & 0.001 & 0.001 & 0.001 & 0.001 & 0.001 & 0.001 & 0.001 & 0.001 & 0.001 & 0.001 \\ 
   &  & p=500 & 0.245 & 0.246 & 0.249 & 0.249 & 0.247 & 0.248 & 0.247 & 0.248 & 0.248 & 0.211 & 0.001 & 0.001 & 0.001 & 0.001 & 0.001 & 0.001 & 0.001 & 0.001 & 0.001 & 0.001 \\ 
   &  & p=996 & 0.245 & 0.246 & 0.63 & 0.994 & 0.247 & 0.248 & 0.247 & 1.233 & 0.248 & 0.471 & 0.001 & 0.001 & 0.026 & 6.81 & 0.001 & 0.001 & 0.001 & 12.404 & 0.001 & 3.809 \\ 
  \cmidrule{2-23} & \multirow{3}[2]{*}{$\rho=0.5$} & p=12 & 0.348 & 0.349 & 0.349 & 0.349 & 0.348 & 0.35 & 0.349 & 0.35 & 0.349 & 0.308 & 0.001 & 0.001 & 0.001 & 0.001 & 0.001 & 0.001 & 0.001 & 0.001 & 0.001 & 0.001 \\ 
   &  & p=500 & 0.358 & 0.359 & 0.363 & 0.365 & 0.36 & 0.361 & 0.361 & 0.362 & 0.361 & 0.308 & 0.001 & 0.001 & 0.001 & 0.001 & 0.001 & 0.001 & 0.001 & 0.001 & 0.001 & 0.001 \\ 
   &  & p=996 & 0.358 & 0.359 & 0.914 & 1.447 & 0.36 & 0.361 & 0.361 & 1.794 & 0.361 & 0.685 & 0.001 & 0.001 & 0.026 & 6.783 & 0.001 & 0.001 & 0.001 & 11.988 & 0.001 & 3.813 \\ 
  \cmidrule{2-23} & \multirow{3}[2]{*}{$\rho=0.9$} & p=12 & 0.475 & 0.476 & 0.476 & 0.477 & 0.475 & 0.477 & 0.477 & 0.477 & 0.477 & 0.42 & 0.001 & 0.001 & 0.001 & 0.001 & 0.001 & 0.001 & 0.001 & 0.001 & 0.001 & 0.001 \\ 
   &  & p=500 & 0.486 & 0.489 & 0.494 & 0.498 & 0.489 & 0.489 & 0.49 & 0.492 & 0.49 & 0.421 & 0.001 & 0.001 & 0.001 & 0.001 & 0.001 & 0.001 & 0.001 & 0.001 & 0.001 & 0.001 \\ 
   &  & p=996 & 0.486 & 0.489 & 1.25 & 1.973 & 0.489 & 0.489 & 0.49 & 2.447 & 0.49 & 0.936 & 0.001 & 0.001 & 0.026 & 6.674 & 0.001 & 0.001 & 0.001 & 11.982 & 0.001 & 3.8 \\ 
  \midrule\multirow{9}[6]{*}{msnr} & \multirow{3}[2]{*}{$\rho=0$} & p=12 & 0.697 & 0.697 & 0.698 & 0.699 & 0.698 & 0.7 & 0.699 & 0.699 & 0.699 & 0.617 & 0.008 & 0.008 & 0.008 & 0.008 & 0.008 & 0.008 & 0.008 & 0.008 & 0.008 & 0.006 \\ 
   &  & p=500 & 0.715 & 0.716 & 0.725 & 0.727 & 0.72 & 0.722 & 0.721 & 0.724 & 0.722 & 0.617 & 0.008 & 0.008 & 0.009 & 0.009 & 0.009 & 0.009 & 0.009 & 0.009 & 0.009 & 0.006 \\ 
   &  & p=996 & 0.715 & 0.716 & 1.837 & 2.898 & 0.72 & 0.722 & 0.721 & 3.596 & 0.722 & 1.372 & 0.008 & 0.008 & 0.222 & 57.889 & 0.009 & 0.009 & 0.009 & 105.436 & 0.009 & 32.38 \\ 
  \cmidrule{2-23} & \multirow{3}[2]{*}{$\rho=0.5$} & p=12 & 1.016 & 1.018 & 1.018 & 1.018 & 1.015 & 1.019 & 1.018 & 1.02 & 1.018 & 0.899 & 0.008 & 0.008 & 0.008 & 0.008 & 0.008 & 0.008 & 0.008 & 0.008 & 0.008 & 0.006 \\ 
   &  & p=500 & 1.043 & 1.046 & 1.06 & 1.063 & 1.049 & 1.054 & 1.052 & 1.056 & 1.052 & 0.899 & 0.008 & 0.008 & 0.009 & 0.009 & 0.009 & 0.009 & 0.009 & 0.009 & 0.009 & 0.006 \\ 
   &  & p=996 & 1.043 & 1.046 & 2.665 & 4.217 & 1.049 & 1.054 & 1.052 & 5.231 & 1.052 & 1.998 & 0.008 & 0.008 & 0.221 & 57.659 & 0.009 & 0.009 & 0.009 & 101.899 & 0.009 & 32.414 \\ 
  \cmidrule{2-23} & \multirow{3}[2]{*}{$\rho=0.9$} & p=12 & 1.389 & 1.39 & 1.391 & 1.395 & 1.388 & 1.395 & 1.395 & 1.395 & 1.395 & 1.3 & 0.008 & 0.008 & 0.008 & 0.008 & 0.008 & 0.008 & 0.008 & 0.008 & 0.008 & 0.007 \\ 
   &  & p=500 & 1.419 & 1.43 & 1.442 & 1.452 & 1.429 & 1.43 & 1.432 & 1.438 & 1.432 & 1.302 & 0.008 & 0.009 & 0.009 & 0.009 & 0.009 & 0.009 & 0.009 & 0.009 & 0.009 & 0.007 \\ 
   &  & p=996 & 1.419 & 1.43 & 3.649 & 5.758 & 1.429 & 1.43 & 1.432 & 7.137 & 1.432 & 2.793 & 0.008 & 0.009 & 0.223 & 56.732 & 0.009 & 0.009 & 0.009 & 101.848 & 0.009 & 32.305 \\ 
  \midrule\multirow{9}[6]{*}{lsnr} & \multirow{3}[2]{*}{$\rho=0$} & p=12 & 1.558 & 1.559 & 1.561 & 1.563 & 1.56 & 1.565 & 1.563 & 1.564 & 1.563 & 1.379 & 0.039 & 0.039 & 0.039 & 0.039 & 0.039 & 0.039 & 0.039 & 0.039 & 0.039 & 0.03 \\ 
   &  & p=500 & 1.598 & 1.602 & 1.62 & 1.625 & 1.611 & 1.615 & 1.611 & 1.619 & 1.614 & 1.379 & 0.042 & 0.042 & 0.043 & 0.044 & 0.043 & 0.043 & 0.043 & 0.043 & 0.043 & 0.03 \\ 
   &  & p=996 & 1.598 & 1.602 & 4.108 & 6.48 & 1.611 & 1.615 & 1.611 & 8.041 & 1.614 & 3.068 & 0.042 & 0.042 & 1.108 & 289.446 & 0.043 & 0.043 & 0.043 & 527.182 & 0.043 & 161.899 \\ 
  \cmidrule{2-23} & \multirow{3}[2]{*}{$\rho=0.5$} & p=12 & 2.272 & 2.275 & 2.276 & 2.277 & 2.269 & 2.279 & 2.277 & 2.281 & 2.277 & 2.035 & 0.039 & 0.039 & 0.039 & 0.039 & 0.039 & 0.039 & 0.039 & 0.039 & 0.039 & 0.031 \\ 
   &  & p=500 & 2.333 & 2.338 & 2.37 & 2.377 & 2.346 & 2.356 & 2.352 & 2.361 & 2.352 & 2.035 & 0.042 & 0.042 & 0.044 & 0.044 & 0.043 & 0.043 & 0.043 & 0.044 & 0.043 & 0.031 \\ 
   &  & p=996 & 2.333 & 2.338 & 5.96 & 9.432 & 2.346 & 2.356 & 2.352 & 11.696 & 2.352 & 4.486 & 0.042 & 0.042 & 1.104 & 288.297 & 0.043 & 0.043 & 0.043 & 509.496 & 0.043 & 162.07 \\ 
  \cmidrule{2-23} & \multirow{3}[2]{*}{$\rho=0.9$} & p=12 & 3.171 & 3.176 & 3.177 & 3.18 & 3.217 & 3.189 & 3.18 & 3.179 & 3.18 & 3.559 & 0.04 & 0.041 & 0.041 & 0.041 & 0.041 & 0.041 & 0.041 & 0.041 & 0.041 & 0.05 \\ 
   &  & p=500 & 3.254 & 3.267 & 3.304 & 3.33 & 3.296 & 3.28 & 3.272 & 3.289 & 3.272 & 3.559 & 0.043 & 0.044 & 0.045 & 0.046 & 0.044 & 0.044 & 0.044 & 0.045 & 0.044 & 0.05 \\ 
   &  & p=996 & 3.254 & 3.267 & 8.257 & 12.968 & 3.296 & 3.28 & 3.272 & 16.003 & 3.272 & 6.727 & 0.043 & 0.044 & 1.115 & 283.661 & 0.044 & 0.044 & 0.044 & 509.239 & 0.044 & 157.615 \\ 
   \midrule 
 \multicolumn{1}{|c}{} & \multicolumn{1}{c}{} &       & \multicolumn{10}{c||}{Relative Test Error}                                    & \multicolumn{10}{c|}{Proportion of Variance Explained} \\
\midrule\multirow{9}[6]{*}{hsnr} & \multirow{3}[2]{*}{$\rho=0$} & p=12 & 1.008 & 1.008 & 1.008 & 1.008 & 1.008 & 1.008 & 1.008 & 1.008 & 1.008 & 1.006 & 0.894 & 0.894 & 0.894 & 0.894 & 0.894 & 0.894 & 0.894 & 0.894 & 0.894 & 0.894 \\ 
   &  & p=500 & 1.008 & 1.008 & 1.009 & 1.009 & 1.009 & 1.009 & 1.009 & 1.009 & 1.009 & 1.006 & 0.894 & 0.894 & 0.894 & 0.894 & 0.894 & 0.894 & 0.894 & 0.894 & 0.894 & 0.894 \\ 
   &  & p=996 & 1.008 & 1.008 & 1.222 & 58.89 & 1.009 & 1.009 & 1.009 & 106.503 & 1.009 & 33.197 & 0.894 & 0.894 & 0.872 & -5.198 & 0.894 & 0.894 & 0.894 & -10.203 & 0.894 & -2.511 \\ 
  \cmidrule{2-23} & \multirow{3}[2]{*}{$\rho=0.5$} & p=12 & 1.008 & 1.008 & 1.008 & 1.008 & 1.008 & 1.008 & 1.008 & 1.008 & 1.008 & 1.006 & 0.894 & 0.894 & 0.894 & 0.894 & 0.894 & 0.894 & 0.894 & 0.894 & 0.894 & 0.894 \\ 
   &  & p=500 & 1.008 & 1.009 & 1.009 & 1.009 & 1.009 & 1.009 & 1.009 & 1.009 & 1.009 & 1.006 & 0.894 & 0.894 & 0.894 & 0.894 & 0.894 & 0.894 & 0.894 & 0.894 & 0.894 & 0.894 \\ 
   &  & p=996 & 1.008 & 1.009 & 1.221 & 58.571 & 1.009 & 1.009 & 1.009 & 102.753 & 1.009 & 33.19 & 0.894 & 0.894 & 0.872 & -5.173 & 0.894 & 0.894 & 0.894 & -9.828 & 0.894 & -2.514 \\ 
  \cmidrule{2-23} & \multirow{3}[2]{*}{$\rho=0.9$} & p=12 & 1.008 & 1.008 & 1.008 & 1.008 & 1.008 & 1.008 & 1.008 & 1.008 & 1.008 & 1.006 & 0.894 & 0.894 & 0.894 & 0.894 & 0.894 & 0.894 & 0.894 & 0.894 & 0.894 & 0.894 \\ 
   &  & p=500 & 1.008 & 1.008 & 1.009 & 1.009 & 1.008 & 1.008 & 1.009 & 1.009 & 1.009 & 1.006 & 0.894 & 0.894 & 0.894 & 0.894 & 0.894 & 0.894 & 0.894 & 0.894 & 0.894 & 0.894 \\ 
   &  & p=996 & 1.008 & 1.008 & 1.223 & 57.707 & 1.008 & 1.008 & 1.009 & 102.803 & 1.009 & 33.184 & 0.894 & 0.894 & 0.871 & -5.076 & 0.894 & 0.894 & 0.894 & -9.824 & 0.894 & -2.504 \\ 
  \midrule\multirow{9}[6]{*}{msnr} & \multirow{3}[2]{*}{$\rho=0$} & p=12 & 1.008 & 1.008 & 1.008 & 1.008 & 1.008 & 1.008 & 1.008 & 1.008 & 1.008 & 1.006 & 0.497 & 0.497 & 0.497 & 0.497 & 0.497 & 0.497 & 0.497 & 0.497 & 0.497 & 0.498 \\ 
   &  & p=500 & 1.008 & 1.008 & 1.009 & 1.009 & 1.009 & 1.009 & 1.009 & 1.009 & 1.009 & 1.006 & 0.496 & 0.496 & 0.496 & 0.496 & 0.496 & 0.496 & 0.496 & 0.496 & 0.496 & 0.498 \\ 
   &  & p=996 & 1.008 & 1.008 & 1.222 & 58.89 & 1.009 & 1.009 & 1.009 & 106.503 & 1.009 & 33.197 & 0.496 & 0.496 & 0.39 & -28.432 & 0.496 & 0.496 & 0.496 & -52.211 & 0.496 & -15.636 \\ 
  \cmidrule{2-23} & \multirow{3}[2]{*}{$\rho=0.5$} & p=12 & 1.008 & 1.008 & 1.008 & 1.008 & 1.008 & 1.008 & 1.008 & 1.008 & 1.008 & 1.006 & 0.497 & 0.497 & 0.497 & 0.497 & 0.497 & 0.497 & 0.497 & 0.497 & 0.497 & 0.497 \\ 
   &  & p=500 & 1.008 & 1.009 & 1.009 & 1.009 & 1.009 & 1.009 & 1.009 & 1.009 & 1.009 & 1.006 & 0.496 & 0.496 & 0.496 & 0.496 & 0.496 & 0.496 & 0.496 & 0.496 & 0.496 & 0.497 \\ 
   &  & p=996 & 1.008 & 1.009 & 1.221 & 58.571 & 1.009 & 1.009 & 1.009 & 102.753 & 1.009 & 33.19 & 0.496 & 0.496 & 0.39 & -28.294 & 0.496 & 0.496 & 0.496 & -50.389 & 0.496 & -15.642 \\ 
  \cmidrule{2-23} & \multirow{3}[2]{*}{$\rho=0.9$} & p=12 & 1.008 & 1.008 & 1.008 & 1.008 & 1.008 & 1.008 & 1.008 & 1.008 & 1.008 & 1.007 & 0.496 & 0.496 & 0.496 & 0.496 & 0.496 & 0.496 & 0.496 & 0.496 & 0.496 & 0.497 \\ 
   &  & p=500 & 1.008 & 1.009 & 1.009 & 1.009 & 1.009 & 1.009 & 1.009 & 1.009 & 1.009 & 1.007 & 0.496 & 0.496 & 0.496 & 0.496 & 0.496 & 0.496 & 0.496 & 0.496 & 0.496 & 0.497 \\ 
   &  & p=996 & 1.008 & 1.009 & 1.223 & 57.707 & 1.009 & 1.009 & 1.009 & 102.803 & 1.009 & 33.185 & 0.496 & 0.496 & 0.389 & -27.846 & 0.496 & 0.496 & 0.496 & -50.388 & 0.496 & -15.614 \\ 
  \midrule\multirow{9}[6]{*}{lsnr} & \multirow{3}[2]{*}{$\rho=0$} & p=12 & 1.008 & 1.008 & 1.008 & 1.008 & 1.008 & 1.008 & 1.008 & 1.008 & 1.008 & 1.006 & 0.161 & 0.161 & 0.161 & 0.16 & 0.161 & 0.16 & 0.16 & 0.16 & 0.16 & 0.162 \\ 
   &  & p=500 & 1.008 & 1.008 & 1.009 & 1.009 & 1.009 & 1.009 & 1.009 & 1.009 & 1.009 & 1.006 & 0.16 & 0.16 & 0.16 & 0.16 & 0.16 & 0.16 & 0.16 & 0.16 & 0.16 & 0.162 \\ 
   &  & p=996 & 1.008 & 1.008 & 1.222 & 58.89 & 1.009 & 1.009 & 1.009 & 106.503 & 1.009 & 33.197 & 0.16 & 0.16 & -0.018 & -48.062 & 0.16 & 0.16 & 0.16 & -87.721 & 0.16 & -26.682 \\ 
  \cmidrule{2-23} & \multirow{3}[2]{*}{$\rho=0.5$} & p=12 & 1.008 & 1.008 & 1.008 & 1.008 & 1.008 & 1.008 & 1.008 & 1.008 & 1.008 & 1.006 & 0.16 & 0.16 & 0.16 & 0.16 & 0.161 & 0.16 & 0.16 & 0.16 & 0.16 & 0.162 \\ 
   &  & p=500 & 1.008 & 1.009 & 1.009 & 1.009 & 1.009 & 1.009 & 1.009 & 1.009 & 1.009 & 1.006 & 0.16 & 0.16 & 0.16 & 0.16 & 0.16 & 0.16 & 0.16 & 0.16 & 0.16 & 0.162 \\ 
   &  & p=996 & 1.008 & 1.009 & 1.221 & 58.571 & 1.009 & 1.009 & 1.009 & 102.753 & 1.009 & 33.19 & 0.16 & 0.16 & -0.017 & -47.809 & 0.16 & 0.16 & 0.16 & -84.626 & 0.16 & -26.681 \\ 
  \cmidrule{2-23} & \multirow{3}[2]{*}{$\rho=0.9$} & p=12 & 1.008 & 1.008 & 1.008 & 1.008 & 1.008 & 1.008 & 1.008 & 1.008 & 1.008 & 1.01 & 0.16 & 0.16 & 0.16 & 0.16 & 0.16 & 0.16 & 0.16 & 0.16 & 0.16 & 0.159 \\ 
   &  & p=500 & 1.009 & 1.009 & 1.009 & 1.009 & 1.009 & 1.009 & 1.009 & 1.009 & 1.009 & 1.01 & 0.16 & 0.16 & 0.159 & 0.159 & 0.16 & 0.16 & 0.16 & 0.16 & 0.16 & 0.159 \\ 
   &  & p=996 & 1.009 & 1.009 & 1.224 & 57.707 & 1.009 & 1.009 & 1.009 & 102.803 & 1.009 & 32.371 & 0.16 & 0.16 & -0.019 & -47.08 & 0.16 & 0.16 & 0.16 & -84.653 & 0.16 & -25.989 \\ 
   \midrule 
 \multicolumn{1}{|c}{} & \multicolumn{1}{c}{} &       & \multicolumn{10}{c||}{Kullback-Leibler Discrepancy}                                    & \multicolumn{10}{c|}{Number of Variables} \\
\midrule\multirow{9}[6]{*}{hsnr} & \multirow{3}[2]{*}{$\rho=0$} & p=12 & 3.115 & 3.115 & 3.115 & 3.115 & 3.115 & 3.115 & 3.115 & 3.115 & 3.115 & 3.113 & 6.617 & 6.62 & 6.631 & 6.639 & 6.68 & 6.653 & 6.639 & 6.645 & 6.639 & 6.011 \\ 
   &  & p=500 & 3.116 & 3.116 & 3.116 & 3.116 & 3.116 & 3.116 & 3.116 & 3.116 & 3.116 & 3.113 & 6.861 & 6.886 & 7.005 & 7.047 & 7.025 & 6.961 & 6.938 & 6.999 & 6.957 & 6.011 \\ 
   &  & p=996 & 3.116 & 3.116 & 3.912 & 76905.805 & 3.116 & 3.116 & 3.116 & 107437.34 & 3.116 & 68394.068 & 6.861 & 6.886 & 96.618 & 275.033 & 7.025 & 6.961 & 6.938 & 379.786 & 6.957 & 103.022 \\ 
  \cmidrule{2-23} & \multirow{3}[2]{*}{$\rho=0.5$} & p=12 & 3.866 & 3.866 & 3.866 & 3.866 & 3.866 & 3.866 & 3.866 & 3.866 & 3.866 & 3.865 & 6.629 & 6.64 & 6.641 & 6.645 & 6.675 & 6.656 & 6.645 & 6.66 & 6.645 & 6.012 \\ 
   &  & p=500 & 3.867 & 3.867 & 3.867 & 3.868 & 3.867 & 3.867 & 3.867 & 3.867 & 3.867 & 3.865 & 6.889 & 6.914 & 7.071 & 7.108 & 7.03 & 6.983 & 6.972 & 7.022 & 6.972 & 6.012 \\ 
   &  & p=996 & 3.867 & 3.867 & 4.66 & 75602.015 & 3.867 & 3.867 & 3.867 & 104378.88 & 3.867 & 68383.572 & 6.889 & 6.914 & 96.203 & 274.935 & 7.03 & 6.983 & 6.972 & 378.791 & 6.972 & 103.021 \\ 
  \cmidrule{2-23} & \multirow{3}[2]{*}{$\rho=0.9$} & p=12 & 4.489 & 4.489 & 4.489 & 4.489 & 4.489 & 4.489 & 4.489 & 4.489 & 4.489 & 4.487 & 6.613 & 6.623 & 6.629 & 6.649 & 6.669 & 6.649 & 6.647 & 6.649 & 6.647 & 6.011 \\ 
   &  & p=500 & 4.489 & 4.489 & 4.49 & 4.49 & 4.489 & 4.489 & 4.489 & 4.49 & 4.489 & 4.487 & 6.822 & 6.905 & 7.041 & 7.127 & 6.976 & 6.905 & 6.925 & 6.967 & 6.925 & 6.014 \\ 
   &  & p=996 & 4.489 & 4.489 & 5.314 & 75583.261 & 4.489 & 4.489 & 4.489 & 106170.983 & 4.489 & 68386.457 & 6.822 & 6.905 & 96.161 & 274.518 & 6.976 & 6.905 & 6.925 & 378.841 & 6.925 & 103.03 \\ 
  \midrule\multirow{9}[6]{*}{msnr} & \multirow{3}[2]{*}{$\rho=0$} & p=12 & 5.255 & 5.255 & 5.255 & 5.255 & 5.255 & 5.256 & 5.255 & 5.256 & 5.255 & 5.254 & 6.617 & 6.62 & 6.631 & 6.639 & 6.68 & 6.653 & 6.639 & 6.645 & 6.639 & 6.011 \\ 
   &  & p=500 & 5.256 & 5.256 & 5.256 & 5.256 & 5.256 & 5.256 & 5.256 & 5.256 & 5.256 & 5.254 & 6.861 & 6.886 & 7.005 & 7.047 & 7.025 & 6.961 & 6.938 & 6.999 & 6.957 & 6.011 \\ 
   &  & p=996 & 5.256 & 5.256 & 6.052 & 76907.945 & 5.256 & 5.256 & 5.256 & 107439.48 & 5.256 & 68396.208 & 6.861 & 6.886 & 96.618 & 275.033 & 7.025 & 6.961 & 6.938 & 379.786 & 6.957 & 103.022 \\ 
  \cmidrule{2-23} & \multirow{3}[2]{*}{$\rho=0.5$} & p=12 & 6.006 & 6.007 & 6.007 & 6.007 & 6.006 & 6.007 & 6.007 & 6.007 & 6.007 & 6.005 & 6.629 & 6.64 & 6.641 & 6.645 & 6.675 & 6.656 & 6.645 & 6.66 & 6.645 & 6.012 \\ 
   &  & p=500 & 6.007 & 6.007 & 6.008 & 6.008 & 6.007 & 6.007 & 6.007 & 6.007 & 6.007 & 6.005 & 6.889 & 6.914 & 7.071 & 7.108 & 7.03 & 6.983 & 6.972 & 7.022 & 6.972 & 6.012 \\ 
   &  & p=996 & 6.007 & 6.007 & 6.8 & 75604.155 & 6.007 & 6.007 & 6.007 & 104381.02 & 6.007 & 68385.712 & 6.889 & 6.914 & 96.203 & 274.935 & 7.03 & 6.983 & 6.972 & 378.791 & 6.972 & 103.021 \\ 
  \cmidrule{2-23} & \multirow{3}[2]{*}{$\rho=0.9$} & p=12 & 6.629 & 6.629 & 6.629 & 6.629 & 6.629 & 6.629 & 6.629 & 6.629 & 6.629 & 6.628 & 6.602 & 6.612 & 6.618 & 6.638 & 6.663 & 6.642 & 6.636 & 6.638 & 6.636 & 5.927 \\ 
   &  & p=500 & 6.629 & 6.63 & 6.63 & 6.63 & 6.63 & 6.63 & 6.63 & 6.63 & 6.63 & 6.628 & 6.81 & 6.893 & 7.013 & 7.099 & 6.97 & 6.898 & 6.913 & 6.955 & 6.913 & 5.931 \\ 
   &  & p=996 & 6.629 & 6.63 & 7.454 & 75585.401 & 6.63 & 6.63 & 6.63 & 106173.123 & 6.63 & 68388.598 & 6.81 & 6.893 & 96.149 & 274.506 & 6.97 & 6.898 & 6.913 & 378.831 & 6.913 & 102.957 \\ 
  \midrule\multirow{9}[6]{*}{lsnr} & \multirow{3}[2]{*}{$\rho=0$} & p=12 & 6.865 & 6.865 & 6.865 & 6.865 & 6.865 & 6.865 & 6.865 & 6.865 & 6.865 & 6.863 & 6.617 & 6.62 & 6.631 & 6.639 & 6.68 & 6.653 & 6.639 & 6.645 & 6.639 & 6.011 \\ 
   &  & p=500 & 6.866 & 6.866 & 6.866 & 6.866 & 6.866 & 6.866 & 6.866 & 6.866 & 6.866 & 6.863 & 6.861 & 6.886 & 7.005 & 7.047 & 7.025 & 6.961 & 6.938 & 6.999 & 6.957 & 6.011 \\ 
   &  & p=996 & 6.866 & 6.866 & 7.662 & 76909.555 & 6.866 & 6.866 & 6.866 & 107441.089 & 6.866 & 68397.817 & 6.861 & 6.886 & 96.618 & 275.033 & 7.025 & 6.961 & 6.938 & 379.786 & 6.957 & 103.022 \\ 
  \cmidrule{2-23} & \multirow{3}[2]{*}{$\rho=0.5$} & p=12 & 7.616 & 7.616 & 7.616 & 7.616 & 7.616 & 7.616 & 7.616 & 7.616 & 7.616 & 7.614 & 6.629 & 6.64 & 6.641 & 6.645 & 6.675 & 6.656 & 6.645 & 6.66 & 6.645 & 5.997 \\ 
   &  & p=500 & 7.617 & 7.617 & 7.617 & 7.617 & 7.617 & 7.617 & 7.617 & 7.617 & 7.617 & 7.614 & 6.889 & 6.914 & 7.071 & 7.108 & 7.03 & 6.983 & 6.972 & 7.022 & 6.972 & 5.997 \\ 
   &  & p=996 & 7.617 & 7.617 & 8.409 & 75605.765 & 7.617 & 7.617 & 7.617 & 104382.629 & 7.617 & 68387.321 & 6.889 & 6.914 & 96.202 & 274.934 & 7.03 & 6.983 & 6.972 & 378.791 & 6.972 & 103.01 \\ 
  \cmidrule{2-23} & \multirow{3}[2]{*}{$\rho=0.9$} & p=12 & 8.239 & 8.239 & 8.239 & 8.239 & 8.239 & 8.239 & 8.239 & 8.239 & 8.239 & 8.24 & 6.262 & 6.287 & 6.287 & 6.304 & 6.286 & 6.306 & 6.301 & 6.305 & 6.301 & 5.013 \\ 
   &  & p=500 & 8.239 & 8.239 & 8.24 & 8.24 & 8.239 & 8.239 & 8.239 & 8.239 & 8.239 & 8.24 & 6.507 & 6.561 & 6.716 & 6.823 & 6.565 & 6.594 & 6.586 & 6.656 & 6.586 & 5.013 \\ 
   &  & p=996 & 8.239 & 8.239 & 9.064 & 75587.011 & 8.239 & 8.239 & 8.239 & 106174.732 & 8.239 & 67638.755 & 6.507 & 6.561 & 95.876 & 274.236 & 6.565 & 6.594 & 6.586 & 378.646 & 6.586 & 99.162 \\ 
   \bottomrule 
\end{tabular}
}
\end{table}

%% latex table generated in R 4.0.2 by xtable 1.8-4 package
% Fri Jun 26 10:43:45 2020
\begin{table}[ht]
\centering
\caption{Fixed X, Sparse-Ex3, n=40} 
\scalebox{0.52}{
\begin{tabular}{|c|c|c|cc|cc|cc|ccc|c||cc|cc|cc|ccc|c|}
  \toprule 
 \multicolumn{1}{|c}{} & \multicolumn{1}{c}{} &       & RAICc & AICc  & RC$_p$ & C$_p$ & 10F CV & LOO CV & S$_p$ & $\widehat{\text{C}}_p$ & GCV   & BIC   & RAICc & AICc  & RC$_p$ & C$_p$ & 10F CV & LOO CV & S$_p$ & $\widehat{\text{C}}_p$ & GCV   & BIC  \\
 \cmidrule{4-23}\multicolumn{1}{|c}{} & \multicolumn{1}{c}{} &       & \multicolumn{10}{c||}{RMSE}                                                   & \multicolumn{10}{c|}{Relative Risk}       \\
 \midrule\multirow{9}[6]{*}{hsnr} & \multirow{3}[2]{*}{$\rho=0$} & p=12 & 0.282 & 0.291 & 0.295 & 0.311 & 0.309 & 0.308 & 0.309 & 0.322 & 0.31 & 0.287 & 0.02 & 0.022 & 0.022 & 0.025 & 0.025 & 0.025 & 0.025 & 0.027 & 0.025 & 0.021 \\ 
   &  & p=20 & 0.282 & 0.292 & 0.302 & 0.331 & 0.321 & 0.321 & 0.322 & 0.358 & 0.324 & 0.291 & 0.02 & 0.022 & 0.024 & 0.034 & 0.029 & 0.03 & 0.03 & 0.042 & 0.03 & 0.022 \\ 
   &  & p=36 & 0.282 & 0.292 & 0.339 & 0.412 & 0.325 & 0.333 & 0.332 & 0.492 & 0.341 & 0.4 & 0.02 & 0.022 & 0.044 & 0.197 & 0.032 & 0.046 & 0.046 & 0.315 & 0.058 & 0.234 \\ 
  \cmidrule{2-23} & \multirow{3}[2]{*}{$\rho=0.5$} & p=12 & 0.304 & 0.313 & 0.318 & 0.334 & 0.332 & 0.333 & 0.331 & 0.347 & 0.332 & 0.309 & 0.024 & 0.026 & 0.027 & 0.029 & 0.029 & 0.029 & 0.029 & 0.032 & 0.029 & 0.025 \\ 
   &  & p=20 & 0.304 & 0.314 & 0.323 & 0.355 & 0.344 & 0.347 & 0.345 & 0.387 & 0.348 & 0.312 & 0.024 & 0.026 & 0.028 & 0.039 & 0.034 & 0.035 & 0.034 & 0.049 & 0.035 & 0.026 \\ 
   &  & p=36 & 0.304 & 0.314 & 0.364 & 0.444 & 0.349 & 0.359 & 0.357 & 0.529 & 0.365 & 0.429 & 0.024 & 0.026 & 0.052 & 0.23 & 0.038 & 0.053 & 0.053 & 0.364 & 0.068 & 0.271 \\ 
  \cmidrule{2-23} & \multirow{3}[2]{*}{$\rho=0.9$} & p=12 & 0.339 & 0.348 & 0.355 & 0.374 & 0.37 & 0.371 & 0.369 & 0.387 & 0.37 & 0.343 & 0.028 & 0.03 & 0.032 & 0.036 & 0.035 & 0.035 & 0.035 & 0.039 & 0.035 & 0.029 \\ 
   &  & p=20 & 0.339 & 0.35 & 0.359 & 0.397 & 0.383 & 0.388 & 0.385 & 0.432 & 0.386 & 0.348 & 0.028 & 0.031 & 0.034 & 0.048 & 0.041 & 0.043 & 0.042 & 0.062 & 0.043 & 0.031 \\ 
   &  & p=36 & 0.339 & 0.35 & 0.405 & 0.496 & 0.387 & 0.402 & 0.399 & 0.592 & 0.406 & 0.478 & 0.028 & 0.031 & 0.064 & 0.287 & 0.045 & 0.068 & 0.067 & 0.456 & 0.082 & 0.337 \\ 
  \midrule\multirow{9}[6]{*}{msnr} & \multirow{3}[2]{*}{$\rho=0$} & p=12 & 0.901 & 0.897 & 0.897 & 0.93 & 0.946 & 0.924 & 0.922 & 0.953 & 0.925 & 0.986 & 0.212 & 0.21 & 0.208 & 0.226 & 0.234 & 0.221 & 0.222 & 0.239 & 0.223 & 0.265 \\ 
   &  & p=20 & 0.901 & 0.9 & 0.92 & 0.99 & 0.977 & 0.961 & 0.959 & 1.057 & 0.963 & 0.995 & 0.212 & 0.213 & 0.227 & 0.301 & 0.268 & 0.264 & 0.264 & 0.366 & 0.267 & 0.275 \\ 
   &  & p=36 & 0.901 & 0.9 & 1.082 & 1.268 & 0.99 & 0.996 & 0.987 & 1.443 & 1.011 & 1.276 & 0.212 & 0.213 & 0.43 & 1.715 & 0.297 & 0.403 & 0.391 & 2.679 & 0.501 & 1.978 \\ 
  \cmidrule{2-23} & \multirow{3}[2]{*}{$\rho=0.5$} & p=12 & 0.975 & 0.976 & 0.975 & 1.006 & 1.021 & 1.004 & 0.995 & 1.028 & 0.998 & 1.066 & 0.251 & 0.252 & 0.251 & 0.267 & 0.274 & 0.266 & 0.261 & 0.281 & 0.262 & 0.304 \\ 
   &  & p=20 & 0.975 & 0.977 & 0.992 & 1.067 & 1.052 & 1.045 & 1.035 & 1.141 & 1.043 & 1.074 & 0.251 & 0.253 & 0.267 & 0.349 & 0.312 & 0.314 & 0.309 & 0.426 & 0.317 & 0.313 \\ 
   &  & p=36 & 0.975 & 0.977 & 1.162 & 1.368 & 1.066 & 1.08 & 1.069 & 1.552 & 1.092 & 1.369 & 0.251 & 0.253 & 0.497 & 1.999 & 0.344 & 0.472 & 0.468 & 3.099 & 0.593 & 2.283 \\ 
  \cmidrule{2-23} & \multirow{3}[2]{*}{$\rho=0.9$} & p=12 & 1.114 & 1.098 & 1.104 & 1.132 & 1.15 & 1.13 & 1.119 & 1.155 & 1.122 & 1.214 & 0.304 & 0.298 & 0.304 & 0.329 & 0.332 & 0.324 & 0.319 & 0.347 & 0.321 & 0.356 \\ 
   &  & p=20 & 1.114 & 1.102 & 1.127 & 1.199 & 1.184 & 1.178 & 1.167 & 1.28 & 1.17 & 1.225 & 0.304 & 0.302 & 0.325 & 0.429 & 0.377 & 0.386 & 0.384 & 0.532 & 0.387 & 0.37 \\ 
   &  & p=36 & 1.114 & 1.102 & 1.314 & 1.537 & 1.195 & 1.219 & 1.205 & 1.74 & 1.225 & 1.551 & 0.304 & 0.302 & 0.608 & 2.487 & 0.407 & 0.603 & 0.583 & 3.884 & 0.716 & 2.869 \\ 
  \midrule\multirow{9}[6]{*}{lsnr} & \multirow{3}[2]{*}{$\rho=0$} & p=12 & 2.13 & 2.138 & 2.141 & 2.175 & 2.163 & 2.165 & 2.169 & 2.202 & 2.172 & 2.163 & 1.019 & 1.051 & 1.067 & 1.126 & 1.107 & 1.105 & 1.113 & 1.173 & 1.12 & 1.049 \\ 
   &  & p=20 & 2.13 & 2.14 & 2.167 & 2.28 & 2.209 & 2.223 & 2.215 & 2.373 & 2.222 & 2.167 & 1.019 & 1.054 & 1.131 & 1.441 & 1.227 & 1.256 & 1.228 & 1.671 & 1.242 & 1.058 \\ 
   &  & p=36 & 2.13 & 2.14 & 2.385 & 2.817 & 2.227 & 2.283 & 2.272 & 3.209 & 2.302 & 2.651 & 1.019 & 1.054 & 1.972 & 8.408 & 1.326 & 1.86 & 1.828 & 12.862 & 2.209 & 7.651 \\ 
  \cmidrule{2-23} & \multirow{3}[2]{*}{$\rho=0.5$} & p=12 & 2.295 & 2.302 & 2.291 & 2.337 & 2.313 & 2.319 & 2.323 & 2.365 & 2.326 & 2.325 & 1.101 & 1.144 & 1.169 & 1.261 & 1.21 & 1.227 & 1.23 & 1.312 & 1.238 & 1.094 \\ 
   &  & p=20 & 2.295 & 2.304 & 2.322 & 2.451 & 2.363 & 2.383 & 2.375 & 2.552 & 2.384 & 2.328 & 1.101 & 1.151 & 1.241 & 1.615 & 1.35 & 1.401 & 1.371 & 1.886 & 1.397 & 1.101 \\ 
   &  & p=36 & 2.295 & 2.304 & 2.566 & 3.037 & 2.383 & 2.452 & 2.434 & 3.445 & 2.469 & 2.85 & 1.101 & 1.151 & 2.245 & 9.788 & 1.47 & 2.153 & 2.039 & 14.842 & 2.496 & 8.723 \\ 
  \cmidrule{2-23} & \multirow{3}[2]{*}{$\rho=0.9$} & p=12 & 2.542 & 2.542 & 2.537 & 2.596 & 2.552 & 2.562 & 2.565 & 2.628 & 2.572 & 2.579 & 1.232 & 1.276 & 1.329 & 1.482 & 1.373 & 1.388 & 1.388 & 1.55 & 1.408 & 1.17 \\ 
   &  & p=20 & 2.542 & 2.548 & 2.575 & 2.727 & 2.606 & 2.632 & 2.627 & 2.841 & 2.641 & 2.583 & 1.232 & 1.294 & 1.436 & 1.935 & 1.543 & 1.593 & 1.569 & 2.282 & 1.611 & 1.18 \\ 
   &  & p=36 & 2.542 & 2.548 & 2.853 & 3.391 & 2.629 & 2.703 & 2.691 & 3.83 & 2.73 & 3.164 & 1.232 & 1.294 & 2.708 & 12.144 & 1.695 & 2.409 & 2.407 & 18.385 & 2.921 & 10.625 \\ 
   \midrule 
 \multicolumn{1}{|c}{} & \multicolumn{1}{c}{} &       & \multicolumn{10}{c||}{Relative Test Error}                                    & \multicolumn{10}{c|}{Proportion of Variance Explained} \\
\midrule\multirow{9}[6]{*}{hsnr} & \multirow{3}[2]{*}{$\rho=0$} & p=12 & 1.229 & 1.247 & 1.256 & 1.288 & 1.287 & 1.281 & 1.284 & 1.311 & 1.286 & 1.241 & 0.901 & 0.9 & 0.899 & 0.897 & 0.897 & 0.897 & 0.897 & 0.895 & 0.897 & 0.9 \\ 
   &  & p=20 & 1.229 & 1.251 & 1.278 & 1.387 & 1.333 & 1.34 & 1.342 & 1.484 & 1.346 & 1.255 & 0.901 & 0.9 & 0.897 & 0.889 & 0.893 & 0.892 & 0.892 & 0.881 & 0.892 & 0.899 \\ 
   &  & p=36 & 1.229 & 1.251 & 1.507 & 3.263 & 1.372 & 1.527 & 1.526 & 4.606 & 1.67 & 3.682 & 0.901 & 0.9 & 0.879 & 0.738 & 0.89 & 0.877 & 0.878 & 0.63 & 0.866 & 0.704 \\ 
  \cmidrule{2-23} & \multirow{3}[2]{*}{$\rho=0.5$} & p=12 & 1.236 & 1.252 & 1.262 & 1.29 & 1.288 & 1.288 & 1.283 & 1.315 & 1.286 & 1.245 & 0.886 & 0.885 & 0.884 & 0.881 & 0.881 & 0.881 & 0.882 & 0.879 & 0.882 & 0.885 \\ 
   &  & p=20 & 1.236 & 1.254 & 1.277 & 1.385 & 1.333 & 1.343 & 1.34 & 1.487 & 1.35 & 1.256 & 0.886 & 0.885 & 0.882 & 0.872 & 0.877 & 0.876 & 0.877 & 0.863 & 0.876 & 0.884 \\ 
   &  & p=36 & 1.236 & 1.254 & 1.512 & 3.271 & 1.371 & 1.527 & 1.525 & 4.589 & 1.67 & 3.675 & 0.886 & 0.885 & 0.861 & 0.699 & 0.874 & 0.859 & 0.86 & 0.577 & 0.846 & 0.662 \\ 
  \cmidrule{2-23} & \multirow{3}[2]{*}{$\rho=0.9$} & p=12 & 1.223 & 1.238 & 1.25 & 1.286 & 1.277 & 1.278 & 1.275 & 1.311 & 1.278 & 1.232 & 0.862 & 0.861 & 0.859 & 0.855 & 0.856 & 0.856 & 0.857 & 0.853 & 0.856 & 0.861 \\ 
   &  & p=20 & 1.223 & 1.242 & 1.264 & 1.378 & 1.32 & 1.337 & 1.335 & 1.485 & 1.339 & 1.246 & 0.862 & 0.86 & 0.858 & 0.845 & 0.851 & 0.85 & 0.85 & 0.833 & 0.849 & 0.86 \\ 
   &  & p=36 & 1.223 & 1.242 & 1.502 & 3.265 & 1.352 & 1.539 & 1.526 & 4.597 & 1.645 & 3.66 & 0.862 & 0.86 & 0.831 & 0.633 & 0.848 & 0.827 & 0.828 & 0.483 & 0.815 & 0.588 \\ 
  \midrule\multirow{9}[6]{*}{msnr} & \multirow{3}[2]{*}{$\rho=0$} & p=12 & 1.286 & 1.283 & 1.281 & 1.304 & 1.316 & 1.298 & 1.299 & 1.322 & 1.3 & 1.357 & 0.452 & 0.454 & 0.454 & 0.445 & 0.44 & 0.447 & 0.447 & 0.437 & 0.446 & 0.422 \\ 
   &  & p=20 & 1.286 & 1.287 & 1.306 & 1.406 & 1.362 & 1.355 & 1.355 & 1.493 & 1.359 & 1.37 & 0.452 & 0.452 & 0.444 & 0.401 & 0.42 & 0.423 & 0.423 & 0.364 & 0.421 & 0.416 \\ 
   &  & p=36 & 1.286 & 1.287 & 1.58 & 3.312 & 1.4 & 1.543 & 1.527 & 4.612 & 1.676 & 3.667 & 0.452 & 0.452 & 0.327 & -0.411 & 0.404 & 0.343 & 0.35 & -0.964 & 0.286 & -0.562 \\ 
  \cmidrule{2-23} & \multirow{3}[2]{*}{$\rho=0.5$} & p=12 & 1.291 & 1.292 & 1.292 & 1.31 & 1.318 & 1.309 & 1.303 & 1.326 & 1.304 & 1.353 & 0.402 & 0.402 & 0.402 & 0.394 & 0.39 & 0.394 & 0.397 & 0.386 & 0.396 & 0.374 \\ 
   &  & p=20 & 1.291 & 1.294 & 1.31 & 1.405 & 1.361 & 1.364 & 1.358 & 1.495 & 1.368 & 1.363 & 0.402 & 0.401 & 0.394 & 0.349 & 0.37 & 0.368 & 0.371 & 0.308 & 0.367 & 0.369 \\ 
   &  & p=36 & 1.291 & 1.294 & 1.576 & 3.319 & 1.399 & 1.548 & 1.543 & 4.595 & 1.687 & 3.648 & 0.402 & 0.401 & 0.27 & -0.536 & 0.352 & 0.283 & 0.285 & -1.127 & 0.219 & -0.689 \\ 
  \cmidrule{2-23} & \multirow{3}[2]{*}{$\rho=0.9$} & p=12 & 1.282 & 1.276 & 1.282 & 1.305 & 1.308 & 1.3 & 1.296 & 1.323 & 1.298 & 1.331 & 0.335 & 0.338 & 0.335 & 0.323 & 0.322 & 0.326 & 0.328 & 0.314 & 0.327 & 0.31 \\ 
   &  & p=20 & 1.282 & 1.28 & 1.302 & 1.398 & 1.35 & 1.359 & 1.356 & 1.494 & 1.359 & 1.343 & 0.335 & 0.336 & 0.325 & 0.275 & 0.3 & 0.295 & 0.297 & 0.225 & 0.295 & 0.303 \\ 
   &  & p=36 & 1.282 & 1.28 & 1.564 & 3.308 & 1.378 & 1.559 & 1.541 & 4.605 & 1.665 & 3.662 & 0.335 & 0.336 & 0.189 & -0.716 & 0.285 & 0.191 & 0.201 & -1.388 & 0.137 & -0.9 \\ 
  \midrule\multirow{9}[6]{*}{lsnr} & \multirow{3}[2]{*}{$\rho=0$} & p=12 & 1.275 & 1.283 & 1.288 & 1.304 & 1.299 & 1.298 & 1.3 & 1.316 & 1.302 & 1.283 & -0.004 & -0.011 & -0.014 & -0.027 & -0.023 & -0.022 & -0.024 & -0.037 & -0.025 & -0.01 \\ 
   &  & p=20 & 1.275 & 1.284 & 1.305 & 1.388 & 1.331 & 1.339 & 1.331 & 1.451 & 1.335 & 1.285 & -0.004 & -0.012 & -0.028 & -0.094 & -0.048 & -0.054 & -0.048 & -0.142 & -0.051 & -0.012 \\ 
   &  & p=36 & 1.275 & 1.284 & 1.532 & 3.267 & 1.358 & 1.502 & 1.493 & 4.468 & 1.595 & 3.063 & -0.004 & -0.012 & -0.207 & -1.573 & -0.069 & -0.183 & -0.176 & -2.519 & -0.257 & -1.413 \\ 
  \cmidrule{2-23} & \multirow{3}[2]{*}{$\rho=0.5$} & p=12 & 1.255 & 1.265 & 1.271 & 1.293 & 1.281 & 1.285 & 1.285 & 1.304 & 1.287 & 1.254 & -0.019 & -0.027 & -0.032 & -0.049 & -0.04 & -0.043 & -0.043 & -0.059 & -0.045 & -0.018 \\ 
   &  & p=20 & 1.255 & 1.267 & 1.288 & 1.375 & 1.313 & 1.325 & 1.318 & 1.438 & 1.324 & 1.255 & -0.019 & -0.028 & -0.045 & -0.116 & -0.066 & -0.075 & -0.07 & -0.167 & -0.075 & -0.019 \\ 
   &  & p=36 & 1.255 & 1.267 & 1.521 & 3.271 & 1.341 & 1.499 & 1.473 & 4.443 & 1.579 & 3.024 & -0.019 & -0.028 & -0.234 & -1.655 & -0.088 & -0.217 & -0.196 & -2.607 & -0.282 & -1.454 \\ 
  \cmidrule{2-23} & \multirow{3}[2]{*}{$\rho=0.9$} & p=12 & 1.229 & 1.237 & 1.247 & 1.275 & 1.255 & 1.258 & 1.258 & 1.288 & 1.261 & 1.217 & -0.036 & -0.043 & -0.051 & -0.076 & -0.058 & -0.061 & -0.061 & -0.086 & -0.064 & -0.027 \\ 
   &  & p=20 & 1.229 & 1.24 & 1.267 & 1.359 & 1.286 & 1.296 & 1.291 & 1.424 & 1.299 & 1.219 & -0.036 & -0.046 & -0.068 & -0.146 & -0.085 & -0.093 & -0.089 & -0.201 & -0.096 & -0.028 \\ 
   &  & p=36 & 1.229 & 1.24 & 1.503 & 3.254 & 1.315 & 1.447 & 1.447 & 4.413 & 1.542 & 2.972 & -0.036 & -0.046 & -0.267 & -1.745 & -0.109 & -0.221 & -0.22 & -2.722 & -0.301 & -1.507 \\ 
   \midrule 
 \multicolumn{1}{|c}{} & \multicolumn{1}{c}{} &       & \multicolumn{10}{c||}{Kullback-Leibler Discrepancy}                                    & \multicolumn{10}{c|}{Number of Variables} \\
\midrule\multirow{9}[6]{*}{hsnr} & \multirow{3}[2]{*}{$\rho=0$} & p=12 & 0.64 & 0.674 & 0.689 & 0.739 & 0.73 & 0.725 & 0.733 & 0.772 & 0.735 & 0.669 & 6.148 & 6.319 & 6.418 & 6.815 & 6.868 & 6.758 & 6.758 & 7.138 & 6.778 & 6.274 \\ 
   &  & p=20 & 0.64 & 0.682 & 0.741 & 0.955 & 0.842 & 0.875 & 0.873 & 1.143 & 0.881 & 0.715 & 6.148 & 6.343 & 6.624 & 7.65 & 7.301 & 7.289 & 7.241 & 8.775 & 7.29 & 6.391 \\ 
   &  & p=36 & 0.64 & 0.682 & 2.736 & 18.615 & 0.97 & 6.127 & 7.065 & 24.747 & 8.614 & 23.616 & 6.148 & 6.343 & 8.522 & 13.668 & 7.526 & 8.022 & 7.939 & 18.913 & 8.414 & 14.278 \\ 
  \cmidrule{2-23} & \multirow{3}[2]{*}{$\rho=0.5$} & p=12 & 0.792 & 0.823 & 0.84 & 0.889 & 0.879 & 0.88 & 0.877 & 0.926 & 0.881 & 0.812 & 6.149 & 6.314 & 6.418 & 6.813 & 6.846 & 6.797 & 6.714 & 7.164 & 6.745 & 6.24 \\ 
   &  & p=20 & 0.792 & 0.834 & 0.88 & 1.107 & 0.993 & 1.027 & 1.02 & 1.293 & 1.036 & 0.849 & 6.149 & 6.333 & 6.569 & 7.636 & 7.271 & 7.304 & 7.199 & 8.799 & 7.292 & 6.33 \\ 
   &  & p=36 & 0.792 & 0.834 & 2.882 & 18.802 & 1.123 & 6.27 & 7.229 & 24.872 & 8.748 & 23.717 & 6.149 & 6.333 & 8.502 & 13.647 & 7.503 & 8.027 & 7.927 & 18.825 & 8.378 & 14.176 \\ 
  \cmidrule{2-23} & \multirow{3}[2]{*}{$\rho=0.9$} & p=12 & 1.013 & 1.041 & 1.061 & 1.116 & 1.098 & 1.105 & 1.097 & 1.154 & 1.103 & 1.033 & 6.147 & 6.295 & 6.421 & 6.844 & 6.847 & 6.802 & 6.699 & 7.199 & 6.741 & 6.232 \\ 
   &  & p=20 & 1.013 & 1.057 & 1.101 & 1.329 & 1.206 & 1.256 & 1.246 & 1.515 & 1.252 & 1.084 & 6.147 & 6.332 & 6.558 & 7.635 & 7.249 & 7.335 & 7.201 & 8.808 & 7.243 & 6.355 \\ 
   &  & p=36 & 1.013 & 1.057 & 3.1 & 19.028 & 1.316 & 6.619 & 7.516 & 25.097 & 8.753 & 23.889 & 6.147 & 6.332 & 8.502 & 13.672 & 7.434 & 8.116 & 7.969 & 18.856 & 8.311 & 14.163 \\ 
  \midrule\multirow{9}[6]{*}{msnr} & \multirow{3}[2]{*}{$\rho=0$} & p=12 & 2.813 & 2.836 & 2.845 & 2.889 & 2.886 & 2.876 & 2.882 & 2.919 & 2.883 & 2.87 & 5.81 & 6.08 & 6.25 & 6.691 & 6.646 & 6.607 & 6.641 & 7.037 & 6.661 & 5.603 \\ 
   &  & p=20 & 2.813 & 2.844 & 2.898 & 3.107 & 2.996 & 3.024 & 3.02 & 3.289 & 3.029 & 2.915 & 5.81 & 6.104 & 6.433 & 7.525 & 7.075 & 7.133 & 7.118 & 8.684 & 7.169 & 5.713 \\ 
   &  & p=36 & 2.813 & 2.844 & 4.914 & 20.781 & 3.124 & 8.276 & 9.091 & 26.891 & 10.575 & 25.567 & 5.81 & 6.104 & 8.122 & 13.362 & 7.3 & 7.866 & 7.787 & 18.868 & 8.265 & 13.551 \\ 
  \cmidrule{2-23} & \multirow{3}[2]{*}{$\rho=0.5$} & p=12 & 2.965 & 2.988 & 2.999 & 3.042 & 3.037 & 3.033 & 3.029 & 3.072 & 3.032 & 3.017 & 5.788 & 6.024 & 6.196 & 6.639 & 6.581 & 6.619 & 6.549 & 7.054 & 6.581 & 5.541 \\ 
   &  & p=20 & 2.965 & 2.999 & 3.041 & 3.259 & 3.148 & 3.179 & 3.171 & 3.437 & 3.187 & 3.054 & 5.788 & 6.044 & 6.347 & 7.471 & 6.991 & 7.137 & 7.039 & 8.691 & 7.131 & 5.632 \\ 
   &  & p=36 & 2.965 & 2.999 & 5.06 & 20.97 & 3.278 & 8.422 & 9.38 & 27.015 & 10.898 & 25.616 & 5.788 & 6.044 & 8.109 & 13.311 & 7.224 & 7.861 & 7.768 & 18.756 & 8.223 & 13.414 \\ 
  \cmidrule{2-23} & \multirow{3}[2]{*}{$\rho=0.9$} & p=12 & 3.194 & 3.206 & 3.225 & 3.269 & 3.259 & 3.26 & 3.253 & 3.301 & 3.257 & 3.242 & 5.658 & 5.912 & 6.126 & 6.601 & 6.503 & 6.537 & 6.456 & 7.021 & 6.5 & 5.407 \\ 
   &  & p=20 & 3.194 & 3.222 & 3.268 & 3.483 & 3.366 & 3.411 & 3.401 & 3.66 & 3.406 & 3.286 & 5.658 & 5.949 & 6.249 & 7.405 & 6.9 & 7.083 & 6.971 & 8.63 & 7.008 & 5.517 \\ 
   &  & p=36 & 3.194 & 3.222 & 5.284 & 21.197 & 3.473 & 8.767 & 9.615 & 27.241 & 10.907 & 25.881 & 5.658 & 5.949 & 8.001 & 13.301 & 7.069 & 7.85 & 7.719 & 18.754 & 8.086 & 13.36 \\ 
  \midrule\multirow{9}[6]{*}{lsnr} & \multirow{3}[2]{*}{$\rho=0$} & p=12 & 4.383 & 4.41 & 4.433 & 4.47 & 4.449 & 4.456 & 4.46 & 4.498 & 4.463 & 4.362 & 2.091 & 2.577 & 3.188 & 3.821 & 3.514 & 3.542 & 3.578 & 4.312 & 3.658 & 0.988 \\ 
   &  & p=20 & 4.383 & 4.415 & 4.483 & 4.667 & 4.538 & 4.576 & 4.565 & 4.818 & 4.575 & 4.38 & 2.091 & 2.593 & 3.432 & 4.669 & 3.876 & 3.995 & 3.939 & 5.867 & 4.045 & 1.029 \\ 
   &  & p=36 & 4.383 & 4.415 & 6.472 & 22.323 & 4.639 & 9.644 & 10.615 & 28.253 & 11.724 & 24.815 & 2.091 & 2.593 & 5.665 & 11.044 & 4.071 & 4.747 & 4.66 & 16.683 & 5.059 & 7.973 \\ 
  \cmidrule{2-23} & \multirow{3}[2]{*}{$\rho=0.5$} & p=12 & 4.533 & 4.561 & 4.582 & 4.622 & 4.596 & 4.607 & 4.608 & 4.648 & 4.611 & 4.516 & 2.053 & 2.587 & 3.247 & 3.907 & 3.545 & 3.629 & 3.599 & 4.313 & 3.685 & 1.022 \\ 
   &  & p=20 & 4.533 & 4.567 & 4.625 & 4.816 & 4.687 & 4.729 & 4.717 & 4.963 & 4.725 & 4.531 & 2.053 & 2.604 & 3.438 & 4.706 & 3.908 & 4.082 & 3.966 & 5.837 & 4.082 & 1.05 \\ 
   &  & p=36 & 4.533 & 4.567 & 6.622 & 22.536 & 4.787 & 9.956 & 10.743 & 28.354 & 11.715 & 24.912 & 2.053 & 2.604 & 5.672 & 11.124 & 4.124 & 4.85 & 4.646 & 16.56 & 5.061 & 7.955 \\ 
  \cmidrule{2-23} & \multirow{3}[2]{*}{$\rho=0.9$} & p=12 & 4.753 & 4.775 & 4.799 & 4.845 & 4.811 & 4.818 & 4.82 & 4.869 & 4.825 & 4.741 & 2.13 & 2.56 & 3.14 & 3.846 & 3.477 & 3.439 & 3.418 & 4.234 & 3.51 & 1.043 \\ 
   &  & p=20 & 4.753 & 4.789 & 4.842 & 5.044 & 4.901 & 4.933 & 4.925 & 5.18 & 4.941 & 4.756 & 2.13 & 2.604 & 3.401 & 4.705 & 3.825 & 3.885 & 3.777 & 5.789 & 3.918 & 1.065 \\ 
   &  & p=36 & 4.753 & 4.789 & 6.843 & 22.75 & 5.002 & 10.041 & 10.952 & 28.503 & 11.885 & 25.045 & 2.13 & 2.604 & 5.698 & 11.107 & 4.04 & 4.603 & 4.463 & 16.412 & 4.844 & 7.879 \\ 
   \bottomrule 
\end{tabular}
}
\end{table}

%% latex table generated in R 4.0.2 by xtable 1.8-4 package
% Fri Jun 26 10:43:46 2020
\begin{table}[ht]
\centering
\caption{Fixed X, Sparse-Ex3, n=200} 
\scalebox{0.52}{
\begin{tabular}{|c|c|c|cc|cc|cc|ccc|c||cc|cc|cc|ccc|c|}
  \toprule 
 \multicolumn{1}{|c}{} & \multicolumn{1}{c}{} &       & RAICc & AICc  & RC$_p$ & C$_p$ & 10F CV & LOO CV & S$_p$ & $\widehat{\text{C}}_p$ & GCV   & BIC   & RAICc & AICc  & RC$_p$ & C$_p$ & 10F CV & LOO CV & S$_p$ & $\widehat{\text{C}}_p$ & GCV   & BIC  \\
 \cmidrule{4-23}\multicolumn{1}{|c}{} & \multicolumn{1}{c}{} &       & \multicolumn{10}{c||}{RMSE}                                                   & \multicolumn{10}{c|}{Relative Risk}       \\
 \midrule\multirow{9}[6]{*}{hsnr} & \multirow{3}[2]{*}{$\rho=0$} & p=12 & 0.16 & 0.162 & 0.162 & 0.164 & 0.163 & 0.163 & 0.163 & 0.165 & 0.163 & 0.146 & 0.005 & 0.005 & 0.005 & 0.005 & 0.005 & 0.005 & 0.005 & 0.005 & 0.005 & 0.004 \\ 
   &  & p=100 & 0.163 & 0.164 & 0.166 & 0.17 & 0.169 & 0.169 & 0.168 & 0.173 & 0.168 & 0.146 & 0.005 & 0.006 & 0.006 & 0.006 & 0.006 & 0.006 & 0.006 & 0.006 & 0.006 & 0.004 \\ 
   &  & p=196 & 0.163 & 0.164 & 0.271 & 0.379 & 0.169 & 0.169 & 0.168 & 0.46 & 0.169 & 0.257 & 0.005 & 0.006 & 0.041 & 1.771 & 0.006 & 0.006 & 0.006 & 2.982 & 0.009 & 1.419 \\ 
  \cmidrule{2-23} & \multirow{3}[2]{*}{$\rho=0.5$} & p=12 & 0.158 & 0.159 & 0.159 & 0.161 & 0.16 & 0.16 & 0.16 & 0.162 & 0.161 & 0.145 & 0.005 & 0.005 & 0.005 & 0.005 & 0.005 & 0.005 & 0.005 & 0.005 & 0.005 & 0.004 \\ 
   &  & p=100 & 0.159 & 0.162 & 0.163 & 0.167 & 0.166 & 0.166 & 0.165 & 0.17 & 0.166 & 0.145 & 0.005 & 0.005 & 0.005 & 0.006 & 0.006 & 0.006 & 0.006 & 0.006 & 0.006 & 0.004 \\ 
   &  & p=196 & 0.159 & 0.162 & 0.267 & 0.373 & 0.166 & 0.166 & 0.165 & 0.45 & 0.166 & 0.254 & 0.005 & 0.005 & 0.039 & 1.709 & 0.006 & 0.006 & 0.006 & 2.86 & 0.008 & 1.363 \\ 
  \cmidrule{2-23} & \multirow{3}[2]{*}{$\rho=0.9$} & p=12 & 0.155 & 0.157 & 0.157 & 0.158 & 0.159 & 0.158 & 0.158 & 0.159 & 0.158 & 0.143 & 0.005 & 0.005 & 0.005 & 0.005 & 0.005 & 0.005 & 0.005 & 0.005 & 0.005 & 0.004 \\ 
   &  & p=100 & 0.157 & 0.16 & 0.161 & 0.165 & 0.164 & 0.165 & 0.163 & 0.168 & 0.164 & 0.143 & 0.005 & 0.005 & 0.005 & 0.006 & 0.005 & 0.006 & 0.005 & 0.006 & 0.005 & 0.004 \\ 
   &  & p=196 & 0.157 & 0.16 & 0.263 & 0.367 & 0.164 & 0.165 & 0.163 & 0.444 & 0.164 & 0.251 & 0.005 & 0.005 & 0.038 & 1.66 & 0.005 & 0.006 & 0.005 & 2.805 & 0.008 & 1.336 \\ 
  \midrule\multirow{9}[6]{*}{msnr} & \multirow{3}[2]{*}{$\rho=0$} & p=12 & 0.468 & 0.471 & 0.473 & 0.477 & 0.476 & 0.476 & 0.476 & 0.48 & 0.476 & 0.427 & 0.044 & 0.045 & 0.045 & 0.046 & 0.045 & 0.046 & 0.046 & 0.047 & 0.046 & 0.037 \\ 
   &  & p=100 & 0.474 & 0.48 & 0.484 & 0.496 & 0.492 & 0.494 & 0.491 & 0.504 & 0.491 & 0.427 & 0.046 & 0.047 & 0.049 & 0.052 & 0.051 & 0.052 & 0.051 & 0.055 & 0.051 & 0.037 \\ 
   &  & p=196 & 0.474 & 0.48 & 0.791 & 1.104 & 0.492 & 0.494 & 0.491 & 1.341 & 0.493 & 0.751 & 0.046 & 0.047 & 0.345 & 15.052 & 0.051 & 0.052 & 0.051 & 25.351 & 0.075 & 12.059 \\ 
  \cmidrule{2-23} & \multirow{3}[2]{*}{$\rho=0.5$} & p=12 & 0.459 & 0.463 & 0.465 & 0.469 & 0.467 & 0.468 & 0.467 & 0.471 & 0.468 & 0.423 & 0.042 & 0.043 & 0.043 & 0.044 & 0.043 & 0.043 & 0.043 & 0.044 & 0.044 & 0.035 \\ 
   &  & p=100 & 0.465 & 0.472 & 0.474 & 0.487 & 0.484 & 0.484 & 0.482 & 0.497 & 0.483 & 0.423 & 0.043 & 0.045 & 0.046 & 0.05 & 0.048 & 0.049 & 0.048 & 0.053 & 0.048 & 0.035 \\ 
   &  & p=196 & 0.465 & 0.472 & 0.779 & 1.089 & 0.484 & 0.484 & 0.482 & 1.313 & 0.485 & 0.74 & 0.043 & 0.045 & 0.333 & 14.523 & 0.048 & 0.049 & 0.048 & 24.306 & 0.072 & 11.585 \\ 
  \cmidrule{2-23} & \multirow{3}[2]{*}{$\rho=0.9$} & p=12 & 0.453 & 0.457 & 0.459 & 0.461 & 0.463 & 0.461 & 0.461 & 0.464 & 0.461 & 0.418 & 0.04 & 0.04 & 0.041 & 0.041 & 0.041 & 0.041 & 0.041 & 0.042 & 0.041 & 0.033 \\ 
   &  & p=100 & 0.459 & 0.467 & 0.47 & 0.481 & 0.478 & 0.48 & 0.476 & 0.489 & 0.477 & 0.418 & 0.041 & 0.043 & 0.044 & 0.047 & 0.046 & 0.047 & 0.046 & 0.05 & 0.046 & 0.033 \\ 
   &  & p=196 & 0.459 & 0.467 & 0.768 & 1.071 & 0.478 & 0.48 & 0.476 & 1.293 & 0.479 & 0.731 & 0.041 & 0.043 & 0.323 & 14.107 & 0.046 & 0.047 & 0.046 & 23.843 & 0.069 & 11.357 \\ 
  \midrule\multirow{9}[6]{*}{lsnr} & \multirow{3}[2]{*}{$\rho=0$} & p=12 & 1.085 & 1.09 & 1.095 & 1.1 & 1.104 & 1.102 & 1.098 & 1.108 & 1.098 & 1.409 & 0.237 & 0.24 & 0.242 & 0.244 & 0.244 & 0.245 & 0.243 & 0.247 & 0.243 & 0.395 \\ 
   &  & p=100 & 1.098 & 1.109 & 1.121 & 1.147 & 1.139 & 1.139 & 1.131 & 1.159 & 1.131 & 1.409 & 0.247 & 0.253 & 0.261 & 0.278 & 0.27 & 0.272 & 0.268 & 0.286 & 0.268 & 0.395 \\ 
   &  & p=196 & 1.098 & 1.109 & 1.838 & 2.535 & 1.139 & 1.139 & 1.131 & 3.017 & 1.136 & 2.062 & 0.247 & 0.253 & 1.756 & 75.29 & 0.27 & 0.272 & 0.268 & 126.763 & 0.389 & 60.014 \\ 
  \cmidrule{2-23} & \multirow{3}[2]{*}{$\rho=0.5$} & p=12 & 1.062 & 1.066 & 1.069 & 1.075 & 1.08 & 1.079 & 1.071 & 1.079 & 1.073 & 1.386 & 0.224 & 0.226 & 0.228 & 0.23 & 0.231 & 0.231 & 0.228 & 0.232 & 0.229 & 0.383 \\ 
   &  & p=100 & 1.072 & 1.085 & 1.094 & 1.119 & 1.118 & 1.114 & 1.104 & 1.135 & 1.105 & 1.386 & 0.231 & 0.238 & 0.244 & 0.262 & 0.256 & 0.256 & 0.252 & 0.273 & 0.252 & 0.383 \\ 
   &  & p=196 & 1.072 & 1.085 & 1.81 & 2.493 & 1.118 & 1.114 & 1.104 & 2.948 & 1.11 & 2.034 & 0.231 & 0.238 & 1.694 & 72.642 & 0.256 & 0.256 & 0.252 & 121.536 & 0.369 & 58.105 \\ 
  \cmidrule{2-23} & \multirow{3}[2]{*}{$\rho=0.9$} & p=12 & 1.047 & 1.053 & 1.057 & 1.061 & 1.07 & 1.06 & 1.059 & 1.063 & 1.059 & 1.349 & 0.212 & 0.215 & 0.217 & 0.219 & 0.222 & 0.218 & 0.218 & 0.22 & 0.218 & 0.363 \\ 
   &  & p=100 & 1.06 & 1.076 & 1.081 & 1.103 & 1.1 & 1.101 & 1.092 & 1.118 & 1.093 & 1.349 & 0.22 & 0.229 & 0.234 & 0.248 & 0.242 & 0.247 & 0.242 & 0.26 & 0.242 & 0.363 \\ 
   &  & p=196 & 1.06 & 1.076 & 1.782 & 2.451 & 1.1 & 1.101 & 1.092 & 2.906 & 1.098 & 1.987 & 0.22 & 0.229 & 1.643 & 70.559 & 0.242 & 0.247 & 0.242 & 119.22 & 0.356 & 56.434 \\ 
   \midrule 
 \multicolumn{1}{|c}{} & \multicolumn{1}{c}{} &       & \multicolumn{10}{c||}{Relative Test Error}                                    & \multicolumn{10}{c|}{Proportion of Variance Explained} \\
\midrule\multirow{9}[6]{*}{hsnr} & \multirow{3}[2]{*}{$\rho=0$} & p=12 & 1.041 & 1.042 & 1.042 & 1.043 & 1.043 & 1.043 & 1.043 & 1.044 & 1.043 & 1.034 & 0.884 & 0.884 & 0.884 & 0.884 & 0.884 & 0.884 & 0.884 & 0.883 & 0.884 & 0.885 \\ 
   &  & p=100 & 1.043 & 1.044 & 1.046 & 1.049 & 1.048 & 1.048 & 1.048 & 1.051 & 1.048 & 1.034 & 0.884 & 0.883 & 0.883 & 0.883 & 0.883 & 0.883 & 0.883 & 0.883 & 0.883 & 0.885 \\ 
   &  & p=196 & 1.043 & 1.044 & 1.323 & 15.091 & 1.048 & 1.048 & 1.048 & 24.733 & 1.07 & 12.289 & 0.884 & 0.883 & 0.852 & -0.685 & 0.883 & 0.883 & 0.883 & -1.761 & 0.881 & -0.372 \\ 
  \cmidrule{2-23} & \multirow{3}[2]{*}{$\rho=0.5$} & p=12 & 1.041 & 1.041 & 1.042 & 1.042 & 1.042 & 1.042 & 1.042 & 1.043 & 1.042 & 1.034 & 0.887 & 0.887 & 0.887 & 0.887 & 0.887 & 0.887 & 0.887 & 0.887 & 0.887 & 0.888 \\ 
   &  & p=100 & 1.042 & 1.044 & 1.045 & 1.048 & 1.047 & 1.047 & 1.047 & 1.051 & 1.047 & 1.034 & 0.887 & 0.887 & 0.887 & 0.887 & 0.887 & 0.887 & 0.887 & 0.886 & 0.887 & 0.888 \\ 
   &  & p=196 & 1.042 & 1.044 & 1.323 & 15.091 & 1.047 & 1.047 & 1.047 & 24.583 & 1.07 & 12.24 & 0.887 & 0.887 & 0.857 & -0.632 & 0.887 & 0.887 & 0.887 & -1.658 & 0.884 & -0.324 \\ 
  \cmidrule{2-23} & \multirow{3}[2]{*}{$\rho=0.9$} & p=12 & 1.04 & 1.04 & 1.041 & 1.041 & 1.041 & 1.041 & 1.041 & 1.042 & 1.041 & 1.033 & 0.89 & 0.89 & 0.89 & 0.89 & 0.89 & 0.89 & 0.89 & 0.89 & 0.89 & 0.891 \\ 
   &  & p=100 & 1.041 & 1.043 & 1.044 & 1.047 & 1.046 & 1.047 & 1.046 & 1.05 & 1.046 & 1.033 & 0.89 & 0.89 & 0.89 & 0.89 & 0.89 & 0.89 & 0.89 & 0.889 & 0.89 & 0.891 \\ 
   &  & p=196 & 1.041 & 1.043 & 1.322 & 15.09 & 1.046 & 1.047 & 1.046 & 24.815 & 1.069 & 12.344 & 0.89 & 0.89 & 0.861 & -0.59 & 0.89 & 0.89 & 0.89 & -1.615 & 0.887 & -0.301 \\ 
  \midrule\multirow{9}[6]{*}{msnr} & \multirow{3}[2]{*}{$\rho=0$} & p=12 & 1.041 & 1.042 & 1.042 & 1.043 & 1.043 & 1.043 & 1.043 & 1.044 & 1.043 & 1.034 & 0.462 & 0.462 & 0.462 & 0.461 & 0.462 & 0.461 & 0.461 & 0.461 & 0.461 & 0.466 \\ 
   &  & p=100 & 1.043 & 1.044 & 1.046 & 1.049 & 1.048 & 1.048 & 1.048 & 1.051 & 1.048 & 1.034 & 0.461 & 0.461 & 0.46 & 0.458 & 0.459 & 0.459 & 0.459 & 0.457 & 0.459 & 0.466 \\ 
   &  & p=196 & 1.043 & 1.044 & 1.323 & 15.091 & 1.048 & 1.048 & 1.048 & 24.733 & 1.07 & 12.29 & 0.461 & 0.461 & 0.317 & -6.794 & 0.459 & 0.459 & 0.459 & -11.774 & 0.447 & -5.347 \\ 
  \cmidrule{2-23} & \multirow{3}[2]{*}{$\rho=0.5$} & p=12 & 1.041 & 1.041 & 1.042 & 1.042 & 1.042 & 1.042 & 1.042 & 1.043 & 1.042 & 1.034 & 0.472 & 0.471 & 0.471 & 0.471 & 0.471 & 0.471 & 0.471 & 0.471 & 0.471 & 0.475 \\ 
   &  & p=100 & 1.042 & 1.044 & 1.045 & 1.048 & 1.047 & 1.047 & 1.047 & 1.051 & 1.047 & 1.034 & 0.471 & 0.47 & 0.47 & 0.468 & 0.469 & 0.469 & 0.469 & 0.467 & 0.469 & 0.475 \\ 
   &  & p=196 & 1.042 & 1.044 & 1.323 & 15.091 & 1.047 & 1.047 & 1.047 & 24.583 & 1.07 & 12.24 & 0.471 & 0.47 & 0.329 & -6.66 & 0.469 & 0.469 & 0.469 & -11.477 & 0.457 & -5.213 \\ 
  \cmidrule{2-23} & \multirow{3}[2]{*}{$\rho=0.9$} & p=12 & 1.04 & 1.04 & 1.041 & 1.041 & 1.041 & 1.041 & 1.041 & 1.042 & 1.041 & 1.033 & 0.48 & 0.48 & 0.479 & 0.479 & 0.479 & 0.479 & 0.479 & 0.479 & 0.479 & 0.483 \\ 
   &  & p=100 & 1.041 & 1.043 & 1.044 & 1.047 & 1.046 & 1.047 & 1.046 & 1.05 & 1.046 & 1.033 & 0.479 & 0.478 & 0.478 & 0.476 & 0.477 & 0.476 & 0.477 & 0.475 & 0.477 & 0.483 \\ 
   &  & p=196 & 1.041 & 1.043 & 1.322 & 15.09 & 1.046 & 1.047 & 1.046 & 24.815 & 1.069 & 12.344 & 0.479 & 0.478 & 0.338 & -6.549 & 0.477 & 0.476 & 0.477 & -11.415 & 0.465 & -5.176 \\ 
  \midrule\multirow{9}[6]{*}{lsnr} & \multirow{3}[2]{*}{$\rho=0$} & p=12 & 1.044 & 1.045 & 1.045 & 1.046 & 1.046 & 1.046 & 1.045 & 1.046 & 1.045 & 1.074 & 0.12 & 0.12 & 0.12 & 0.119 & 0.119 & 0.119 & 0.119 & 0.119 & 0.119 & 0.095 \\ 
   &  & p=100 & 1.046 & 1.047 & 1.049 & 1.052 & 1.05 & 1.051 & 1.05 & 1.054 & 1.05 & 1.074 & 0.119 & 0.118 & 0.117 & 0.114 & 0.115 & 0.115 & 0.115 & 0.113 & 0.115 & 0.095 \\ 
   &  & p=196 & 1.046 & 1.047 & 1.329 & 15.097 & 1.05 & 1.051 & 1.05 & 24.734 & 1.073 & 12.237 & 0.119 & 0.118 & -0.119 & -11.716 & 0.115 & 0.115 & 0.115 & -19.833 & 0.096 & -9.307 \\ 
  \cmidrule{2-23} & \multirow{3}[2]{*}{$\rho=0.5$} & p=12 & 1.043 & 1.044 & 1.044 & 1.045 & 1.045 & 1.045 & 1.044 & 1.045 & 1.044 & 1.074 & 0.126 & 0.126 & 0.126 & 0.125 & 0.125 & 0.125 & 0.125 & 0.125 & 0.125 & 0.1 \\ 
   &  & p=100 & 1.045 & 1.046 & 1.047 & 1.051 & 1.05 & 1.05 & 1.049 & 1.053 & 1.049 & 1.074 & 0.125 & 0.124 & 0.123 & 0.12 & 0.121 & 0.121 & 0.122 & 0.118 & 0.121 & 0.1 \\ 
   &  & p=196 & 1.045 & 1.046 & 1.329 & 15.096 & 1.05 & 1.05 & 1.049 & 24.584 & 1.072 & 12.275 & 0.125 & 0.124 & -0.113 & -11.643 & 0.121 & 0.121 & 0.122 & -19.589 & 0.102 & -9.28 \\ 
  \cmidrule{2-23} & \multirow{3}[2]{*}{$\rho=0.9$} & p=12 & 1.042 & 1.043 & 1.043 & 1.044 & 1.044 & 1.044 & 1.044 & 1.044 & 1.044 & 1.073 & 0.131 & 0.131 & 0.13 & 0.13 & 0.13 & 0.13 & 0.13 & 0.13 & 0.13 & 0.106 \\ 
   &  & p=100 & 1.044 & 1.046 & 1.047 & 1.05 & 1.048 & 1.049 & 1.048 & 1.052 & 1.048 & 1.073 & 0.13 & 0.128 & 0.128 & 0.125 & 0.126 & 0.125 & 0.126 & 0.123 & 0.126 & 0.106 \\ 
   &  & p=196 & 1.044 & 1.046 & 1.328 & 15.095 & 1.048 & 1.049 & 1.048 & 24.816 & 1.071 & 12.273 & 0.13 & 0.128 & -0.107 & -11.582 & 0.126 & 0.125 & 0.126 & -19.684 & 0.107 & -9.23 \\ 
   \midrule 
 \multicolumn{1}{|c}{} & \multicolumn{1}{c}{} &       & \multicolumn{10}{c||}{Kullback-Leibler Discrepancy}                                    & \multicolumn{10}{c|}{Number of Variables} \\
\midrule\multirow{9}[6]{*}{hsnr} & \multirow{3}[2]{*}{$\rho=0$} & p=12 & 0.764 & 0.765 & 0.765 & 0.766 & 0.766 & 0.766 & 0.766 & 0.767 & 0.766 & 0.756 & 6.508 & 6.562 & 6.595 & 6.657 & 6.7 & 6.66 & 6.639 & 6.726 & 6.64 & 6.032 \\ 
   &  & p=100 & 0.766 & 0.768 & 0.77 & 0.774 & 0.772 & 0.773 & 0.772 & 0.776 & 0.772 & 0.756 & 6.628 & 6.728 & 6.867 & 7.144 & 7.101 & 7.068 & 6.993 & 7.323 & 6.994 & 6.032 \\ 
   &  & p=196 & 0.766 & 0.768 & 1.212 & 105.183 & 0.772 & 0.773 & 0.772 & 144.58 & 2.017 & 108.091 & 6.628 & 6.728 & 25.908 & 59.763 & 7.101 & 7.068 & 6.993 & 84.379 & 7.18 & 35.067 \\ 
  \cmidrule{2-23} & \multirow{3}[2]{*}{$\rho=0.5$} & p=12 & 0.728 & 0.729 & 0.73 & 0.73 & 0.73 & 0.73 & 0.73 & 0.731 & 0.73 & 0.721 & 6.464 & 6.529 & 6.552 & 6.625 & 6.651 & 6.61 & 6.597 & 6.672 & 6.615 & 6.033 \\ 
   &  & p=100 & 0.73 & 0.732 & 0.733 & 0.737 & 0.736 & 0.736 & 0.736 & 0.741 & 0.736 & 0.721 & 6.56 & 6.703 & 6.77 & 7.084 & 7.074 & 6.988 & 6.944 & 7.322 & 6.957 & 6.033 \\ 
   &  & p=196 & 0.73 & 0.732 & 1.177 & 105.147 & 0.736 & 0.736 & 0.736 & 144.347 & 1.981 & 108.028 & 6.56 & 6.703 & 25.926 & 59.793 & 7.074 & 6.988 & 6.944 & 83.869 & 7.143 & 35.067 \\ 
  \cmidrule{2-23} & \multirow{3}[2]{*}{$\rho=0.9$} & p=12 & 0.699 & 0.7 & 0.701 & 0.701 & 0.701 & 0.701 & 0.701 & 0.702 & 0.701 & 0.692 & 6.441 & 6.507 & 6.538 & 6.587 & 6.676 & 6.584 & 6.578 & 6.63 & 6.579 & 6.025 \\ 
   &  & p=100 & 0.701 & 0.703 & 0.705 & 0.709 & 0.707 & 0.708 & 0.707 & 0.711 & 0.707 & 0.692 & 6.55 & 6.705 & 6.802 & 7.075 & 7.043 & 7.022 & 6.936 & 7.264 & 6.953 & 6.025 \\ 
   &  & p=196 & 0.701 & 0.703 & 1.148 & 105.118 & 0.707 & 0.708 & 0.707 & 144.556 & 1.952 & 108.319 & 6.55 & 6.705 & 25.926 & 59.713 & 7.043 & 7.022 & 6.936 & 83.901 & 7.139 & 35.249 \\ 
  \midrule\multirow{9}[6]{*}{msnr} & \multirow{3}[2]{*}{$\rho=0$} & p=12 & 2.904 & 2.905 & 2.905 & 2.906 & 2.906 & 2.906 & 2.906 & 2.907 & 2.906 & 2.896 & 6.508 & 6.562 & 6.595 & 6.657 & 6.7 & 6.66 & 6.639 & 6.726 & 6.64 & 6.03 \\ 
   &  & p=100 & 2.906 & 2.908 & 2.91 & 2.914 & 2.912 & 2.913 & 2.912 & 2.916 & 2.912 & 2.896 & 6.628 & 6.728 & 6.867 & 7.144 & 7.101 & 7.068 & 6.993 & 7.323 & 6.994 & 6.03 \\ 
   &  & p=196 & 2.906 & 2.908 & 3.352 & 107.323 & 2.912 & 2.913 & 2.912 & 146.72 & 4.157 & 110.231 & 6.628 & 6.728 & 25.908 & 59.763 & 7.101 & 7.068 & 6.993 & 84.379 & 7.18 & 35.065 \\ 
  \cmidrule{2-23} & \multirow{3}[2]{*}{$\rho=0.5$} & p=12 & 2.869 & 2.869 & 2.87 & 2.871 & 2.87 & 2.87 & 2.87 & 2.871 & 2.87 & 2.861 & 6.464 & 6.529 & 6.552 & 6.625 & 6.651 & 6.61 & 6.597 & 6.672 & 6.615 & 6.031 \\ 
   &  & p=100 & 2.87 & 2.872 & 2.873 & 2.878 & 2.876 & 2.876 & 2.876 & 2.881 & 2.876 & 2.861 & 6.56 & 6.703 & 6.77 & 7.084 & 7.074 & 6.988 & 6.944 & 7.322 & 6.957 & 6.031 \\ 
   &  & p=196 & 2.87 & 2.872 & 3.317 & 107.287 & 2.876 & 2.876 & 2.876 & 146.487 & 4.121 & 110.169 & 6.56 & 6.703 & 25.926 & 59.793 & 7.074 & 6.988 & 6.944 & 83.869 & 7.143 & 35.066 \\ 
  \cmidrule{2-23} & \multirow{3}[2]{*}{$\rho=0.9$} & p=12 & 2.84 & 2.84 & 2.841 & 2.841 & 2.841 & 2.841 & 2.841 & 2.842 & 2.841 & 2.832 & 6.441 & 6.507 & 6.538 & 6.587 & 6.676 & 6.584 & 6.578 & 6.63 & 6.579 & 6.025 \\ 
   &  & p=100 & 2.841 & 2.844 & 2.845 & 2.849 & 2.847 & 2.848 & 2.847 & 2.852 & 2.847 & 2.832 & 6.55 & 6.705 & 6.802 & 7.075 & 7.043 & 7.022 & 6.936 & 7.264 & 6.953 & 6.025 \\ 
   &  & p=196 & 2.841 & 2.844 & 3.288 & 107.258 & 2.847 & 2.848 & 2.847 & 146.696 & 4.092 & 110.459 & 6.55 & 6.705 & 25.926 & 59.713 & 7.043 & 7.022 & 6.936 & 83.901 & 7.139 & 35.249 \\ 
  \midrule\multirow{9}[6]{*}{lsnr} & \multirow{3}[2]{*}{$\rho=0$} & p=12 & 4.517 & 4.517 & 4.518 & 4.518 & 4.518 & 4.518 & 4.518 & 4.519 & 4.518 & 4.545 & 6.304 & 6.373 & 6.414 & 6.481 & 6.493 & 6.463 & 6.465 & 6.553 & 6.466 & 4.471 \\ 
   &  & p=100 & 4.518 & 4.52 & 4.522 & 4.526 & 4.524 & 4.525 & 4.524 & 4.528 & 4.524 & 4.545 & 6.42 & 6.551 & 6.683 & 6.963 & 6.887 & 6.865 & 6.827 & 7.147 & 6.828 & 4.471 \\ 
   &  & p=196 & 4.518 & 4.52 & 4.967 & 108.938 & 4.524 & 4.525 & 4.524 & 148.331 & 5.769 & 111.571 & 6.42 & 6.551 & 25.638 & 59.456 & 6.887 & 6.865 & 6.827 & 84.267 & 7.014 & 33.529 \\ 
  \cmidrule{2-23} & \multirow{3}[2]{*}{$\rho=0.5$} & p=12 & 4.481 & 4.481 & 4.481 & 4.482 & 4.482 & 4.482 & 4.482 & 4.482 & 4.482 & 4.51 & 6.28 & 6.356 & 6.382 & 6.463 & 6.454 & 6.423 & 6.435 & 6.517 & 6.448 & 4.497 \\ 
   &  & p=100 & 4.482 & 4.484 & 4.485 & 4.489 & 4.488 & 4.488 & 4.487 & 4.492 & 4.487 & 4.51 & 6.368 & 6.523 & 6.585 & 6.917 & 6.888 & 6.806 & 6.787 & 7.177 & 6.8 & 4.497 \\ 
   &  & p=196 & 4.482 & 4.484 & 4.932 & 108.901 & 4.488 & 4.488 & 4.487 & 148.097 & 5.732 & 111.812 & 6.368 & 6.523 & 25.654 & 59.485 & 6.888 & 6.806 & 6.787 & 83.78 & 6.986 & 33.734 \\ 
  \cmidrule{2-23} & \multirow{3}[2]{*}{$\rho=0.9$} & p=12 & 4.452 & 4.452 & 4.453 & 4.453 & 4.453 & 4.453 & 4.453 & 4.453 & 4.453 & 4.48 & 6.245 & 6.325 & 6.357 & 6.428 & 6.482 & 6.413 & 6.402 & 6.485 & 6.411 & 4.559 \\ 
   &  & p=100 & 4.453 & 4.455 & 4.456 & 4.46 & 4.459 & 4.46 & 4.458 & 4.463 & 4.459 & 4.48 & 6.355 & 6.524 & 6.634 & 6.892 & 6.826 & 6.859 & 6.753 & 7.121 & 6.778 & 4.559 \\ 
   &  & p=196 & 4.453 & 4.455 & 4.903 & 108.872 & 4.459 & 4.46 & 4.458 & 148.306 & 5.703 & 111.782 & 6.355 & 6.524 & 25.633 & 59.439 & 6.826 & 6.859 & 6.753 & 83.814 & 6.964 & 33.789 \\ 
   \bottomrule 
\end{tabular}
}
\end{table}

%% latex table generated in R 4.0.2 by xtable 1.8-4 package
% Fri Jun 26 10:43:51 2020
\begin{table}[ht]
\centering
\caption{Random X, Sparse-Ex3, n=1000} 
\scalebox{0.52}{
\begin{tabular}{|c|c|c|cc|cc|cc|ccc|c||cc|cc|cc|ccc|c|}
  \toprule 
 \multicolumn{1}{|c}{} & \multicolumn{1}{c}{} &       & RAICc & AICc  & RC$_p$ & C$_p$ & 10F CV & LOO CV & S$_p$ & $\widehat{\text{C}}_p$ & GCV   & BIC   & RAICc & AICc  & RC$_p$ & C$_p$ & 10F CV & LOO CV & S$_p$ & $\widehat{\text{C}}_p$ & GCV   & BIC  \\
 \cmidrule{4-23}\multicolumn{1}{|c}{} & \multicolumn{1}{c}{} &       & \multicolumn{10}{c||}{RMSE}                                                   & \multicolumn{10}{c|}{Relative Risk}       \\
 \midrule\multirow{9}[6]{*}{hsnr} & \multirow{3}[2]{*}{$\rho=0$} & p=12 & 0.07 & 0.07 & 0.07 & 0.07 & 0.07 & 0.07 & 0.07 & 0.07 & 0.07 & 0.062 & 0.001 & 0.001 & 0.001 & 0.001 & 0.001 & 0.001 & 0.001 & 0.001 & 0.001 & 0.001 \\ 
   &  & p=500 & 0.072 & 0.072 & 0.073 & 0.073 & 0.072 & 0.073 & 0.072 & 0.073 & 0.073 & 0.062 & 0.001 & 0.001 & 0.001 & 0.001 & 0.001 & 0.001 & 0.001 & 0.001 & 0.001 & 0.001 \\ 
   &  & p=996 & 0.072 & 0.072 & 0.184 & 0.291 & 0.072 & 0.073 & 0.072 & 0.361 & 0.073 & 0.138 & 0.001 & 0.001 & 0.026 & 6.795 & 0.001 & 0.001 & 0.001 & 12.379 & 0.001 & 3.79 \\ 
  \cmidrule{2-23} & \multirow{3}[2]{*}{$\rho=0.5$} & p=12 & 0.07 & 0.07 & 0.07 & 0.07 & 0.07 & 0.07 & 0.07 & 0.07 & 0.07 & 0.062 & 0.001 & 0.001 & 0.001 & 0.001 & 0.001 & 0.001 & 0.001 & 0.001 & 0.001 & 0.001 \\ 
   &  & p=500 & 0.072 & 0.072 & 0.072 & 0.073 & 0.072 & 0.072 & 0.072 & 0.072 & 0.072 & 0.062 & 0.001 & 0.001 & 0.001 & 0.001 & 0.001 & 0.001 & 0.001 & 0.001 & 0.001 & 0.001 \\ 
   &  & p=996 & 0.072 & 0.072 & 0.185 & 0.291 & 0.072 & 0.072 & 0.072 & 0.36 & 0.072 & 0.138 & 0.001 & 0.001 & 0.026 & 6.806 & 0.001 & 0.001 & 0.001 & 12.326 & 0.001 & 3.806 \\ 
  \cmidrule{2-23} & \multirow{3}[2]{*}{$\rho=0.9$} & p=12 & 0.07 & 0.07 & 0.07 & 0.07 & 0.07 & 0.07 & 0.07 & 0.07 & 0.07 & 0.062 & 0.001 & 0.001 & 0.001 & 0.001 & 0.001 & 0.001 & 0.001 & 0.001 & 0.001 & 0.001 \\ 
   &  & p=500 & 0.071 & 0.072 & 0.072 & 0.073 & 0.072 & 0.072 & 0.072 & 0.073 & 0.072 & 0.062 & 0.001 & 0.001 & 0.001 & 0.001 & 0.001 & 0.001 & 0.001 & 0.001 & 0.001 & 0.001 \\ 
   &  & p=996 & 0.071 & 0.072 & 0.185 & 0.292 & 0.072 & 0.072 & 0.072 & 0.36 & 0.072 & 0.138 & 0.001 & 0.001 & 0.026 & 6.822 & 0.001 & 0.001 & 0.001 & 12.334 & 0.001 & 3.821 \\ 
  \midrule\multirow{9}[6]{*}{msnr} & \multirow{3}[2]{*}{$\rho=0$} & p=12 & 0.204 & 0.204 & 0.204 & 0.205 & 0.204 & 0.205 & 0.205 & 0.205 & 0.205 & 0.181 & 0.008 & 0.008 & 0.008 & 0.008 & 0.008 & 0.008 & 0.008 & 0.008 & 0.008 & 0.006 \\ 
   &  & p=500 & 0.209 & 0.21 & 0.212 & 0.213 & 0.211 & 0.212 & 0.211 & 0.212 & 0.211 & 0.181 & 0.008 & 0.008 & 0.009 & 0.009 & 0.009 & 0.009 & 0.009 & 0.009 & 0.009 & 0.006 \\ 
   &  & p=996 & 0.209 & 0.21 & 0.538 & 0.848 & 0.211 & 0.212 & 0.211 & 1.053 & 0.211 & 0.401 & 0.008 & 0.008 & 0.221 & 57.756 & 0.009 & 0.009 & 0.009 & 105.22 & 0.009 & 32.212 \\ 
  \cmidrule{2-23} & \multirow{3}[2]{*}{$\rho=0.5$} & p=12 & 0.203 & 0.204 & 0.204 & 0.204 & 0.205 & 0.204 & 0.204 & 0.204 & 0.204 & 0.18 & 0.008 & 0.008 & 0.008 & 0.008 & 0.008 & 0.008 & 0.008 & 0.008 & 0.008 & 0.006 \\ 
   &  & p=500 & 0.209 & 0.21 & 0.211 & 0.212 & 0.211 & 0.211 & 0.21 & 0.211 & 0.21 & 0.18 & 0.008 & 0.008 & 0.009 & 0.009 & 0.009 & 0.009 & 0.008 & 0.009 & 0.008 & 0.006 \\ 
   &  & p=996 & 0.209 & 0.21 & 0.538 & 0.848 & 0.211 & 0.211 & 0.21 & 1.049 & 0.21 & 0.401 & 0.008 & 0.008 & 0.221 & 57.85 & 0.009 & 0.009 & 0.008 & 104.772 & 0.008 & 32.349 \\ 
  \cmidrule{2-23} & \multirow{3}[2]{*}{$\rho=0.9$} & p=12 & 0.203 & 0.203 & 0.203 & 0.204 & 0.203 & 0.204 & 0.203 & 0.204 & 0.204 & 0.181 & 0.008 & 0.008 & 0.008 & 0.008 & 0.008 & 0.008 & 0.008 & 0.008 & 0.008 & 0.006 \\ 
   &  & p=500 & 0.208 & 0.209 & 0.21 & 0.212 & 0.211 & 0.211 & 0.21 & 0.212 & 0.21 & 0.181 & 0.008 & 0.008 & 0.009 & 0.009 & 0.009 & 0.009 & 0.008 & 0.009 & 0.008 & 0.006 \\ 
   &  & p=996 & 0.208 & 0.209 & 0.54 & 0.85 & 0.211 & 0.211 & 0.21 & 1.049 & 0.21 & 0.402 & 0.008 & 0.008 & 0.222 & 57.983 & 0.009 & 0.009 & 0.008 & 104.836 & 0.008 & 32.482 \\ 
  \midrule\multirow{9}[6]{*}{lsnr} & \multirow{3}[2]{*}{$\rho=0$} & p=12 & 0.456 & 0.456 & 0.457 & 0.458 & 0.457 & 0.458 & 0.458 & 0.458 & 0.458 & 0.404 & 0.039 & 0.039 & 0.039 & 0.039 & 0.039 & 0.039 & 0.039 & 0.039 & 0.039 & 0.03 \\ 
   &  & p=500 & 0.468 & 0.469 & 0.475 & 0.476 & 0.472 & 0.473 & 0.472 & 0.474 & 0.473 & 0.404 & 0.042 & 0.042 & 0.043 & 0.044 & 0.043 & 0.043 & 0.043 & 0.043 & 0.043 & 0.03 \\ 
   &  & p=996 & 0.468 & 0.469 & 1.203 & 1.897 & 0.472 & 0.473 & 0.472 & 2.354 & 0.473 & 0.898 & 0.042 & 0.042 & 1.106 & 288.778 & 0.043 & 0.043 & 0.043 & 526.102 & 0.043 & 161.058 \\ 
  \cmidrule{2-23} & \multirow{3}[2]{*}{$\rho=0.5$} & p=12 & 0.455 & 0.455 & 0.456 & 0.456 & 0.457 & 0.457 & 0.456 & 0.457 & 0.456 & 0.404 & 0.039 & 0.039 & 0.039 & 0.039 & 0.039 & 0.039 & 0.039 & 0.039 & 0.039 & 0.03 \\ 
   &  & p=500 & 0.468 & 0.469 & 0.472 & 0.474 & 0.472 & 0.471 & 0.47 & 0.472 & 0.47 & 0.404 & 0.042 & 0.042 & 0.043 & 0.044 & 0.043 & 0.043 & 0.042 & 0.043 & 0.042 & 0.03 \\ 
   &  & p=996 & 0.468 & 0.469 & 1.203 & 1.896 & 0.472 & 0.471 & 0.47 & 2.345 & 0.47 & 0.897 & 0.042 & 0.042 & 1.107 & 289.249 & 0.043 & 0.043 & 0.042 & 523.862 & 0.042 & 161.744 \\ 
  \cmidrule{2-23} & \multirow{3}[2]{*}{$\rho=0.9$} & p=12 & 0.453 & 0.454 & 0.454 & 0.455 & 0.455 & 0.456 & 0.455 & 0.456 & 0.455 & 0.404 & 0.038 & 0.038 & 0.038 & 0.039 & 0.039 & 0.039 & 0.039 & 0.039 & 0.039 & 0.03 \\ 
   &  & p=500 & 0.466 & 0.468 & 0.47 & 0.475 & 0.472 & 0.472 & 0.47 & 0.474 & 0.47 & 0.404 & 0.041 & 0.042 & 0.043 & 0.044 & 0.043 & 0.043 & 0.042 & 0.043 & 0.042 & 0.03 \\ 
   &  & p=996 & 0.466 & 0.468 & 1.208 & 1.901 & 0.472 & 0.472 & 0.47 & 2.346 & 0.47 & 0.899 & 0.041 & 0.042 & 1.108 & 289.917 & 0.043 & 0.043 & 0.042 & 524.181 & 0.042 & 162.41 \\ 
   \midrule 
 \multicolumn{1}{|c}{} & \multicolumn{1}{c}{} &       & \multicolumn{10}{c||}{Relative Test Error}                                    & \multicolumn{10}{c|}{Proportion of Variance Explained} \\
\midrule\multirow{9}[6]{*}{hsnr} & \multirow{3}[2]{*}{$\rho=0$} & p=12 & 1.008 & 1.008 & 1.008 & 1.008 & 1.008 & 1.008 & 1.008 & 1.008 & 1.008 & 1.006 & 0.894 & 0.894 & 0.894 & 0.894 & 0.894 & 0.894 & 0.894 & 0.894 & 0.894 & 0.894 \\ 
   &  & p=500 & 1.008 & 1.008 & 1.009 & 1.009 & 1.009 & 1.009 & 1.009 & 1.009 & 1.009 & 1.006 & 0.894 & 0.894 & 0.894 & 0.894 & 0.894 & 0.894 & 0.894 & 0.894 & 0.894 & 0.894 \\ 
   &  & p=996 & 1.008 & 1.008 & 1.222 & 58.89 & 1.009 & 1.009 & 1.009 & 106.503 & 1.009 & 33.197 & 0.894 & 0.894 & 0.872 & -5.185 & 0.894 & 0.894 & 0.894 & -10.182 & 0.894 & -2.495 \\ 
  \cmidrule{2-23} & \multirow{3}[2]{*}{$\rho=0.5$} & p=12 & 1.008 & 1.008 & 1.008 & 1.008 & 1.008 & 1.008 & 1.008 & 1.008 & 1.008 & 1.006 & 0.894 & 0.894 & 0.894 & 0.894 & 0.894 & 0.894 & 0.894 & 0.894 & 0.894 & 0.894 \\ 
   &  & p=500 & 1.008 & 1.008 & 1.009 & 1.009 & 1.009 & 1.009 & 1.008 & 1.009 & 1.008 & 1.006 & 0.894 & 0.894 & 0.894 & 0.894 & 0.894 & 0.894 & 0.894 & 0.894 & 0.894 & 0.894 \\ 
   &  & p=996 & 1.008 & 1.008 & 1.222 & 58.889 & 1.009 & 1.009 & 1.008 & 106.067 & 1.008 & 33.197 & 0.894 & 0.894 & 0.872 & -5.194 & 0.894 & 0.894 & 0.894 & -10.135 & 0.894 & -2.508 \\ 
  \cmidrule{2-23} & \multirow{3}[2]{*}{$\rho=0.9$} & p=12 & 1.008 & 1.008 & 1.008 & 1.008 & 1.008 & 1.008 & 1.008 & 1.008 & 1.008 & 1.006 & 0.894 & 0.894 & 0.894 & 0.894 & 0.894 & 0.894 & 0.894 & 0.894 & 0.894 & 0.894 \\ 
   &  & p=500 & 1.008 & 1.008 & 1.009 & 1.009 & 1.009 & 1.009 & 1.009 & 1.009 & 1.009 & 1.006 & 0.894 & 0.894 & 0.894 & 0.894 & 0.894 & 0.894 & 0.894 & 0.894 & 0.894 & 0.894 \\ 
   &  & p=996 & 1.008 & 1.008 & 1.222 & 58.89 & 1.009 & 1.009 & 1.009 & 106.067 & 1.009 & 33.197 & 0.894 & 0.894 & 0.872 & -5.206 & 0.894 & 0.894 & 0.894 & -10.141 & 0.894 & -2.52 \\ 
  \midrule\multirow{9}[6]{*}{msnr} & \multirow{3}[2]{*}{$\rho=0$} & p=12 & 1.008 & 1.008 & 1.008 & 1.008 & 1.008 & 1.008 & 1.008 & 1.008 & 1.008 & 1.006 & 0.496 & 0.496 & 0.496 & 0.496 & 0.496 & 0.496 & 0.496 & 0.496 & 0.496 & 0.497 \\ 
   &  & p=500 & 1.008 & 1.008 & 1.009 & 1.009 & 1.009 & 1.009 & 1.009 & 1.009 & 1.009 & 1.006 & 0.496 & 0.496 & 0.496 & 0.496 & 0.496 & 0.496 & 0.496 & 0.496 & 0.496 & 0.497 \\ 
   &  & p=996 & 1.008 & 1.008 & 1.222 & 58.89 & 1.009 & 1.009 & 1.009 & 106.503 & 1.009 & 33.197 & 0.496 & 0.496 & 0.39 & -28.398 & 0.496 & 0.496 & 0.496 & -52.155 & 0.496 & -15.594 \\ 
  \cmidrule{2-23} & \multirow{3}[2]{*}{$\rho=0.5$} & p=12 & 1.008 & 1.008 & 1.008 & 1.008 & 1.008 & 1.008 & 1.008 & 1.008 & 1.008 & 1.006 & 0.497 & 0.497 & 0.497 & 0.497 & 0.497 & 0.497 & 0.497 & 0.497 & 0.497 & 0.497 \\ 
   &  & p=500 & 1.008 & 1.008 & 1.009 & 1.009 & 1.009 & 1.009 & 1.008 & 1.009 & 1.008 & 1.006 & 0.496 & 0.496 & 0.496 & 0.496 & 0.496 & 0.496 & 0.496 & 0.496 & 0.496 & 0.497 \\ 
   &  & p=996 & 1.008 & 1.008 & 1.222 & 58.889 & 1.009 & 1.009 & 1.008 & 106.067 & 1.008 & 33.197 & 0.496 & 0.496 & 0.39 & -28.421 & 0.496 & 0.496 & 0.496 & -51.932 & 0.496 & -15.628 \\ 
  \cmidrule{2-23} & \multirow{3}[2]{*}{$\rho=0.9$} & p=12 & 1.008 & 1.008 & 1.008 & 1.008 & 1.008 & 1.008 & 1.008 & 1.008 & 1.008 & 1.006 & 0.497 & 0.497 & 0.497 & 0.497 & 0.497 & 0.497 & 0.497 & 0.497 & 0.497 & 0.498 \\ 
   &  & p=500 & 1.008 & 1.008 & 1.009 & 1.009 & 1.009 & 1.009 & 1.009 & 1.009 & 1.009 & 1.006 & 0.496 & 0.496 & 0.496 & 0.496 & 0.496 & 0.496 & 0.496 & 0.496 & 0.496 & 0.498 \\ 
   &  & p=996 & 1.008 & 1.008 & 1.222 & 58.89 & 1.009 & 1.009 & 1.009 & 106.067 & 1.009 & 33.197 & 0.496 & 0.496 & 0.39 & -28.453 & 0.496 & 0.496 & 0.496 & -51.946 & 0.496 & -15.66 \\ 
  \midrule\multirow{9}[6]{*}{lsnr} & \multirow{3}[2]{*}{$\rho=0$} & p=12 & 1.008 & 1.008 & 1.008 & 1.008 & 1.008 & 1.008 & 1.008 & 1.008 & 1.008 & 1.006 & 0.16 & 0.16 & 0.16 & 0.16 & 0.16 & 0.16 & 0.16 & 0.16 & 0.16 & 0.162 \\ 
   &  & p=500 & 1.008 & 1.008 & 1.009 & 1.009 & 1.009 & 1.009 & 1.009 & 1.009 & 1.009 & 1.006 & 0.16 & 0.16 & 0.16 & 0.16 & 0.16 & 0.16 & 0.16 & 0.16 & 0.16 & 0.162 \\ 
   &  & p=996 & 1.008 & 1.008 & 1.222 & 58.89 & 1.009 & 1.009 & 1.009 & 106.503 & 1.009 & 33.197 & 0.16 & 0.16 & -0.018 & -48.044 & 0.16 & 0.16 & 0.16 & -87.69 & 0.16 & -26.659 \\ 
  \cmidrule{2-23} & \multirow{3}[2]{*}{$\rho=0.5$} & p=12 & 1.008 & 1.008 & 1.008 & 1.008 & 1.008 & 1.008 & 1.008 & 1.008 & 1.008 & 1.006 & 0.161 & 0.161 & 0.161 & 0.161 & 0.16 & 0.16 & 0.161 & 0.16 & 0.161 & 0.162 \\ 
   &  & p=500 & 1.008 & 1.008 & 1.009 & 1.009 & 1.009 & 1.009 & 1.008 & 1.009 & 1.008 & 1.006 & 0.16 & 0.16 & 0.16 & 0.16 & 0.16 & 0.16 & 0.16 & 0.16 & 0.16 & 0.162 \\ 
   &  & p=996 & 1.008 & 1.008 & 1.222 & 58.889 & 1.009 & 1.009 & 1.008 & 106.067 & 1.008 & 33.197 & 0.16 & 0.16 & -0.018 & -48.056 & 0.16 & 0.16 & 0.16 & -87.323 & 0.16 & -26.677 \\ 
  \cmidrule{2-23} & \multirow{3}[2]{*}{$\rho=0.9$} & p=12 & 1.008 & 1.008 & 1.008 & 1.008 & 1.008 & 1.008 & 1.008 & 1.008 & 1.008 & 1.006 & 0.161 & 0.161 & 0.161 & 0.161 & 0.161 & 0.161 & 0.161 & 0.161 & 0.161 & 0.162 \\ 
   &  & p=500 & 1.008 & 1.008 & 1.009 & 1.009 & 1.009 & 1.009 & 1.009 & 1.009 & 1.009 & 1.006 & 0.16 & 0.16 & 0.16 & 0.16 & 0.16 & 0.16 & 0.16 & 0.16 & 0.16 & 0.162 \\ 
   &  & p=996 & 1.008 & 1.008 & 1.222 & 58.89 & 1.009 & 1.009 & 1.009 & 106.067 & 1.009 & 33.197 & 0.16 & 0.16 & -0.018 & -48.074 & 0.16 & 0.16 & 0.16 & -87.33 & 0.16 & -26.695 \\ 
   \midrule 
 \multicolumn{1}{|c}{} & \multicolumn{1}{c}{} &       & \multicolumn{10}{c||}{Kullback-Leibler Discrepancy}                                    & \multicolumn{10}{c|}{Number of Variables} \\
\midrule\multirow{9}[6]{*}{hsnr} & \multirow{3}[2]{*}{$\rho=0$} & p=12 & 0.659 & 0.659 & 0.659 & 0.659 & 0.659 & 0.659 & 0.659 & 0.659 & 0.659 & 0.657 & 6.617 & 6.62 & 6.631 & 6.639 & 6.68 & 6.653 & 6.639 & 6.645 & 6.639 & 6.011 \\ 
   &  & p=500 & 0.66 & 0.66 & 0.66 & 0.66 & 0.66 & 0.66 & 0.66 & 0.66 & 0.66 & 0.657 & 6.861 & 6.886 & 7.005 & 7.047 & 7.025 & 6.961 & 6.938 & 6.999 & 6.957 & 6.011 \\ 
   &  & p=996 & 0.66 & 0.66 & 1.456 & 76903.349 & 0.66 & 0.66 & 0.66 & 107434.883 & 0.66 & 68391.611 & 6.861 & 6.886 & 96.618 & 275.033 & 7.025 & 6.961 & 6.938 & 379.786 & 6.957 & 103.022 \\ 
  \cmidrule{2-23} & \multirow{3}[2]{*}{$\rho=0.5$} & p=12 & 0.659 & 0.659 & 0.659 & 0.659 & 0.659 & 0.659 & 0.659 & 0.659 & 0.659 & 0.657 & 6.598 & 6.609 & 6.616 & 6.622 & 6.703 & 6.638 & 6.622 & 6.636 & 6.622 & 6.011 \\ 
   &  & p=500 & 0.66 & 0.66 & 0.66 & 0.66 & 0.66 & 0.66 & 0.66 & 0.66 & 0.66 & 0.657 & 6.859 & 6.883 & 6.974 & 7.031 & 7.054 & 6.925 & 6.902 & 6.967 & 6.902 & 6.011 \\ 
   &  & p=996 & 0.66 & 0.66 & 1.456 & 76903.349 & 0.66 & 0.66 & 0.66 & 107212.014 & 0.66 & 68391.611 & 6.859 & 6.883 & 96.635 & 274.995 & 7.054 & 6.925 & 6.902 & 378.725 & 6.902 & 103.02 \\ 
  \cmidrule{2-23} & \multirow{3}[2]{*}{$\rho=0.9$} & p=12 & 0.658 & 0.658 & 0.658 & 0.659 & 0.658 & 0.659 & 0.659 & 0.659 & 0.659 & 0.657 & 6.569 & 6.585 & 6.585 & 6.606 & 6.656 & 6.621 & 6.598 & 6.62 & 6.607 & 6.01 \\ 
   &  & p=500 & 0.659 & 0.659 & 0.659 & 0.66 & 0.659 & 0.659 & 0.659 & 0.66 & 0.659 & 0.657 & 6.823 & 6.863 & 6.931 & 7.058 & 7.054 & 6.963 & 6.914 & 7.003 & 6.923 & 6.01 \\ 
   &  & p=996 & 0.659 & 0.659 & 1.456 & 76903.349 & 0.659 & 0.659 & 0.659 & 107212.013 & 0.659 & 68391.611 & 6.823 & 6.863 & 96.769 & 275.168 & 7.054 & 6.963 & 6.914 & 378.769 & 6.923 & 103.02 \\ 
  \midrule\multirow{9}[6]{*}{msnr} & \multirow{3}[2]{*}{$\rho=0$} & p=12 & 2.799 & 2.799 & 2.799 & 2.799 & 2.799 & 2.799 & 2.799 & 2.799 & 2.799 & 2.797 & 6.617 & 6.62 & 6.631 & 6.639 & 6.68 & 6.653 & 6.639 & 6.645 & 6.639 & 6.011 \\ 
   &  & p=500 & 2.8 & 2.8 & 2.8 & 2.8 & 2.8 & 2.8 & 2.8 & 2.8 & 2.8 & 2.797 & 6.861 & 6.886 & 7.005 & 7.047 & 7.025 & 6.961 & 6.938 & 6.999 & 6.957 & 6.011 \\ 
   &  & p=996 & 2.8 & 2.8 & 3.596 & 76905.489 & 2.8 & 2.8 & 2.8 & 107437.024 & 2.8 & 68393.751 & 6.861 & 6.886 & 96.618 & 275.033 & 7.025 & 6.961 & 6.938 & 379.786 & 6.957 & 103.022 \\ 
  \cmidrule{2-23} & \multirow{3}[2]{*}{$\rho=0.5$} & p=12 & 2.799 & 2.799 & 2.799 & 2.799 & 2.799 & 2.799 & 2.799 & 2.799 & 2.799 & 2.797 & 6.598 & 6.609 & 6.616 & 6.622 & 6.703 & 6.638 & 6.622 & 6.636 & 6.622 & 6.011 \\ 
   &  & p=500 & 2.8 & 2.8 & 2.8 & 2.8 & 2.8 & 2.8 & 2.8 & 2.8 & 2.8 & 2.797 & 6.859 & 6.883 & 6.974 & 7.031 & 7.054 & 6.925 & 6.902 & 6.967 & 6.902 & 6.011 \\ 
   &  & p=996 & 2.8 & 2.8 & 3.596 & 76905.489 & 2.8 & 2.8 & 2.8 & 107214.154 & 2.8 & 68393.751 & 6.859 & 6.883 & 96.635 & 274.995 & 7.054 & 6.925 & 6.902 & 378.725 & 6.902 & 103.02 \\ 
  \cmidrule{2-23} & \multirow{3}[2]{*}{$\rho=0.9$} & p=12 & 2.799 & 2.799 & 2.799 & 2.799 & 2.799 & 2.799 & 2.799 & 2.799 & 2.799 & 2.797 & 6.569 & 6.585 & 6.585 & 6.606 & 6.656 & 6.621 & 6.598 & 6.62 & 6.607 & 6.01 \\ 
   &  & p=500 & 2.799 & 2.799 & 2.799 & 2.8 & 2.799 & 2.799 & 2.799 & 2.8 & 2.799 & 2.797 & 6.823 & 6.863 & 6.931 & 7.058 & 7.054 & 6.963 & 6.914 & 7.003 & 6.923 & 6.01 \\ 
   &  & p=996 & 2.799 & 2.799 & 3.596 & 76905.489 & 2.799 & 2.799 & 2.799 & 107214.153 & 2.799 & 68393.751 & 6.823 & 6.863 & 96.769 & 275.168 & 7.054 & 6.963 & 6.914 & 378.769 & 6.923 & 103.02 \\ 
  \midrule\multirow{9}[6]{*}{lsnr} & \multirow{3}[2]{*}{$\rho=0$} & p=12 & 4.409 & 4.409 & 4.409 & 4.409 & 4.409 & 4.409 & 4.409 & 4.409 & 4.409 & 4.407 & 6.617 & 6.62 & 6.631 & 6.639 & 6.68 & 6.653 & 6.639 & 6.645 & 6.639 & 6.011 \\ 
   &  & p=500 & 4.409 & 4.409 & 4.41 & 4.41 & 4.409 & 4.41 & 4.409 & 4.41 & 4.41 & 4.407 & 6.861 & 6.886 & 7.005 & 7.047 & 7.025 & 6.961 & 6.938 & 6.999 & 6.957 & 6.011 \\ 
   &  & p=996 & 4.409 & 4.409 & 5.206 & 76907.098 & 4.409 & 4.41 & 4.409 & 107438.633 & 4.41 & 68395.361 & 6.861 & 6.886 & 96.618 & 275.033 & 7.025 & 6.961 & 6.938 & 379.786 & 6.957 & 103.022 \\ 
  \cmidrule{2-23} & \multirow{3}[2]{*}{$\rho=0.5$} & p=12 & 4.408 & 4.408 & 4.408 & 4.408 & 4.408 & 4.408 & 4.408 & 4.408 & 4.408 & 4.407 & 6.598 & 6.609 & 6.616 & 6.622 & 6.703 & 6.638 & 6.622 & 6.636 & 6.622 & 6.01 \\ 
   &  & p=500 & 4.409 & 4.409 & 4.409 & 4.409 & 4.409 & 4.409 & 4.409 & 4.409 & 4.409 & 4.407 & 6.859 & 6.883 & 6.974 & 7.031 & 7.054 & 6.925 & 6.902 & 6.967 & 6.902 & 6.01 \\ 
   &  & p=996 & 4.409 & 4.409 & 5.205 & 76907.098 & 4.409 & 4.409 & 4.409 & 107215.763 & 4.409 & 68395.361 & 6.859 & 6.883 & 96.635 & 274.995 & 7.054 & 6.925 & 6.902 & 378.725 & 6.902 & 103.019 \\ 
  \cmidrule{2-23} & \multirow{3}[2]{*}{$\rho=0.9$} & p=12 & 4.408 & 4.408 & 4.408 & 4.408 & 4.408 & 4.408 & 4.408 & 4.408 & 4.408 & 4.406 & 6.569 & 6.585 & 6.585 & 6.606 & 6.656 & 6.621 & 6.598 & 6.62 & 6.607 & 6.01 \\ 
   &  & p=500 & 4.409 & 4.409 & 4.409 & 4.409 & 4.409 & 4.409 & 4.409 & 4.409 & 4.409 & 4.406 & 6.823 & 6.863 & 6.931 & 7.058 & 7.054 & 6.963 & 6.914 & 7.003 & 6.923 & 6.01 \\ 
   &  & p=996 & 4.409 & 4.409 & 5.205 & 76907.098 & 4.409 & 4.409 & 4.409 & 107215.763 & 4.409 & 68395.36 & 6.823 & 6.863 & 96.769 & 275.168 & 7.054 & 6.963 & 6.914 & 378.769 & 6.923 & 103.02 \\ 
   \bottomrule 
\end{tabular}
}
\end{table}

%% latex table generated in R 4.0.2 by xtable 1.8-4 package
% Fri Jun 26 10:43:46 2020
\begin{table}[ht]
\centering
\caption{Fixed X, Sparse-Ex4, n=40} 
\scalebox{0.52}{
\begin{tabular}{|c|c|c|cc|cc|cc|ccc|c||cc|cc|cc|ccc|c|}
  \toprule 
 \multicolumn{1}{|c}{} & \multicolumn{1}{c}{} &       & RAICc & AICc  & RC$_p$ & C$_p$ & 10F CV & LOO CV & S$_p$ & $\widehat{\text{C}}_p$ & GCV   & BIC   & RAICc & AICc  & RC$_p$ & C$_p$ & 10F CV & LOO CV & S$_p$ & $\widehat{\text{C}}_p$ & GCV   & BIC  \\
 \cmidrule{4-23}\multicolumn{1}{|c}{} & \multicolumn{1}{c}{} &       & \multicolumn{10}{c||}{RMSE}                                                   & \multicolumn{10}{c|}{Relative Risk}       \\
 \midrule\multirow{9}[6]{*}{hsnr} & \multirow{3}[2]{*}{$\rho=0$} & p=12 & 1.082 & 1.116 & 1.135 & 1.195 & 1.189 & 1.182 & 1.187 & 1.235 & 1.191 & 1.104 & 0.025 & 0.027 & 0.028 & 0.032 & 0.032 & 0.031 & 0.031 & 0.034 & 0.032 & 0.027 \\ 
   &  & p=20 & 1.082 & 1.12 & 1.159 & 1.271 & 1.231 & 1.234 & 1.238 & 1.374 & 1.243 & 1.118 & 0.025 & 0.028 & 0.031 & 0.043 & 0.037 & 0.037 & 0.038 & 0.053 & 0.038 & 0.028 \\ 
   &  & p=36 & 1.082 & 1.12 & 1.301 & 1.584 & 1.248 & 1.279 & 1.277 & 1.891 & 1.308 & 1.538 & 0.025 & 0.028 & 0.056 & 0.25 & 0.041 & 0.058 & 0.058 & 0.398 & 0.074 & 0.296 \\ 
  \cmidrule{2-23} & \multirow{3}[2]{*}{$\rho=0.5$} & p=12 & 0.765 & 0.789 & 0.802 & 0.845 & 0.84 & 0.836 & 0.839 & 0.873 & 0.842 & 0.781 & 0.025 & 0.027 & 0.028 & 0.032 & 0.032 & 0.031 & 0.031 & 0.034 & 0.032 & 0.027 \\ 
   &  & p=20 & 0.765 & 0.792 & 0.819 & 0.899 & 0.871 & 0.872 & 0.875 & 0.972 & 0.879 & 0.79 & 0.025 & 0.028 & 0.031 & 0.043 & 0.037 & 0.037 & 0.038 & 0.053 & 0.038 & 0.028 \\ 
   &  & p=36 & 0.765 & 0.792 & 0.92 & 1.12 & 0.882 & 0.904 & 0.903 & 1.337 & 0.925 & 1.087 & 0.025 & 0.028 & 0.056 & 0.25 & 0.041 & 0.058 & 0.058 & 0.398 & 0.074 & 0.296 \\ 
  \cmidrule{2-23} & \multirow{3}[2]{*}{$\rho=0.9$} & p=12 & 0.342 & 0.353 & 0.359 & 0.378 & 0.376 & 0.374 & 0.375 & 0.391 & 0.377 & 0.349 & 0.025 & 0.027 & 0.028 & 0.032 & 0.032 & 0.031 & 0.031 & 0.034 & 0.032 & 0.027 \\ 
   &  & p=20 & 0.342 & 0.354 & 0.366 & 0.402 & 0.389 & 0.39 & 0.391 & 0.435 & 0.393 & 0.353 & 0.025 & 0.028 & 0.031 & 0.043 & 0.037 & 0.037 & 0.038 & 0.053 & 0.038 & 0.028 \\ 
   &  & p=36 & 0.342 & 0.354 & 0.411 & 0.501 & 0.395 & 0.404 & 0.404 & 0.598 & 0.414 & 0.486 & 0.025 & 0.028 & 0.056 & 0.25 & 0.041 & 0.058 & 0.058 & 0.398 & 0.074 & 0.296 \\ 
  \midrule\multirow{9}[6]{*}{msnr} & \multirow{3}[2]{*}{$\rho=0$} & p=12 & 3.276 & 3.325 & 3.371 & 3.505 & 3.526 & 3.477 & 3.493 & 3.618 & 3.504 & 3.569 & 0.241 & 0.248 & 0.255 & 0.275 & 0.282 & 0.271 & 0.274 & 0.296 & 0.276 & 0.292 \\ 
   &  & p=20 & 3.276 & 3.336 & 3.43 & 3.737 & 3.65 & 3.622 & 3.639 & 4.022 & 3.656 & 3.608 & 0.241 & 0.252 & 0.273 & 0.37 & 0.326 & 0.323 & 0.328 & 0.457 & 0.332 & 0.305 \\ 
   &  & p=36 & 3.276 & 3.336 & 4.019 & 4.755 & 3.698 & 3.755 & 3.753 & 5.518 & 3.844 & 4.762 & 0.241 & 0.252 & 0.521 & 2.15 & 0.362 & 0.5 & 0.5 & 3.376 & 0.636 & 2.551 \\ 
  \cmidrule{2-23} & \multirow{3}[2]{*}{$\rho=0.5$} & p=12 & 2.299 & 2.335 & 2.358 & 2.469 & 2.476 & 2.447 & 2.457 & 2.551 & 2.464 & 2.5 & 0.236 & 0.243 & 0.247 & 0.273 & 0.278 & 0.267 & 0.27 & 0.294 & 0.272 & 0.284 \\ 
   &  & p=20 & 2.299 & 2.343 & 2.405 & 2.628 & 2.563 & 2.549 & 2.56 & 2.836 & 2.572 & 2.527 & 0.236 & 0.246 & 0.266 & 0.366 & 0.321 & 0.32 & 0.324 & 0.455 & 0.328 & 0.297 \\ 
   &  & p=36 & 2.299 & 2.343 & 2.813 & 3.35 & 2.598 & 2.644 & 2.64 & 3.9 & 2.705 & 3.352 & 0.236 & 0.246 & 0.511 & 2.145 & 0.357 & 0.497 & 0.496 & 3.383 & 0.632 & 2.556 \\ 
  \cmidrule{2-23} & \multirow{3}[2]{*}{$\rho=0.9$} & p=12 & 1.024 & 1.042 & 1.053 & 1.104 & 1.106 & 1.094 & 1.098 & 1.141 & 1.102 & 1.115 & 0.233 & 0.241 & 0.246 & 0.272 & 0.277 & 0.267 & 0.27 & 0.294 & 0.272 & 0.281 \\ 
   &  & p=20 & 1.024 & 1.046 & 1.074 & 1.175 & 1.145 & 1.14 & 1.145 & 1.268 & 1.15 & 1.127 & 0.233 & 0.244 & 0.265 & 0.365 & 0.321 & 0.32 & 0.324 & 0.455 & 0.328 & 0.294 \\ 
   &  & p=36 & 1.024 & 1.046 & 1.254 & 1.497 & 1.161 & 1.182 & 1.18 & 1.744 & 1.209 & 1.497 & 0.233 & 0.244 & 0.508 & 2.144 & 0.357 & 0.496 & 0.496 & 3.383 & 0.631 & 2.554 \\ 
  \midrule\multirow{9}[6]{*}{lsnr} & \multirow{3}[2]{*}{$\rho=0$} & p=12 & 8.458 & 8.428 & 8.417 & 8.547 & 8.505 & 8.518 & 8.513 & 8.621 & 8.498 & 8.506 & 1.306 & 1.324 & 1.38 & 1.469 & 1.42 & 1.424 & 1.431 & 1.516 & 1.435 & 1.2 \\ 
   &  & p=20 & 8.458 & 8.441 & 8.505 & 8.962 & 8.675 & 8.734 & 8.711 & 9.283 & 8.713 & 8.527 & 1.306 & 1.333 & 1.441 & 1.867 & 1.57 & 1.608 & 1.598 & 2.154 & 1.615 & 1.218 \\ 
   &  & p=36 & 8.458 & 8.441 & 9.267 & 10.921 & 8.73 & 8.915 & 8.897 & 12.368 & 8.993 & 10.351 & 1.306 & 1.333 & 2.466 & 10.573 & 1.682 & 2.216 & 2.337 & 16.202 & 2.848 & 9.784 \\ 
  \cmidrule{2-23} & \multirow{3}[2]{*}{$\rho=0.5$} & p=12 & 6.037 & 5.993 & 5.978 & 6.082 & 6.047 & 6.055 & 6.059 & 6.136 & 6.053 & 6.023 & 1.317 & 1.332 & 1.391 & 1.492 & 1.435 & 1.436 & 1.449 & 1.535 & 1.456 & 1.199 \\ 
   &  & p=20 & 6.037 & 5.999 & 6.041 & 6.372 & 6.17 & 6.2 & 6.195 & 6.598 & 6.205 & 6.041 & 1.317 & 1.339 & 1.454 & 1.88 & 1.588 & 1.614 & 1.616 & 2.173 & 1.64 & 1.22 \\ 
   &  & p=36 & 6.037 & 5.999 & 6.571 & 7.738 & 6.211 & 6.333 & 6.324 & 8.77 & 6.41 & 7.335 & 1.317 & 1.339 & 2.476 & 10.58 & 1.705 & 2.23 & 2.354 & 16.229 & 2.888 & 9.802 \\ 
  \cmidrule{2-23} & \multirow{3}[2]{*}{$\rho=0.9$} & p=12 & 2.703 & 2.68 & 2.677 & 2.716 & 2.709 & 2.714 & 2.715 & 2.744 & 2.711 & 2.687 & 1.306 & 1.322 & 1.387 & 1.485 & 1.435 & 1.439 & 1.449 & 1.535 & 1.457 & 1.177 \\ 
   &  & p=20 & 2.703 & 2.683 & 2.708 & 2.853 & 2.761 & 2.779 & 2.777 & 2.952 & 2.779 & 2.695 & 1.306 & 1.329 & 1.451 & 1.883 & 1.581 & 1.619 & 1.619 & 2.174 & 1.641 & 1.198 \\ 
   &  & p=36 & 2.703 & 2.683 & 2.939 & 3.458 & 2.781 & 2.838 & 2.835 & 3.929 & 2.869 & 3.279 & 1.306 & 1.329 & 2.473 & 10.578 & 1.709 & 2.233 & 2.357 & 16.292 & 2.887 & 9.833 \\ 
   \midrule 
 \multicolumn{1}{|c}{} & \multicolumn{1}{c}{} &       & \multicolumn{10}{c||}{Relative Test Error}                                    & \multicolumn{10}{c|}{Proportion of Variance Explained} \\
\midrule\multirow{9}[6]{*}{hsnr} & \multirow{3}[2]{*}{$\rho=0$} & p=12 & 1.229 & 1.247 & 1.256 & 1.288 & 1.287 & 1.281 & 1.284 & 1.311 & 1.286 & 1.241 & 0.878 & 0.876 & 0.875 & 0.872 & 0.872 & 0.873 & 0.872 & 0.87 & 0.872 & 0.877 \\ 
   &  & p=20 & 1.229 & 1.251 & 1.278 & 1.387 & 1.333 & 1.34 & 1.342 & 1.484 & 1.346 & 1.255 & 0.878 & 0.876 & 0.873 & 0.862 & 0.867 & 0.867 & 0.867 & 0.853 & 0.866 & 0.875 \\ 
   &  & p=36 & 1.229 & 1.251 & 1.507 & 3.263 & 1.372 & 1.527 & 1.526 & 4.606 & 1.67 & 3.682 & 0.878 & 0.876 & 0.85 & 0.676 & 0.864 & 0.848 & 0.848 & 0.542 & 0.834 & 0.634 \\ 
  \cmidrule{2-23} & \multirow{3}[2]{*}{$\rho=0.5$} & p=12 & 1.229 & 1.247 & 1.256 & 1.288 & 1.287 & 1.281 & 1.284 & 1.311 & 1.286 & 1.241 & 0.878 & 0.876 & 0.875 & 0.872 & 0.872 & 0.873 & 0.872 & 0.87 & 0.872 & 0.877 \\ 
   &  & p=20 & 1.229 & 1.251 & 1.278 & 1.387 & 1.333 & 1.34 & 1.342 & 1.484 & 1.346 & 1.255 & 0.878 & 0.876 & 0.873 & 0.862 & 0.867 & 0.867 & 0.867 & 0.853 & 0.866 & 0.875 \\ 
   &  & p=36 & 1.229 & 1.251 & 1.507 & 3.263 & 1.372 & 1.527 & 1.526 & 4.606 & 1.67 & 3.682 & 0.878 & 0.876 & 0.85 & 0.676 & 0.864 & 0.848 & 0.848 & 0.542 & 0.834 & 0.634 \\ 
  \cmidrule{2-23} & \multirow{3}[2]{*}{$\rho=0.9$} & p=12 & 1.229 & 1.247 & 1.256 & 1.288 & 1.287 & 1.281 & 1.284 & 1.311 & 1.286 & 1.241 & 0.878 & 0.876 & 0.875 & 0.872 & 0.872 & 0.873 & 0.872 & 0.87 & 0.872 & 0.877 \\ 
   &  & p=20 & 1.229 & 1.251 & 1.278 & 1.387 & 1.333 & 1.34 & 1.342 & 1.484 & 1.346 & 1.255 & 0.878 & 0.876 & 0.873 & 0.862 & 0.867 & 0.867 & 0.867 & 0.853 & 0.866 & 0.875 \\ 
   &  & p=36 & 1.229 & 1.251 & 1.507 & 3.263 & 1.372 & 1.527 & 1.526 & 4.606 & 1.67 & 3.682 & 0.878 & 0.876 & 0.85 & 0.676 & 0.864 & 0.848 & 0.848 & 0.542 & 0.834 & 0.634 \\ 
  \midrule\multirow{9}[6]{*}{msnr} & \multirow{3}[2]{*}{$\rho=0$} & p=12 & 1.256 & 1.265 & 1.272 & 1.293 & 1.301 & 1.288 & 1.292 & 1.316 & 1.294 & 1.311 & 0.392 & 0.388 & 0.385 & 0.374 & 0.371 & 0.376 & 0.374 & 0.363 & 0.374 & 0.365 \\ 
   &  & p=20 & 1.256 & 1.269 & 1.291 & 1.395 & 1.347 & 1.345 & 1.35 & 1.488 & 1.354 & 1.325 & 0.392 & 0.386 & 0.375 & 0.325 & 0.348 & 0.349 & 0.347 & 0.28 & 0.345 & 0.359 \\ 
   &  & p=36 & 1.256 & 1.269 & 1.555 & 3.292 & 1.385 & 1.533 & 1.534 & 4.6 & 1.678 & 3.72 & 0.392 & 0.386 & 0.247 & -0.593 & 0.329 & 0.258 & 0.258 & -1.226 & 0.188 & -0.8 \\ 
  \cmidrule{2-23} & \multirow{3}[2]{*}{$\rho=0.5$} & p=12 & 1.252 & 1.259 & 1.263 & 1.291 & 1.296 & 1.285 & 1.288 & 1.313 & 1.29 & 1.303 & 0.394 & 0.391 & 0.389 & 0.375 & 0.373 & 0.378 & 0.377 & 0.364 & 0.376 & 0.369 \\ 
   &  & p=20 & 1.252 & 1.263 & 1.284 & 1.39 & 1.343 & 1.341 & 1.346 & 1.485 & 1.35 & 1.317 & 0.394 & 0.389 & 0.379 & 0.327 & 0.35 & 0.351 & 0.349 & 0.281 & 0.347 & 0.363 \\ 
   &  & p=36 & 1.252 & 1.263 & 1.545 & 3.287 & 1.381 & 1.53 & 1.529 & 4.607 & 1.673 & 3.725 & 0.394 & 0.389 & 0.252 & -0.591 & 0.332 & 0.26 & 0.26 & -1.23 & 0.19 & -0.803 \\ 
  \cmidrule{2-23} & \multirow{3}[2]{*}{$\rho=0.9$} & p=12 & 1.248 & 1.257 & 1.262 & 1.29 & 1.295 & 1.285 & 1.288 & 1.314 & 1.29 & 1.299 & 0.396 & 0.392 & 0.389 & 0.375 & 0.373 & 0.378 & 0.377 & 0.364 & 0.376 & 0.371 \\ 
   &  & p=20 & 1.248 & 1.26 & 1.283 & 1.39 & 1.342 & 1.341 & 1.346 & 1.485 & 1.35 & 1.314 & 0.396 & 0.39 & 0.379 & 0.327 & 0.35 & 0.351 & 0.349 & 0.281 & 0.347 & 0.364 \\ 
   &  & p=36 & 1.248 & 1.26 & 1.542 & 3.286 & 1.38 & 1.529 & 1.529 & 4.607 & 1.673 & 3.723 & 0.396 & 0.39 & 0.254 & -0.59 & 0.332 & 0.26 & 0.26 & -1.23 & 0.19 & -0.802 \\ 
  \midrule\multirow{9}[6]{*}{lsnr} & \multirow{3}[2]{*}{$\rho=0$} & p=12 & 1.279 & 1.282 & 1.294 & 1.313 & 1.303 & 1.304 & 1.305 & 1.323 & 1.306 & 1.256 & -0.054 & -0.057 & -0.067 & -0.082 & -0.074 & -0.075 & -0.076 & -0.091 & -0.076 & -0.035 \\ 
   &  & p=20 & 1.279 & 1.284 & 1.307 & 1.398 & 1.335 & 1.343 & 1.341 & 1.459 & 1.344 & 1.26 & -0.054 & -0.059 & -0.078 & -0.152 & -0.1 & -0.107 & -0.105 & -0.203 & -0.108 & -0.038 \\ 
   &  & p=36 & 1.279 & 1.284 & 1.526 & 3.255 & 1.359 & 1.473 & 1.498 & 4.455 & 1.607 & 3.086 & -0.054 & -0.059 & -0.258 & -1.683 & -0.12 & -0.214 & -0.235 & -2.672 & -0.325 & -1.544 \\ 
  \cmidrule{2-23} & \multirow{3}[2]{*}{$\rho=0.5$} & p=12 & 1.281 & 1.284 & 1.297 & 1.318 & 1.306 & 1.306 & 1.309 & 1.327 & 1.311 & 1.256 & -0.056 & -0.058 & -0.069 & -0.086 & -0.076 & -0.077 & -0.079 & -0.094 & -0.08 & -0.035 \\ 
   &  & p=20 & 1.281 & 1.286 & 1.31 & 1.401 & 1.339 & 1.344 & 1.345 & 1.463 & 1.35 & 1.26 & -0.056 & -0.06 & -0.08 & -0.155 & -0.103 & -0.108 & -0.108 & -0.206 & -0.112 & -0.039 \\ 
   &  & p=36 & 1.281 & 1.286 & 1.528 & 3.256 & 1.364 & 1.476 & 1.502 & 4.461 & 1.616 & 3.09 & -0.056 & -0.06 & -0.259 & -1.684 & -0.124 & -0.216 & -0.238 & -2.677 & -0.332 & -1.547 \\ 
  \cmidrule{2-23} & \multirow{3}[2]{*}{$\rho=0.9$} & p=12 & 1.279 & 1.282 & 1.296 & 1.317 & 1.306 & 1.307 & 1.309 & 1.327 & 1.311 & 1.251 & -0.054 & -0.057 & -0.068 & -0.085 & -0.076 & -0.077 & -0.079 & -0.094 & -0.08 & -0.031 \\ 
   &  & p=20 & 1.279 & 1.283 & 1.309 & 1.401 & 1.337 & 1.345 & 1.345 & 1.464 & 1.35 & 1.256 & -0.054 & -0.058 & -0.079 & -0.155 & -0.102 & -0.109 & -0.109 & -0.206 & -0.113 & -0.035 \\ 
   &  & p=36 & 1.279 & 1.283 & 1.527 & 3.256 & 1.365 & 1.476 & 1.503 & 4.474 & 1.616 & 3.097 & -0.054 & -0.058 & -0.259 & -1.683 & -0.125 & -0.217 & -0.238 & -2.688 & -0.332 & -1.553 \\ 
   \midrule 
 \multicolumn{1}{|c}{} & \multicolumn{1}{c}{} &       & \multicolumn{10}{c||}{Kullback-Leibler Discrepancy}                                    & \multicolumn{10}{c|}{Number of Variables} \\
\midrule\multirow{9}[6]{*}{hsnr} & \multirow{3}[2]{*}{$\rho=0$} & p=12 & 3.332 & 3.366 & 3.38 & 3.43 & 3.421 & 3.417 & 3.424 & 3.464 & 3.426 & 3.361 & 6.148 & 6.319 & 6.418 & 6.815 & 6.868 & 6.758 & 6.758 & 7.138 & 6.778 & 6.274 \\ 
   &  & p=20 & 3.332 & 3.374 & 3.432 & 3.647 & 3.534 & 3.566 & 3.564 & 3.834 & 3.573 & 3.406 & 6.148 & 6.343 & 6.624 & 7.65 & 7.301 & 7.289 & 7.241 & 8.775 & 7.29 & 6.391 \\ 
   &  & p=36 & 3.332 & 3.374 & 5.427 & 21.306 & 3.662 & 8.818 & 9.757 & 27.438 & 11.305 & 26.308 & 6.148 & 6.343 & 8.522 & 13.668 & 7.526 & 8.022 & 7.939 & 18.913 & 8.414 & 14.278 \\ 
  \cmidrule{2-23} & \multirow{3}[2]{*}{$\rho=0.5$} & p=12 & 2.639 & 2.672 & 2.687 & 2.737 & 2.728 & 2.724 & 2.731 & 2.771 & 2.733 & 2.667 & 6.148 & 6.319 & 6.418 & 6.815 & 6.868 & 6.758 & 6.758 & 7.138 & 6.778 & 6.274 \\ 
   &  & p=20 & 2.639 & 2.68 & 2.739 & 2.954 & 2.841 & 2.873 & 2.871 & 3.141 & 2.88 & 2.713 & 6.148 & 6.343 & 6.624 & 7.65 & 7.301 & 7.289 & 7.241 & 8.775 & 7.29 & 6.391 \\ 
   &  & p=36 & 2.639 & 2.68 & 4.734 & 20.613 & 2.969 & 8.125 & 9.063 & 26.745 & 10.612 & 25.615 & 6.148 & 6.343 & 8.522 & 13.668 & 7.526 & 8.022 & 7.939 & 18.913 & 8.414 & 14.278 \\ 
  \cmidrule{2-23} & \multirow{3}[2]{*}{$\rho=0.9$} & p=12 & 1.029 & 1.063 & 1.077 & 1.128 & 1.119 & 1.114 & 1.122 & 1.161 & 1.123 & 1.058 & 6.148 & 6.319 & 6.418 & 6.815 & 6.868 & 6.758 & 6.758 & 7.138 & 6.778 & 6.274 \\ 
   &  & p=20 & 1.029 & 1.071 & 1.129 & 1.344 & 1.231 & 1.264 & 1.262 & 1.532 & 1.27 & 1.103 & 6.148 & 6.343 & 6.624 & 7.65 & 7.301 & 7.289 & 7.241 & 8.775 & 7.29 & 6.391 \\ 
   &  & p=36 & 1.029 & 1.071 & 3.124 & 19.003 & 1.359 & 6.516 & 7.454 & 25.136 & 9.002 & 24.005 & 6.148 & 6.343 & 8.522 & 13.668 & 7.526 & 8.022 & 7.939 & 18.913 & 8.414 & 14.278 \\ 
  \midrule\multirow{9}[6]{*}{msnr} & \multirow{3}[2]{*}{$\rho=0$} & p=12 & 5.487 & 5.515 & 5.528 & 5.574 & 5.569 & 5.561 & 5.569 & 5.606 & 5.571 & 5.541 & 6.027 & 6.242 & 6.35 & 6.79 & 6.805 & 6.719 & 6.72 & 7.118 & 6.74 & 5.886 \\ 
   &  & p=20 & 5.487 & 5.523 & 5.579 & 5.789 & 5.681 & 5.703 & 5.708 & 5.977 & 5.717 & 5.586 & 6.027 & 6.266 & 6.569 & 7.602 & 7.238 & 7.226 & 7.205 & 8.76 & 7.254 & 6.003 \\ 
   &  & p=36 & 5.487 & 5.523 & 7.592 & 23.462 & 5.809 & 10.958 & 11.901 & 29.559 & 13.449 & 28.422 & 6.027 & 6.266 & 8.301 & 13.526 & 7.463 & 7.972 & 7.904 & 18.874 & 8.379 & 13.896 \\ 
  \cmidrule{2-23} & \multirow{3}[2]{*}{$\rho=0.5$} & p=12 & 4.79 & 4.819 & 4.831 & 4.879 & 4.873 & 4.866 & 4.873 & 4.912 & 4.875 & 4.843 & 6.052 & 6.265 & 6.387 & 6.804 & 6.832 & 6.74 & 6.744 & 7.127 & 6.764 & 5.917 \\ 
   &  & p=20 & 4.79 & 4.827 & 4.882 & 5.095 & 4.985 & 5.008 & 5.013 & 5.282 & 5.021 & 4.889 & 6.052 & 6.289 & 6.6 & 7.639 & 7.265 & 7.247 & 7.229 & 8.769 & 7.278 & 6.034 \\ 
   &  & p=36 & 4.79 & 4.827 & 6.894 & 22.766 & 5.113 & 10.262 & 11.205 & 28.886 & 12.753 & 27.76 & 6.052 & 6.289 & 8.33 & 13.539 & 7.49 & 7.993 & 7.928 & 18.909 & 8.403 & 13.95 \\ 
  \cmidrule{2-23} & \multirow{3}[2]{*}{$\rho=0.9$} & p=12 & 3.179 & 3.208 & 3.221 & 3.269 & 3.263 & 3.256 & 3.264 & 3.302 & 3.265 & 3.233 & 6.062 & 6.274 & 6.394 & 6.805 & 6.837 & 6.741 & 6.744 & 7.126 & 6.764 & 5.921 \\ 
   &  & p=20 & 3.179 & 3.216 & 3.272 & 3.486 & 3.375 & 3.398 & 3.403 & 3.673 & 3.412 & 3.278 & 6.062 & 6.298 & 6.604 & 7.64 & 7.27 & 7.248 & 7.229 & 8.769 & 7.278 & 6.038 \\ 
   &  & p=36 & 3.179 & 3.216 & 5.284 & 21.157 & 3.503 & 8.653 & 9.595 & 27.276 & 11.144 & 26.15 & 6.062 & 6.298 & 8.339 & 13.54 & 7.495 & 7.994 & 7.929 & 18.909 & 8.404 & 13.952 \\ 
  \midrule\multirow{9}[6]{*}{lsnr} & \multirow{3}[2]{*}{$\rho=0$} & p=12 & 7.097 & 7.112 & 7.138 & 7.18 & 7.157 & 7.162 & 7.169 & 7.201 & 7.17 & 7.079 & 2.11 & 2.575 & 3.361 & 4.103 & 3.665 & 3.682 & 3.747 & 4.456 & 3.842 & 0.971 \\ 
   &  & p=20 & 7.097 & 7.12 & 7.181 & 7.381 & 7.249 & 7.28 & 7.278 & 7.525 & 7.284 & 7.101 & 2.11 & 2.602 & 3.491 & 4.884 & 4.048 & 4.142 & 4.15 & 6.037 & 4.278 & 1.031 \\ 
   &  & p=36 & 7.097 & 7.12 & 9.173 & 25 & 7.341 & 12.025 & 13.138 & 30.888 & 14.41 & 27.428 & 2.11 & 2.602 & 5.792 & 11.159 & 4.223 & 4.759 & 4.786 & 16.74 & 5.245 & 7.967 \\ 
  \cmidrule{2-23} & \multirow{3}[2]{*}{$\rho=0.5$} & p=12 & 6.407 & 6.423 & 6.451 & 6.491 & 6.469 & 6.471 & 6.48 & 6.513 & 6.481 & 6.387 & 2.057 & 2.582 & 3.412 & 4.133 & 3.7 & 3.676 & 3.755 & 4.499 & 3.855 & 0.966 \\ 
   &  & p=20 & 6.407 & 6.43 & 6.49 & 6.694 & 6.561 & 6.585 & 6.59 & 6.836 & 6.598 & 6.411 & 2.057 & 2.603 & 3.489 & 4.95 & 4.099 & 4.112 & 4.175 & 6.076 & 4.316 & 1.039 \\ 
   &  & p=36 & 6.407 & 6.43 & 8.483 & 24.311 & 6.656 & 11.344 & 12.448 & 30.204 & 13.746 & 26.784 & 2.057 & 2.603 & 5.828 & 11.191 & 4.287 & 4.756 & 4.799 & 16.796 & 5.315 & 7.978 \\ 
  \cmidrule{2-23} & \multirow{3}[2]{*}{$\rho=0.9$} & p=12 & 4.799 & 4.814 & 4.843 & 4.883 & 4.862 & 4.864 & 4.873 & 4.905 & 4.875 & 4.776 & 2.054 & 2.592 & 3.454 & 4.205 & 3.721 & 3.722 & 3.79 & 4.55 & 3.912 & 0.932 \\ 
   &  & p=20 & 4.799 & 4.821 & 4.885 & 5.086 & 4.95 & 4.977 & 4.984 & 5.227 & 4.991 & 4.8 & 2.054 & 2.613 & 3.512 & 4.984 & 4.103 & 4.147 & 4.215 & 6.109 & 4.37 & 1.005 \\ 
   &  & p=36 & 4.799 & 4.821 & 6.873 & 22.704 & 5.047 & 9.736 & 10.842 & 28.614 & 12.139 & 25.21 & 2.054 & 2.613 & 5.824 & 11.236 & 4.3 & 4.784 & 4.839 & 16.87 & 5.37 & 7.988 \\ 
   \bottomrule 
\end{tabular}
}
\end{table}

%% latex table generated in R 4.0.2 by xtable 1.8-4 package
% Fri Jun 26 10:43:51 2020
\begin{table}[ht]
\centering
\caption{Random X, Sparse-Ex4, n=200} 
\scalebox{0.52}{
\begin{tabular}{|c|c|c|cc|cc|cc|ccc|c||cc|cc|cc|ccc|c|}
  \toprule 
 \multicolumn{1}{|c}{} & \multicolumn{1}{c}{} &       & RAICc & AICc  & RC$_p$ & C$_p$ & 10F CV & LOO CV & S$_p$ & $\widehat{\text{C}}_p$ & GCV   & BIC   & RAICc & AICc  & RC$_p$ & C$_p$ & 10F CV & LOO CV & S$_p$ & $\widehat{\text{C}}_p$ & GCV   & BIC  \\
 \cmidrule{4-23}\multicolumn{1}{|c}{} & \multicolumn{1}{c}{} &       & \multicolumn{10}{c||}{RMSE}                                                   & \multicolumn{10}{c|}{Relative Risk}       \\
 \midrule\multirow{9}[6]{*}{hsnr} & \multirow{3}[2]{*}{$\rho=0$} & p=12 & 0.534 & 0.539 & 0.541 & 0.547 & 0.548 & 0.544 & 0.546 & 0.549 & 0.546 & 0.494 & 0.005 & 0.005 & 0.005 & 0.005 & 0.005 & 0.005 & 0.005 & 0.005 & 0.005 & 0.004 \\ 
   &  & p=100 & 0.539 & 0.545 & 0.548 & 0.565 & 0.565 & 0.557 & 0.557 & 0.578 & 0.558 & 0.494 & 0.005 & 0.005 & 0.005 & 0.006 & 0.005 & 0.005 & 0.005 & 0.006 & 0.005 & 0.004 \\ 
   &  & p=196 & 0.539 & 0.545 & 0.852 & 1.181 & 0.565 & 0.557 & 0.557 & 1.449 & 0.558 & 0.791 & 0.005 & 0.005 & 0.029 & 1.343 & 0.005 & 0.005 & 0.005 & 2.346 & 0.005 & 0.894 \\ 
  \cmidrule{2-23} & \multirow{3}[2]{*}{$\rho=0.5$} & p=12 & 0.377 & 0.381 & 0.383 & 0.387 & 0.388 & 0.385 & 0.386 & 0.388 & 0.386 & 0.349 & 0.005 & 0.005 & 0.005 & 0.005 & 0.005 & 0.005 & 0.005 & 0.005 & 0.005 & 0.004 \\ 
   &  & p=100 & 0.381 & 0.385 & 0.388 & 0.4 & 0.399 & 0.394 & 0.394 & 0.408 & 0.395 & 0.349 & 0.005 & 0.005 & 0.005 & 0.006 & 0.005 & 0.005 & 0.005 & 0.006 & 0.005 & 0.004 \\ 
   &  & p=196 & 0.381 & 0.385 & 0.602 & 0.835 & 0.399 & 0.394 & 0.394 & 1.024 & 0.395 & 0.559 & 0.005 & 0.005 & 0.029 & 1.343 & 0.005 & 0.005 & 0.005 & 2.346 & 0.005 & 0.894 \\ 
  \cmidrule{2-23} & \multirow{3}[2]{*}{$\rho=0.9$} & p=12 & 0.169 & 0.17 & 0.171 & 0.173 & 0.173 & 0.172 & 0.173 & 0.174 & 0.173 & 0.156 & 0.005 & 0.005 & 0.005 & 0.005 & 0.005 & 0.005 & 0.005 & 0.005 & 0.005 & 0.004 \\ 
   &  & p=100 & 0.17 & 0.172 & 0.173 & 0.179 & 0.179 & 0.176 & 0.176 & 0.183 & 0.176 & 0.156 & 0.005 & 0.005 & 0.005 & 0.006 & 0.005 & 0.005 & 0.005 & 0.006 & 0.005 & 0.004 \\ 
   &  & p=196 & 0.17 & 0.172 & 0.269 & 0.373 & 0.179 & 0.176 & 0.176 & 0.458 & 0.176 & 0.25 & 0.005 & 0.005 & 0.029 & 1.343 & 0.005 & 0.005 & 0.005 & 2.346 & 0.005 & 0.894 \\ 
  \midrule\multirow{9}[6]{*}{msnr} & \multirow{3}[2]{*}{$\rho=0$} & p=12 & 1.556 & 1.571 & 1.578 & 1.595 & 1.598 & 1.587 & 1.593 & 1.601 & 1.593 & 1.44 & 0.04 & 0.041 & 0.041 & 0.042 & 0.042 & 0.042 & 0.042 & 0.043 & 0.042 & 0.034 \\ 
   &  & p=100 & 1.571 & 1.589 & 1.598 & 1.648 & 1.647 & 1.623 & 1.624 & 1.684 & 1.627 & 1.44 & 0.041 & 0.043 & 0.043 & 0.048 & 0.046 & 0.045 & 0.045 & 0.051 & 0.045 & 0.034 \\ 
   &  & p=196 & 1.571 & 1.589 & 2.483 & 3.442 & 1.647 & 1.623 & 1.624 & 4.224 & 1.627 & 2.305 & 0.041 & 0.043 & 0.245 & 11.416 & 0.046 & 0.045 & 0.045 & 19.938 & 0.045 & 7.599 \\ 
  \cmidrule{2-23} & \multirow{3}[2]{*}{$\rho=0.5$} & p=12 & 1.1 & 1.111 & 1.116 & 1.128 & 1.13 & 1.122 & 1.127 & 1.132 & 1.127 & 1.018 & 0.04 & 0.041 & 0.041 & 0.042 & 0.042 & 0.042 & 0.042 & 0.043 & 0.042 & 0.034 \\ 
   &  & p=100 & 1.111 & 1.124 & 1.13 & 1.165 & 1.165 & 1.148 & 1.148 & 1.191 & 1.15 & 1.018 & 0.041 & 0.043 & 0.043 & 0.048 & 0.046 & 0.045 & 0.045 & 0.051 & 0.045 & 0.034 \\ 
   &  & p=196 & 1.111 & 1.124 & 1.755 & 2.434 & 1.165 & 1.148 & 1.148 & 2.987 & 1.15 & 1.63 & 0.041 & 0.043 & 0.245 & 11.416 & 0.046 & 0.045 & 0.045 & 19.938 & 0.045 & 7.599 \\ 
  \cmidrule{2-23} & \multirow{3}[2]{*}{$\rho=0.9$} & p=12 & 0.492 & 0.497 & 0.499 & 0.504 & 0.505 & 0.502 & 0.504 & 0.506 & 0.504 & 0.455 & 0.04 & 0.041 & 0.041 & 0.042 & 0.042 & 0.042 & 0.042 & 0.043 & 0.042 & 0.034 \\ 
   &  & p=100 & 0.497 & 0.503 & 0.505 & 0.521 & 0.521 & 0.513 & 0.514 & 0.533 & 0.514 & 0.455 & 0.041 & 0.043 & 0.043 & 0.048 & 0.046 & 0.045 & 0.045 & 0.051 & 0.045 & 0.034 \\ 
   &  & p=196 & 0.497 & 0.503 & 0.785 & 1.088 & 0.521 & 0.513 & 0.514 & 1.336 & 0.514 & 0.729 & 0.041 & 0.043 & 0.245 & 11.416 & 0.046 & 0.045 & 0.045 & 19.938 & 0.045 & 7.599 \\ 
  \midrule\multirow{9}[6]{*}{lsnr} & \multirow{3}[2]{*}{$\rho=0$} & p=12 & 3.496 & 3.527 & 3.542 & 3.579 & 3.596 & 3.562 & 3.573 & 3.591 & 3.573 & 4.217 & 0.203 & 0.206 & 0.208 & 0.213 & 0.214 & 0.211 & 0.212 & 0.214 & 0.212 & 0.33 \\ 
   &  & p=100 & 3.531 & 3.568 & 3.584 & 3.697 & 3.705 & 3.642 & 3.642 & 3.776 & 3.648 & 4.217 & 0.21 & 0.215 & 0.218 & 0.242 & 0.236 & 0.228 & 0.227 & 0.257 & 0.228 & 0.33 \\ 
   &  & p=196 & 3.531 & 3.568 & 5.64 & 7.781 & 3.705 & 3.642 & 3.642 & 9.453 & 3.648 & 6.039 & 0.21 & 0.215 & 1.242 & 57.096 & 0.236 & 0.228 & 0.227 & 99.689 & 0.228 & 38.138 \\ 
  \cmidrule{2-23} & \multirow{3}[2]{*}{$\rho=0.5$} & p=12 & 2.461 & 2.485 & 2.496 & 2.523 & 2.531 & 2.511 & 2.52 & 2.533 & 2.52 & 2.926 & 0.2 & 0.204 & 0.206 & 0.211 & 0.212 & 0.209 & 0.211 & 0.213 & 0.211 & 0.319 \\ 
   &  & p=100 & 2.485 & 2.514 & 2.527 & 2.607 & 2.608 & 2.567 & 2.569 & 2.664 & 2.573 & 2.926 & 0.207 & 0.213 & 0.216 & 0.24 & 0.233 & 0.226 & 0.226 & 0.255 & 0.227 & 0.319 \\ 
   &  & p=196 & 2.485 & 2.514 & 3.97 & 5.483 & 2.608 & 2.567 & 2.569 & 6.68 & 2.573 & 4.221 & 0.207 & 0.213 & 1.238 & 57.092 & 0.233 & 0.226 & 0.226 & 99.688 & 0.227 & 38.128 \\ 
  \cmidrule{2-23} & \multirow{3}[2]{*}{$\rho=0.9$} & p=12 & 1.1 & 1.111 & 1.116 & 1.128 & 1.131 & 1.122 & 1.127 & 1.132 & 1.127 & 1.305 & 0.2 & 0.204 & 0.206 & 0.211 & 0.211 & 0.208 & 0.21 & 0.213 & 0.21 & 0.317 \\ 
   &  & p=100 & 1.111 & 1.124 & 1.13 & 1.165 & 1.165 & 1.148 & 1.148 & 1.191 & 1.15 & 1.305 & 0.207 & 0.213 & 0.216 & 0.24 & 0.232 & 0.226 & 0.226 & 0.255 & 0.227 & 0.317 \\ 
   &  & p=196 & 1.111 & 1.124 & 1.775 & 2.452 & 1.165 & 1.148 & 1.148 & 2.987 & 1.15 & 1.883 & 0.207 & 0.213 & 1.238 & 57.092 & 0.232 & 0.226 & 0.226 & 99.688 & 0.227 & 38.125 \\ 
   \midrule 
 \multicolumn{1}{|c}{} & \multicolumn{1}{c}{} &       & \multicolumn{10}{c||}{Relative Test Error}                                    & \multicolumn{10}{c|}{Proportion of Variance Explained} \\
\midrule\multirow{9}[6]{*}{hsnr} & \multirow{3}[2]{*}{$\rho=0$} & p=12 & 1.04 & 1.041 & 1.041 & 1.042 & 1.042 & 1.042 & 1.042 & 1.042 & 1.042 & 1.034 & 0.89 & 0.89 & 0.89 & 0.89 & 0.89 & 0.89 & 0.89 & 0.89 & 0.89 & 0.891 \\ 
   &  & p=100 & 1.041 & 1.042 & 1.043 & 1.048 & 1.046 & 1.045 & 1.045 & 1.051 & 1.045 & 1.034 & 0.89 & 0.89 & 0.89 & 0.89 & 0.89 & 0.89 & 0.89 & 0.889 & 0.89 & 0.891 \\ 
   &  & p=196 & 1.041 & 1.042 & 1.244 & 12.282 & 1.046 & 1.045 & 1.045 & 20.791 & 1.045 & 8.534 & 0.89 & 0.89 & 0.869 & -0.304 & 0.89 & 0.89 & 0.89 & -1.2 & 0.89 & 0.096 \\ 
  \cmidrule{2-23} & \multirow{3}[2]{*}{$\rho=0.5$} & p=12 & 1.04 & 1.041 & 1.041 & 1.042 & 1.042 & 1.042 & 1.042 & 1.042 & 1.042 & 1.034 & 0.89 & 0.89 & 0.89 & 0.89 & 0.89 & 0.89 & 0.89 & 0.89 & 0.89 & 0.891 \\ 
   &  & p=100 & 1.041 & 1.042 & 1.043 & 1.048 & 1.046 & 1.045 & 1.045 & 1.051 & 1.045 & 1.034 & 0.89 & 0.89 & 0.89 & 0.89 & 0.89 & 0.89 & 0.89 & 0.889 & 0.89 & 0.891 \\ 
   &  & p=196 & 1.041 & 1.042 & 1.244 & 12.282 & 1.046 & 1.045 & 1.045 & 20.791 & 1.045 & 8.534 & 0.89 & 0.89 & 0.869 & -0.304 & 0.89 & 0.89 & 0.89 & -1.2 & 0.89 & 0.096 \\ 
  \cmidrule{2-23} & \multirow{3}[2]{*}{$\rho=0.9$} & p=12 & 1.04 & 1.041 & 1.041 & 1.042 & 1.042 & 1.042 & 1.042 & 1.042 & 1.042 & 1.034 & 0.89 & 0.89 & 0.89 & 0.89 & 0.89 & 0.89 & 0.89 & 0.89 & 0.89 & 0.891 \\ 
   &  & p=100 & 1.041 & 1.042 & 1.043 & 1.048 & 1.046 & 1.045 & 1.045 & 1.051 & 1.045 & 1.034 & 0.89 & 0.89 & 0.89 & 0.89 & 0.89 & 0.89 & 0.89 & 0.889 & 0.89 & 0.891 \\ 
   &  & p=196 & 1.041 & 1.042 & 1.244 & 12.282 & 1.046 & 1.045 & 1.045 & 20.791 & 1.045 & 8.534 & 0.89 & 0.89 & 0.869 & -0.304 & 0.89 & 0.89 & 0.89 & -1.2 & 0.89 & 0.096 \\ 
  \midrule\multirow{9}[6]{*}{msnr} & \multirow{3}[2]{*}{$\rho=0$} & p=12 & 1.04 & 1.041 & 1.041 & 1.042 & 1.042 & 1.042 & 1.042 & 1.042 & 1.042 & 1.034 & 0.481 & 0.48 & 0.48 & 0.48 & 0.48 & 0.48 & 0.48 & 0.479 & 0.48 & 0.484 \\ 
   &  & p=100 & 1.041 & 1.042 & 1.043 & 1.048 & 1.046 & 1.045 & 1.045 & 1.051 & 1.045 & 1.034 & 0.48 & 0.48 & 0.479 & 0.477 & 0.478 & 0.478 & 0.478 & 0.475 & 0.478 & 0.484 \\ 
   &  & p=196 & 1.041 & 1.042 & 1.244 & 12.282 & 1.046 & 1.045 & 1.045 & 20.791 & 1.045 & 8.534 & 0.48 & 0.48 & 0.379 & -5.155 & 0.478 & 0.478 & 0.478 & -9.402 & 0.478 & -3.271 \\ 
  \cmidrule{2-23} & \multirow{3}[2]{*}{$\rho=0.5$} & p=12 & 1.04 & 1.041 & 1.041 & 1.042 & 1.042 & 1.042 & 1.042 & 1.042 & 1.042 & 1.034 & 0.481 & 0.48 & 0.48 & 0.48 & 0.48 & 0.48 & 0.48 & 0.479 & 0.48 & 0.484 \\ 
   &  & p=100 & 1.041 & 1.042 & 1.043 & 1.048 & 1.046 & 1.045 & 1.045 & 1.051 & 1.045 & 1.034 & 0.48 & 0.48 & 0.479 & 0.477 & 0.478 & 0.478 & 0.478 & 0.475 & 0.478 & 0.484 \\ 
   &  & p=196 & 1.041 & 1.042 & 1.244 & 12.282 & 1.046 & 1.045 & 1.045 & 20.791 & 1.045 & 8.534 & 0.48 & 0.48 & 0.379 & -5.155 & 0.478 & 0.478 & 0.478 & -9.402 & 0.478 & -3.271 \\ 
  \cmidrule{2-23} & \multirow{3}[2]{*}{$\rho=0.9$} & p=12 & 1.04 & 1.041 & 1.041 & 1.042 & 1.042 & 1.042 & 1.042 & 1.042 & 1.042 & 1.034 & 0.481 & 0.48 & 0.48 & 0.48 & 0.48 & 0.48 & 0.48 & 0.479 & 0.48 & 0.484 \\ 
   &  & p=100 & 1.041 & 1.042 & 1.043 & 1.048 & 1.046 & 1.045 & 1.045 & 1.051 & 1.045 & 1.034 & 0.48 & 0.48 & 0.479 & 0.477 & 0.478 & 0.478 & 0.478 & 0.475 & 0.478 & 0.484 \\ 
   &  & p=196 & 1.041 & 1.042 & 1.244 & 12.282 & 1.046 & 1.045 & 1.045 & 20.791 & 1.045 & 8.534 & 0.48 & 0.48 & 0.379 & -5.155 & 0.478 & 0.478 & 0.478 & -9.402 & 0.478 & -3.271 \\ 
  \midrule\multirow{9}[6]{*}{lsnr} & \multirow{3}[2]{*}{$\rho=0$} & p=12 & 1.04 & 1.041 & 1.042 & 1.042 & 1.043 & 1.042 & 1.042 & 1.043 & 1.042 & 1.067 & 0.134 & 0.133 & 0.133 & 0.132 & 0.132 & 0.133 & 0.132 & 0.132 & 0.132 & 0.112 \\ 
   &  & p=100 & 1.042 & 1.043 & 1.043 & 1.048 & 1.047 & 1.045 & 1.045 & 1.051 & 1.046 & 1.067 & 0.133 & 0.132 & 0.131 & 0.127 & 0.128 & 0.13 & 0.13 & 0.125 & 0.13 & 0.112 \\ 
   &  & p=196 & 1.042 & 1.043 & 1.247 & 12.284 & 1.047 & 1.045 & 1.045 & 20.792 & 1.046 & 8.564 & 0.133 & 0.132 & -0.038 & -9.237 & 0.128 & 0.13 & 0.13 & -16.319 & 0.13 & -6.134 \\ 
  \cmidrule{2-23} & \multirow{3}[2]{*}{$\rho=0.5$} & p=12 & 1.04 & 1.041 & 1.041 & 1.042 & 1.042 & 1.042 & 1.042 & 1.043 & 1.042 & 1.064 & 0.134 & 0.134 & 0.133 & 0.132 & 0.132 & 0.133 & 0.133 & 0.132 & 0.133 & 0.114 \\ 
   &  & p=100 & 1.041 & 1.042 & 1.043 & 1.048 & 1.047 & 1.045 & 1.045 & 1.051 & 1.045 & 1.064 & 0.133 & 0.132 & 0.132 & 0.128 & 0.129 & 0.13 & 0.13 & 0.125 & 0.13 & 0.114 \\ 
   &  & p=196 & 1.041 & 1.042 & 1.246 & 12.284 & 1.047 & 1.045 & 1.045 & 20.791 & 1.045 & 8.562 & 0.133 & 0.132 & -0.038 & -9.237 & 0.129 & 0.13 & 0.13 & -16.318 & 0.13 & -6.132 \\ 
  \cmidrule{2-23} & \multirow{3}[2]{*}{$\rho=0.9$} & p=12 & 1.04 & 1.041 & 1.041 & 1.042 & 1.042 & 1.042 & 1.042 & 1.042 & 1.042 & 1.064 & 0.134 & 0.134 & 0.133 & 0.133 & 0.132 & 0.133 & 0.133 & 0.132 & 0.133 & 0.114 \\ 
   &  & p=100 & 1.041 & 1.042 & 1.043 & 1.048 & 1.046 & 1.045 & 1.045 & 1.051 & 1.045 & 1.064 & 0.133 & 0.132 & 0.132 & 0.128 & 0.129 & 0.13 & 0.13 & 0.125 & 0.13 & 0.114 \\ 
   &  & p=196 & 1.041 & 1.042 & 1.246 & 12.284 & 1.046 & 1.045 & 1.045 & 20.791 & 1.045 & 8.561 & 0.133 & 0.132 & -0.038 & -9.237 & 0.129 & 0.13 & 0.13 & -16.318 & 0.13 & -6.132 \\ 
   \midrule 
 \multicolumn{1}{|c}{} & \multicolumn{1}{c}{} &       & \multicolumn{10}{c||}{Kullback-Leibler Discrepancy}                                    & \multicolumn{10}{c|}{Number of Variables} \\
\midrule\multirow{9}[6]{*}{hsnr} & \multirow{3}[2]{*}{$\rho=0$} & p=12 & 3.154 & 3.155 & 3.156 & 3.157 & 3.157 & 3.156 & 3.157 & 3.157 & 3.157 & 3.147 & 6.434 & 6.501 & 6.54 & 6.628 & 6.72 & 6.592 & 6.619 & 6.662 & 6.619 & 6.036 \\ 
   &  & p=100 & 3.156 & 3.158 & 3.159 & 3.166 & 3.163 & 3.161 & 3.161 & 3.172 & 3.162 & 3.147 & 6.528 & 6.615 & 6.692 & 7.107 & 7.073 & 6.855 & 6.843 & 7.385 & 6.862 & 6.036 \\ 
   &  & p=196 & 3.156 & 3.158 & 3.809 & 2586.131 & 3.163 & 3.161 & 3.161 & 3618.445 & 3.162 & 2526.058 & 6.528 & 6.615 & 22.429 & 54.127 & 7.073 & 6.855 & 6.843 & 78.427 & 6.862 & 29.957 \\ 
  \cmidrule{2-23} & \multirow{3}[2]{*}{$\rho=0.5$} & p=12 & 2.461 & 2.462 & 2.463 & 2.464 & 2.464 & 2.463 & 2.464 & 2.464 & 2.464 & 2.454 & 6.434 & 6.501 & 6.54 & 6.628 & 6.72 & 6.592 & 6.619 & 6.662 & 6.619 & 6.036 \\ 
   &  & p=100 & 2.463 & 2.465 & 2.466 & 2.473 & 2.47 & 2.468 & 2.468 & 2.479 & 2.468 & 2.454 & 6.528 & 6.615 & 6.692 & 7.107 & 7.073 & 6.855 & 6.843 & 7.385 & 6.862 & 6.036 \\ 
   &  & p=196 & 2.463 & 2.465 & 3.116 & 2585.437 & 2.47 & 2.468 & 2.468 & 3617.752 & 2.468 & 2525.365 & 6.528 & 6.615 & 22.429 & 54.127 & 7.073 & 6.855 & 6.843 & 78.427 & 6.862 & 29.957 \\ 
  \cmidrule{2-23} & \multirow{3}[2]{*}{$\rho=0.9$} & p=12 & 0.852 & 0.853 & 0.853 & 0.854 & 0.854 & 0.854 & 0.854 & 0.855 & 0.854 & 0.844 & 6.434 & 6.501 & 6.54 & 6.628 & 6.72 & 6.592 & 6.619 & 6.662 & 6.619 & 6.036 \\ 
   &  & p=100 & 0.854 & 0.855 & 0.856 & 0.864 & 0.86 & 0.859 & 0.859 & 0.87 & 0.859 & 0.844 & 6.528 & 6.615 & 6.692 & 7.107 & 7.073 & 6.855 & 6.843 & 7.385 & 6.862 & 6.036 \\ 
   &  & p=196 & 0.854 & 0.855 & 1.507 & 2583.828 & 0.86 & 0.859 & 0.859 & 3616.143 & 0.859 & 2523.755 & 6.528 & 6.615 & 22.429 & 54.127 & 7.073 & 6.855 & 6.843 & 78.427 & 6.862 & 29.957 \\ 
  \midrule\multirow{9}[6]{*}{msnr} & \multirow{3}[2]{*}{$\rho=0$} & p=12 & 5.294 & 5.295 & 5.296 & 5.297 & 5.297 & 5.296 & 5.297 & 5.297 & 5.297 & 5.287 & 6.434 & 6.501 & 6.54 & 6.628 & 6.72 & 6.592 & 6.619 & 6.662 & 6.619 & 6.036 \\ 
   &  & p=100 & 5.297 & 5.298 & 5.299 & 5.306 & 5.303 & 5.302 & 5.301 & 5.312 & 5.302 & 5.287 & 6.528 & 6.615 & 6.692 & 7.107 & 7.073 & 6.855 & 6.843 & 7.385 & 6.862 & 6.036 \\ 
   &  & p=196 & 5.297 & 5.298 & 5.95 & 2588.271 & 5.303 & 5.302 & 5.301 & 3620.585 & 5.302 & 2528.198 & 6.528 & 6.615 & 22.429 & 54.127 & 7.073 & 6.855 & 6.843 & 78.427 & 6.862 & 29.957 \\ 
  \cmidrule{2-23} & \multirow{3}[2]{*}{$\rho=0.5$} & p=12 & 4.601 & 4.602 & 4.603 & 4.604 & 4.604 & 4.603 & 4.604 & 4.604 & 4.604 & 4.594 & 6.434 & 6.501 & 6.54 & 6.628 & 6.72 & 6.592 & 6.619 & 6.662 & 6.619 & 6.036 \\ 
   &  & p=100 & 4.603 & 4.605 & 4.606 & 4.613 & 4.61 & 4.608 & 4.608 & 4.619 & 4.608 & 4.594 & 6.528 & 6.615 & 6.692 & 7.107 & 7.073 & 6.855 & 6.843 & 7.385 & 6.862 & 6.036 \\ 
   &  & p=196 & 4.603 & 4.605 & 5.256 & 2587.578 & 4.61 & 4.608 & 4.608 & 3619.892 & 4.608 & 2527.505 & 6.528 & 6.615 & 22.429 & 54.127 & 7.073 & 6.855 & 6.843 & 78.427 & 6.862 & 29.957 \\ 
  \cmidrule{2-23} & \multirow{3}[2]{*}{$\rho=0.9$} & p=12 & 2.992 & 2.993 & 2.993 & 2.994 & 2.994 & 2.994 & 2.994 & 2.995 & 2.994 & 2.984 & 6.434 & 6.501 & 6.54 & 6.628 & 6.72 & 6.592 & 6.619 & 6.662 & 6.619 & 6.036 \\ 
   &  & p=100 & 2.994 & 2.995 & 2.996 & 3.004 & 3 & 2.999 & 2.999 & 3.01 & 2.999 & 2.984 & 6.528 & 6.615 & 6.692 & 7.107 & 7.073 & 6.855 & 6.843 & 7.385 & 6.862 & 6.036 \\ 
   &  & p=196 & 2.994 & 2.995 & 3.647 & 2585.968 & 3 & 2.999 & 2.999 & 3618.283 & 2.999 & 2525.895 & 6.528 & 6.615 & 22.429 & 54.127 & 7.073 & 6.855 & 6.843 & 78.427 & 6.862 & 29.957 \\ 
  \midrule\multirow{9}[6]{*}{lsnr} & \multirow{3}[2]{*}{$\rho=0$} & p=12 & 6.904 & 6.905 & 6.906 & 6.907 & 6.907 & 6.906 & 6.907 & 6.907 & 6.907 & 6.926 & 6.418 & 6.488 & 6.528 & 6.612 & 6.687 & 6.58 & 6.604 & 6.647 & 6.604 & 4.946 \\ 
   &  & p=100 & 6.906 & 6.908 & 6.909 & 6.916 & 6.913 & 6.911 & 6.911 & 6.922 & 6.911 & 6.926 & 6.512 & 6.602 & 6.681 & 7.096 & 7.04 & 6.843 & 6.828 & 7.37 & 6.847 & 4.946 \\ 
   &  & p=196 & 6.906 & 6.908 & 7.562 & 2589.883 & 6.913 & 6.911 & 6.911 & 3622.195 & 6.911 & 2529.834 & 6.512 & 6.602 & 22.333 & 54.037 & 7.04 & 6.843 & 6.828 & 78.421 & 6.847 & 28.987 \\ 
  \cmidrule{2-23} & \multirow{3}[2]{*}{$\rho=0.5$} & p=12 & 6.211 & 6.212 & 6.212 & 6.213 & 6.214 & 6.213 & 6.213 & 6.214 & 6.213 & 6.231 & 6.433 & 6.5 & 6.539 & 6.627 & 6.705 & 6.591 & 6.618 & 6.661 & 6.618 & 4.98 \\ 
   &  & p=100 & 6.213 & 6.214 & 6.215 & 6.223 & 6.219 & 6.218 & 6.218 & 6.229 & 6.218 & 6.231 & 6.527 & 6.614 & 6.691 & 7.106 & 7.058 & 6.854 & 6.842 & 7.384 & 6.861 & 4.98 \\ 
   &  & p=196 & 6.213 & 6.214 & 6.868 & 2589.189 & 6.219 & 6.218 & 6.218 & 3621.502 & 6.218 & 2529.139 & 6.527 & 6.614 & 22.354 & 54.058 & 7.058 & 6.854 & 6.842 & 78.426 & 6.861 & 29.016 \\ 
  \cmidrule{2-23} & \multirow{3}[2]{*}{$\rho=0.9$} & p=12 & 4.601 & 4.602 & 4.603 & 4.604 & 4.604 & 4.603 & 4.604 & 4.604 & 4.604 & 4.621 & 6.434 & 6.501 & 6.54 & 6.628 & 6.712 & 6.592 & 6.619 & 6.662 & 6.619 & 4.976 \\ 
   &  & p=100 & 4.603 & 4.605 & 4.606 & 4.613 & 4.61 & 4.608 & 4.608 & 4.619 & 4.608 & 4.621 & 6.528 & 6.615 & 6.692 & 7.107 & 7.065 & 6.855 & 6.843 & 7.385 & 6.862 & 4.976 \\ 
   &  & p=196 & 4.603 & 4.605 & 5.258 & 2587.579 & 4.61 & 4.608 & 4.608 & 3619.892 & 4.608 & 2527.529 & 6.528 & 6.615 & 22.355 & 54.059 & 7.065 & 6.855 & 6.843 & 78.427 & 6.862 & 29.017 \\ 
   \bottomrule 
\end{tabular}
}
\end{table}

%% latex table generated in R 4.0.2 by xtable 1.8-4 package
% Fri Jun 26 10:43:47 2020
\begin{table}[ht]
\centering
\caption{Fixed X, Sparse-Ex4, n=1000} 
\scalebox{0.52}{
\begin{tabular}{|c|c|c|cc|cc|cc|ccc|c||cc|cc|cc|ccc|c|}
  \toprule 
 \multicolumn{1}{|c}{} & \multicolumn{1}{c}{} &       & RAICc & AICc  & RC$_p$ & C$_p$ & 10F CV & LOO CV & S$_p$ & $\widehat{\text{C}}_p$ & GCV   & BIC   & RAICc & AICc  & RC$_p$ & C$_p$ & 10F CV & LOO CV & S$_p$ & $\widehat{\text{C}}_p$ & GCV   & BIC  \\
 \cmidrule{4-23}\multicolumn{1}{|c}{} & \multicolumn{1}{c}{} &       & \multicolumn{10}{c||}{RMSE}                                                   & \multicolumn{10}{c|}{Relative Risk}       \\
 \midrule\multirow{9}[6]{*}{hsnr} & \multirow{3}[2]{*}{$\rho=0$} & p=12 & 0.241 & 0.241 & 0.241 & 0.242 & 0.242 & 0.242 & 0.241 & 0.242 & 0.241 & 0.212 & 0.001 & 0.001 & 0.001 & 0.001 & 0.001 & 0.001 & 0.001 & 0.001 & 0.001 & 0.001 \\ 
   &  & p=500 & 0.248 & 0.249 & 0.249 & 0.251 & 0.251 & 0.251 & 0.251 & 0.253 & 0.251 & 0.212 & 0.001 & 0.001 & 0.001 & 0.001 & 0.001 & 0.001 & 0.001 & 0.001 & 0.001 & 0.001 \\ 
   &  & p=996 & 0.248 & 0.249 & 0.619 & 0.953 & 0.251 & 0.251 & 0.251 & 1.181 & 0.251 & 0.473 & 0.001 & 0.001 & 0.024 & 5.508 & 0.001 & 0.001 & 0.001 & 9.651 & 0.001 & 3.042 \\ 
  \cmidrule{2-23} & \multirow{3}[2]{*}{$\rho=0.5$} & p=12 & 0.17 & 0.17 & 0.17 & 0.171 & 0.171 & 0.171 & 0.171 & 0.171 & 0.171 & 0.15 & 0.001 & 0.001 & 0.001 & 0.001 & 0.001 & 0.001 & 0.001 & 0.001 & 0.001 & 0.001 \\ 
   &  & p=500 & 0.175 & 0.176 & 0.176 & 0.178 & 0.178 & 0.178 & 0.177 & 0.179 & 0.177 & 0.15 & 0.001 & 0.001 & 0.001 & 0.001 & 0.001 & 0.001 & 0.001 & 0.001 & 0.001 & 0.001 \\ 
   &  & p=996 & 0.175 & 0.176 & 0.437 & 0.674 & 0.178 & 0.178 & 0.177 & 0.835 & 0.177 & 0.335 & 0.001 & 0.001 & 0.024 & 5.508 & 0.001 & 0.001 & 0.001 & 9.651 & 0.001 & 3.042 \\ 
  \cmidrule{2-23} & \multirow{3}[2]{*}{$\rho=0.9$} & p=12 & 0.076 & 0.076 & 0.076 & 0.076 & 0.076 & 0.076 & 0.076 & 0.077 & 0.076 & 0.067 & 0.001 & 0.001 & 0.001 & 0.001 & 0.001 & 0.001 & 0.001 & 0.001 & 0.001 & 0.001 \\ 
   &  & p=500 & 0.078 & 0.079 & 0.079 & 0.079 & 0.079 & 0.079 & 0.079 & 0.08 & 0.079 & 0.067 & 0.001 & 0.001 & 0.001 & 0.001 & 0.001 & 0.001 & 0.001 & 0.001 & 0.001 & 0.001 \\ 
   &  & p=996 & 0.078 & 0.079 & 0.196 & 0.301 & 0.079 & 0.079 & 0.079 & 0.373 & 0.079 & 0.15 & 0.001 & 0.001 & 0.024 & 5.508 & 0.001 & 0.001 & 0.001 & 9.651 & 0.001 & 3.042 \\ 
  \midrule\multirow{9}[6]{*}{msnr} & \multirow{3}[2]{*}{$\rho=0$} & p=12 & 0.701 & 0.703 & 0.703 & 0.704 & 0.705 & 0.704 & 0.704 & 0.706 & 0.704 & 0.619 & 0.008 & 0.008 & 0.008 & 0.008 & 0.008 & 0.008 & 0.008 & 0.008 & 0.008 & 0.006 \\ 
   &  & p=500 & 0.723 & 0.727 & 0.727 & 0.732 & 0.732 & 0.732 & 0.732 & 0.737 & 0.732 & 0.619 & 0.009 & 0.009 & 0.009 & 0.009 & 0.009 & 0.009 & 0.009 & 0.009 & 0.009 & 0.006 \\ 
   &  & p=996 & 0.723 & 0.727 & 1.804 & 2.778 & 0.732 & 0.732 & 0.732 & 3.442 & 0.732 & 1.38 & 0.009 & 0.009 & 0.208 & 46.819 & 0.009 & 0.009 & 0.009 & 82.035 & 0.009 & 25.859 \\ 
  \cmidrule{2-23} & \multirow{3}[2]{*}{$\rho=0.5$} & p=12 & 0.496 & 0.497 & 0.497 & 0.498 & 0.498 & 0.498 & 0.498 & 0.499 & 0.498 & 0.438 & 0.008 & 0.008 & 0.008 & 0.008 & 0.008 & 0.008 & 0.008 & 0.008 & 0.008 & 0.006 \\ 
   &  & p=500 & 0.511 & 0.514 & 0.514 & 0.518 & 0.518 & 0.518 & 0.517 & 0.521 & 0.517 & 0.438 & 0.009 & 0.009 & 0.009 & 0.009 & 0.009 & 0.009 & 0.009 & 0.009 & 0.009 & 0.006 \\ 
   &  & p=996 & 0.511 & 0.514 & 1.275 & 1.964 & 0.518 & 0.518 & 0.517 & 2.434 & 0.517 & 0.975 & 0.009 & 0.009 & 0.208 & 46.819 & 0.009 & 0.009 & 0.009 & 82.035 & 0.009 & 25.859 \\ 
  \cmidrule{2-23} & \multirow{3}[2]{*}{$\rho=0.9$} & p=12 & 0.222 & 0.222 & 0.222 & 0.223 & 0.223 & 0.223 & 0.223 & 0.223 & 0.223 & 0.196 & 0.008 & 0.008 & 0.008 & 0.008 & 0.008 & 0.008 & 0.008 & 0.008 & 0.008 & 0.006 \\ 
   &  & p=500 & 0.229 & 0.23 & 0.23 & 0.231 & 0.232 & 0.232 & 0.231 & 0.233 & 0.231 & 0.196 & 0.009 & 0.009 & 0.009 & 0.009 & 0.009 & 0.009 & 0.009 & 0.009 & 0.009 & 0.006 \\ 
   &  & p=996 & 0.229 & 0.23 & 0.57 & 0.878 & 0.232 & 0.232 & 0.231 & 1.088 & 0.231 & 0.436 & 0.009 & 0.009 & 0.208 & 46.819 & 0.009 & 0.009 & 0.009 & 82.035 & 0.009 & 25.859 \\ 
  \midrule\multirow{9}[6]{*}{lsnr} & \multirow{3}[2]{*}{$\rho=0$} & p=12 & 1.568 & 1.571 & 1.572 & 1.575 & 1.575 & 1.575 & 1.574 & 1.579 & 1.574 & 1.385 & 0.04 & 0.04 & 0.04 & 0.04 & 0.04 & 0.04 & 0.04 & 0.04 & 0.04 & 0.031 \\ 
   &  & p=500 & 1.616 & 1.626 & 1.625 & 1.637 & 1.637 & 1.638 & 1.636 & 1.647 & 1.636 & 1.385 & 0.044 & 0.044 & 0.044 & 0.045 & 0.045 & 0.045 & 0.045 & 0.046 & 0.045 & 0.031 \\ 
   &  & p=996 & 1.616 & 1.626 & 4.033 & 6.211 & 1.637 & 1.638 & 1.636 & 7.697 & 1.636 & 3.085 & 0.044 & 0.044 & 1.04 & 234.093 & 0.045 & 0.045 & 0.045 & 410.176 & 0.045 & 129.296 \\ 
  \cmidrule{2-23} & \multirow{3}[2]{*}{$\rho=0.5$} & p=12 & 1.109 & 1.111 & 1.111 & 1.114 & 1.114 & 1.114 & 1.113 & 1.116 & 1.113 & 0.979 & 0.04 & 0.04 & 0.04 & 0.04 & 0.04 & 0.04 & 0.04 & 0.04 & 0.04 & 0.031 \\ 
   &  & p=500 & 1.143 & 1.149 & 1.149 & 1.157 & 1.158 & 1.158 & 1.157 & 1.165 & 1.157 & 0.979 & 0.044 & 0.044 & 0.044 & 0.045 & 0.045 & 0.045 & 0.045 & 0.046 & 0.045 & 0.031 \\ 
   &  & p=996 & 1.143 & 1.149 & 2.852 & 4.392 & 1.158 & 1.158 & 1.157 & 5.442 & 1.157 & 2.181 & 0.044 & 0.044 & 1.04 & 234.093 & 0.045 & 0.045 & 0.045 & 410.176 & 0.045 & 129.296 \\ 
  \cmidrule{2-23} & \multirow{3}[2]{*}{$\rho=0.9$} & p=12 & 0.496 & 0.497 & 0.497 & 0.498 & 0.498 & 0.498 & 0.498 & 0.499 & 0.498 & 0.438 & 0.04 & 0.04 & 0.04 & 0.04 & 0.04 & 0.04 & 0.04 & 0.04 & 0.04 & 0.031 \\ 
   &  & p=500 & 0.511 & 0.514 & 0.514 & 0.518 & 0.518 & 0.518 & 0.517 & 0.521 & 0.517 & 0.438 & 0.044 & 0.044 & 0.044 & 0.045 & 0.045 & 0.045 & 0.045 & 0.046 & 0.045 & 0.031 \\ 
   &  & p=996 & 0.511 & 0.514 & 1.275 & 1.964 & 0.518 & 0.518 & 0.517 & 2.434 & 0.517 & 0.975 & 0.044 & 0.044 & 1.04 & 234.093 & 0.045 & 0.045 & 0.045 & 410.176 & 0.045 & 129.296 \\ 
   \midrule 
 \multicolumn{1}{|c}{} & \multicolumn{1}{c}{} &       & \multicolumn{10}{c||}{Relative Test Error}                                    & \multicolumn{10}{c|}{Proportion of Variance Explained} \\
\midrule\multirow{9}[6]{*}{hsnr} & \multirow{3}[2]{*}{$\rho=0$} & p=12 & 1.008 & 1.008 & 1.008 & 1.008 & 1.008 & 1.008 & 1.008 & 1.008 & 1.008 & 1.006 & 0.895 & 0.895 & 0.895 & 0.895 & 0.895 & 0.895 & 0.895 & 0.895 & 0.895 & 0.896 \\ 
   &  & p=500 & 1.009 & 1.009 & 1.009 & 1.009 & 1.009 & 1.009 & 1.009 & 1.009 & 1.009 & 1.006 & 0.895 & 0.895 & 0.895 & 0.895 & 0.895 & 0.895 & 0.895 & 0.895 & 0.895 & 0.896 \\ 
   &  & p=996 & 1.009 & 1.009 & 1.212 & 48.611 & 1.009 & 1.009 & 1.009 & 84.423 & 1.009 & 27.297 & 0.895 & 0.895 & 0.874 & -4.041 & 0.895 & 0.895 & 0.895 & -7.754 & 0.895 & -1.83 \\ 
  \cmidrule{2-23} & \multirow{3}[2]{*}{$\rho=0.5$} & p=12 & 1.008 & 1.008 & 1.008 & 1.008 & 1.008 & 1.008 & 1.008 & 1.008 & 1.008 & 1.006 & 0.895 & 0.895 & 0.895 & 0.895 & 0.895 & 0.895 & 0.895 & 0.895 & 0.895 & 0.896 \\ 
   &  & p=500 & 1.009 & 1.009 & 1.009 & 1.009 & 1.009 & 1.009 & 1.009 & 1.009 & 1.009 & 1.006 & 0.895 & 0.895 & 0.895 & 0.895 & 0.895 & 0.895 & 0.895 & 0.895 & 0.895 & 0.896 \\ 
   &  & p=996 & 1.009 & 1.009 & 1.212 & 48.611 & 1.009 & 1.009 & 1.009 & 84.423 & 1.009 & 27.297 & 0.895 & 0.895 & 0.874 & -4.041 & 0.895 & 0.895 & 0.895 & -7.754 & 0.895 & -1.83 \\ 
  \cmidrule{2-23} & \multirow{3}[2]{*}{$\rho=0.9$} & p=12 & 1.008 & 1.008 & 1.008 & 1.008 & 1.008 & 1.008 & 1.008 & 1.008 & 1.008 & 1.006 & 0.895 & 0.895 & 0.895 & 0.895 & 0.895 & 0.895 & 0.895 & 0.895 & 0.895 & 0.896 \\ 
   &  & p=500 & 1.009 & 1.009 & 1.009 & 1.009 & 1.009 & 1.009 & 1.009 & 1.009 & 1.009 & 1.006 & 0.895 & 0.895 & 0.895 & 0.895 & 0.895 & 0.895 & 0.895 & 0.895 & 0.895 & 0.896 \\ 
   &  & p=996 & 1.009 & 1.009 & 1.212 & 48.611 & 1.009 & 1.009 & 1.009 & 84.423 & 1.009 & 27.297 & 0.895 & 0.895 & 0.874 & -4.041 & 0.895 & 0.895 & 0.895 & -7.754 & 0.895 & -1.83 \\ 
  \midrule\multirow{9}[6]{*}{msnr} & \multirow{3}[2]{*}{$\rho=0$} & p=12 & 1.008 & 1.008 & 1.008 & 1.008 & 1.008 & 1.008 & 1.008 & 1.008 & 1.008 & 1.006 & 0.5 & 0.5 & 0.5 & 0.5 & 0.5 & 0.5 & 0.5 & 0.5 & 0.5 & 0.501 \\ 
   &  & p=500 & 1.009 & 1.009 & 1.009 & 1.009 & 1.009 & 1.009 & 1.009 & 1.009 & 1.009 & 1.006 & 0.5 & 0.5 & 0.5 & 0.5 & 0.5 & 0.5 & 0.5 & 0.5 & 0.5 & 0.501 \\ 
   &  & p=996 & 1.009 & 1.009 & 1.212 & 48.611 & 1.009 & 1.009 & 1.009 & 84.423 & 1.009 & 27.297 & 0.5 & 0.5 & 0.399 & -23.102 & 0.5 & 0.5 & 0.5 & -40.858 & 0.5 & -12.534 \\ 
  \cmidrule{2-23} & \multirow{3}[2]{*}{$\rho=0.5$} & p=12 & 1.008 & 1.008 & 1.008 & 1.008 & 1.008 & 1.008 & 1.008 & 1.008 & 1.008 & 1.006 & 0.5 & 0.5 & 0.5 & 0.5 & 0.5 & 0.5 & 0.5 & 0.5 & 0.5 & 0.501 \\ 
   &  & p=500 & 1.009 & 1.009 & 1.009 & 1.009 & 1.009 & 1.009 & 1.009 & 1.009 & 1.009 & 1.006 & 0.5 & 0.5 & 0.5 & 0.5 & 0.5 & 0.5 & 0.5 & 0.5 & 0.5 & 0.501 \\ 
   &  & p=996 & 1.009 & 1.009 & 1.212 & 48.611 & 1.009 & 1.009 & 1.009 & 84.423 & 1.009 & 27.297 & 0.5 & 0.5 & 0.399 & -23.102 & 0.5 & 0.5 & 0.5 & -40.858 & 0.5 & -12.534 \\ 
  \cmidrule{2-23} & \multirow{3}[2]{*}{$\rho=0.9$} & p=12 & 1.008 & 1.008 & 1.008 & 1.008 & 1.008 & 1.008 & 1.008 & 1.008 & 1.008 & 1.006 & 0.5 & 0.5 & 0.5 & 0.5 & 0.5 & 0.5 & 0.5 & 0.5 & 0.5 & 0.501 \\ 
   &  & p=500 & 1.009 & 1.009 & 1.009 & 1.009 & 1.009 & 1.009 & 1.009 & 1.009 & 1.009 & 1.006 & 0.5 & 0.5 & 0.5 & 0.5 & 0.5 & 0.5 & 0.5 & 0.5 & 0.5 & 0.501 \\ 
   &  & p=996 & 1.009 & 1.009 & 1.212 & 48.611 & 1.009 & 1.009 & 1.009 & 84.423 & 1.009 & 27.297 & 0.5 & 0.5 & 0.399 & -23.102 & 0.5 & 0.5 & 0.5 & -40.858 & 0.5 & -12.534 \\ 
  \midrule\multirow{9}[6]{*}{lsnr} & \multirow{3}[2]{*}{$\rho=0$} & p=12 & 1.008 & 1.008 & 1.008 & 1.008 & 1.008 & 1.008 & 1.008 & 1.008 & 1.008 & 1.006 & 0.162 & 0.162 & 0.162 & 0.162 & 0.162 & 0.162 & 0.162 & 0.162 & 0.162 & 0.164 \\ 
   &  & p=500 & 1.009 & 1.009 & 1.009 & 1.009 & 1.009 & 1.009 & 1.009 & 1.009 & 1.009 & 1.006 & 0.162 & 0.162 & 0.162 & 0.161 & 0.161 & 0.161 & 0.161 & 0.161 & 0.161 & 0.164 \\ 
   &  & p=996 & 1.009 & 1.009 & 1.212 & 48.611 & 1.009 & 1.009 & 1.009 & 84.423 & 1.009 & 27.297 & 0.162 & 0.162 & -0.007 & -39.395 & 0.161 & 0.161 & 0.161 & -69.155 & 0.161 & -21.683 \\ 
  \cmidrule{2-23} & \multirow{3}[2]{*}{$\rho=0.5$} & p=12 & 1.008 & 1.008 & 1.008 & 1.008 & 1.008 & 1.008 & 1.008 & 1.008 & 1.008 & 1.006 & 0.162 & 0.162 & 0.162 & 0.162 & 0.162 & 0.162 & 0.162 & 0.162 & 0.162 & 0.164 \\ 
   &  & p=500 & 1.009 & 1.009 & 1.009 & 1.009 & 1.009 & 1.009 & 1.009 & 1.009 & 1.009 & 1.006 & 0.162 & 0.162 & 0.162 & 0.161 & 0.161 & 0.161 & 0.161 & 0.161 & 0.161 & 0.164 \\ 
   &  & p=996 & 1.009 & 1.009 & 1.212 & 48.611 & 1.009 & 1.009 & 1.009 & 84.423 & 1.009 & 27.297 & 0.162 & 0.162 & -0.007 & -39.395 & 0.161 & 0.161 & 0.161 & -69.155 & 0.161 & -21.683 \\ 
  \cmidrule{2-23} & \multirow{3}[2]{*}{$\rho=0.9$} & p=12 & 1.008 & 1.008 & 1.008 & 1.008 & 1.008 & 1.008 & 1.008 & 1.008 & 1.008 & 1.006 & 0.162 & 0.162 & 0.162 & 0.162 & 0.162 & 0.162 & 0.162 & 0.162 & 0.162 & 0.164 \\ 
   &  & p=500 & 1.009 & 1.009 & 1.009 & 1.009 & 1.009 & 1.009 & 1.009 & 1.009 & 1.009 & 1.006 & 0.162 & 0.162 & 0.162 & 0.161 & 0.161 & 0.161 & 0.161 & 0.161 & 0.161 & 0.164 \\ 
   &  & p=996 & 1.009 & 1.009 & 1.212 & 48.611 & 1.009 & 1.009 & 1.009 & 84.423 & 1.009 & 27.297 & 0.162 & 0.162 & -0.007 & -39.395 & 0.161 & 0.161 & 0.161 & -69.155 & 0.161 & -21.683 \\ 
   \midrule 
 \multicolumn{1}{|c}{} & \multicolumn{1}{c}{} &       & \multicolumn{10}{c||}{Kullback-Leibler Discrepancy}                                    & \multicolumn{10}{c|}{Number of Variables} \\
\midrule\multirow{9}[6]{*}{hsnr} & \multirow{3}[2]{*}{$\rho=0$} & p=12 & 3.101 & 3.101 & 3.101 & 3.101 & 3.101 & 3.101 & 3.101 & 3.101 & 3.101 & 3.099 & 6.643 & 6.658 & 6.659 & 6.677 & 6.72 & 6.68 & 6.672 & 6.698 & 6.673 & 6.01 \\ 
   &  & p=500 & 3.102 & 3.102 & 3.102 & 3.102 & 3.102 & 3.102 & 3.102 & 3.102 & 3.102 & 3.099 & 6.96 & 7.02 & 7.018 & 7.101 & 7.174 & 7.117 & 7.098 & 7.169 & 7.099 & 6.01 \\ 
   &  & p=996 & 3.102 & 3.102 & 3.379 & 390.736 & 3.102 & 3.102 & 3.102 & 589.682 & 3.102 & 371.017 & 6.96 & 7.02 & 95.117 & 262.275 & 7.174 & 7.117 & 7.098 & 360.193 & 7.099 & 104.015 \\ 
  \cmidrule{2-23} & \multirow{3}[2]{*}{$\rho=0.5$} & p=12 & 2.408 & 2.408 & 2.408 & 2.408 & 2.408 & 2.408 & 2.408 & 2.408 & 2.408 & 2.406 & 6.643 & 6.658 & 6.659 & 6.677 & 6.72 & 6.68 & 6.672 & 6.698 & 6.673 & 6.01 \\ 
   &  & p=500 & 2.408 & 2.409 & 2.409 & 2.409 & 2.409 & 2.409 & 2.409 & 2.409 & 2.409 & 2.406 & 6.96 & 7.02 & 7.018 & 7.101 & 7.174 & 7.117 & 7.098 & 7.169 & 7.099 & 6.01 \\ 
   &  & p=996 & 2.408 & 2.409 & 2.685 & 390.043 & 2.409 & 2.409 & 2.409 & 588.988 & 2.409 & 370.324 & 6.96 & 7.02 & 95.117 & 262.275 & 7.174 & 7.117 & 7.098 & 360.193 & 7.099 & 104.015 \\ 
  \cmidrule{2-23} & \multirow{3}[2]{*}{$\rho=0.9$} & p=12 & 0.798 & 0.798 & 0.798 & 0.798 & 0.798 & 0.798 & 0.798 & 0.798 & 0.798 & 0.796 & 6.643 & 6.658 & 6.659 & 6.677 & 6.72 & 6.68 & 6.672 & 6.698 & 6.673 & 6.01 \\ 
   &  & p=500 & 0.799 & 0.799 & 0.799 & 0.799 & 0.799 & 0.799 & 0.799 & 0.799 & 0.799 & 0.796 & 6.96 & 7.02 & 7.018 & 7.101 & 7.174 & 7.117 & 7.098 & 7.169 & 7.099 & 6.01 \\ 
   &  & p=996 & 0.799 & 0.799 & 1.076 & 388.434 & 0.799 & 0.799 & 0.799 & 587.379 & 0.799 & 368.715 & 6.96 & 7.02 & 95.117 & 262.275 & 7.174 & 7.117 & 7.098 & 360.193 & 7.099 & 104.015 \\ 
  \midrule\multirow{9}[6]{*}{msnr} & \multirow{3}[2]{*}{$\rho=0$} & p=12 & 5.241 & 5.241 & 5.241 & 5.241 & 5.241 & 5.241 & 5.241 & 5.241 & 5.241 & 5.239 & 6.643 & 6.658 & 6.659 & 6.677 & 6.72 & 6.68 & 6.672 & 6.698 & 6.673 & 6.01 \\ 
   &  & p=500 & 5.242 & 5.242 & 5.242 & 5.242 & 5.242 & 5.242 & 5.242 & 5.242 & 5.242 & 5.239 & 6.96 & 7.02 & 7.018 & 7.101 & 7.174 & 7.117 & 7.098 & 7.169 & 7.099 & 6.01 \\ 
   &  & p=996 & 5.242 & 5.242 & 5.519 & 392.876 & 5.242 & 5.242 & 5.242 & 591.822 & 5.242 & 373.157 & 6.96 & 7.02 & 95.117 & 262.275 & 7.174 & 7.117 & 7.098 & 360.193 & 7.099 & 104.015 \\ 
  \cmidrule{2-23} & \multirow{3}[2]{*}{$\rho=0.5$} & p=12 & 4.548 & 4.548 & 4.548 & 4.548 & 4.548 & 4.548 & 4.548 & 4.548 & 4.548 & 4.546 & 6.643 & 6.658 & 6.659 & 6.677 & 6.72 & 6.68 & 6.672 & 6.698 & 6.673 & 6.01 \\ 
   &  & p=500 & 4.548 & 4.549 & 4.549 & 4.549 & 4.549 & 4.549 & 4.549 & 4.549 & 4.549 & 4.546 & 6.96 & 7.02 & 7.018 & 7.101 & 7.174 & 7.117 & 7.098 & 7.169 & 7.099 & 6.01 \\ 
   &  & p=996 & 4.548 & 4.549 & 4.825 & 392.183 & 4.549 & 4.549 & 4.549 & 591.128 & 4.549 & 372.464 & 6.96 & 7.02 & 95.117 & 262.275 & 7.174 & 7.117 & 7.098 & 360.193 & 7.099 & 104.015 \\ 
  \cmidrule{2-23} & \multirow{3}[2]{*}{$\rho=0.9$} & p=12 & 2.938 & 2.938 & 2.938 & 2.938 & 2.938 & 2.938 & 2.938 & 2.938 & 2.938 & 2.936 & 6.643 & 6.658 & 6.659 & 6.677 & 6.72 & 6.68 & 6.672 & 6.698 & 6.673 & 6.01 \\ 
   &  & p=500 & 2.939 & 2.939 & 2.939 & 2.939 & 2.939 & 2.939 & 2.939 & 2.939 & 2.939 & 2.936 & 6.96 & 7.02 & 7.018 & 7.101 & 7.174 & 7.117 & 7.098 & 7.169 & 7.099 & 6.01 \\ 
   &  & p=996 & 2.939 & 2.939 & 3.216 & 390.574 & 2.939 & 2.939 & 2.939 & 589.519 & 2.939 & 370.855 & 6.96 & 7.02 & 95.117 & 262.275 & 7.174 & 7.117 & 7.098 & 360.193 & 7.099 & 104.015 \\ 
  \midrule\multirow{9}[6]{*}{lsnr} & \multirow{3}[2]{*}{$\rho=0$} & p=12 & 6.85 & 6.85 & 6.85 & 6.85 & 6.85 & 6.85 & 6.85 & 6.85 & 6.85 & 6.848 & 6.643 & 6.658 & 6.659 & 6.677 & 6.72 & 6.68 & 6.672 & 6.698 & 6.673 & 6.01 \\ 
   &  & p=500 & 6.851 & 6.851 & 6.851 & 6.851 & 6.851 & 6.851 & 6.851 & 6.852 & 6.851 & 6.848 & 6.96 & 7.02 & 7.018 & 7.101 & 7.174 & 7.117 & 7.098 & 7.169 & 7.099 & 6.01 \\ 
   &  & p=996 & 6.851 & 6.851 & 7.128 & 394.486 & 6.851 & 6.851 & 6.851 & 593.431 & 6.851 & 374.767 & 6.96 & 7.02 & 95.117 & 262.275 & 7.174 & 7.117 & 7.098 & 360.193 & 7.099 & 104.015 \\ 
  \cmidrule{2-23} & \multirow{3}[2]{*}{$\rho=0.5$} & p=12 & 6.157 & 6.157 & 6.157 & 6.157 & 6.157 & 6.157 & 6.157 & 6.157 & 6.157 & 6.155 & 6.643 & 6.658 & 6.659 & 6.677 & 6.72 & 6.68 & 6.672 & 6.698 & 6.673 & 6.01 \\ 
   &  & p=500 & 6.158 & 6.158 & 6.158 & 6.158 & 6.158 & 6.158 & 6.158 & 6.158 & 6.158 & 6.155 & 6.96 & 7.02 & 7.018 & 7.101 & 7.174 & 7.117 & 7.098 & 7.169 & 7.099 & 6.01 \\ 
   &  & p=996 & 6.158 & 6.158 & 6.435 & 393.793 & 6.158 & 6.158 & 6.158 & 592.738 & 6.158 & 374.073 & 6.96 & 7.02 & 95.117 & 262.275 & 7.174 & 7.117 & 7.098 & 360.193 & 7.099 & 104.015 \\ 
  \cmidrule{2-23} & \multirow{3}[2]{*}{$\rho=0.9$} & p=12 & 4.548 & 4.548 & 4.548 & 4.548 & 4.548 & 4.548 & 4.548 & 4.548 & 4.548 & 4.546 & 6.643 & 6.658 & 6.659 & 6.677 & 6.72 & 6.68 & 6.672 & 6.698 & 6.673 & 6.01 \\ 
   &  & p=500 & 4.548 & 4.549 & 4.549 & 4.549 & 4.549 & 4.549 & 4.549 & 4.549 & 4.549 & 4.546 & 6.96 & 7.02 & 7.018 & 7.101 & 7.174 & 7.117 & 7.098 & 7.169 & 7.099 & 6.01 \\ 
   &  & p=996 & 4.548 & 4.549 & 4.825 & 392.183 & 4.549 & 4.549 & 4.549 & 591.128 & 4.549 & 372.464 & 6.96 & 7.02 & 95.117 & 262.275 & 7.174 & 7.117 & 7.098 & 360.193 & 7.099 & 104.015 \\ 
   \bottomrule 
\end{tabular}
}
\end{table}

% latex table generated in R 4.0.2 by xtable 1.8-4 package
% Fri Jun 26 10:43:52 2020
\begin{table}[ht]
\centering
\caption{Random X, Dense-Ex1, n=40} 
\scalebox{0.52}{
\begin{tabular}{|c|c|c|cc|cc|cc|ccc|c||cc|cc|cc|ccc|c|}
  \toprule 
 \multicolumn{1}{|c}{} & \multicolumn{1}{c}{} &       & RAICc & AICc  & RC$_p$ & C$_p$ & 10F CV & LOO CV & S$_p$ & $\widehat{\text{C}}_p$ & GCV   & BIC   & RAICc & AICc  & RC$_p$ & C$_p$ & 10F CV & LOO CV & S$_p$ & $\widehat{\text{C}}_p$ & GCV   & BIC  \\
 \cmidrule{4-23}\multicolumn{1}{|c}{} & \multicolumn{1}{c}{} &       & \multicolumn{10}{c||}{RMSE}                                                   & \multicolumn{10}{c|}{Relative Risk}       \\
 \midrule\multirow{9}[6]{*}{hsnr} & \multirow{3}[2]{*}{$\rho=0$} & p=12 & 0.434 & 0.402 & 0.394 & 0.382 & 0.391 & 0.385 & 0.383 & 0.377 & 0.382 & 0.395 & 0.083 & 0.068 & 0.064 & 0.058 & 0.062 & 0.059 & 0.059 & 0.056 & 0.058 & 0.065 \\ 
   &  & p=20 & 0.685 & 0.595 & 0.555 & 0.523 & 0.536 & 0.53 & 0.528 & 0.512 & 0.527 & 0.544 & 0.21 & 0.165 & 0.146 & 0.133 & 0.138 & 0.137 & 0.136 & 0.129 & 0.135 & 0.145 \\ 
   &  & p=36 & 0.75 & 0.655 & 0.603 & 0.616 & 0.591 & 0.59 & 0.592 & 0.638 & 0.594 & 0.659 & 0.245 & 0.199 & 0.196 & 0.516 & 0.194 & 0.236 & 0.262 & 0.79 & 0.291 & 0.943 \\ 
  \cmidrule{2-23} & \multirow{3}[2]{*}{$\rho=0.5$} & p=12 & 0.263 & 0.243 & 0.24 & 0.234 & 0.239 & 0.236 & 0.235 & 0.232 & 0.234 & 0.24 & 0.079 & 0.064 & 0.061 & 0.056 & 0.06 & 0.058 & 0.057 & 0.054 & 0.057 & 0.061 \\ 
   &  & p=20 & 0.424 & 0.37 & 0.344 & 0.323 & 0.332 & 0.328 & 0.327 & 0.317 & 0.327 & 0.338 & 0.21 & 0.169 & 0.149 & 0.135 & 0.14 & 0.139 & 0.138 & 0.13 & 0.137 & 0.147 \\ 
   &  & p=36 & 0.454 & 0.398 & 0.368 & 0.377 & 0.36 & 0.36 & 0.362 & 0.391 & 0.363 & 0.404 & 0.236 & 0.193 & 0.19 & 0.505 & 0.192 & 0.234 & 0.256 & 0.781 & 0.277 & 0.938 \\ 
  \cmidrule{2-23} & \multirow{3}[2]{*}{$\rho=0.9$} & p=12 & 0.144 & 0.13 & 0.126 & 0.121 & 0.125 & 0.123 & 0.122 & 0.121 & 0.122 & 0.128 & 0.089 & 0.069 & 0.063 & 0.057 & 0.062 & 0.059 & 0.058 & 0.055 & 0.058 & 0.066 \\ 
   &  & p=20 & 0.209 & 0.189 & 0.176 & 0.168 & 0.173 & 0.171 & 0.17 & 0.164 & 0.169 & 0.18 & 0.172 & 0.149 & 0.135 & 0.129 & 0.134 & 0.133 & 0.132 & 0.126 & 0.131 & 0.144 \\ 
   &  & p=36 & 0.213 & 0.193 & 0.184 & 0.191 & 0.18 & 0.181 & 0.181 & 0.2 & 0.182 & 0.206 & 0.175 & 0.152 & 0.161 & 0.457 & 0.156 & 0.202 & 0.232 & 0.735 & 0.251 & 0.85 \\ 
  \midrule\multirow{9}[6]{*}{msnr} & \multirow{3}[2]{*}{$\rho=0$} & p=12 & 1.279 & 1.229 & 1.186 & 1.151 & 1.181 & 1.169 & 1.167 & 1.135 & 1.164 & 1.364 & 0.555 & 0.527 & 0.502 & 0.488 & 0.497 & 0.494 & 0.497 & 0.477 & 0.494 & 0.618 \\ 
   &  & p=20 & 1.438 & 1.38 & 1.335 & 1.353 & 1.357 & 1.35 & 1.347 & 1.372 & 1.346 & 1.535 & 0.64 & 0.621 & 0.618 & 0.694 & 0.651 & 0.657 & 0.656 & 0.77 & 0.66 & 0.729 \\ 
   &  & p=36 & 1.471 & 1.417 & 1.425 & 1.58 & 1.407 & 1.412 & 1.415 & 1.704 & 1.427 & 1.72 & 0.658 & 0.642 & 0.844 & 3.473 & 0.755 & 0.995 & 1.108 & 5.231 & 1.253 & 4.24 \\ 
  \cmidrule{2-23} & \multirow{3}[2]{*}{$\rho=0.5$} & p=12 & 0.803 & 0.77 & 0.743 & 0.722 & 0.74 & 0.733 & 0.733 & 0.714 & 0.729 & 0.845 & 0.561 & 0.536 & 0.509 & 0.494 & 0.504 & 0.502 & 0.504 & 0.489 & 0.5 & 0.613 \\ 
   &  & p=20 & 0.873 & 0.844 & 0.816 & 0.832 & 0.833 & 0.826 & 0.826 & 0.843 & 0.824 & 0.928 & 0.62 & 0.607 & 0.607 & 0.699 & 0.639 & 0.646 & 0.651 & 0.766 & 0.65 & 0.706 \\ 
   &  & p=36 & 0.887 & 0.863 & 0.869 & 0.965 & 0.856 & 0.86 & 0.861 & 1.044 & 0.868 & 1.04 & 0.634 & 0.623 & 0.825 & 3.416 & 0.728 & 1.002 & 1.076 & 5.338 & 1.251 & 4.165 \\ 
  \cmidrule{2-23} & \multirow{3}[2]{*}{$\rho=0.9$} & p=12 & 0.39 & 0.382 & 0.371 & 0.367 & 0.374 & 0.372 & 0.371 & 0.366 & 0.37 & 0.408 & 0.468 & 0.464 & 0.456 & 0.458 & 0.459 & 0.46 & 0.461 & 0.461 & 0.46 & 0.501 \\ 
   &  & p=20 & 0.404 & 0.4 & 0.395 & 0.409 & 0.402 & 0.403 & 0.403 & 0.416 & 0.403 & 0.425 & 0.465 & 0.47 & 0.495 & 0.598 & 0.528 & 0.541 & 0.54 & 0.655 & 0.544 & 0.508 \\ 
   &  & p=36 & 0.405 & 0.402 & 0.418 & 0.473 & 0.408 & 0.414 & 0.416 & 0.521 & 0.419 & 0.485 & 0.459 & 0.468 & 0.698 & 3.168 & 0.569 & 0.837 & 0.96 & 4.751 & 1.029 & 3.481 \\ 
  \midrule\multirow{9}[6]{*}{lsnr} & \multirow{3}[2]{*}{$\rho=0$} & p=12 & 2.032 & 2.042 & 2.065 & 2.112 & 2.081 & 2.094 & 2.09 & 2.136 & 2.094 & 2.049 & 1.185 & 1.22 & 1.321 & 1.431 & 1.34 & 1.37 & 1.373 & 1.473 & 1.381 & 1.14 \\ 
   &  & p=20 & 2.126 & 2.152 & 2.202 & 2.341 & 2.23 & 2.253 & 2.255 & 2.428 & 2.264 & 2.145 & 1.197 & 1.267 & 1.432 & 1.857 & 1.495 & 1.568 & 1.591 & 2.139 & 1.623 & 1.171 \\ 
   &  & p=36 & 2.15 & 2.175 & 2.424 & 2.862 & 2.279 & 2.343 & 2.348 & 3.294 & 2.389 & 2.642 & 1.201 & 1.268 & 2.599 & 14.951 & 1.653 & 2.835 & 3.014 & 21.828 & 3.639 & 12.954 \\ 
  \cmidrule{2-23} & \multirow{3}[2]{*}{$\rho=0.5$} & p=12 & 1.242 & 1.253 & 1.268 & 1.298 & 1.282 & 1.287 & 1.285 & 1.314 & 1.286 & 1.261 & 1.153 & 1.204 & 1.288 & 1.415 & 1.317 & 1.355 & 1.355 & 1.468 & 1.362 & 1.116 \\ 
   &  & p=20 & 1.292 & 1.303 & 1.342 & 1.422 & 1.354 & 1.371 & 1.364 & 1.482 & 1.376 & 1.307 & 1.171 & 1.223 & 1.418 & 1.829 & 1.459 & 1.554 & 1.536 & 2.141 & 1.597 & 1.14 \\ 
   &  & p=36 & 1.304 & 1.315 & 1.477 & 1.752 & 1.382 & 1.427 & 1.425 & 2.018 & 1.446 & 1.606 & 1.177 & 1.23 & 2.553 & 14.954 & 1.624 & 2.915 & 3.046 & 21.762 & 3.702 & 12.774 \\ 
  \cmidrule{2-23} & \multirow{3}[2]{*}{$\rho=0.9$} & p=12 & 0.577 & 0.586 & 0.599 & 0.616 & 0.605 & 0.607 & 0.604 & 0.626 & 0.607 & 0.592 & 0.941 & 1 & 1.096 & 1.19 & 1.118 & 1.141 & 1.126 & 1.257 & 1.138 & 0.936 \\ 
   &  & p=20 & 0.591 & 0.598 & 0.622 & 0.667 & 0.634 & 0.635 & 0.636 & 0.7 & 0.638 & 0.607 & 0.922 & 0.97 & 1.138 & 1.566 & 1.222 & 1.25 & 1.269 & 1.844 & 1.295 & 0.931 \\ 
   &  & p=36 & 0.592 & 0.601 & 0.707 & 0.85 & 0.641 & 0.659 & 0.66 & 0.991 & 0.672 & 0.771 & 0.913 & 0.96 & 2.352 & 14.113 & 1.324 & 2.457 & 2.668 & 20.66 & 3.342 & 12.332 \\ 
   \midrule 
 \multicolumn{1}{|c}{} & \multicolumn{1}{c}{} &       & \multicolumn{10}{c||}{Relative Test Error}                                    & \multicolumn{10}{c|}{Proportion of Variance Explained} \\
\midrule\multirow{9}[6]{*}{hsnr} & \multirow{3}[2]{*}{$\rho=0$} & p=12 & 1.703 & 1.574 & 1.543 & 1.495 & 1.526 & 1.503 & 1.5 & 1.475 & 1.497 & 1.548 & 0.821 & 0.835 & 0.838 & 0.844 & 0.84 & 0.843 & 0.843 & 0.846 & 0.843 & 0.838 \\ 
   &  & p=20 & 2.851 & 2.436 & 2.268 & 2.135 & 2.188 & 2.169 & 2.164 & 2.095 & 2.157 & 2.236 & 0.708 & 0.748 & 0.765 & 0.777 & 0.772 & 0.774 & 0.774 & 0.781 & 0.775 & 0.767 \\ 
   &  & p=36 & 3.161 & 2.75 & 2.714 & 5.384 & 2.687 & 3.064 & 3.303 & 7.681 & 3.558 & 9.11 & 0.677 & 0.717 & 0.721 & 0.436 & 0.722 & 0.684 & 0.661 & 0.192 & 0.635 & 0.055 \\ 
  \cmidrule{2-23} & \multirow{3}[2]{*}{$\rho=0.5$} & p=12 & 1.668 & 1.542 & 1.517 & 1.477 & 1.511 & 1.49 & 1.486 & 1.463 & 1.482 & 1.518 & 0.825 & 0.838 & 0.841 & 0.845 & 0.842 & 0.844 & 0.845 & 0.847 & 0.845 & 0.841 \\ 
   &  & p=20 & 2.862 & 2.469 & 2.294 & 2.151 & 2.206 & 2.187 & 2.18 & 2.113 & 2.177 & 2.261 & 0.708 & 0.745 & 0.762 & 0.776 & 0.77 & 0.772 & 0.773 & 0.779 & 0.773 & 0.765 \\ 
   &  & p=36 & 3.081 & 2.694 & 2.665 & 5.272 & 2.674 & 3.034 & 3.243 & 7.589 & 3.413 & 9.038 & 0.685 & 0.723 & 0.726 & 0.446 & 0.724 & 0.686 & 0.666 & 0.201 & 0.648 & 0.06 \\ 
  \cmidrule{2-23} & \multirow{3}[2]{*}{$\rho=0.9$} & p=12 & 1.748 & 1.586 & 1.537 & 1.483 & 1.527 & 1.503 & 1.494 & 1.473 & 1.492 & 1.561 & 0.817 & 0.834 & 0.839 & 0.845 & 0.84 & 0.843 & 0.844 & 0.846 & 0.844 & 0.837 \\ 
   &  & p=20 & 2.519 & 2.303 & 2.173 & 2.108 & 2.158 & 2.148 & 2.136 & 2.082 & 2.13 & 2.244 & 0.742 & 0.763 & 0.775 & 0.781 & 0.776 & 0.777 & 0.778 & 0.783 & 0.779 & 0.767 \\ 
   &  & p=36 & 2.544 & 2.338 & 2.41 & 4.916 & 2.356 & 2.748 & 3.004 & 7.278 & 3.166 & 8.318 & 0.739 & 0.76 & 0.752 & 0.488 & 0.757 & 0.715 & 0.689 & 0.241 & 0.672 & 0.137 \\ 
  \midrule\multirow{9}[6]{*}{msnr} & \multirow{3}[2]{*}{$\rho=0$} & p=12 & 1.572 & 1.539 & 1.513 & 1.496 & 1.508 & 1.503 & 1.507 & 1.483 & 1.504 & 1.638 & 0.229 & 0.244 & 0.256 & 0.264 & 0.259 & 0.261 & 0.259 & 0.27 & 0.26 & 0.197 \\ 
   &  & p=20 & 1.666 & 1.643 & 1.636 & 1.709 & 1.666 & 1.673 & 1.673 & 1.782 & 1.676 & 1.762 & 0.185 & 0.196 & 0.198 & 0.161 & 0.182 & 0.179 & 0.18 & 0.124 & 0.178 & 0.14 \\ 
   &  & p=36 & 1.685 & 1.664 & 1.868 & 4.45 & 1.778 & 2.026 & 2.137 & 6.256 & 2.297 & 5.27 & 0.176 & 0.185 & 0.084 & -1.202 & 0.128 & 0.007 & -0.048 & -2.081 & -0.123 & -1.594 \\ 
  \cmidrule{2-23} & \multirow{3}[2]{*}{$\rho=0.5$} & p=12 & 1.581 & 1.55 & 1.521 & 1.503 & 1.515 & 1.512 & 1.514 & 1.497 & 1.51 & 1.636 & 0.225 & 0.239 & 0.253 & 0.261 & 0.255 & 0.257 & 0.256 & 0.264 & 0.258 & 0.199 \\ 
   &  & p=20 & 1.646 & 1.629 & 1.625 & 1.713 & 1.656 & 1.664 & 1.669 & 1.776 & 1.668 & 1.738 & 0.195 & 0.202 & 0.203 & 0.159 & 0.188 & 0.184 & 0.182 & 0.127 & 0.182 & 0.151 \\ 
   &  & p=36 & 1.66 & 1.647 & 1.85 & 4.39 & 1.751 & 2.026 & 2.099 & 6.362 & 2.288 & 5.18 & 0.188 & 0.194 & 0.094 & -1.173 & 0.142 & 0.006 & -0.031 & -2.134 & -0.12 & -1.553 \\ 
  \cmidrule{2-23} & \multirow{3}[2]{*}{$\rho=0.9$} & p=12 & 1.486 & 1.479 & 1.468 & 1.468 & 1.471 & 1.471 & 1.472 & 1.469 & 1.47 & 1.523 & 0.271 & 0.274 & 0.279 & 0.278 & 0.277 & 0.277 & 0.276 & 0.277 & 0.277 & 0.254 \\ 
   &  & p=20 & 1.483 & 1.487 & 1.51 & 1.612 & 1.544 & 1.557 & 1.556 & 1.67 & 1.559 & 1.531 & 0.273 & 0.271 & 0.259 & 0.208 & 0.243 & 0.236 & 0.237 & 0.18 & 0.235 & 0.251 \\ 
   &  & p=36 & 1.478 & 1.485 & 1.719 & 4.19 & 1.591 & 1.855 & 1.982 & 5.803 & 2.055 & 4.51 & 0.276 & 0.272 & 0.158 & -1.065 & 0.221 & 0.089 & 0.026 & -1.853 & -0.009 & -1.221 \\ 
  \midrule\multirow{9}[6]{*}{lsnr} & \multirow{3}[2]{*}{$\rho=0$} & p=12 & 1.246 & 1.253 & 1.271 & 1.293 & 1.275 & 1.282 & 1.282 & 1.302 & 1.284 & 1.238 & -0.029 & -0.035 & -0.05 & -0.068 & -0.053 & -0.059 & -0.059 & -0.076 & -0.061 & -0.022 \\ 
   &  & p=20 & 1.249 & 1.263 & 1.296 & 1.383 & 1.309 & 1.323 & 1.327 & 1.439 & 1.333 & 1.245 & -0.031 & -0.043 & -0.07 & -0.142 & -0.081 & -0.093 & -0.096 & -0.188 & -0.101 & -0.028 \\ 
   &  & p=36 & 1.249 & 1.263 & 1.535 & 3.961 & 1.341 & 1.587 & 1.626 & 5.344 & 1.75 & 3.57 & -0.032 & -0.043 & -0.268 & -2.282 & -0.108 & -0.311 & -0.342 & -3.423 & -0.445 & -1.956 \\ 
  \cmidrule{2-23} & \multirow{3}[2]{*}{$\rho=0.5$} & p=12 & 1.239 & 1.249 & 1.266 & 1.291 & 1.271 & 1.279 & 1.279 & 1.301 & 1.28 & 1.233 & -0.023 & -0.032 & -0.045 & -0.066 & -0.05 & -0.056 & -0.056 & -0.075 & -0.058 & -0.018 \\ 
   &  & p=20 & 1.244 & 1.254 & 1.293 & 1.378 & 1.302 & 1.32 & 1.317 & 1.438 & 1.33 & 1.239 & -0.027 & -0.036 & -0.068 & -0.137 & -0.075 & -0.09 & -0.087 & -0.188 & -0.098 & -0.023 \\ 
   &  & p=36 & 1.245 & 1.255 & 1.525 & 3.951 & 1.339 & 1.602 & 1.629 & 5.335 & 1.751 & 3.535 & -0.028 & -0.037 & -0.259 & -2.274 & -0.105 & -0.323 & -0.345 & -3.414 & -0.447 & -1.927 \\ 
  \cmidrule{2-23} & \multirow{3}[2]{*}{$\rho=0.9$} & p=12 & 1.196 & 1.208 & 1.227 & 1.246 & 1.23 & 1.236 & 1.233 & 1.258 & 1.235 & 1.196 & 0.012 & 0.002 & -0.014 & -0.029 & -0.017 & -0.021 & -0.019 & -0.04 & -0.021 & 0.012 \\ 
   &  & p=20 & 1.191 & 1.201 & 1.235 & 1.322 & 1.251 & 1.257 & 1.261 & 1.379 & 1.267 & 1.195 & 0.016 & 0.008 & -0.02 & -0.092 & -0.033 & -0.039 & -0.042 & -0.139 & -0.047 & 0.013 \\ 
   &  & p=36 & 1.19 & 1.199 & 1.481 & 3.824 & 1.276 & 1.503 & 1.544 & 5.153 & 1.684 & 3.479 & 0.017 & 0.01 & -0.223 & -2.165 & -0.054 & -0.242 & -0.276 & -3.262 & -0.392 & -1.879 \\ 
   \midrule 
 \multicolumn{1}{|c}{} & \multicolumn{1}{c}{} &       & \multicolumn{10}{c||}{Kullback-Leibler Discrepancy}                                    & \multicolumn{10}{c|}{Number of Variables} \\
\midrule\multirow{9}[6]{*}{hsnr} & \multirow{3}[2]{*}{$\rho=0$} & p=12 & 1.263 & 1.176 & 1.153 & 1.114 & 1.134 & 1.118 & 1.118 & 1.096 & 1.116 & 1.153 & 10.934 & 11.396 & 11.514 & 11.723 & 11.583 & 11.674 & 11.7 & 11.801 & 11.714 & 11.498 \\ 
   &  & p=20 & 1.996 & 2.26 & 2.454 & 2.888 & 2.706 & 2.799 & 2.842 & 2.995 & 2.849 & 2.824 & 10.343 & 12.741 & 14.282 & 16.657 & 15.742 & 16.109 & 16.194 & 17.584 & 16.276 & 15.555 \\ 
   &  & p=36 & 2.055 & 2.36 & 4.976 & 145.005 & 6.382 & 61.391 & 76.425 & 188.496 & 88.007 & 215.179 & 9.89 & 12.267 & 15.468 & 21.926 & 16.684 & 17.99 & 18.421 & 27.985 & 19.068 & 27.807 \\ 
  \cmidrule{2-23} & \multirow{3}[2]{*}{$\rho=0.5$} & p=12 & 0.274 & 0.191 & 0.175 & 0.143 & 0.167 & 0.154 & 0.15 & 0.13 & 0.147 & 0.173 & 11.198 & 11.627 & 11.722 & 11.863 & 11.746 & 11.817 & 11.833 & 11.917 & 11.847 & 11.705 \\ 
   &  & p=20 & 1.004 & 1.257 & 1.503 & 1.959 & 1.748 & 1.868 & 1.907 & 2.064 & 1.918 & 1.873 & 10.148 & 12.471 & 14.173 & 16.744 & 15.713 & 16.154 & 16.243 & 17.682 & 16.333 & 15.49 \\ 
   &  & p=36 & 1.044 & 1.321 & 3.479 & 139.29 & 11.947 & 61.21 & 73.707 & 186.652 & 82.713 & 213.044 & 9.798 & 12.135 & 15.332 & 21.703 & 16.633 & 17.914 & 18.202 & 27.923 & 18.811 & 27.581 \\ 
  \cmidrule{2-23} & \multirow{3}[2]{*}{$\rho=0.9$} & p=12 & -1.014 & -1.104 & -1.135 & -1.174 & -1.145 & -1.159 & -1.166 & -1.182 & -1.168 & -1.122 & 10.516 & 11.274 & 11.507 & 11.765 & 11.567 & 11.668 & 11.711 & 11.826 & 11.724 & 11.395 \\ 
   &  & p=20 & -0.578 & -0.322 & -0.093 & 0.491 & 0.243 & 0.366 & 0.407 & 0.659 & 0.428 & 0.301 & 8.848 & 10.931 & 12.863 & 15.68 & 14.306 & 14.768 & 14.932 & 16.976 & 15.082 & 13.411 \\ 
   &  & p=36 & -0.582 & -0.361 & 2.354 & 134.139 & 4.381 & 54.799 & 72.646 & 181.084 & 79.769 & 204.867 & 8.668 & 10.657 & 13.699 & 20.161 & 14.682 & 15.975 & 16.337 & 26.771 & 16.873 & 25.155 \\ 
  \midrule\multirow{9}[6]{*}{msnr} & \multirow{3}[2]{*}{$\rho=0$} & p=12 & 3.056 & 3.09 & 3.113 & 3.154 & 3.125 & 3.134 & 3.148 & 3.168 & 3.148 & 3.12 & 5.222 & 6.19 & 7.132 & 8.143 & 7.552 & 7.684 & 7.773 & 8.714 & 7.838 & 4.827 \\ 
   &  & p=20 & 3.17 & 3.268 & 3.431 & 3.923 & 3.634 & 3.705 & 3.697 & 4.291 & 3.734 & 3.415 & 4.711 & 5.931 & 7.574 & 9.891 & 8.533 & 8.777 & 8.756 & 11.759 & 8.951 & 4.678 \\ 
   &  & p=36 & 3.192 & 3.29 & 5.82 & 138.976 & 5.646 & 51.213 & 61.388 & 177.08 & 70.874 & 171.079 & 4.597 & 5.821 & 9.076 & 15.335 & 8.79 & 9.724 & 10.015 & 22.182 & 10.526 & 14.332 \\ 
  \cmidrule{2-23} & \multirow{3}[2]{*}{$\rho=0.5$} & p=12 & 2.093 & 2.149 & 2.167 & 2.206 & 2.179 & 2.192 & 2.203 & 2.23 & 2.203 & 2.165 & 4.884 & 6.038 & 6.996 & 8.072 & 7.442 & 7.643 & 7.683 & 8.697 & 7.803 & 4.688 \\ 
   &  & p=20 & 2.175 & 2.255 & 2.437 & 2.965 & 2.653 & 2.711 & 2.741 & 3.314 & 2.761 & 2.423 & 4.554 & 5.613 & 7.431 & 9.741 & 8.268 & 8.507 & 8.581 & 11.58 & 8.732 & 4.512 \\ 
   &  & p=36 & 2.2 & 2.283 & 4.511 & 136.01 & 7.669 & 50.208 & 59.465 & 177.14 & 69.19 & 168.09 & 4.554 & 5.521 & 9.042 & 15.303 & 8.534 & 9.593 & 9.807 & 22.337 & 10.415 & 14.124 \\ 
  \cmidrule{2-23} & \multirow{3}[2]{*}{$\rho=0.9$} & p=12 & 0.672 & 0.728 & 0.766 & 0.825 & 0.784 & 0.805 & 0.805 & 0.854 & 0.809 & 0.707 & 3.899 & 4.837 & 5.903 & 7.029 & 6.19 & 6.467 & 6.424 & 7.578 & 6.557 & 3.405 \\ 
   &  & p=20 & 0.724 & 0.79 & 0.949 & 1.462 & 1.133 & 1.205 & 1.218 & 1.757 & 1.243 & 0.842 & 3.569 & 4.306 & 5.785 & 8.104 & 6.646 & 6.782 & 6.782 & 9.679 & 6.967 & 3.199 \\ 
   &  & p=36 & 0.731 & 0.806 & 3.464 & 130.41 & 4.121 & 45.218 & 58.858 & 167.676 & 63.665 & 156.712 & 3.567 & 4.295 & 7.756 & 13.747 & 6.85 & 7.447 & 7.881 & 20.308 & 8.264 & 11.647 \\ 
  \midrule\multirow{9}[6]{*}{lsnr} & \multirow{3}[2]{*}{$\rho=0$} & p=12 & 4.312 & 4.336 & 4.392 & 4.457 & 4.404 & 4.426 & 4.426 & 4.484 & 4.429 & 4.282 & 1.82 & 2.17 & 2.856 & 3.621 & 3.149 & 3.269 & 3.153 & 3.974 & 3.222 & 0.913 \\ 
   &  & p=20 & 4.385 & 4.427 & 4.541 & 4.894 & 4.601 & 4.67 & 4.678 & 5.15 & 4.698 & 4.379 & 1.658 & 2.058 & 2.982 & 4.51 & 3.326 & 3.537 & 3.412 & 5.593 & 3.555 & 0.905 \\ 
   &  & p=36 & 4.403 & 4.444 & 6.989 & 136.329 & 5.64 & 48.271 & 53.915 & 168.706 & 63.972 & 143.251 & 1.64 & 2.042 & 5.426 & 10.95 & 3.584 & 4.328 & 4.32 & 16.973 & 4.853 & 7.587 \\ 
  \cmidrule{2-23} & \multirow{3}[2]{*}{$\rho=0.5$} & p=12 & 3.348 & 3.377 & 3.425 & 3.501 & 3.445 & 3.466 & 3.467 & 3.531 & 3.475 & 3.322 & 1.742 & 2.112 & 2.748 & 3.555 & 3.062 & 3.152 & 3.079 & 3.939 & 3.165 & 0.921 \\ 
   &  & p=20 & 3.407 & 3.442 & 3.568 & 3.927 & 3.629 & 3.701 & 3.697 & 4.188 & 3.751 & 3.4 & 1.661 & 2.001 & 2.968 & 4.396 & 3.286 & 3.516 & 3.365 & 5.601 & 3.564 & 0.94 \\ 
   &  & p=36 & 3.423 & 3.457 & 5.714 & 134.473 & 4.809 & 47.325 & 51.953 & 168.604 & 62.722 & 141.299 & 1.66 & 2.006 & 5.332 & 11.039 & 3.549 & 4.336 & 4.321 & 17.134 & 4.84 & 7.506 \\ 
  \cmidrule{2-23} & \multirow{3}[2]{*}{$\rho=0.9$} & p=12 & 1.98 & 2.01 & 2.06 & 2.113 & 2.071 & 2.09 & 2.081 & 2.149 & 2.088 & 1.961 & 1.636 & 1.907 & 2.44 & 3.017 & 2.734 & 2.789 & 2.669 & 3.377 & 2.748 & 0.974 \\ 
   &  & p=20 & 2.046 & 2.075 & 2.18 & 2.551 & 2.255 & 2.29 & 2.302 & 2.774 & 2.327 & 2.044 & 1.577 & 1.845 & 2.603 & 3.84 & 2.941 & 3.01 & 2.939 & 4.889 & 3.06 & 1.009 \\ 
   &  & p=36 & 2.061 & 2.088 & 4.454 & 129.968 & 5.003 & 44.847 & 49.173 & 162.858 & 62.122 & 139.704 & 1.577 & 1.854 & 5.074 & 10.542 & 3.174 & 3.731 & 3.765 & 16.089 & 4.243 & 7.464 \\ 
   \bottomrule 
\end{tabular}
}
\end{table}

% latex table generated in R 4.0.2 by xtable 1.8-4 package
% Fri Jun 26 10:43:47 2020
\begin{table}[ht]
\centering
\caption{Fixed X, Dense-Ex1, n=200} 
\scalebox{0.52}{
\begin{tabular}{|c|c|c|cc|cc|cc|ccc|c||cc|cc|cc|ccc|c|}
  \toprule 
 \multicolumn{1}{|c}{} & \multicolumn{1}{c}{} &       & RAICc & AICc  & RC$_p$ & C$_p$ & 10F CV & LOO CV & S$_p$ & $\widehat{\text{C}}_p$ & GCV   & BIC   & RAICc & AICc  & RC$_p$ & C$_p$ & 10F CV & LOO CV & S$_p$ & $\widehat{\text{C}}_p$ & GCV   & BIC  \\
 \cmidrule{4-23}\multicolumn{1}{|c}{} & \multicolumn{1}{c}{} &       & \multicolumn{10}{c||}{RMSE}                                                   & \multicolumn{10}{c|}{Relative Risk}       \\
 \midrule\multirow{9}[6]{*}{hsnr} & \multirow{3}[2]{*}{$\rho=0$} & p=12 & 0.163 & 0.163 & 0.163 & 0.163 & 0.163 & 0.163 & 0.163 & 0.163 & 0.163 & 0.163 & 0.007 & 0.007 & 0.007 & 0.007 & 0.007 & 0.007 & 0.007 & 0.007 & 0.007 & 0.007 \\ 
   &  & p=100 & 0.302 & 0.3 & 0.299 & 0.301 & 0.301 & 0.3 & 0.3 & 0.31 & 0.3 & 0.338 & 0.026 & 0.025 & 0.025 & 0.026 & 0.026 & 0.025 & 0.025 & 0.029 & 0.025 & 0.032 \\ 
   &  & p=196 & 0.302 & 0.3 & 0.348 & 0.423 & 0.301 & 0.3 & 0.3 & 0.499 & 0.3 & 0.429 & 0.026 & 0.025 & 0.052 & 1.543 & 0.026 & 0.025 & 0.028 & 2.972 & 0.028 & 1.976 \\ 
  \cmidrule{2-23} & \multirow{3}[2]{*}{$\rho=0.5$} & p=12 & 0.101 & 0.101 & 0.101 & 0.101 & 0.101 & 0.101 & 0.101 & 0.101 & 0.101 & 0.101 & 0.007 & 0.007 & 0.007 & 0.007 & 0.007 & 0.007 & 0.007 & 0.007 & 0.007 & 0.007 \\ 
   &  & p=100 & 0.185 & 0.183 & 0.183 & 0.184 & 0.184 & 0.183 & 0.183 & 0.189 & 0.183 & 0.207 & 0.025 & 0.025 & 0.025 & 0.026 & 0.025 & 0.025 & 0.025 & 0.028 & 0.025 & 0.032 \\ 
   &  & p=196 & 0.185 & 0.183 & 0.214 & 0.259 & 0.184 & 0.183 & 0.184 & 0.305 & 0.184 & 0.263 & 0.025 & 0.025 & 0.052 & 1.566 & 0.025 & 0.025 & 0.028 & 2.969 & 0.028 & 1.984 \\ 
  \cmidrule{2-23} & \multirow{3}[2]{*}{$\rho=0.9$} & p=12 & 0.053 & 0.053 & 0.053 & 0.053 & 0.053 & 0.053 & 0.053 & 0.053 & 0.053 & 0.053 & 0.007 & 0.007 & 0.007 & 0.007 & 0.007 & 0.007 & 0.007 & 0.007 & 0.007 & 0.007 \\ 
   &  & p=100 & 0.094 & 0.094 & 0.093 & 0.094 & 0.094 & 0.094 & 0.094 & 0.096 & 0.094 & 0.107 & 0.024 & 0.024 & 0.024 & 0.025 & 0.024 & 0.024 & 0.024 & 0.027 & 0.024 & 0.032 \\ 
   &  & p=196 & 0.094 & 0.094 & 0.11 & 0.135 & 0.094 & 0.094 & 0.094 & 0.159 & 0.094 & 0.135 & 0.024 & 0.024 & 0.051 & 1.616 & 0.024 & 0.024 & 0.03 & 3.016 & 0.034 & 1.879 \\ 
  \midrule\multirow{9}[6]{*}{msnr} & \multirow{3}[2]{*}{$\rho=0$} & p=12 & 0.498 & 0.495 & 0.494 & 0.493 & 0.498 & 0.494 & 0.493 & 0.492 & 0.493 & 0.612 & 0.068 & 0.067 & 0.067 & 0.066 & 0.068 & 0.067 & 0.066 & 0.066 & 0.066 & 0.108 \\ 
   &  & p=100 & 0.734 & 0.733 & 0.733 & 0.743 & 0.74 & 0.734 & 0.736 & 0.756 & 0.735 & 0.868 & 0.146 & 0.145 & 0.145 & 0.152 & 0.149 & 0.147 & 0.147 & 0.161 & 0.147 & 0.216 \\ 
   &  & p=196 & 0.734 & 0.733 & 0.908 & 1.137 & 0.74 & 0.734 & 0.737 & 1.342 & 0.737 & 1.104 & 0.146 & 0.145 & 0.385 & 13.061 & 0.149 & 0.147 & 0.166 & 23.85 & 0.167 & 13.524 \\ 
  \cmidrule{2-23} & \multirow{3}[2]{*}{$\rho=0.5$} & p=12 & 0.303 & 0.302 & 0.302 & 0.301 & 0.304 & 0.301 & 0.301 & 0.3 & 0.301 & 0.379 & 0.066 & 0.065 & 0.065 & 0.064 & 0.066 & 0.065 & 0.064 & 0.064 & 0.064 & 0.113 \\ 
   &  & p=100 & 0.446 & 0.446 & 0.446 & 0.453 & 0.451 & 0.449 & 0.448 & 0.461 & 0.449 & 0.538 & 0.142 & 0.143 & 0.143 & 0.15 & 0.147 & 0.145 & 0.145 & 0.159 & 0.145 & 0.224 \\ 
   &  & p=196 & 0.446 & 0.446 & 0.555 & 0.698 & 0.451 & 0.449 & 0.449 & 0.818 & 0.45 & 0.681 & 0.142 & 0.143 & 0.381 & 13.256 & 0.147 & 0.145 & 0.164 & 23.429 & 0.165 & 13.374 \\ 
  \cmidrule{2-23} & \multirow{3}[2]{*}{$\rho=0.9$} & p=12 & 0.16 & 0.159 & 0.159 & 0.158 & 0.161 & 0.158 & 0.158 & 0.157 & 0.158 & 0.233 & 0.069 & 0.068 & 0.067 & 0.067 & 0.069 & 0.067 & 0.067 & 0.066 & 0.067 & 0.169 \\ 
   &  & p=100 & 0.223 & 0.223 & 0.223 & 0.227 & 0.225 & 0.224 & 0.224 & 0.231 & 0.224 & 0.283 & 0.13 & 0.13 & 0.131 & 0.137 & 0.134 & 0.133 & 0.133 & 0.146 & 0.133 & 0.239 \\ 
   &  & p=196 & 0.223 & 0.223 & 0.286 & 0.361 & 0.225 & 0.224 & 0.225 & 0.423 & 0.226 & 0.351 & 0.13 & 0.13 & 0.382 & 13.63 & 0.134 & 0.133 & 0.181 & 24.024 & 0.218 & 12.843 \\ 
  \midrule\multirow{9}[6]{*}{lsnr} & \multirow{3}[2]{*}{$\rho=0$} & p=12 & 1.2 & 1.193 & 1.187 & 1.178 & 1.191 & 1.181 & 1.183 & 1.175 & 1.183 & 1.448 & 0.411 & 0.406 & 0.4 & 0.393 & 0.406 & 0.396 & 0.397 & 0.39 & 0.397 & 0.657 \\ 
   &  & p=100 & 1.361 & 1.365 & 1.366 & 1.387 & 1.386 & 1.379 & 1.378 & 1.403 & 1.378 & 1.543 & 0.508 & 0.51 & 0.509 & 0.532 & 0.531 & 0.521 & 0.521 & 0.553 & 0.521 & 0.71 \\ 
   &  & p=196 & 1.361 & 1.365 & 1.802 & 2.344 & 1.386 & 1.379 & 1.378 & 2.763 & 1.382 & 2.011 & 0.508 & 0.51 & 1.735 & 64.807 & 0.531 & 0.521 & 0.521 & 113.551 & 0.625 & 53.57 \\ 
  \cmidrule{2-23} & \multirow{3}[2]{*}{$\rho=0.5$} & p=12 & 0.751 & 0.746 & 0.742 & 0.737 & 0.745 & 0.741 & 0.74 & 0.735 & 0.739 & 0.882 & 0.423 & 0.415 & 0.409 & 0.4 & 0.414 & 0.407 & 0.405 & 0.398 & 0.405 & 0.627 \\ 
   &  & p=100 & 0.827 & 0.83 & 0.831 & 0.844 & 0.838 & 0.838 & 0.839 & 0.852 & 0.839 & 0.924 & 0.499 & 0.501 & 0.501 & 0.527 & 0.514 & 0.512 & 0.513 & 0.542 & 0.513 & 0.665 \\ 
   &  & p=196 & 0.827 & 0.83 & 1.097 & 1.429 & 0.838 & 0.838 & 0.839 & 1.676 & 0.841 & 1.211 & 0.499 & 0.501 & 1.715 & 66.071 & 0.514 & 0.512 & 0.513 & 110.244 & 0.613 & 52.882 \\ 
  \cmidrule{2-23} & \multirow{3}[2]{*}{$\rho=0.9$} & p=12 & 0.375 & 0.375 & 0.374 & 0.374 & 0.375 & 0.373 & 0.374 & 0.373 & 0.374 & 0.418 & 0.397 & 0.394 & 0.392 & 0.39 & 0.394 & 0.39 & 0.391 & 0.388 & 0.391 & 0.491 \\ 
   &  & p=100 & 0.39 & 0.394 & 0.395 & 0.401 & 0.397 & 0.397 & 0.398 & 0.407 & 0.399 & 0.419 & 0.412 & 0.421 & 0.423 & 0.44 & 0.428 & 0.431 & 0.43 & 0.465 & 0.433 & 0.461 \\ 
   &  & p=196 & 0.39 & 0.394 & 0.55 & 0.726 & 0.397 & 0.397 & 0.398 & 0.851 & 0.401 & 0.576 & 0.412 & 0.421 & 1.684 & 67.917 & 0.428 & 0.431 & 0.43 & 114.267 & 0.676 & 52.475 \\ 
   \midrule 
 \multicolumn{1}{|c}{} & \multicolumn{1}{c}{} &       & \multicolumn{10}{c||}{Relative Test Error}                                    & \multicolumn{10}{c|}{Proportion of Variance Explained} \\
\midrule\multirow{9}[6]{*}{hsnr} & \multirow{3}[2]{*}{$\rho=0$} & p=12 & 1.064 & 1.064 & 1.064 & 1.064 & 1.064 & 1.064 & 1.064 & 1.064 & 1.064 & 1.064 & 0.891 & 0.891 & 0.891 & 0.891 & 0.891 & 0.891 & 0.891 & 0.891 & 0.891 & 0.891 \\ 
   &  & p=100 & 1.236 & 1.232 & 1.231 & 1.239 & 1.236 & 1.234 & 1.234 & 1.264 & 1.234 & 1.3 & 0.879 & 0.88 & 0.88 & 0.879 & 0.879 & 0.879 & 0.88 & 0.877 & 0.88 & 0.873 \\ 
   &  & p=196 & 1.236 & 1.232 & 1.479 & 15.263 & 1.236 & 1.234 & 1.256 & 28.469 & 1.257 & 19.259 & 0.879 & 0.88 & 0.856 & -0.49 & 0.879 & 0.879 & 0.877 & -1.78 & 0.877 & -0.88 \\ 
  \cmidrule{2-23} & \multirow{3}[2]{*}{$\rho=0.5$} & p=12 & 1.064 & 1.064 & 1.064 & 1.064 & 1.064 & 1.064 & 1.064 & 1.064 & 1.064 & 1.064 & 0.892 & 0.892 & 0.892 & 0.892 & 0.892 & 0.892 & 0.892 & 0.892 & 0.892 & 0.892 \\ 
   &  & p=100 & 1.234 & 1.231 & 1.23 & 1.238 & 1.235 & 1.233 & 1.233 & 1.261 & 1.233 & 1.298 & 0.88 & 0.88 & 0.88 & 0.879 & 0.88 & 0.88 & 0.88 & 0.877 & 0.88 & 0.874 \\ 
   &  & p=196 & 1.234 & 1.231 & 1.477 & 15.501 & 1.235 & 1.233 & 1.255 & 28.496 & 1.255 & 19.374 & 0.88 & 0.88 & 0.856 & -0.511 & 0.88 & 0.88 & 0.878 & -1.777 & 0.878 & -0.888 \\ 
  \cmidrule{2-23} & \multirow{3}[2]{*}{$\rho=0.9$} & p=12 & 1.064 & 1.064 & 1.064 & 1.064 & 1.064 & 1.064 & 1.064 & 1.064 & 1.064 & 1.064 & 0.89 & 0.89 & 0.89 & 0.89 & 0.89 & 0.89 & 0.89 & 0.89 & 0.89 & 0.89 \\ 
   &  & p=100 & 1.215 & 1.213 & 1.213 & 1.22 & 1.216 & 1.216 & 1.215 & 1.239 & 1.215 & 1.287 & 0.878 & 0.878 & 0.878 & 0.877 & 0.878 & 0.878 & 0.878 & 0.875 & 0.878 & 0.87 \\ 
   &  & p=196 & 1.215 & 1.213 & 1.457 & 15.44 & 1.216 & 1.216 & 1.266 & 27.954 & 1.304 & 17.789 & 0.878 & 0.878 & 0.853 & -0.554 & 0.878 & 0.878 & 0.873 & -1.813 & 0.869 & -0.79 \\ 
  \midrule\multirow{9}[6]{*}{msnr} & \multirow{3}[2]{*}{$\rho=0$} & p=12 & 1.07 & 1.069 & 1.069 & 1.069 & 1.07 & 1.069 & 1.069 & 1.068 & 1.069 & 1.111 & 0.473 & 0.474 & 0.474 & 0.474 & 0.473 & 0.474 & 0.474 & 0.474 & 0.474 & 0.453 \\ 
   &  & p=100 & 1.158 & 1.158 & 1.158 & 1.165 & 1.162 & 1.159 & 1.159 & 1.176 & 1.159 & 1.235 & 0.445 & 0.445 & 0.445 & 0.442 & 0.443 & 0.445 & 0.445 & 0.437 & 0.445 & 0.408 \\ 
   &  & p=196 & 1.158 & 1.158 & 1.418 & 15.201 & 1.162 & 1.159 & 1.181 & 26.933 & 1.182 & 15.705 & 0.445 & 0.445 & 0.321 & -6.283 & 0.443 & 0.445 & 0.434 & -11.903 & 0.434 & -6.524 \\ 
  \cmidrule{2-23} & \multirow{3}[2]{*}{$\rho=0.5$} & p=12 & 1.068 & 1.068 & 1.067 & 1.067 & 1.068 & 1.067 & 1.067 & 1.067 & 1.067 & 1.118 & 0.476 & 0.476 & 0.476 & 0.477 & 0.476 & 0.476 & 0.477 & 0.477 & 0.477 & 0.452 \\ 
   &  & p=100 & 1.155 & 1.155 & 1.156 & 1.163 & 1.16 & 1.158 & 1.158 & 1.173 & 1.158 & 1.244 & 0.447 & 0.447 & 0.447 & 0.443 & 0.445 & 0.446 & 0.446 & 0.438 & 0.446 & 0.405 \\ 
   &  & p=196 & 1.155 & 1.155 & 1.415 & 15.442 & 1.16 & 1.158 & 1.179 & 26.525 & 1.18 & 15.571 & 0.447 & 0.447 & 0.323 & -6.39 & 0.445 & 0.446 & 0.436 & -11.695 & 0.435 & -6.452 \\ 
  \cmidrule{2-23} & \multirow{3}[2]{*}{$\rho=0.9$} & p=12 & 1.07 & 1.069 & 1.069 & 1.068 & 1.071 & 1.069 & 1.068 & 1.068 & 1.068 & 1.173 & 0.471 & 0.472 & 0.472 & 0.472 & 0.471 & 0.472 & 0.472 & 0.473 & 0.472 & 0.421 \\ 
   &  & p=100 & 1.137 & 1.137 & 1.137 & 1.145 & 1.141 & 1.14 & 1.14 & 1.154 & 1.14 & 1.251 & 0.446 & 0.446 & 0.446 & 0.442 & 0.444 & 0.444 & 0.444 & 0.438 & 0.444 & 0.39 \\ 
   &  & p=196 & 1.137 & 1.137 & 1.402 & 15.33 & 1.141 & 1.14 & 1.191 & 26.257 & 1.229 & 14.502 & 0.446 & 0.446 & 0.317 & -6.473 & 0.444 & 0.444 & 0.42 & -11.8 & 0.401 & -6.07 \\ 
  \midrule\multirow{9}[6]{*}{lsnr} & \multirow{3}[2]{*}{$\rho=0$} & p=12 & 1.085 & 1.084 & 1.083 & 1.081 & 1.084 & 1.082 & 1.082 & 1.08 & 1.082 & 1.136 & 0.101 & 0.102 & 0.103 & 0.104 & 0.102 & 0.103 & 0.103 & 0.104 & 0.103 & 0.059 \\ 
   &  & p=100 & 1.111 & 1.111 & 1.111 & 1.116 & 1.115 & 1.113 & 1.113 & 1.12 & 1.113 & 1.154 & 0.088 & 0.088 & 0.088 & 0.084 & 0.084 & 0.085 & 0.086 & 0.08 & 0.086 & 0.052 \\ 
   &  & p=196 & 1.111 & 1.111 & 1.377 & 15.093 & 1.115 & 1.113 & 1.113 & 25.693 & 1.136 & 12.649 & 0.088 & 0.088 & -0.131 & -11.397 & 0.084 & 0.085 & 0.086 & -20.104 & 0.067 & -9.39 \\ 
  \cmidrule{2-23} & \multirow{3}[2]{*}{$\rho=0.5$} & p=12 & 1.088 & 1.086 & 1.085 & 1.083 & 1.086 & 1.085 & 1.084 & 1.083 & 1.084 & 1.13 & 0.099 & 0.101 & 0.102 & 0.103 & 0.101 & 0.102 & 0.102 & 0.103 & 0.102 & 0.064 \\ 
   &  & p=100 & 1.109 & 1.109 & 1.109 & 1.115 & 1.112 & 1.112 & 1.112 & 1.118 & 1.112 & 1.145 & 0.09 & 0.089 & 0.089 & 0.085 & 0.087 & 0.087 & 0.087 & 0.082 & 0.087 & 0.06 \\ 
   &  & p=196 & 1.109 & 1.109 & 1.374 & 15.396 & 1.112 & 1.112 & 1.112 & 25.021 & 1.134 & 12.523 & 0.09 & 0.089 & -0.128 & -11.642 & 0.087 & 0.087 & 0.087 & -19.545 & 0.069 & -9.282 \\ 
  \cmidrule{2-23} & \multirow{3}[2]{*}{$\rho=0.9$} & p=12 & 1.081 & 1.081 & 1.08 & 1.08 & 1.081 & 1.08 & 1.08 & 1.079 & 1.08 & 1.101 & 0.103 & 0.103 & 0.103 & 0.104 & 0.103 & 0.104 & 0.104 & 0.104 & 0.104 & 0.087 \\ 
   &  & p=100 & 1.087 & 1.088 & 1.089 & 1.092 & 1.09 & 1.091 & 1.09 & 1.098 & 1.091 & 1.097 & 0.102 & 0.101 & 0.1 & 0.097 & 0.099 & 0.099 & 0.099 & 0.093 & 0.099 & 0.094 \\ 
   &  & p=196 & 1.087 & 1.088 & 1.354 & 15.28 & 1.09 & 1.091 & 1.09 & 25.026 & 1.142 & 12.033 & 0.102 & 0.101 & -0.119 & -11.626 & 0.099 & 0.099 & 0.099 & -19.678 & 0.056 & -8.943 \\ 
   \midrule 
 \multicolumn{1}{|c}{} & \multicolumn{1}{c}{} &       & \multicolumn{10}{c||}{Kullback-Leibler Discrepancy}                                    & \multicolumn{10}{c|}{Number of Variables} \\
\midrule\multirow{9}[6]{*}{hsnr} & \multirow{3}[2]{*}{$\rho=0$} & p=12 & 0.314 & 0.314 & 0.314 & 0.314 & 0.314 & 0.314 & 0.314 & 0.314 & 0.314 & 0.314 & 12 & 12 & 12 & 12 & 12 & 12 & 12 & 12 & 12 & 12 \\ 
   &  & p=100 & 0.512 & 0.515 & 0.516 & 0.536 & 0.525 & 0.527 & 0.527 & 0.58 & 0.527 & 0.534 & 25.456 & 26.583 & 26.972 & 29.167 & 28.123 & 28.603 & 28.461 & 31.979 & 28.492 & 20.826 \\ 
   &  & p=196 & 0.512 & 0.515 & 0.928 & 104.955 & 0.525 & 0.527 & 1.307 & 151.701 & 1.772 & 132.908 & 25.456 & 26.583 & 41.353 & 74.405 & 28.123 & 28.603 & 28.625 & 111.014 & 28.658 & 65.115 \\ 
  \cmidrule{2-23} & \multirow{3}[2]{*}{$\rho=0.5$} & p=12 & -0.645 & -0.645 & -0.645 & -0.645 & -0.645 & -0.645 & -0.645 & -0.645 & -0.645 & -0.645 & 12 & 12 & 12 & 12 & 12 & 12 & 12 & 12 & 12 & 12 \\ 
   &  & p=100 & -0.467 & -0.465 & -0.464 & -0.443 & -0.454 & -0.452 & -0.451 & -0.402 & -0.451 & -0.446 & 25.109 & 26.232 & 26.639 & 28.858 & 27.82 & 28.27 & 28.251 & 31.622 & 28.346 & 20.527 \\ 
   &  & p=196 & -0.467 & -0.465 & -0.056 & 103.71 & -0.454 & -0.452 & 0.534 & 150.944 & 0.535 & 132.072 & 25.109 & 26.232 & 41.1 & 74.198 & 27.82 & 28.27 & 28.416 & 110.833 & 28.511 & 65.089 \\ 
  \cmidrule{2-23} & \multirow{3}[2]{*}{$\rho=0.9$} & p=12 & -1.953 & -1.953 & -1.953 & -1.953 & -1.953 & -1.953 & -1.953 & -1.953 & -1.953 & -1.953 & 12 & 12 & 12 & 12 & 12 & 12 & 12 & 12 & 12 & 12 \\ 
   &  & p=100 & -1.756 & -1.753 & -1.752 & -1.734 & -1.744 & -1.741 & -1.742 & -1.698 & -1.742 & -1.729 & 22.477 & 23.642 & 24.027 & 25.895 & 24.897 & 25.347 & 25.357 & 28.445 & 25.357 & 17.6 \\ 
   &  & p=196 & -1.756 & -1.753 & -1.365 & 101.575 & -1.744 & -1.741 & 2.006 & 147.281 & 4.07 & 126.559 & 22.477 & 23.642 & 39.012 & 72.64 & 24.897 & 25.347 & 25.692 & 107.566 & 25.865 & 59.17 \\ 
  \midrule\multirow{9}[6]{*}{msnr} & \multirow{3}[2]{*}{$\rho=0$} & p=12 & 2.46 & 2.459 & 2.459 & 2.458 & 2.46 & 2.459 & 2.458 & 2.458 & 2.458 & 2.493 & 11.639 & 11.676 & 11.681 & 11.699 & 11.644 & 11.689 & 11.698 & 11.714 & 11.698 & 10.297 \\ 
   &  & p=100 & 2.575 & 2.577 & 2.579 & 2.593 & 2.585 & 2.583 & 2.584 & 2.612 & 2.584 & 2.61 & 15.105 & 15.882 & 16.336 & 17.808 & 16.918 & 17.031 & 17.137 & 19.29 & 17.148 & 9.943 \\ 
   &  & p=196 & 2.575 & 2.577 & 3.004 & 107.026 & 2.585 & 2.583 & 3.364 & 151.008 & 3.829 & 123.07 & 15.105 & 15.882 & 33.264 & 66.853 & 16.918 & 17.031 & 17.304 & 99.484 & 17.323 & 47.07 \\ 
  \cmidrule{2-23} & \multirow{3}[2]{*}{$\rho=0.5$} & p=12 & 1.499 & 1.498 & 1.498 & 1.498 & 1.499 & 1.498 & 1.498 & 1.497 & 1.498 & 1.536 & 11.822 & 11.844 & 11.851 & 11.867 & 11.823 & 11.859 & 11.867 & 11.878 & 11.867 & 10.42 \\ 
   &  & p=100 & 1.594 & 1.598 & 1.599 & 1.613 & 1.605 & 1.605 & 1.604 & 1.631 & 1.605 & 1.637 & 15 & 15.688 & 16.163 & 17.573 & 16.675 & 16.818 & 16.792 & 19.044 & 16.858 & 9.532 \\ 
   &  & p=196 & 1.594 & 1.598 & 2.02 & 105.783 & 1.605 & 1.605 & 2.59 & 149.546 & 2.591 & 121.753 & 15 & 15.688 & 32.903 & 66.697 & 16.675 & 16.818 & 16.945 & 98.48 & 17.032 & 46.585 \\ 
  \cmidrule{2-23} & \multirow{3}[2]{*}{$\rho=0.9$} & p=12 & 0.193 & 0.192 & 0.191 & 0.191 & 0.193 & 0.191 & 0.191 & 0.19 & 0.191 & 0.259 & 11.622 & 11.675 & 11.698 & 11.729 & 11.61 & 11.714 & 11.719 & 11.755 & 11.721 & 8.005 \\ 
   &  & p=100 & 0.31 & 0.312 & 0.313 & 0.325 & 0.317 & 0.318 & 0.318 & 0.34 & 0.318 & 0.358 & 12.785 & 13.481 & 13.812 & 15.033 & 14.132 & 14.413 & 14.449 & 16.392 & 14.498 & 5.797 \\ 
   &  & p=196 & 0.31 & 0.312 & 0.716 & 103.458 & 0.317 & 0.318 & 4.066 & 146.045 & 6.13 & 115.525 & 12.785 & 13.481 & 30.947 & 64.487 & 14.132 & 14.413 & 14.81 & 95.741 & 15.034 & 40.007 \\ 
  \midrule\multirow{9}[6]{*}{lsnr} & \multirow{3}[2]{*}{$\rho=0$} & p=12 & 4.076 & 4.075 & 4.074 & 4.074 & 4.075 & 4.074 & 4.074 & 4.073 & 4.074 & 4.101 & 7.688 & 7.875 & 7.998 & 8.268 & 7.899 & 8.178 & 8.141 & 8.364 & 8.149 & 2.798 \\ 
   &  & p=100 & 4.135 & 4.138 & 4.139 & 4.147 & 4.144 & 4.143 & 4.143 & 4.155 & 4.143 & 4.151 & 7.798 & 8.374 & 8.689 & 9.799 & 8.993 & 9.288 & 9.322 & 10.5 & 9.353 & 2.511 \\ 
   &  & p=196 & 4.135 & 4.138 & 4.57 & 108.351 & 4.144 & 4.143 & 4.143 & 150.013 & 5.388 & 112.359 & 7.798 & 8.374 & 27.188 & 60.945 & 8.993 & 9.288 & 9.322 & 90.109 & 9.539 & 32.674 \\ 
  \cmidrule{2-23} & \multirow{3}[2]{*}{$\rho=0.5$} & p=12 & 3.118 & 3.117 & 3.117 & 3.116 & 3.117 & 3.117 & 3.116 & 3.116 & 3.116 & 3.139 & 7.411 & 7.651 & 7.805 & 8.1 & 7.698 & 7.934 & 7.97 & 8.189 & 7.987 & 2.809 \\ 
   &  & p=100 & 3.156 & 3.158 & 3.16 & 3.169 & 3.163 & 3.164 & 3.164 & 3.174 & 3.164 & 3.168 & 7.52 & 7.966 & 8.471 & 9.438 & 8.629 & 8.939 & 8.938 & 10.038 & 8.941 & 2.638 \\ 
   &  & p=196 & 3.156 & 3.158 & 3.586 & 107.338 & 3.163 & 3.164 & 3.164 & 148.218 & 4.149 & 110.681 & 7.52 & 7.966 & 26.972 & 60.775 & 8.629 & 8.939 & 8.938 & 88.293 & 9.126 & 32.401 \\ 
  \cmidrule{2-23} & \multirow{3}[2]{*}{$\rho=0.9$} & p=12 & 1.803 & 1.803 & 1.803 & 1.803 & 1.803 & 1.803 & 1.803 & 1.803 & 1.803 & 1.814 & 6.048 & 6.289 & 6.466 & 6.665 & 6.217 & 6.54 & 6.594 & 6.826 & 6.612 & 2.079 \\ 
   &  & p=100 & 1.868 & 1.871 & 1.873 & 1.878 & 1.874 & 1.875 & 1.875 & 1.886 & 1.875 & 1.872 & 5.55 & 5.981 & 6.31 & 6.933 & 6.324 & 6.497 & 6.51 & 7.472 & 6.581 & 2.004 \\ 
   &  & p=196 & 1.868 & 1.871 & 2.282 & 105.013 & 1.874 & 1.875 & 1.875 & 144.855 & 5.624 & 107.671 & 5.55 & 5.981 & 25.737 & 59.467 & 6.324 & 6.497 & 6.51 & 84.918 & 6.956 & 30.695 \\ 
   \bottomrule 
\end{tabular}
}
\end{table}

% latex table generated in R 4.0.2 by xtable 1.8-4 package
% Fri Jun 26 10:43:52 2020
\begin{table}[ht]
\centering
\caption{Random X, Dense-Ex1, n=1000} 
\scalebox{0.5}{
\begin{tabular}{|c|c|c|cc|cc|cc|ccc|c||cc|cc|cc|ccc|c|}
  \toprule 
 \multicolumn{1}{|c}{} & \multicolumn{1}{c}{} &       & RAICc & AICc  & RC$_p$ & C$_p$ & 10F CV & LOO CV & S$_p$ & $\widehat{\text{C}}_p$ & GCV   & BIC   & RAICc & AICc  & RC$_p$ & C$_p$ & 10F CV & LOO CV & S$_p$ & $\widehat{\text{C}}_p$ & GCV   & BIC  \\
 \cmidrule{4-23}\multicolumn{1}{|c}{} & \multicolumn{1}{c}{} &       & \multicolumn{10}{c||}{RMSE}                                                   & \multicolumn{10}{c|}{Relative Risk}       \\
 \midrule\multirow{9}[6]{*}{hsnr} & \multirow{3}[2]{*}{$\rho=0$} & p=12 & 0.074 & 0.074 & 0.074 & 0.074 & 0.074 & 0.074 & 0.074 & 0.074 & 0.074 & 0.074 & 0.001 & 0.001 & 0.001 & 0.001 & 0.001 & 0.001 & 0.001 & 0.001 & 0.001 & 0.001 \\ 
   &  & p=500 & 0.155 & 0.155 & 0.155 & 0.155 & 0.155 & 0.155 & 0.155 & 0.156 & 0.155 & 0.18 & 0.006 & 0.006 & 0.006 & 0.006 & 0.006 & 0.006 & 0.006 & 0.006 & 0.006 & 0.008 \\ 
   &  & p=996 & 0.155 & 0.155 & 0.229 & 0.32 & 0.155 & 0.155 & 0.155 & 0.379 & 0.155 & 0.25 & 0.006 & 0.006 & 0.03 & 6.89 & 0.006 & 0.006 & 0.006 & 12.846 & 0.006 & 4.978 \\ 
  \cmidrule{2-23} & \multirow{3}[2]{*}{$\rho=0.5$} & p=12 & 0.046 & 0.046 & 0.046 & 0.046 & 0.046 & 0.046 & 0.046 & 0.046 & 0.046 & 0.046 & 0.001 & 0.001 & 0.001 & 0.001 & 0.001 & 0.001 & 0.001 & 0.001 & 0.001 & 0.001 \\ 
   &  & p=500 & 0.095 & 0.095 & 0.095 & 0.095 & 0.095 & 0.095 & 0.095 & 0.095 & 0.095 & 0.11 & 0.006 & 0.006 & 0.006 & 0.006 & 0.006 & 0.006 & 0.006 & 0.006 & 0.006 & 0.008 \\ 
   &  & p=996 & 0.095 & 0.095 & 0.14 & 0.196 & 0.095 & 0.095 & 0.095 & 0.234 & 0.095 & 0.153 & 0.006 & 0.006 & 0.03 & 6.845 & 0.006 & 0.006 & 0.006 & 12.315 & 0.006 & 4.965 \\ 
  \cmidrule{2-23} & \multirow{3}[2]{*}{$\rho=0.9$} & p=12 & 0.024 & 0.024 & 0.024 & 0.024 & 0.024 & 0.024 & 0.024 & 0.024 & 0.024 & 0.024 & 0.001 & 0.001 & 0.001 & 0.001 & 0.001 & 0.001 & 0.001 & 0.001 & 0.001 & 0.001 \\ 
   &  & p=500 & 0.048 & 0.048 & 0.048 & 0.048 & 0.048 & 0.048 & 0.048 & 0.048 & 0.048 & 0.056 & 0.005 & 0.005 & 0.005 & 0.005 & 0.005 & 0.005 & 0.005 & 0.005 & 0.005 & 0.007 \\ 
   &  & p=996 & 0.048 & 0.048 & 0.072 & 0.101 & 0.048 & 0.048 & 0.048 & 0.12 & 0.048 & 0.078 & 0.005 & 0.005 & 0.03 & 6.678 & 0.005 & 0.005 & 0.005 & 12.272 & 0.005 & 4.708 \\ 
  \midrule\multirow{9}[6]{*}{msnr} & \multirow{3}[2]{*}{$\rho=0$} & p=12 & 0.216 & 0.216 & 0.216 & 0.216 & 0.216 & 0.216 & 0.216 & 0.216 & 0.216 & 0.218 & 0.012 & 0.012 & 0.012 & 0.012 & 0.012 & 0.012 & 0.012 & 0.012 & 0.012 & 0.012 \\ 
   &  & p=500 & 0.396 & 0.396 & 0.396 & 0.397 & 0.398 & 0.397 & 0.397 & 0.397 & 0.397 & 0.476 & 0.037 & 0.037 & 0.037 & 0.037 & 0.037 & 0.037 & 0.037 & 0.037 & 0.037 & 0.053 \\ 
   &  & p=996 & 0.396 & 0.396 & 0.628 & 0.893 & 0.398 & 0.397 & 0.397 & 1.066 & 0.397 & 0.657 & 0.037 & 0.037 & 0.246 & 58.553 & 0.037 & 0.037 & 0.037 & 108.057 & 0.037 & 36.984 \\ 
  \cmidrule{2-23} & \multirow{3}[2]{*}{$\rho=0.5$} & p=12 & 0.134 & 0.134 & 0.134 & 0.134 & 0.134 & 0.134 & 0.134 & 0.134 & 0.134 & 0.134 & 0.012 & 0.012 & 0.012 & 0.012 & 0.012 & 0.012 & 0.012 & 0.012 & 0.012 & 0.012 \\ 
   &  & p=500 & 0.243 & 0.242 & 0.242 & 0.243 & 0.243 & 0.243 & 0.242 & 0.243 & 0.242 & 0.29 & 0.036 & 0.036 & 0.036 & 0.037 & 0.036 & 0.036 & 0.036 & 0.037 & 0.036 & 0.052 \\ 
   &  & p=996 & 0.243 & 0.242 & 0.385 & 0.548 & 0.243 & 0.243 & 0.242 & 0.655 & 0.242 & 0.401 & 0.036 & 0.036 & 0.245 & 58.167 & 0.036 & 0.036 & 0.036 & 104.242 & 0.036 & 36.878 \\ 
  \cmidrule{2-23} & \multirow{3}[2]{*}{$\rho=0.9$} & p=12 & 0.069 & 0.069 & 0.069 & 0.069 & 0.069 & 0.069 & 0.069 & 0.069 & 0.069 & 0.069 & 0.012 & 0.012 & 0.012 & 0.012 & 0.012 & 0.012 & 0.012 & 0.012 & 0.012 & 0.012 \\ 
   &  & p=500 & 0.122 & 0.122 & 0.122 & 0.122 & 0.122 & 0.122 & 0.122 & 0.122 & 0.122 & 0.148 & 0.034 & 0.034 & 0.034 & 0.034 & 0.034 & 0.034 & 0.034 & 0.034 & 0.034 & 0.049 \\ 
   &  & p=996 & 0.122 & 0.122 & 0.196 & 0.282 & 0.122 & 0.122 & 0.122 & 0.336 & 0.122 & 0.204 & 0.034 & 0.034 & 0.243 & 56.752 & 0.034 & 0.034 & 0.034 & 103.682 & 0.034 & 35.435 \\ 
  \midrule\multirow{9}[6]{*}{lsnr} & \multirow{3}[2]{*}{$\rho=0$} & p=12 & 0.503 & 0.503 & 0.503 & 0.503 & 0.508 & 0.502 & 0.503 & 0.502 & 0.503 & 0.695 & 0.066 & 0.066 & 0.066 & 0.065 & 0.067 & 0.065 & 0.065 & 0.065 & 0.065 & 0.127 \\ 
   &  & p=500 & 0.776 & 0.778 & 0.777 & 0.781 & 0.781 & 0.78 & 0.78 & 0.782 & 0.78 & 0.987 & 0.139 & 0.14 & 0.14 & 0.142 & 0.142 & 0.141 & 0.141 & 0.142 & 0.141 & 0.225 \\ 
   &  & p=996 & 0.776 & 0.778 & 1.32 & 1.922 & 0.781 & 0.78 & 0.78 & 2.304 & 0.78 & 1.367 & 0.139 & 0.14 & 1.192 & 292.752 & 0.142 & 0.141 & 0.141 & 536.268 & 0.141 & 168.413 \\ 
  \cmidrule{2-23} & \multirow{3}[2]{*}{$\rho=0.5$} & p=12 & 0.307 & 0.306 & 0.306 & 0.306 & 0.31 & 0.306 & 0.306 & 0.306 & 0.306 & 0.443 & 0.063 & 0.063 & 0.063 & 0.063 & 0.065 & 0.063 & 0.063 & 0.063 & 0.063 & 0.136 \\ 
   &  & p=500 & 0.475 & 0.475 & 0.476 & 0.477 & 0.479 & 0.477 & 0.478 & 0.479 & 0.478 & 0.604 & 0.138 & 0.139 & 0.14 & 0.14 & 0.141 & 0.14 & 0.14 & 0.142 & 0.14 & 0.223 \\ 
   &  & p=996 & 0.475 & 0.475 & 0.809 & 1.179 & 0.479 & 0.477 & 0.478 & 1.412 & 0.478 & 0.837 & 0.138 & 0.139 & 1.19 & 290.823 & 0.141 & 0.14 & 0.14 & 515.434 & 0.14 & 168.017 \\ 
  \cmidrule{2-23} & \multirow{3}[2]{*}{$\rho=0.9$} & p=12 & 0.162 & 0.161 & 0.161 & 0.161 & 0.163 & 0.161 & 0.161 & 0.161 & 0.161 & 0.248 & 0.066 & 0.066 & 0.066 & 0.065 & 0.067 & 0.065 & 0.065 & 0.065 & 0.065 & 0.156 \\ 
   &  & p=500 & 0.235 & 0.235 & 0.236 & 0.237 & 0.238 & 0.236 & 0.236 & 0.237 & 0.236 & 0.302 & 0.125 & 0.125 & 0.126 & 0.127 & 0.128 & 0.126 & 0.126 & 0.127 & 0.126 & 0.205 \\ 
   &  & p=996 & 0.235 & 0.235 & 0.413 & 0.604 & 0.238 & 0.236 & 0.236 & 0.727 & 0.236 & 0.423 & 0.125 & 0.125 & 1.18 & 283.72 & 0.128 & 0.126 & 0.126 & 517.624 & 0.126 & 164.529 \\ 
   \midrule 
 \multicolumn{1}{|c}{} & \multicolumn{1}{c}{} &       & \multicolumn{10}{c||}{Relative Test Error}                                    & \multicolumn{10}{c|}{Proportion of Variance Explained} \\
\midrule\multirow{9}[6]{*}{hsnr} & \multirow{3}[2]{*}{$\rho=0$} & p=12 & 1.012 & 1.012 & 1.012 & 1.012 & 1.012 & 1.012 & 1.012 & 1.012 & 1.012 & 1.012 & 0.893 & 0.893 & 0.893 & 0.893 & 0.893 & 0.893 & 0.893 & 0.893 & 0.893 & 0.893 \\ 
   &  & p=500 & 1.048 & 1.048 & 1.048 & 1.048 & 1.048 & 1.048 & 1.048 & 1.048 & 1.048 & 1.064 & 0.89 & 0.89 & 0.89 & 0.89 & 0.89 & 0.89 & 0.89 & 0.89 & 0.89 & 0.888 \\ 
   &  & p=996 & 1.048 & 1.048 & 1.253 & 58.92 & 1.048 & 1.048 & 1.048 & 109.434 & 1.048 & 42.921 & 0.89 & 0.89 & 0.868 & -5.262 & 0.89 & 0.89 & 0.89 & -10.589 & 0.89 & -3.554 \\ 
  \cmidrule{2-23} & \multirow{3}[2]{*}{$\rho=0.5$} & p=12 & 1.012 & 1.012 & 1.012 & 1.012 & 1.012 & 1.012 & 1.012 & 1.012 & 1.012 & 1.012 & 0.893 & 0.893 & 0.893 & 0.893 & 0.893 & 0.893 & 0.893 & 0.893 & 0.893 & 0.893 \\ 
   &  & p=500 & 1.047 & 1.047 & 1.047 & 1.048 & 1.048 & 1.047 & 1.047 & 1.048 & 1.047 & 1.064 & 0.89 & 0.89 & 0.89 & 0.89 & 0.89 & 0.89 & 0.89 & 0.89 & 0.89 & 0.888 \\ 
   &  & p=996 & 1.047 & 1.047 & 1.252 & 58.601 & 1.048 & 1.047 & 1.047 & 105.127 & 1.047 & 42.901 & 0.89 & 0.89 & 0.868 & -5.222 & 0.89 & 0.89 & 0.89 & -10.116 & 0.89 & -3.543 \\ 
  \cmidrule{2-23} & \multirow{3}[2]{*}{$\rho=0.9$} & p=12 & 1.012 & 1.012 & 1.012 & 1.012 & 1.012 & 1.012 & 1.012 & 1.012 & 1.012 & 1.012 & 0.893 & 0.893 & 0.893 & 0.893 & 0.893 & 0.893 & 0.893 & 0.893 & 0.893 & 0.893 \\ 
   &  & p=500 & 1.045 & 1.045 & 1.045 & 1.045 & 1.045 & 1.045 & 1.045 & 1.045 & 1.045 & 1.061 & 0.89 & 0.89 & 0.89 & 0.89 & 0.89 & 0.89 & 0.89 & 0.89 & 0.89 & 0.888 \\ 
   &  & p=996 & 1.045 & 1.045 & 1.252 & 57.736 & 1.045 & 1.045 & 1.045 & 105.429 & 1.045 & 41.155 & 0.89 & 0.89 & 0.868 & -5.078 & 0.89 & 0.89 & 0.89 & -10.085 & 0.89 & -3.319 \\ 
  \midrule\multirow{9}[6]{*}{msnr} & \multirow{3}[2]{*}{$\rho=0$} & p=12 & 1.012 & 1.012 & 1.012 & 1.012 & 1.012 & 1.012 & 1.012 & 1.012 & 1.012 & 1.012 & 0.494 & 0.494 & 0.494 & 0.494 & 0.494 & 0.494 & 0.494 & 0.494 & 0.494 & 0.493 \\ 
   &  & p=500 & 1.037 & 1.037 & 1.037 & 1.037 & 1.037 & 1.037 & 1.037 & 1.037 & 1.037 & 1.053 & 0.482 & 0.482 & 0.482 & 0.481 & 0.481 & 0.482 & 0.481 & 0.481 & 0.481 & 0.474 \\ 
   &  & p=996 & 1.037 & 1.037 & 1.244 & 58.907 & 1.037 & 1.037 & 1.037 & 108.265 & 1.037 & 37.594 & 0.482 & 0.482 & 0.377 & -28.597 & 0.481 & 0.482 & 0.481 & -53.302 & 0.481 & -17.882 \\ 
  \cmidrule{2-23} & \multirow{3}[2]{*}{$\rho=0.5$} & p=12 & 1.012 & 1.012 & 1.012 & 1.012 & 1.012 & 1.012 & 1.012 & 1.012 & 1.012 & 1.012 & 0.494 & 0.494 & 0.494 & 0.494 & 0.494 & 0.494 & 0.494 & 0.494 & 0.494 & 0.494 \\ 
   &  & p=500 & 1.036 & 1.036 & 1.036 & 1.036 & 1.036 & 1.036 & 1.036 & 1.037 & 1.036 & 1.052 & 0.482 & 0.482 & 0.482 & 0.482 & 0.482 & 0.482 & 0.482 & 0.482 & 0.482 & 0.474 \\ 
   &  & p=996 & 1.036 & 1.036 & 1.244 & 58.589 & 1.036 & 1.036 & 1.036 & 104.685 & 1.036 & 37.574 & 0.482 & 0.482 & 0.378 & -28.421 & 0.482 & 0.482 & 0.482 & -51.454 & 0.482 & -17.85 \\ 
  \cmidrule{2-23} & \multirow{3}[2]{*}{$\rho=0.9$} & p=12 & 1.012 & 1.012 & 1.012 & 1.012 & 1.012 & 1.012 & 1.012 & 1.012 & 1.012 & 1.012 & 0.494 & 0.494 & 0.494 & 0.494 & 0.494 & 0.494 & 0.494 & 0.494 & 0.494 & 0.494 \\ 
   &  & p=500 & 1.034 & 1.034 & 1.034 & 1.034 & 1.034 & 1.034 & 1.034 & 1.034 & 1.034 & 1.049 & 0.483 & 0.483 & 0.483 & 0.483 & 0.483 & 0.483 & 0.483 & 0.483 & 0.483 & 0.476 \\ 
   &  & p=996 & 1.034 & 1.034 & 1.243 & 57.728 & 1.034 & 1.034 & 1.034 & 104.775 & 1.034 & 36.479 & 0.483 & 0.483 & 0.379 & -27.855 & 0.483 & 0.483 & 0.483 & -51.34 & 0.483 & -17.218 \\ 
  \midrule\multirow{9}[6]{*}{lsnr} & \multirow{3}[2]{*}{$\rho=0$} & p=12 & 1.013 & 1.013 & 1.013 & 1.013 & 1.013 & 1.013 & 1.013 & 1.013 & 1.013 & 1.025 & 0.156 & 0.156 & 0.156 & 0.156 & 0.155 & 0.156 & 0.156 & 0.156 & 0.156 & 0.145 \\ 
   &  & p=500 & 1.028 & 1.028 & 1.028 & 1.028 & 1.028 & 1.028 & 1.028 & 1.028 & 1.028 & 1.045 & 0.143 & 0.143 & 0.143 & 0.143 & 0.143 & 0.143 & 0.143 & 0.143 & 0.143 & 0.129 \\ 
   &  & p=996 & 1.028 & 1.028 & 1.237 & 58.905 & 1.028 & 1.028 & 1.028 & 107.494 & 1.028 & 34.284 & 0.143 & 0.143 & -0.031 & -48.161 & 0.143 & 0.143 & 0.143 & -88.657 & 0.143 & -27.614 \\ 
  \cmidrule{2-23} & \multirow{3}[2]{*}{$\rho=0.5$} & p=12 & 1.013 & 1.013 & 1.013 & 1.013 & 1.013 & 1.013 & 1.013 & 1.013 & 1.013 & 1.027 & 0.156 & 0.156 & 0.156 & 0.156 & 0.156 & 0.156 & 0.156 & 0.156 & 0.156 & 0.144 \\ 
   &  & p=500 & 1.028 & 1.028 & 1.028 & 1.028 & 1.028 & 1.028 & 1.028 & 1.028 & 1.028 & 1.045 & 0.144 & 0.144 & 0.143 & 0.143 & 0.143 & 0.143 & 0.143 & 0.143 & 0.143 & 0.13 \\ 
   &  & p=996 & 1.028 & 1.028 & 1.236 & 58.587 & 1.028 & 1.028 & 1.028 & 103.508 & 1.028 & 34.276 & 0.144 & 0.144 & -0.03 & -47.887 & 0.143 & 0.143 & 0.143 & -85.311 & 0.143 & -27.598 \\ 
  \cmidrule{2-23} & \multirow{3}[2]{*}{$\rho=0.9$} & p=12 & 1.013 & 1.013 & 1.013 & 1.013 & 1.013 & 1.013 & 1.013 & 1.013 & 1.013 & 1.031 & 0.156 & 0.156 & 0.156 & 0.156 & 0.156 & 0.156 & 0.156 & 0.156 & 0.156 & 0.141 \\ 
   &  & p=500 & 1.025 & 1.025 & 1.025 & 1.025 & 1.026 & 1.025 & 1.025 & 1.025 & 1.025 & 1.041 & 0.146 & 0.146 & 0.146 & 0.146 & 0.146 & 0.146 & 0.146 & 0.146 & 0.146 & 0.133 \\ 
   &  & p=996 & 1.025 & 1.025 & 1.237 & 57.72 & 1.026 & 1.025 & 1.025 & 104.609 & 1.025 & 33.894 & 0.146 & 0.146 & -0.03 & -47.089 & 0.146 & 0.146 & 0.146 & -86.14 & 0.146 & -27.238 \\ 
   \midrule 
 \multicolumn{1}{|c}{} & \multicolumn{1}{c}{} &       & \multicolumn{10}{c||}{Kullback-Leibler Discrepancy}                                    & \multicolumn{10}{c|}{Number of Variables} \\
\midrule\multirow{9}[6]{*}{hsnr} & \multirow{3}[2]{*}{$\rho=0$} & p=12 & 0.287 & 0.287 & 0.287 & 0.287 & 0.287 & 0.287 & 0.287 & 0.287 & 0.287 & 0.287 & 12 & 12 & 12 & 12 & 12 & 12 & 12 & 12 & 12 & 12 \\ 
   &  & p=500 & 0.42 & 0.42 & 0.42 & 0.42 & 0.42 & 0.42 & 0.42 & 0.421 & 0.42 & 0.434 & 35.97 & 36.344 & 36.441 & 36.976 & 36.492 & 36.873 & 36.843 & 37.417 & 36.847 & 28.028 \\ 
   &  & p=996 & 0.42 & 0.42 & 1.207 & 76903.101 & 0.42 & 0.42 & 0.42 & 108415.913 & 0.42 & 76781.638 & 35.97 & 36.344 & 118.541 & 296.26 & 36.492 & 36.873 & 36.843 & 410.064 & 36.847 & 152.871 \\ 
  \cmidrule{2-23} & \multirow{3}[2]{*}{$\rho=0.5$} & p=12 & -0.665 & -0.665 & -0.665 & -0.665 & -0.665 & -0.665 & -0.665 & -0.665 & -0.665 & -0.665 & 12 & 12 & 12 & 12 & 12 & 12 & 12 & 12 & 12 & 12 \\ 
   &  & p=500 & -0.557 & -0.556 & -0.556 & -0.556 & -0.556 & -0.556 & -0.556 & -0.556 & -0.556 & -0.542 & 35.752 & 36.096 & 36.245 & 36.572 & 36.012 & 36.49 & 36.501 & 37.051 & 36.501 & 27.723 \\ 
   &  & p=996 & -0.557 & -0.556 & 0.228 & 75597.583 & -0.556 & -0.556 & -0.556 & 105005.233 & -0.556 & 76757.66 & 35.752 & 36.096 & 117.97 & 295.91 & 36.012 & 36.49 & 36.501 & 413.719 & 36.501 & 152.629 \\ 
  \cmidrule{2-23} & \multirow{3}[2]{*}{$\rho=0.9$} & p=12 & -1.989 & -1.989 & -1.989 & -1.989 & -1.989 & -1.989 & -1.989 & -1.989 & -1.989 & -1.989 & 12 & 12 & 12 & 12 & 12 & 12 & 12 & 12 & 12 & 12 \\ 
   &  & p=500 & -1.865 & -1.865 & -1.865 & -1.865 & -1.865 & -1.865 & -1.865 & -1.864 & -1.865 & -1.851 & 33.178 & 33.561 & 33.711 & 34.19 & 33.652 & 34.007 & 34.038 & 34.583 & 34.038 & 25.179 \\ 
   &  & p=996 & -1.865 & -1.865 & -1.048 & 75576.902 & -1.865 & -1.865 & -1.865 & 106902.398 & -1.865 & 75342.334 & 33.178 & 33.561 & 115.955 & 294.011 & 33.652 & 34.007 & 34.038 & 407.529 & 34.038 & 144.612 \\ 
  \midrule\multirow{9}[6]{*}{msnr} & \multirow{3}[2]{*}{$\rho=0$} & p=12 & 2.427 & 2.427 & 2.427 & 2.427 & 2.427 & 2.427 & 2.427 & 2.427 & 2.427 & 2.428 & 12 & 12 & 12 & 12 & 12 & 12 & 12 & 12 & 12 & 11.982 \\ 
   &  & p=500 & 2.547 & 2.547 & 2.547 & 2.548 & 2.548 & 2.548 & 2.548 & 2.548 & 2.548 & 2.562 & 25.616 & 25.934 & 25.895 & 26.211 & 26.071 & 26.148 & 26.173 & 26.478 & 26.175 & 17.392 \\ 
   &  & p=996 & 2.547 & 2.547 & 3.338 & 76905.201 & 2.548 & 2.548 & 2.548 & 107911.181 & 2.548 & 72300.1 & 25.616 & 25.934 & 110.363 & 287.913 & 26.071 & 26.148 & 26.173 & 401.555 & 26.175 & 125.025 \\ 
  \cmidrule{2-23} & \multirow{3}[2]{*}{$\rho=0.5$} & p=12 & 1.475 & 1.475 & 1.475 & 1.475 & 1.475 & 1.475 & 1.475 & 1.475 & 1.475 & 1.475 & 12 & 12 & 12 & 12 & 12 & 12 & 12 & 12 & 12 & 12 \\ 
   &  & p=500 & 1.571 & 1.571 & 1.571 & 1.572 & 1.572 & 1.572 & 1.572 & 1.572 & 1.572 & 1.585 & 25.336 & 25.563 & 25.687 & 26.032 & 25.794 & 25.954 & 25.97 & 26.281 & 25.97 & 17.165 \\ 
   &  & p=996 & 1.571 & 1.571 & 2.359 & 75599.684 & 1.572 & 1.572 & 1.572 & 104930.642 & 1.572 & 72276.122 & 25.336 & 25.563 & 110.129 & 287.9 & 25.794 & 25.954 & 25.97 & 402.418 & 25.97 & 124.812 \\ 
  \cmidrule{2-23} & \multirow{3}[2]{*}{$\rho=0.9$} & p=12 & 0.151 & 0.151 & 0.151 & 0.151 & 0.151 & 0.151 & 0.151 & 0.151 & 0.151 & 0.151 & 12 & 12 & 12 & 12 & 12 & 12 & 12 & 12 & 12 & 11.998 \\ 
   &  & p=500 & 0.263 & 0.263 & 0.263 & 0.263 & 0.263 & 0.263 & 0.263 & 0.263 & 0.263 & 0.277 & 22.994 & 23.242 & 23.353 & 23.616 & 23.385 & 23.559 & 23.524 & 23.805 & 23.534 & 14.494 \\ 
   &  & p=996 & 0.263 & 0.263 & 1.082 & 75579.034 & 0.263 & 0.263 & 0.263 & 106722.711 & 0.263 & 71332.246 & 22.994 & 23.242 & 107.748 & 286.465 & 23.385 & 23.559 & 23.524 & 397.024 & 23.534 & 119.516 \\ 
  \midrule\multirow{9}[6]{*}{lsnr} & \multirow{3}[2]{*}{$\rho=0$} & p=12 & 4.038 & 4.038 & 4.038 & 4.038 & 4.038 & 4.038 & 4.038 & 4.038 & 4.038 & 4.05 & 11.618 & 11.622 & 11.622 & 11.628 & 11.573 & 11.636 & 11.628 & 11.64 & 11.628 & 9.407 \\ 
   &  & p=500 & 4.147 & 4.148 & 4.148 & 4.148 & 4.148 & 4.148 & 4.148 & 4.148 & 4.148 & 4.163 & 17.999 & 18.283 & 18.256 & 18.568 & 18.37 & 18.528 & 18.573 & 18.833 & 18.573 & 9.184 \\ 
   &  & p=996 & 4.147 & 4.148 & 4.939 & 76906.834 & 4.148 & 4.148 & 4.148 & 107778.903 & 4.148 & 69388.889 & 17.999 & 18.283 & 103.999 & 282.923 & 18.37 & 18.528 & 18.573 & 392.937 & 18.573 & 108.847 \\ 
  \cmidrule{2-23} & \multirow{3}[2]{*}{$\rho=0.5$} & p=12 & 3.085 & 3.085 & 3.085 & 3.085 & 3.085 & 3.085 & 3.085 & 3.085 & 3.085 & 3.099 & 11.844 & 11.848 & 11.849 & 11.852 & 11.782 & 11.846 & 11.852 & 11.854 & 11.852 & 9.435 \\ 
   &  & p=500 & 3.172 & 3.172 & 3.172 & 3.172 & 3.172 & 3.172 & 3.172 & 3.172 & 3.172 & 3.187 & 17.664 & 17.876 & 18.047 & 18.303 & 18.103 & 18.182 & 18.211 & 18.55 & 18.212 & 8.899 \\ 
   &  & p=996 & 3.172 & 3.172 & 3.96 & 75601.317 & 3.172 & 3.172 & 3.172 & 104560.814 & 3.172 & 69376.666 & 17.664 & 17.876 & 103.758 & 282.767 & 18.103 & 18.182 & 18.211 & 391.794 & 18.212 & 108.592 \\ 
  \cmidrule{2-23} & \multirow{3}[2]{*}{$\rho=0.9$} & p=12 & 1.762 & 1.762 & 1.762 & 1.762 & 1.762 & 1.762 & 1.762 & 1.762 & 1.762 & 1.78 & 11.601 & 11.622 & 11.622 & 11.628 & 11.543 & 11.632 & 11.628 & 11.636 & 11.628 & 7.766 \\ 
   &  & p=500 & 1.863 & 1.863 & 1.863 & 1.864 & 1.864 & 1.863 & 1.863 & 1.864 & 1.863 & 1.878 & 15.296 & 15.505 & 15.632 & 15.889 & 15.601 & 15.809 & 15.774 & 16.02 & 15.779 & 6.376 \\ 
   &  & p=996 & 1.863 & 1.863 & 2.685 & 75580.631 & 1.864 & 1.863 & 1.863 & 106684.517 & 1.863 & 69022.699 & 15.296 & 15.505 & 101.677 & 279.857 & 15.601 & 15.809 & 15.774 & 391.233 & 15.779 & 105.349 \\ 
   \bottomrule 
\end{tabular}
}
\end{table}

%% latex table generated in R 4.0.2 by xtable 1.8-4 package
% Fri Jun 26 10:43:47 2020
\begin{table}[ht]
\centering
\caption{Fixed X, Dense-Ex2, n=40} 
\scalebox{0.52}{
\begin{tabular}{|c|c|c|cc|cc|cc|ccc|c||cc|cc|cc|ccc|c|}
  \toprule 
 \multicolumn{1}{|c}{} & \multicolumn{1}{c}{} &       & RAICc & AICc  & RC$_p$ & C$_p$ & 10F CV & LOO CV & S$_p$ & $\widehat{\text{C}}_p$ & GCV   & BIC   & RAICc & AICc  & RC$_p$ & C$_p$ & 10F CV & LOO CV & S$_p$ & $\widehat{\text{C}}_p$ & GCV   & BIC  \\
 \cmidrule{4-23}\multicolumn{1}{|c}{} & \multicolumn{1}{c}{} &       & \multicolumn{10}{c||}{RMSE}                                                   & \multicolumn{10}{c|}{Relative Risk}       \\
 \midrule\multirow{9}[6]{*}{hsnr} & \multirow{3}[2]{*}{$\rho=0$} & p=12 & 0.675 & 0.657 & 0.654 & 0.646 & 0.656 & 0.649 & 0.646 & 0.645 & 0.646 & 0.651 & 0.083 & 0.075 & 0.074 & 0.072 & 0.075 & 0.073 & 0.072 & 0.071 & 0.072 & 0.073 \\ 
   &  & p=20 & 2.141 & 1.003 & 0.845 & 0.805 & 0.816 & 0.81 & 0.807 & 0.801 & 0.806 & 0.805 & 0.65 & 0.191 & 0.132 & 0.111 & 0.117 & 0.114 & 0.112 & 0.108 & 0.112 & 0.111 \\ 
   &  & p=36 & 2.51 & 2.113 & 1.624 & 1.437 & 1.694 & 1.576 & 1.574 & 1.377 & 1.545 & 1.392 & 0.894 & 0.785 & 0.666 & 0.824 & 0.686 & 0.721 & 0.73 & 0.898 & 0.748 & 0.936 \\ 
  \cmidrule{2-23} & \multirow{3}[2]{*}{$\rho=0.5$} & p=12 & 0.403 & 0.397 & 0.397 & 0.396 & 0.398 & 0.397 & 0.396 & 0.396 & 0.396 & 0.397 & 0.077 & 0.072 & 0.072 & 0.072 & 0.073 & 0.072 & 0.072 & 0.072 & 0.072 & 0.072 \\ 
   &  & p=20 & 1.336 & 0.609 & 0.48 & 0.472 & 0.476 & 0.472 & 0.472 & 0.472 & 0.472 & 0.472 & 0.709 & 0.197 & 0.11 & 0.104 & 0.107 & 0.104 & 0.104 & 0.104 & 0.104 & 0.104 \\ 
   &  & p=36 & 1.44 & 1.253 & 0.963 & 0.844 & 1.008 & 0.93 & 0.935 & 0.811 & 0.916 & 0.821 & 0.864 & 0.787 & 0.663 & 0.856 & 0.691 & 0.731 & 0.744 & 0.922 & 0.76 & 0.95 \\ 
  \cmidrule{2-23} & \multirow{3}[2]{*}{$\rho=0.9$} & p=12 & 0.185 & 0.182 & 0.182 & 0.181 & 0.182 & 0.182 & 0.181 & 0.181 & 0.181 & 0.182 & 0.073 & 0.068 & 0.067 & 0.067 & 0.069 & 0.067 & 0.067 & 0.067 & 0.067 & 0.068 \\ 
   &  & p=20 & 0.568 & 0.255 & 0.194 & 0.193 & 0.194 & 0.193 & 0.193 & 0.193 & 0.193 & 0.194 & 0.703 & 0.183 & 0.083 & 0.082 & 0.083 & 0.082 & 0.082 & 0.082 & 0.082 & 0.083 \\ 
   &  & p=36 & 0.682 & 0.57 & 0.415 & 0.349 & 0.446 & 0.4 & 0.403 & 0.333 & 0.39 & 0.341 & 0.753 & 0.682 & 0.588 & 0.802 & 0.625 & 0.676 & 0.69 & 0.854 & 0.706 & 0.844 \\ 
  \midrule\multirow{9}[6]{*}{msnr} & \multirow{3}[2]{*}{$\rho=0$} & p=12 & 2.446 & 2.304 & 2.169 & 2.073 & 2.148 & 2.127 & 2.116 & 2.029 & 2.107 & 2.524 & 0.916 & 0.848 & 0.79 & 0.737 & 0.769 & 0.76 & 0.756 & 0.705 & 0.752 & 0.914 \\ 
   &  & p=20 & 2.865 & 2.803 & 2.619 & 2.524 & 2.654 & 2.628 & 2.634 & 2.491 & 2.623 & 2.98 & 0.951 & 0.936 & 0.906 & 0.931 & 0.923 & 0.93 & 0.938 & 0.939 & 0.94 & 0.978 \\ 
   &  & p=36 & 3.673 & 3.579 & 3.39 & 3.574 & 3.395 & 3.407 & 3.409 & 3.683 & 3.415 & 3.938 & 1.126 & 1.152 & 1.405 & 3.343 & 1.313 & 1.664 & 1.667 & 4.775 & 1.798 & 3.834 \\ 
  \cmidrule{2-23} & \multirow{3}[2]{*}{$\rho=0.5$} & p=12 & 1.506 & 1.434 & 1.352 & 1.292 & 1.342 & 1.325 & 1.319 & 1.263 & 1.314 & 1.563 & 0.948 & 0.9 & 0.842 & 0.778 & 0.82 & 0.808 & 0.8 & 0.743 & 0.795 & 0.942 \\ 
   &  & p=20 & 1.681 & 1.653 & 1.564 & 1.506 & 1.581 & 1.567 & 1.571 & 1.486 & 1.567 & 1.763 & 0.942 & 0.93 & 0.918 & 0.939 & 0.931 & 0.937 & 0.937 & 0.941 & 0.938 & 0.97 \\ 
   &  & p=36 & 2.144 & 2.084 & 1.988 & 2.108 & 1.996 & 2.003 & 2.007 & 2.173 & 2.005 & 2.313 & 1.109 & 1.138 & 1.383 & 3.319 & 1.293 & 1.632 & 1.618 & 4.833 & 1.771 & 3.809 \\ 
  \cmidrule{2-23} & \multirow{3}[2]{*}{$\rho=0.9$} & p=12 & 0.669 & 0.642 & 0.612 & 0.588 & 0.609 & 0.602 & 0.599 & 0.577 & 0.597 & 0.698 & 0.892 & 0.847 & 0.801 & 0.736 & 0.778 & 0.764 & 0.761 & 0.704 & 0.756 & 0.889 \\ 
   &  & p=20 & 0.667 & 0.664 & 0.646 & 0.62 & 0.649 & 0.641 & 0.642 & 0.611 & 0.64 & 0.704 & 0.814 & 0.812 & 0.805 & 0.792 & 0.803 & 0.799 & 0.804 & 0.78 & 0.803 & 0.858 \\ 
   &  & p=36 & 0.858 & 0.846 & 0.823 & 0.875 & 0.827 & 0.832 & 0.83 & 0.903 & 0.832 & 0.937 & 0.915 & 0.939 & 1.157 & 2.839 & 1.068 & 1.363 & 1.337 & 4.161 & 1.462 & 3.271 \\ 
  \midrule\multirow{9}[6]{*}{lsnr} & \multirow{3}[2]{*}{$\rho=0$} & p=12 & 3.508 & 3.545 & 3.623 & 3.727 & 3.657 & 3.66 & 3.665 & 3.774 & 3.674 & 3.555 & 1.472 & 1.564 & 1.786 & 1.981 & 1.842 & 1.857 & 1.859 & 2.073 & 1.881 & 1.311 \\ 
   &  & p=20 & 3.482 & 3.518 & 3.683 & 3.959 & 3.723 & 3.745 & 3.73 & 4.07 & 3.764 & 3.454 & 1.222 & 1.268 & 1.509 & 2.038 & 1.578 & 1.617 & 1.573 & 2.245 & 1.629 & 1.132 \\ 
   &  & p=36 & 4.436 & 4.5 & 5.118 & 5.94 & 4.754 & 4.872 & 4.873 & 6.709 & 4.972 & 5.404 & 1.344 & 1.434 & 2.935 & 11.622 & 1.926 & 2.661 & 2.755 & 17.317 & 3.532 & 10.462 \\ 
  \cmidrule{2-23} & \multirow{3}[2]{*}{$\rho=0.5$} & p=12 & 2.157 & 2.185 & 2.239 & 2.307 & 2.249 & 2.256 & 2.269 & 2.328 & 2.271 & 2.197 & 1.472 & 1.584 & 1.837 & 2.051 & 1.856 & 1.888 & 1.921 & 2.123 & 1.934 & 1.323 \\ 
   &  & p=20 & 2.062 & 2.082 & 2.175 & 2.342 & 2.187 & 2.202 & 2.206 & 2.408 & 2.222 & 2.04 & 1.21 & 1.254 & 1.484 & 1.999 & 1.529 & 1.569 & 1.563 & 2.228 & 1.616 & 1.123 \\ 
   &  & p=36 & 2.617 & 2.648 & 3.016 & 3.501 & 2.809 & 2.873 & 2.873 & 3.96 & 2.918 & 3.168 & 1.336 & 1.415 & 2.895 & 11.419 & 1.929 & 2.773 & 2.802 & 17.419 & 3.334 & 10.205 \\ 
  \cmidrule{2-23} & \multirow{3}[2]{*}{$\rho=0.9$} & p=12 & 0.997 & 1.007 & 1.031 & 1.059 & 1.032 & 1.037 & 1.046 & 1.071 & 1.047 & 1.006 & 1.402 & 1.498 & 1.732 & 1.929 & 1.729 & 1.77 & 1.82 & 2.006 & 1.837 & 1.246 \\ 
   &  & p=20 & 0.83 & 0.84 & 0.878 & 0.944 & 0.88 & 0.882 & 0.883 & 0.973 & 0.888 & 0.828 & 1.049 & 1.086 & 1.256 & 1.632 & 1.272 & 1.294 & 1.283 & 1.801 & 1.305 & 1.03 \\ 
   &  & p=36 & 1.061 & 1.074 & 1.235 & 1.441 & 1.131 & 1.152 & 1.156 & 1.608 & 1.178 & 1.297 & 1.165 & 1.223 & 2.439 & 9.906 & 1.598 & 2.118 & 2.281 & 14.538 & 2.797 & 8.727 \\ 
   \midrule 
 \multicolumn{1}{|c}{} & \multicolumn{1}{c}{} &       & \multicolumn{10}{c||}{Relative Test Error}                                    & \multicolumn{10}{c|}{Proportion of Variance Explained} \\
\midrule\multirow{9}[6]{*}{hsnr} & \multirow{3}[2]{*}{$\rho=0$} & p=12 & 1.546 & 1.499 & 1.491 & 1.474 & 1.497 & 1.48 & 1.474 & 1.472 & 1.474 & 1.485 & 0.797 & 0.803 & 0.804 & 0.806 & 0.803 & 0.806 & 0.806 & 0.807 & 0.806 & 0.805 \\ 
   &  & p=20 & 7.756 & 2.988 & 2.369 & 2.156 & 2.219 & 2.181 & 2.168 & 2.126 & 2.159 & 2.152 & 0.32 & 0.738 & 0.792 & 0.811 & 0.805 & 0.809 & 0.81 & 0.813 & 0.811 & 0.811 \\ 
   &  & p=36 & 8.967 & 7.999 & 6.941 & 8.344 & 7.119 & 7.428 & 7.512 & 9.01 & 7.67 & 9.344 & 0.096 & 0.193 & 0.3 & 0.159 & 0.282 & 0.251 & 0.242 & 0.091 & 0.226 & 0.058 \\ 
  \cmidrule{2-23} & \multirow{3}[2]{*}{$\rho=0.5$} & p=12 & 1.497 & 1.467 & 1.467 & 1.465 & 1.472 & 1.467 & 1.465 & 1.463 & 1.465 & 1.467 & 0.799 & 0.803 & 0.803 & 0.803 & 0.802 & 0.803 & 0.803 & 0.804 & 0.803 & 0.803 \\ 
   &  & p=20 & 8.495 & 3.078 & 2.167 & 2.101 & 2.132 & 2.105 & 2.101 & 2.101 & 2.101 & 2.101 & 0.266 & 0.734 & 0.813 & 0.818 & 0.816 & 0.818 & 0.818 & 0.818 & 0.818 & 0.818 \\ 
   &  & p=36 & 8.722 & 8.034 & 6.922 & 8.654 & 7.172 & 7.531 & 7.653 & 9.238 & 7.787 & 9.485 & 0.122 & 0.191 & 0.303 & 0.129 & 0.278 & 0.242 & 0.23 & 0.07 & 0.216 & 0.045 \\ 
  \cmidrule{2-23} & \multirow{3}[2]{*}{$\rho=0.9$} & p=12 & 1.502 & 1.47 & 1.466 & 1.464 & 1.474 & 1.466 & 1.464 & 1.463 & 1.464 & 1.467 & 0.81 & 0.814 & 0.815 & 0.815 & 0.814 & 0.815 & 0.815 & 0.815 & 0.815 & 0.815 \\ 
   &  & p=20 & 10.464 & 3.461 & 2.12 & 2.101 & 2.116 & 2.101 & 2.101 & 2.101 & 2.101 & 2.118 & 0.277 & 0.761 & 0.853 & 0.855 & 0.854 & 0.855 & 0.855 & 0.855 & 0.855 & 0.854 \\ 
   &  & p=36 & 8.783 & 8.045 & 7.074 & 9.288 & 7.459 & 7.983 & 8.126 & 9.818 & 8.298 & 9.721 & 0.225 & 0.29 & 0.376 & 0.18 & 0.342 & 0.295 & 0.283 & 0.133 & 0.268 & 0.142 \\ 
  \midrule\multirow{9}[6]{*}{msnr} & \multirow{3}[2]{*}{$\rho=0$} & p=12 & 1.713 & 1.66 & 1.615 & 1.574 & 1.599 & 1.592 & 1.589 & 1.549 & 1.585 & 1.712 & 0.037 & 0.066 & 0.092 & 0.115 & 0.101 & 0.105 & 0.107 & 0.129 & 0.109 & 0.038 \\ 
   &  & p=20 & 2.164 & 2.145 & 2.108 & 2.138 & 2.129 & 2.138 & 2.148 & 2.149 & 2.149 & 2.197 & 0.027 & 0.035 & 0.052 & 0.038 & 0.043 & 0.038 & 0.034 & 0.034 & 0.033 & 0.012 \\ 
   &  & p=36 & 2.181 & 2.208 & 2.474 & 4.506 & 2.377 & 2.746 & 2.748 & 6.009 & 2.885 & 5.022 & -0.065 & -0.078 & -0.207 & -1.199 & -0.16 & -0.34 & -0.341 & -1.933 & -0.408 & -1.451 \\ 
  \cmidrule{2-23} & \multirow{3}[2]{*}{$\rho=0.5$} & p=12 & 1.72 & 1.683 & 1.639 & 1.59 & 1.622 & 1.613 & 1.607 & 1.564 & 1.604 & 1.715 & 0.022 & 0.043 & 0.068 & 0.096 & 0.078 & 0.083 & 0.086 & 0.111 & 0.088 & 0.025 \\ 
   &  & p=20 & 2.171 & 2.156 & 2.142 & 2.168 & 2.157 & 2.165 & 2.165 & 2.17 & 2.166 & 2.206 & 0.032 & 0.039 & 0.045 & 0.034 & 0.038 & 0.035 & 0.035 & 0.033 & 0.035 & 0.017 \\ 
   &  & p=36 & 2.166 & 2.196 & 2.454 & 4.489 & 2.36 & 2.716 & 2.701 & 6.081 & 2.862 & 5.004 & -0.056 & -0.071 & -0.196 & -1.188 & -0.15 & -0.324 & -0.317 & -1.964 & -0.395 & -1.44 \\ 
  \cmidrule{2-23} & \multirow{3}[2]{*}{$\rho=0.9$} & p=12 & 1.725 & 1.688 & 1.651 & 1.598 & 1.632 & 1.621 & 1.619 & 1.572 & 1.615 & 1.722 & 0.048 & 0.069 & 0.089 & 0.118 & 0.099 & 0.106 & 0.107 & 0.133 & 0.109 & 0.05 \\ 
   &  & p=20 & 2.29 & 2.287 & 2.276 & 2.255 & 2.272 & 2.266 & 2.274 & 2.236 & 2.272 & 2.359 & 0.114 & 0.115 & 0.119 & 0.128 & 0.121 & 0.123 & 0.12 & 0.135 & 0.121 & 0.087 \\ 
   &  & p=36 & 2.112 & 2.141 & 2.407 & 4.45 & 2.298 & 2.657 & 2.625 & 6.056 & 2.777 & 4.975 & 0.047 & 0.034 & -0.086 & -1.009 & -0.037 & -0.199 & -0.185 & -1.734 & -0.254 & -1.246 \\ 
  \midrule\multirow{9}[6]{*}{lsnr} & \multirow{3}[2]{*}{$\rho=0$} & p=12 & 1.229 & 1.244 & 1.278 & 1.308 & 1.287 & 1.289 & 1.289 & 1.323 & 1.293 & 1.204 & -0.064 & -0.076 & -0.106 & -0.132 & -0.113 & -0.115 & -0.116 & -0.145 & -0.119 & -0.042 \\ 
   &  & p=20 & 1.299 & 1.31 & 1.369 & 1.499 & 1.386 & 1.396 & 1.385 & 1.549 & 1.398 & 1.277 & -0.044 & -0.053 & -0.1 & -0.204 & -0.114 & -0.121 & -0.113 & -0.245 & -0.124 & -0.026 \\ 
   &  & p=36 & 1.282 & 1.301 & 1.616 & 3.438 & 1.404 & 1.558 & 1.578 & 4.633 & 1.741 & 3.195 & -0.06 & -0.075 & -0.336 & -1.842 & -0.161 & -0.288 & -0.304 & -2.829 & -0.439 & -1.641 \\ 
  \cmidrule{2-23} & \multirow{3}[2]{*}{$\rho=0.5$} & p=12 & 1.223 & 1.24 & 1.279 & 1.311 & 1.282 & 1.287 & 1.292 & 1.322 & 1.294 & 1.201 & -0.062 & -0.077 & -0.11 & -0.139 & -0.113 & -0.117 & -0.121 & -0.148 & -0.123 & -0.043 \\ 
   &  & p=20 & 1.301 & 1.312 & 1.369 & 1.497 & 1.38 & 1.39 & 1.389 & 1.554 & 1.402 & 1.279 & -0.042 & -0.051 & -0.096 & -0.199 & -0.105 & -0.113 & -0.112 & -0.245 & -0.123 & -0.024 \\ 
   &  & p=36 & 1.281 & 1.297 & 1.609 & 3.401 & 1.406 & 1.583 & 1.589 & 4.663 & 1.701 & 3.146 & -0.058 & -0.072 & -0.329 & -1.81 & -0.161 & -0.308 & -0.313 & -2.852 & -0.406 & -1.599 \\ 
  \cmidrule{2-23} & \multirow{3}[2]{*}{$\rho=0.9$} & p=12 & 1.228 & 1.244 & 1.282 & 1.314 & 1.281 & 1.288 & 1.296 & 1.326 & 1.299 & 1.202 & -0.056 & -0.07 & -0.102 & -0.13 & -0.102 & -0.108 & -0.115 & -0.141 & -0.117 & -0.034 \\ 
   &  & p=20 & 1.332 & 1.344 & 1.398 & 1.517 & 1.403 & 1.41 & 1.406 & 1.571 & 1.414 & 1.327 & -0.012 & -0.021 & -0.062 & -0.152 & -0.065 & -0.071 & -0.068 & -0.193 & -0.073 & -0.007 \\ 
   &  & p=36 & 1.283 & 1.297 & 1.593 & 3.408 & 1.388 & 1.515 & 1.554 & 4.534 & 1.68 & 3.121 & -0.032 & -0.044 & -0.281 & -1.741 & -0.117 & -0.219 & -0.251 & -2.647 & -0.351 & -1.511 \\ 
   \midrule 
 \multicolumn{1}{|c}{} & \multicolumn{1}{c}{} &       & \multicolumn{10}{c||}{Kullback-Leibler Discrepancy}                                    & \multicolumn{10}{c|}{Number of Variables} \\
\midrule\multirow{9}[6]{*}{hsnr} & \multirow{3}[2]{*}{$\rho=0$} & p=12 & 1.993 & 1.977 & 1.974 & 1.967 & 1.975 & 1.969 & 1.967 & 1.966 & 1.967 & 1.972 & 11.825 & 11.914 & 11.93 & 11.967 & 11.927 & 11.958 & 11.967 & 11.973 & 11.967 & 11.942 \\ 
   &  & p=20 & 2.878 & 2.967 & 2.89 & 2.886 & 2.881 & 2.886 & 2.887 & 2.882 & 2.887 & 2.886 & 7.247 & 18.078 & 19.56 & 19.886 & 19.807 & 19.847 & 19.867 & 19.931 & 19.879 & 19.889 \\ 
   &  & p=36 & 3.171 & 3.009 & 5.146 & 28.073 & 5.63 & 18.576 & 19.911 & 32.846 & 22.496 & 35.236 & 8.313 & 11.56 & 18.996 & 28.897 & 18.487 & 22.452 & 22.2 & 32.863 & 23.496 & 33.963 \\ 
  \cmidrule{2-23} & \multirow{3}[2]{*}{$\rho=0.5$} & p=12 & 1.007 & 0.998 & 0.998 & 0.997 & 1 & 0.998 & 0.997 & 0.997 & 0.997 & 0.998 & 11.95 & 11.995 & 11.995 & 11.998 & 11.988 & 11.995 & 11.998 & 12 & 11.998 & 11.995 \\ 
   &  & p=20 & 1.875 & 1.954 & 1.839 & 1.829 & 1.833 & 1.83 & 1.829 & 1.829 & 1.829 & 1.829 & 5.917 & 17.857 & 19.915 & 19.999 & 19.958 & 19.995 & 19.999 & 20 & 19.999 & 19.999 \\ 
   &  & p=36 & 2.071 & 1.937 & 3.02 & 27.329 & 4.862 & 17.3 & 19.134 & 31.888 & 21.574 & 34.153 & 8.563 & 11.071 & 19.058 & 29.147 & 18.49 & 22.611 & 22.385 & 32.879 & 23.702 & 33.981 \\ 
  \cmidrule{2-23} & \multirow{3}[2]{*}{$\rho=0.9$} & p=12 & -0.555 & -0.563 & -0.565 & -0.565 & -0.562 & -0.565 & -0.565 & -0.565 & -0.565 & -0.564 & 11.938 & 11.988 & 11.996 & 11.998 & 11.982 & 11.996 & 11.998 & 12 & 11.998 & 11.994 \\ 
   &  & p=20 & 0.133 & 0.193 & 0.042 & 0.04 & 0.041 & 0.04 & 0.04 & 0.04 & 0.04 & 0.042 & 4.054 & 17.435 & 19.974 & 20 & 19.977 & 20 & 20 & 20 & 20 & 19.968 \\ 
   &  & p=36 & 0.499 & 0.263 & 1.824 & 25.732 & 3.111 & 16.989 & 18.051 & 30.228 & 20.598 & 32.546 & 6.005 & 9.75 & 19.442 & 29.901 & 18.046 & 22.819 & 22.603 & 33.416 & 24.232 & 34.16 \\ 
  \midrule\multirow{9}[6]{*}{msnr} & \multirow{3}[2]{*}{$\rho=0$} & p=12 & 4.061 & 4.095 & 4.095 & 4.109 & 4.1 & 4.106 & 4.109 & 4.113 & 4.109 & 4.127 & 4.939 & 6.499 & 7.854 & 9.056 & 8.221 & 8.421 & 8.559 & 9.585 & 8.664 & 4.995 \\ 
   &  & p=20 & 4.041 & 4.093 & 4.318 & 4.724 & 4.473 & 4.563 & 4.582 & 4.854 & 4.604 & 4.297 & 3.049 & 4.265 & 8.426 & 12.73 & 8.731 & 9.55 & 9.554 & 14.344 & 9.893 & 3.471 \\ 
   &  & p=36 & 4.54 & 4.574 & 6.767 & 24.49 & 5.182 & 13.139 & 13.617 & 30.955 & 15.096 & 30.291 & 3.187 & 4.32 & 10.024 & 16.956 & 8.612 & 10.287 & 10.073 & 24.51 & 10.948 & 14.936 \\ 
  \cmidrule{2-23} & \multirow{3}[2]{*}{$\rho=0.5$} & p=12 & 3.082 & 3.125 & 3.132 & 3.151 & 3.137 & 3.147 & 3.149 & 3.153 & 3.151 & 3.152 & 4.796 & 6.294 & 7.739 & 9.059 & 8.021 & 8.345 & 8.49 & 9.603 & 8.605 & 4.839 \\ 
   &  & p=20 & 2.981 & 3.024 & 3.276 & 3.685 & 3.403 & 3.499 & 3.528 & 3.807 & 3.543 & 3.242 & 3.048 & 4.061 & 8.072 & 12.632 & 8.297 & 9.147 & 9.165 & 14.234 & 9.414 & 3.352 \\ 
   &  & p=36 & 3.474 & 3.504 & 4.217 & 23.166 & 4.039 & 11.595 & 12.166 & 30.051 & 14.128 & 29.175 & 3.344 & 4.499 & 9.999 & 16.837 & 8.375 & 9.981 & 9.835 & 24.582 & 10.775 & 15.039 \\ 
  \cmidrule{2-23} & \multirow{3}[2]{*}{$\rho=0.9$} & p=12 & 1.499 & 1.545 & 1.557 & 1.583 & 1.566 & 1.575 & 1.578 & 1.59 & 1.58 & 1.574 & 4.931 & 6.339 & 7.609 & 8.946 & 7.923 & 8.223 & 8.384 & 9.488 & 8.469 & 4.989 \\ 
   &  & p=20 & 1.146 & 1.181 & 1.514 & 1.952 & 1.592 & 1.705 & 1.737 & 2.041 & 1.757 & 1.393 & 2.799 & 3.301 & 7.531 & 12.685 & 7.393 & 8.411 & 8.497 & 14.035 & 8.738 & 2.967 \\ 
   &  & p=36 & 1.662 & 1.7 & 2.591 & 21.705 & 2.217 & 10.04 & 10.417 & 28.333 & 11.985 & 27.446 & 2.984 & 3.941 & 9.577 & 16.567 & 7.306 & 8.908 & 8.831 & 24.101 & 9.724 & 14.897 \\ 
  \midrule\multirow{9}[6]{*}{lsnr} & \multirow{3}[2]{*}{$\rho=0$} & p=12 & 5.364 & 5.385 & 5.43 & 5.488 & 5.45 & 5.457 & 5.458 & 5.514 & 5.465 & 5.35 & 1.595 & 1.975 & 2.849 & 3.713 & 3.13 & 3.186 & 3.13 & 4.134 & 3.256 & 0.82 \\ 
   &  & p=20 & 5.277 & 5.296 & 5.415 & 5.707 & 5.467 & 5.523 & 5.5 & 5.829 & 5.528 & 5.278 & 1.214 & 1.459 & 2.765 & 4.84 & 3.057 & 3.228 & 3.007 & 5.776 & 3.25 & 0.573 \\ 
   &  & p=36 & 5.758 & 5.783 & 7.948 & 24.157 & 6.018 & 11.034 & 11.894 & 30.132 & 14.028 & 26.762 & 1.134 & 1.467 & 5.655 & 11.336 & 3.084 & 3.845 & 3.772 & 16.844 & 4.49 & 8.131 \\ 
  \cmidrule{2-23} & \multirow{3}[2]{*}{$\rho=0.5$} & p=12 & 4.392 & 4.418 & 4.465 & 4.528 & 4.48 & 4.492 & 4.498 & 4.548 & 4.502 & 4.387 & 1.579 & 1.956 & 2.849 & 3.765 & 3.097 & 3.172 & 3.213 & 4.133 & 3.308 & 0.838 \\ 
   &  & p=20 & 4.231 & 4.252 & 4.366 & 4.659 & 4.421 & 4.459 & 4.447 & 4.791 & 4.477 & 4.224 & 1.226 & 1.492 & 2.742 & 4.78 & 2.93 & 3.022 & 3.022 & 5.758 & 3.22 & 0.581 \\ 
   &  & p=36 & 4.703 & 4.726 & 6.824 & 22.86 & 4.988 & 10.301 & 10.907 & 29.176 & 12.606 & 25.491 & 1.176 & 1.484 & 5.633 & 11.354 & 3.202 & 3.932 & 3.836 & 16.996 & 4.375 & 7.954 \\ 
  \cmidrule{2-23} & \multirow{3}[2]{*}{$\rho=0.9$} & p=12 & 2.832 & 2.859 & 2.909 & 2.967 & 2.919 & 2.931 & 2.941 & 2.992 & 2.945 & 2.825 & 1.642 & 2.009 & 2.958 & 3.881 & 3.182 & 3.225 & 3.342 & 4.321 & 3.428 & 0.831 \\ 
   &  & p=20 & 2.429 & 2.453 & 2.56 & 2.856 & 2.609 & 2.628 & 2.621 & 2.992 & 2.632 & 2.432 & 1.199 & 1.496 & 2.597 & 4.627 & 2.739 & 2.754 & 2.668 & 5.552 & 2.796 & 0.593 \\ 
   &  & p=36 & 2.921 & 2.941 & 3.682 & 21.217 & 3.173 & 8.346 & 8.933 & 27.112 & 10.72 & 23.433 & 1.2 & 1.471 & 5.554 & 11.465 & 2.949 & 3.433 & 3.494 & 16.569 & 4.091 & 7.774 \\ 
   \bottomrule 
\end{tabular}
}
\end{table}

%% latex table generated in R 4.0.2 by xtable 1.8-4 package
% Fri Jun 26 10:43:53 2020
\begin{table}[ht]
\centering
\caption{Random X, Dense-Ex2, n=200} 
\scalebox{0.51}{
\begin{tabular}{|c|c|c|cc|cc|cc|ccc|c||cc|cc|cc|ccc|c|}
  \toprule 
 \multicolumn{1}{|c}{} & \multicolumn{1}{c}{} &       & RAICc & AICc  & RC$_p$ & C$_p$ & 10F CV & LOO CV & S$_p$ & $\widehat{\text{C}}_p$ & GCV   & BIC   & RAICc & AICc  & RC$_p$ & C$_p$ & 10F CV & LOO CV & S$_p$ & $\widehat{\text{C}}_p$ & GCV   & BIC  \\
 \cmidrule{4-23}\multicolumn{1}{|c}{} & \multicolumn{1}{c}{} &       & \multicolumn{10}{c||}{RMSE}                                                   & \multicolumn{10}{c|}{Relative Risk}       \\
 \midrule\multirow{9}[6]{*}{hsnr} & \multirow{3}[2]{*}{$\rho=0$} & p=12 & 0.252 & 0.252 & 0.252 & 0.252 & 0.252 & 0.252 & 0.252 & 0.252 & 0.252 & 0.252 & 0.008 & 0.008 & 0.008 & 0.008 & 0.008 & 0.008 & 0.008 & 0.008 & 0.008 & 0.008 \\ 
   &  & p=100 & 1.527 & 1.332 & 1.27 & 1.202 & 1.221 & 1.21 & 1.209 & 1.191 & 1.209 & 1.487 & 0.179 & 0.145 & 0.133 & 0.122 & 0.125 & 0.124 & 0.124 & 0.12 & 0.124 & 0.176 \\ 
   &  & p=196 & 1.671 & 1.471 & 1.394 & 1.444 & 1.343 & 1.325 & 1.325 & 1.592 & 1.324 & 1.695 & 0.21 & 0.177 & 0.176 & 1.557 & 0.161 & 0.169 & 0.179 & 3.994 & 0.179 & 6.364 \\ 
  \cmidrule{2-23} & \multirow{3}[2]{*}{$\rho=0.5$} & p=12 & 0.154 & 0.154 & 0.154 & 0.154 & 0.154 & 0.154 & 0.154 & 0.154 & 0.154 & 0.154 & 0.008 & 0.008 & 0.008 & 0.008 & 0.008 & 0.008 & 0.008 & 0.008 & 0.008 & 0.008 \\ 
   &  & p=100 & 0.894 & 0.781 & 0.744 & 0.705 & 0.717 & 0.71 & 0.71 & 0.698 & 0.709 & 0.879 & 0.179 & 0.145 & 0.134 & 0.123 & 0.127 & 0.125 & 0.125 & 0.121 & 0.125 & 0.179 \\ 
   &  & p=196 & 0.976 & 0.859 & 0.815 & 0.844 & 0.784 & 0.774 & 0.774 & 0.93 & 0.774 & 0.991 & 0.21 & 0.176 & 0.175 & 1.588 & 0.16 & 0.168 & 0.178 & 3.958 & 0.187 & 6.354 \\ 
  \cmidrule{2-23} & \multirow{3}[2]{*}{$\rho=0.9$} & p=12 & 0.073 & 0.073 & 0.073 & 0.073 & 0.073 & 0.073 & 0.073 & 0.073 & 0.073 & 0.073 & 0.008 & 0.008 & 0.008 & 0.008 & 0.008 & 0.008 & 0.008 & 0.008 & 0.008 & 0.008 \\ 
   &  & p=100 & 0.38 & 0.338 & 0.32 & 0.303 & 0.31 & 0.306 & 0.306 & 0.3 & 0.306 & 0.392 & 0.172 & 0.144 & 0.132 & 0.124 & 0.127 & 0.126 & 0.126 & 0.122 & 0.126 & 0.185 \\ 
   &  & p=196 & 0.405 & 0.361 & 0.344 & 0.358 & 0.332 & 0.328 & 0.328 & 0.396 & 0.328 & 0.425 & 0.193 & 0.165 & 0.166 & 1.541 & 0.151 & 0.159 & 0.177 & 3.939 & 0.177 & 6.212 \\ 
  \midrule\multirow{9}[6]{*}{msnr} & \multirow{3}[2]{*}{$\rho=0$} & p=12 & 0.74 & 0.739 & 0.739 & 0.739 & 0.744 & 0.739 & 0.739 & 0.738 & 0.739 & 0.768 & 0.066 & 0.066 & 0.066 & 0.066 & 0.067 & 0.066 & 0.066 & 0.066 & 0.066 & 0.073 \\ 
   &  & p=100 & 3.205 & 3.09 & 2.994 & 2.988 & 3.015 & 2.997 & 3 & 3.077 & 2.999 & 4.445 & 0.57 & 0.548 & 0.531 & 0.566 & 0.544 & 0.548 & 0.549 & 0.653 & 0.549 & 0.899 \\ 
   &  & p=196 & 3.293 & 3.171 & 3.262 & 3.664 & 3.1 & 3.078 & 3.081 & 4.231 & 3.083 & 4.629 & 0.589 & 0.567 & 0.713 & 11.964 & 0.565 & 0.568 & 0.569 & 29.288 & 0.629 & 17.897 \\ 
  \cmidrule{2-23} & \multirow{3}[2]{*}{$\rho=0.5$} & p=12 & 0.448 & 0.448 & 0.448 & 0.448 & 0.448 & 0.448 & 0.448 & 0.448 & 0.448 & 0.449 & 0.065 & 0.065 & 0.065 & 0.065 & 0.065 & 0.065 & 0.065 & 0.065 & 0.065 & 0.066 \\ 
   &  & p=100 & 1.872 & 1.807 & 1.751 & 1.747 & 1.766 & 1.753 & 1.755 & 1.798 & 1.755 & 2.586 & 0.567 & 0.545 & 0.529 & 0.563 & 0.544 & 0.547 & 0.548 & 0.649 & 0.548 & 0.889 \\ 
   &  & p=196 & 1.919 & 1.851 & 1.905 & 2.142 & 1.81 & 1.799 & 1.8 & 2.472 & 1.801 & 2.697 & 0.585 & 0.562 & 0.712 & 12.108 & 0.562 & 0.567 & 0.566 & 29.57 & 0.625 & 17.92 \\ 
  \cmidrule{2-23} & \multirow{3}[2]{*}{$\rho=0.9$} & p=12 & 0.213 & 0.213 & 0.213 & 0.213 & 0.213 & 0.213 & 0.213 & 0.213 & 0.213 & 0.215 & 0.065 & 0.065 & 0.065 & 0.065 & 0.065 & 0.065 & 0.065 & 0.065 & 0.065 & 0.068 \\ 
   &  & p=100 & 0.784 & 0.758 & 0.737 & 0.737 & 0.745 & 0.74 & 0.74 & 0.76 & 0.74 & 1.047 & 0.529 & 0.511 & 0.5 & 0.534 & 0.512 & 0.517 & 0.517 & 0.617 & 0.517 & 0.78 \\ 
   &  & p=196 & 0.798 & 0.773 & 0.799 & 0.905 & 0.758 & 0.753 & 0.753 & 1.046 & 0.753 & 1.103 & 0.542 & 0.525 & 0.672 & 12.205 & 0.525 & 0.528 & 0.528 & 28.767 & 0.582 & 16.895 \\ 
  \midrule\multirow{9}[6]{*}{lsnr} & \multirow{3}[2]{*}{$\rho=0$} & p=12 & 1.844 & 1.82 & 1.809 & 1.792 & 1.834 & 1.805 & 1.798 & 1.783 & 1.796 & 2.773 & 0.413 & 0.402 & 0.398 & 0.39 & 0.408 & 0.395 & 0.393 & 0.386 & 0.392 & 0.87 \\ 
   &  & p=100 & 4.826 & 4.845 & 4.847 & 4.925 & 4.862 & 4.876 & 4.881 & 5.039 & 4.882 & 4.919 & 1.065 & 1.088 & 1.119 & 1.193 & 1.115 & 1.133 & 1.135 & 1.298 & 1.136 & 1.015 \\ 
   &  & p=196 & 4.884 & 4.901 & 5.501 & 6.687 & 4.918 & 4.935 & 4.938 & 7.837 & 4.939 & 5.831 & 1.069 & 1.093 & 2.021 & 57.941 & 1.121 & 1.139 & 1.139 & 116.827 & 1.141 & 44.203 \\ 
  \cmidrule{2-23} & \multirow{3}[2]{*}{$\rho=0.5$} & p=12 & 1.126 & 1.109 & 1.101 & 1.089 & 1.116 & 1.095 & 1.094 & 1.081 & 1.093 & 1.698 & 0.418 & 0.406 & 0.399 & 0.39 & 0.409 & 0.394 & 0.394 & 0.385 & 0.394 & 0.876 \\ 
   &  & p=100 & 2.811 & 2.82 & 2.827 & 2.871 & 2.837 & 2.847 & 2.844 & 2.941 & 2.845 & 2.873 & 1.055 & 1.074 & 1.11 & 1.181 & 1.111 & 1.127 & 1.124 & 1.293 & 1.125 & 1.013 \\ 
   &  & p=196 & 2.844 & 2.85 & 3.209 & 3.893 & 2.869 & 2.878 & 2.878 & 4.572 & 2.878 & 3.406 & 1.059 & 1.079 & 2.011 & 58.924 & 1.116 & 1.129 & 1.13 & 115.687 & 1.131 & 44.261 \\ 
  \cmidrule{2-23} & \multirow{3}[2]{*}{$\rho=0.9$} & p=12 & 0.539 & 0.533 & 0.527 & 0.521 & 0.535 & 0.525 & 0.523 & 0.518 & 0.523 & 0.78 & 0.423 & 0.414 & 0.405 & 0.395 & 0.418 & 0.402 & 0.4 & 0.391 & 0.4 & 0.818 \\ 
   &  & p=100 & 1.143 & 1.149 & 1.158 & 1.176 & 1.156 & 1.161 & 1.161 & 1.198 & 1.162 & 1.171 & 0.943 & 0.966 & 1.004 & 1.067 & 0.995 & 1.009 & 1.008 & 1.148 & 1.013 & 0.925 \\ 
   &  & p=196 & 1.153 & 1.159 & 1.326 & 1.623 & 1.167 & 1.171 & 1.172 & 1.909 & 1.173 & 1.4 & 0.946 & 0.969 & 1.902 & 58.963 & 1.001 & 1.013 & 1.015 & 114.298 & 1.018 & 43.301 \\ 
   \midrule 
 \multicolumn{1}{|c}{} & \multicolumn{1}{c}{} &       & \multicolumn{10}{c||}{Relative Test Error}                                    & \multicolumn{10}{c|}{Proportion of Variance Explained} \\
\midrule\multirow{9}[6]{*}{hsnr} & \multirow{3}[2]{*}{$\rho=0$} & p=12 & 1.065 & 1.065 & 1.065 & 1.065 & 1.065 & 1.065 & 1.065 & 1.065 & 1.065 & 1.065 & 0.888 & 0.888 & 0.888 & 0.888 & 0.888 & 0.888 & 0.888 & 0.888 & 0.888 & 0.888 \\ 
   &  & p=100 & 2.524 & 2.225 & 2.129 & 2.034 & 2.062 & 2.048 & 2.046 & 2.015 & 2.046 & 2.487 & 0.734 & 0.765 & 0.775 & 0.785 & 0.782 & 0.784 & 0.784 & 0.787 & 0.784 & 0.737 \\ 
   &  & p=196 & 2.788 & 2.503 & 2.495 & 14.184 & 2.366 & 2.435 & 2.517 & 34.59 & 2.517 & 54.937 & 0.706 & 0.736 & 0.737 & -0.497 & 0.75 & 0.743 & 0.734 & -2.672 & 0.734 & -4.793 \\ 
  \cmidrule{2-23} & \multirow{3}[2]{*}{$\rho=0.5$} & p=12 & 1.065 & 1.065 & 1.065 & 1.065 & 1.065 & 1.065 & 1.065 & 1.065 & 1.065 & 1.065 & 0.888 & 0.888 & 0.888 & 0.888 & 0.888 & 0.888 & 0.888 & 0.888 & 0.888 & 0.888 \\ 
   &  & p=100 & 2.524 & 2.231 & 2.134 & 2.043 & 2.072 & 2.057 & 2.057 & 2.022 & 2.056 & 2.51 & 0.734 & 0.764 & 0.775 & 0.784 & 0.781 & 0.783 & 0.783 & 0.786 & 0.783 & 0.735 \\ 
   &  & p=196 & 2.782 & 2.494 & 2.487 & 14.436 & 2.357 & 2.43 & 2.51 & 34.395 & 2.585 & 54.863 & 0.707 & 0.737 & 0.738 & -0.525 & 0.751 & 0.744 & 0.735 & -2.641 & 0.727 & -4.784 \\ 
  \cmidrule{2-23} & \multirow{3}[2]{*}{$\rho=0.9$} & p=12 & 1.065 & 1.065 & 1.065 & 1.065 & 1.065 & 1.065 & 1.065 & 1.065 & 1.065 & 1.065 & 0.888 & 0.888 & 0.888 & 0.888 & 0.888 & 0.888 & 0.888 & 0.888 & 0.888 & 0.888 \\ 
   &  & p=100 & 2.464 & 2.222 & 2.123 & 2.05 & 2.077 & 2.064 & 2.065 & 2.03 & 2.065 & 2.566 & 0.74 & 0.766 & 0.776 & 0.784 & 0.781 & 0.782 & 0.782 & 0.786 & 0.782 & 0.729 \\ 
   &  & p=196 & 2.64 & 2.403 & 2.41 & 14.053 & 2.284 & 2.36 & 2.509 & 34.294 & 2.512 & 53.645 & 0.722 & 0.747 & 0.746 & -0.483 & 0.759 & 0.752 & 0.736 & -2.624 & 0.736 & -4.656 \\ 
  \midrule\multirow{9}[6]{*}{msnr} & \multirow{3}[2]{*}{$\rho=0$} & p=12 & 1.066 & 1.066 & 1.066 & 1.066 & 1.067 & 1.066 & 1.066 & 1.066 & 1.066 & 1.072 & 0.469 & 0.469 & 0.469 & 0.469 & 0.468 & 0.469 & 0.469 & 0.469 & 0.469 & 0.465 \\ 
   &  & p=100 & 1.569 & 1.547 & 1.53 & 1.565 & 1.543 & 1.547 & 1.548 & 1.651 & 1.548 & 1.903 & 0.216 & 0.227 & 0.235 & 0.218 & 0.228 & 0.226 & 0.226 & 0.174 & 0.226 & 0.05 \\ 
   &  & p=196 & 1.589 & 1.566 & 1.712 & 12.927 & 1.563 & 1.567 & 1.568 & 30.003 & 1.633 & 18.819 & 0.206 & 0.217 & 0.144 & -5.46 & 0.218 & 0.216 & 0.216 & -14.037 & 0.185 & -8.41 \\ 
  \cmidrule{2-23} & \multirow{3}[2]{*}{$\rho=0.5$} & p=12 & 1.065 & 1.065 & 1.065 & 1.065 & 1.065 & 1.065 & 1.065 & 1.065 & 1.065 & 1.066 & 0.469 & 0.469 & 0.469 & 0.469 & 0.469 & 0.469 & 0.469 & 0.469 & 0.469 & 0.469 \\ 
   &  & p=100 & 1.567 & 1.544 & 1.528 & 1.562 & 1.543 & 1.546 & 1.547 & 1.648 & 1.548 & 1.894 & 0.217 & 0.228 & 0.236 & 0.219 & 0.229 & 0.227 & 0.227 & 0.176 & 0.226 & 0.055 \\ 
   &  & p=196 & 1.584 & 1.562 & 1.711 & 13.061 & 1.561 & 1.566 & 1.565 & 30.292 & 1.63 & 18.844 & 0.208 & 0.219 & 0.145 & -5.529 & 0.22 & 0.217 & 0.217 & -14.178 & 0.187 & -8.422 \\ 
  \cmidrule{2-23} & \multirow{3}[2]{*}{$\rho=0.9$} & p=12 & 1.065 & 1.065 & 1.065 & 1.065 & 1.065 & 1.065 & 1.065 & 1.065 & 1.065 & 1.068 & 0.469 & 0.469 & 0.469 & 0.469 & 0.469 & 0.469 & 0.469 & 0.469 & 0.469 & 0.467 \\ 
   &  & p=100 & 1.529 & 1.511 & 1.5 & 1.534 & 1.511 & 1.517 & 1.516 & 1.617 & 1.516 & 1.785 & 0.236 & 0.245 & 0.251 & 0.234 & 0.245 & 0.242 & 0.242 & 0.192 & 0.242 & 0.11 \\ 
   &  & p=196 & 1.542 & 1.525 & 1.673 & 13.152 & 1.524 & 1.528 & 1.528 & 29.619 & 1.592 & 17.835 & 0.23 & 0.238 & 0.164 & -5.576 & 0.238 & 0.237 & 0.236 & -13.808 & 0.207 & -7.914 \\ 
  \midrule\multirow{9}[6]{*}{lsnr} & \multirow{3}[2]{*}{$\rho=0$} & p=12 & 1.083 & 1.08 & 1.079 & 1.078 & 1.082 & 1.079 & 1.078 & 1.077 & 1.078 & 1.176 & 0.099 & 0.101 & 0.102 & 0.103 & 0.1 & 0.102 & 0.103 & 0.104 & 0.103 & 0.022 \\ 
   &  & p=100 & 1.213 & 1.217 & 1.223 & 1.238 & 1.222 & 1.226 & 1.226 & 1.259 & 1.227 & 1.204 & -0.01 & -0.014 & -0.019 & -0.031 & -0.018 & -0.021 & -0.022 & -0.049 & -0.022 & -0.003 \\ 
   &  & p=196 & 1.214 & 1.218 & 1.403 & 12.549 & 1.224 & 1.227 & 1.227 & 24.281 & 1.228 & 9.769 & -0.011 & -0.015 & -0.169 & -9.451 & -0.019 & -0.022 & -0.022 & -19.218 & -0.023 & -7.141 \\ 
  \cmidrule{2-23} & \multirow{3}[2]{*}{$\rho=0.5$} & p=12 & 1.084 & 1.081 & 1.08 & 1.078 & 1.082 & 1.079 & 1.079 & 1.077 & 1.079 & 1.177 & 0.098 & 0.101 & 0.102 & 0.103 & 0.1 & 0.102 & 0.102 & 0.104 & 0.103 & 0.021 \\ 
   &  & p=100 & 1.211 & 1.215 & 1.222 & 1.236 & 1.222 & 1.225 & 1.224 & 1.258 & 1.225 & 1.203 & -0.009 & -0.012 & -0.018 & -0.029 & -0.018 & -0.02 & -0.02 & -0.048 & -0.02 & -0.002 \\ 
   &  & p=196 & 1.212 & 1.216 & 1.401 & 12.73 & 1.223 & 1.225 & 1.226 & 24.023 & 1.226 & 9.781 & -0.009 & -0.012 & -0.167 & -9.604 & -0.018 & -0.021 & -0.021 & -19.008 & -0.021 & -7.151 \\ 
  \cmidrule{2-23} & \multirow{3}[2]{*}{$\rho=0.9$} & p=12 & 1.084 & 1.083 & 1.081 & 1.079 & 1.083 & 1.08 & 1.08 & 1.078 & 1.08 & 1.165 & 0.098 & 0.099 & 0.101 & 0.102 & 0.098 & 0.101 & 0.101 & 0.103 & 0.101 & 0.031 \\ 
   &  & p=100 & 1.189 & 1.193 & 1.2 & 1.213 & 1.199 & 1.202 & 1.201 & 1.229 & 1.202 & 1.186 & 0.01 & 0.006 & 0 & -0.01 & 0.002 & -0.001 & 0 & -0.024 & -0.001 & 0.013 \\ 
   &  & p=196 & 1.189 & 1.194 & 1.381 & 12.729 & 1.2 & 1.202 & 1.203 & 23.838 & 1.203 & 9.618 & 0.01 & 0.006 & -0.15 & -9.604 & 0.001 & -0.001 & -0.002 & -18.84 & -0.002 & -7.012 \\ 
   \midrule 
 \multicolumn{1}{|c}{} & \multicolumn{1}{c}{} &       & \multicolumn{10}{c||}{Kullback-Leibler Discrepancy}                                    & \multicolumn{10}{c|}{Number of Variables} \\
\midrule\multirow{9}[6]{*}{hsnr} & \multirow{3}[2]{*}{$\rho=0$} & p=12 & 1.167 & 1.167 & 1.167 & 1.167 & 1.167 & 1.167 & 1.167 & 1.167 & 1.167 & 1.167 & 12 & 12 & 12 & 12 & 12 & 12 & 12 & 12 & 12 & 12 \\ 
   &  & p=100 & 3.358 & 3.496 & 3.564 & 4.103 & 3.917 & 4.04 & 4.046 & 4.301 & 4.047 & 3.506 & 55.826 & 67.582 & 72.802 & 88.015 & 83.049 & 85.879 & 86.006 & 94.321 & 86.044 & 59.991 \\ 
   &  & p=196 & 3.442 & 3.575 & 4.49 & 2652.866 & 4.14 & 116.711 & 225.922 & 4380.182 & 225.933 & 5790.051 & 53.883 & 65.085 & 80.745 & 117.638 & 81.695 & 86.345 & 86.387 & 169.849 & 86.585 & 169.976 \\ 
  \cmidrule{2-23} & \multirow{3}[2]{*}{$\rho=0.5$} & p=12 & 0.175 & 0.175 & 0.175 & 0.175 & 0.175 & 0.175 & 0.175 & 0.175 & 0.175 & 0.175 & 12 & 12 & 12 & 12 & 12 & 12 & 12 & 12 & 12 & 12 \\ 
   &  & p=100 & 2.284 & 2.423 & 2.491 & 3.054 & 2.844 & 2.977 & 2.983 & 3.243 & 2.988 & 2.422 & 55.66 & 67.417 & 72.672 & 88.34 & 82.784 & 85.833 & 85.932 & 94.551 & 86.108 & 59.11 \\ 
   &  & p=196 & 2.365 & 2.494 & 3.388 & 2723.87 & 3.057 & 115.643 & 224.829 & 4396.314 & 277.914 & 5786.601 & 53.649 & 64.901 & 80.478 & 117.462 & 81.525 & 86.18 & 86.141 & 169.289 & 86.453 & 169.333 \\ 
  \cmidrule{2-23} & \multirow{3}[2]{*}{$\rho=0.9$} & p=12 & -1.317 & -1.317 & -1.317 & -1.317 & -1.317 & -1.317 & -1.317 & -1.317 & -1.317 & -1.317 & 12 & 12 & 12 & 12 & 12 & 12 & 12 & 12 & 12 & 12 \\ 
   &  & p=100 & 0.529 & 0.645 & 0.733 & 1.348 & 1.094 & 1.24 & 1.252 & 1.559 & 1.257 & 0.639 & 53.618 & 64.265 & 70.539 & 87.61 & 80.67 & 84.158 & 84.354 & 94.462 & 84.465 & 53.586 \\ 
   &  & p=196 & 0.59 & 0.71 & 1.578 & 2631.15 & 1.177 & 113.727 & 284.973 & 4360.501 & 285.012 & 5738.959 & 52.281 & 62.89 & 78.332 & 115.356 & 78.433 & 83.125 & 83.312 & 167.957 & 83.617 & 164.742 \\ 
  \midrule\multirow{9}[6]{*}{msnr} & \multirow{3}[2]{*}{$\rho=0$} & p=12 & 3.308 & 3.307 & 3.307 & 3.307 & 3.308 & 3.307 & 3.307 & 3.307 & 3.307 & 3.314 & 11.974 & 11.977 & 11.977 & 11.979 & 11.958 & 11.977 & 11.979 & 11.98 & 11.979 & 11.853 \\ 
   &  & p=100 & 4.723 & 4.748 & 4.786 & 4.973 & 4.834 & 4.876 & 4.877 & 5.316 & 4.878 & 4.837 & 27.204 & 32.404 & 38.352 & 48.658 & 40.212 & 43.12 & 43.027 & 59.917 & 43.179 & 4.723 \\ 
   &  & p=196 & 4.752 & 4.777 & 5.462 & 2603.495 & 4.856 & 4.896 & 4.897 & 4175.849 & 117.165 & 3700.415 & 26.635 & 31.957 & 48.554 & 84.64 & 39.34 & 42.117 & 42.006 & 140.212 & 42.347 & 57.391 \\ 
  \cmidrule{2-23} & \multirow{3}[2]{*}{$\rho=0.5$} & p=12 & 2.315 & 2.315 & 2.315 & 2.315 & 2.315 & 2.315 & 2.315 & 2.315 & 2.315 & 2.316 & 11.999 & 11.999 & 11.999 & 11.999 & 11.997 & 11.999 & 11.999 & 11.999 & 11.999 & 11.989 \\ 
   &  & p=100 & 3.649 & 3.672 & 3.711 & 3.893 & 3.761 & 3.799 & 3.803 & 4.237 & 3.806 & 3.761 & 27.025 & 32.107 & 38.149 & 48.392 & 39.984 & 42.584 & 42.646 & 59.56 & 42.881 & 4.732 \\ 
   &  & p=196 & 3.676 & 3.699 & 4.394 & 2679.53 & 3.78 & 3.826 & 3.821 & 4221.818 & 116.087 & 3702.617 & 26.595 & 31.74 & 48.634 & 84.323 & 39.128 & 41.951 & 41.659 & 139.794 & 41.903 & 57.606 \\ 
  \cmidrule{2-23} & \multirow{3}[2]{*}{$\rho=0.9$} & p=12 & 0.823 & 0.823 & 0.823 & 0.823 & 0.823 & 0.823 & 0.823 & 0.823 & 0.823 & 0.825 & 11.996 & 11.996 & 11.996 & 11.998 & 11.996 & 11.996 & 11.998 & 11.998 & 11.998 & 11.952 \\ 
   &  & p=100 & 1.92 & 1.943 & 1.976 & 2.143 & 2.008 & 2.05 & 2.05 & 2.456 & 2.05 & 2.005 & 24.918 & 30.079 & 35.452 & 45.565 & 36.438 & 39.344 & 39.199 & 55.872 & 39.242 & 4.114 \\ 
   &  & p=196 & 1.941 & 1.962 & 2.637 & 2623.849 & 2.025 & 2.064 & 2.063 & 4122.364 & 114.329 & 3605.77 & 24.731 & 29.51 & 46.436 & 82.212 & 35.892 & 38.593 & 38.704 & 137.113 & 38.995 & 53.653 \\ 
  \midrule\multirow{9}[6]{*}{lsnr} & \multirow{3}[2]{*}{$\rho=0$} & p=12 & 4.933 & 4.931 & 4.93 & 4.928 & 4.932 & 4.929 & 4.929 & 4.928 & 4.929 & 5.01 & 10.138 & 10.32 & 10.408 & 10.545 & 10.214 & 10.443 & 10.486 & 10.608 & 10.499 & 2.528 \\ 
   &  & p=100 & 6.003 & 6.013 & 6.027 & 6.064 & 6.025 & 6.035 & 6.034 & 6.121 & 6.034 & 5.982 & 6.766 & 8.183 & 11.193 & 14.521 & 10.066 & 10.796 & 10.821 & 16.359 & 10.868 & 0.323 \\ 
   &  & p=196 & 6.021 & 6.031 & 6.698 & 2600.217 & 6.046 & 6.054 & 6.053 & 3847.648 & 6.054 & 2701.179 & 6.536 & 7.878 & 27.802 & 61.646 & 9.888 & 10.564 & 10.65 & 98.208 & 10.719 & 28.266 \\ 
  \cmidrule{2-23} & \multirow{3}[2]{*}{$\rho=0.5$} & p=12 & 3.942 & 3.94 & 3.938 & 3.937 & 3.94 & 3.938 & 3.937 & 3.936 & 3.937 & 4.019 & 10.304 & 10.514 & 10.636 & 10.789 & 10.482 & 10.711 & 10.719 & 10.88 & 10.727 & 2.382 \\ 
   &  & p=100 & 4.929 & 4.937 & 4.954 & 4.988 & 4.953 & 4.962 & 4.96 & 5.05 & 4.96 & 4.91 & 6.716 & 7.945 & 11.052 & 14.268 & 10.05 & 10.832 & 10.731 & 16.353 & 10.786 & 0.371 \\ 
   &  & p=196 & 4.947 & 4.955 & 5.621 & 2686.135 & 4.971 & 4.978 & 4.979 & 3869.607 & 4.979 & 2701.919 & 6.522 & 7.867 & 27.658 & 61.116 & 9.903 & 10.414 & 10.527 & 97.598 & 10.56 & 28.285 \\ 
  \cmidrule{2-23} & \multirow{3}[2]{*}{$\rho=0.9$} & p=12 & 2.45 & 2.449 & 2.447 & 2.445 & 2.449 & 2.446 & 2.446 & 2.444 & 2.446 & 2.515 & 10.042 & 10.213 & 10.387 & 10.59 & 10.183 & 10.457 & 10.489 & 10.662 & 10.49 & 2.137 \\ 
   &  & p=100 & 3.214 & 3.222 & 3.237 & 3.265 & 3.234 & 3.24 & 3.239 & 3.314 & 3.241 & 3.201 & 6.143 & 7.236 & 9.672 & 12.358 & 8.747 & 9.489 & 9.304 & 13.891 & 9.484 & 0.806 \\ 
   &  & p=196 & 3.227 & 3.235 & 3.898 & 2615.265 & 3.248 & 3.253 & 3.253 & 3799.924 & 3.255 & 2673.846 & 6.005 & 7.139 & 26.722 & 60.066 & 8.7 & 9.35 & 9.31 & 95.353 & 9.475 & 27.9 \\ 
   \bottomrule 
\end{tabular}
}
\end{table}

%% latex table generated in R 4.0.2 by xtable 1.8-4 package
% Fri Jun 26 10:43:48 2020
\begin{table}[ht]
\centering
\caption{Fixed X, Dense-Ex2, n=1000} 
\scalebox{0.52}{
\begin{tabular}{|c|c|c|cc|cc|cc|ccc|c||cc|cc|cc|ccc|c|}
  \toprule 
 \multicolumn{1}{|c}{} & \multicolumn{1}{c}{} &       & RAICc & AICc  & RC$_p$ & C$_p$ & 10F CV & LOO CV & S$_p$ & $\widehat{\text{C}}_p$ & GCV   & BIC   & RAICc & AICc  & RC$_p$ & C$_p$ & 10F CV & LOO CV & S$_p$ & $\widehat{\text{C}}_p$ & GCV   & BIC  \\
 \cmidrule{4-23}\multicolumn{1}{|c}{} & \multicolumn{1}{c}{} &       & \multicolumn{10}{c||}{RMSE}                                                   & \multicolumn{10}{c|}{Relative Risk}       \\
 \midrule\multirow{9}[6]{*}{hsnr} & \multirow{3}[2]{*}{$\rho=0$} & p=12 & 0.112 & 0.112 & 0.112 & 0.112 & 0.112 & 0.112 & 0.112 & 0.112 & 0.112 & 0.112 & 0.001 & 0.001 & 0.001 & 0.001 & 0.001 & 0.001 & 0.001 & 0.001 & 0.001 & 0.001 \\ 
   &  & p=500 & 0.711 & 0.704 & 0.703 & 0.7 & 0.702 & 0.7 & 0.7 & 0.704 & 0.7 & 0.861 & 0.025 & 0.024 & 0.024 & 0.024 & 0.024 & 0.024 & 0.024 & 0.025 & 0.024 & 0.035 \\ 
   &  & p=996 & 0.711 & 0.704 & 0.799 & 0.981 & 0.702 & 0.7 & 0.7 & 1.151 & 0.7 & 1.098 & 0.025 & 0.024 & 0.044 & 5.721 & 0.024 & 0.024 & 0.024 & 12.06 & 0.024 & 8.646 \\ 
  \cmidrule{2-23} & \multirow{3}[2]{*}{$\rho=0.5$} & p=12 & 0.068 & 0.068 & 0.068 & 0.068 & 0.068 & 0.068 & 0.068 & 0.068 & 0.068 & 0.068 & 0.001 & 0.001 & 0.001 & 0.001 & 0.001 & 0.001 & 0.001 & 0.001 & 0.001 & 0.001 \\ 
   &  & p=500 & 0.416 & 0.412 & 0.411 & 0.41 & 0.411 & 0.41 & 0.41 & 0.412 & 0.41 & 0.506 & 0.025 & 0.024 & 0.024 & 0.024 & 0.024 & 0.024 & 0.024 & 0.025 & 0.024 & 0.036 \\ 
   &  & p=996 & 0.416 & 0.412 & 0.467 & 0.574 & 0.411 & 0.41 & 0.41 & 0.679 & 0.41 & 0.644 & 0.025 & 0.024 & 0.044 & 5.73 & 0.024 & 0.024 & 0.024 & 12.4 & 0.024 & 8.669 \\ 
  \cmidrule{2-23} & \multirow{3}[2]{*}{$\rho=0.9$} & p=12 & 0.032 & 0.032 & 0.032 & 0.032 & 0.032 & 0.032 & 0.032 & 0.032 & 0.032 & 0.032 & 0.001 & 0.001 & 0.001 & 0.001 & 0.001 & 0.001 & 0.001 & 0.001 & 0.001 & 0.001 \\ 
   &  & p=500 & 0.176 & 0.175 & 0.175 & 0.174 & 0.175 & 0.174 & 0.174 & 0.175 & 0.174 & 0.218 & 0.024 & 0.024 & 0.024 & 0.024 & 0.024 & 0.024 & 0.024 & 0.024 & 0.024 & 0.036 \\ 
   &  & p=996 & 0.176 & 0.175 & 0.199 & 0.245 & 0.175 & 0.174 & 0.174 & 0.286 & 0.174 & 0.274 & 0.024 & 0.024 & 0.044 & 5.883 & 0.024 & 0.024 & 0.024 & 12.096 & 0.024 & 8.454 \\ 
  \midrule\multirow{9}[6]{*}{msnr} & \multirow{3}[2]{*}{$\rho=0$} & p=12 & 0.327 & 0.327 & 0.327 & 0.327 & 0.327 & 0.327 & 0.327 & 0.327 & 0.327 & 0.327 & 0.012 & 0.012 & 0.012 & 0.012 & 0.012 & 0.012 & 0.012 & 0.012 & 0.012 & 0.012 \\ 
   &  & p=500 & 1.733 & 1.726 & 1.724 & 1.724 & 1.724 & 1.724 & 1.726 & 1.731 & 1.725 & 2.334 & 0.137 & 0.136 & 0.136 & 0.137 & 0.136 & 0.136 & 0.137 & 0.138 & 0.137 & 0.237 \\ 
   &  & p=996 & 1.733 & 1.726 & 2.084 & 2.624 & 1.724 & 1.724 & 1.726 & 3.094 & 1.725 & 2.798 & 0.137 & 0.136 & 0.317 & 48.012 & 0.136 & 0.136 & 0.137 & 98.667 & 0.137 & 48.066 \\ 
  \cmidrule{2-23} & \multirow{3}[2]{*}{$\rho=0.5$} & p=12 & 0.199 & 0.199 & 0.199 & 0.199 & 0.199 & 0.199 & 0.199 & 0.199 & 0.199 & 0.199 & 0.012 & 0.012 & 0.012 & 0.012 & 0.012 & 0.012 & 0.012 & 0.012 & 0.012 & 0.012 \\ 
   &  & p=500 & 1.012 & 1.009 & 1.007 & 1.009 & 1.008 & 1.008 & 1.009 & 1.012 & 1.009 & 1.365 & 0.137 & 0.136 & 0.136 & 0.137 & 0.136 & 0.136 & 0.137 & 0.138 & 0.137 & 0.237 \\ 
   &  & p=996 & 1.012 & 1.009 & 1.22 & 1.535 & 1.008 & 1.008 & 1.009 & 1.807 & 1.009 & 1.638 & 0.137 & 0.136 & 0.317 & 48.032 & 0.136 & 0.136 & 0.137 & 98.408 & 0.137 & 48.176 \\ 
  \cmidrule{2-23} & \multirow{3}[2]{*}{$\rho=0.9$} & p=12 & 0.094 & 0.094 & 0.094 & 0.094 & 0.094 & 0.094 & 0.094 & 0.094 & 0.094 & 0.094 & 0.012 & 0.012 & 0.012 & 0.012 & 0.012 & 0.012 & 0.012 & 0.012 & 0.012 & 0.012 \\ 
   &  & p=500 & 0.426 & 0.425 & 0.425 & 0.426 & 0.425 & 0.425 & 0.425 & 0.428 & 0.426 & 0.582 & 0.132 & 0.131 & 0.131 & 0.133 & 0.132 & 0.132 & 0.132 & 0.134 & 0.132 & 0.235 \\ 
   &  & p=996 & 0.426 & 0.425 & 0.518 & 0.654 & 0.425 & 0.425 & 0.425 & 0.77 & 0.426 & 0.693 & 0.132 & 0.131 & 0.315 & 49.284 & 0.132 & 0.132 & 0.132 & 97.977 & 0.132 & 45.934 \\ 
  \midrule\multirow{9}[6]{*}{lsnr} & \multirow{3}[2]{*}{$\rho=0$} & p=12 & 0.733 & 0.733 & 0.733 & 0.733 & 0.735 & 0.733 & 0.733 & 0.733 & 0.733 & 0.791 & 0.061 & 0.061 & 0.061 & 0.061 & 0.061 & 0.061 & 0.061 & 0.061 & 0.061 & 0.073 \\ 
   &  & p=500 & 3.182 & 3.174 & 3.17 & 3.173 & 3.186 & 3.175 & 3.176 & 3.182 & 3.176 & 4.543 & 0.438 & 0.437 & 0.437 & 0.44 & 0.44 & 0.439 & 0.439 & 0.443 & 0.439 & 0.863 \\ 
   &  & p=996 & 3.182 & 3.174 & 4.158 & 5.411 & 3.186 & 3.175 & 3.176 & 6.334 & 3.176 & 5.31 & 0.438 & 0.437 & 1.392 & 239.878 & 0.44 & 0.439 & 0.439 & 469.897 & 0.439 & 165.154 \\ 
  \cmidrule{2-23} & \multirow{3}[2]{*}{$\rho=0.5$} & p=12 & 0.445 & 0.445 & 0.445 & 0.445 & 0.445 & 0.445 & 0.445 & 0.445 & 0.445 & 0.45 & 0.061 & 0.061 & 0.061 & 0.061 & 0.06 & 0.061 & 0.061 & 0.061 & 0.061 & 0.062 \\ 
   &  & p=500 & 1.859 & 1.857 & 1.853 & 1.854 & 1.863 & 1.857 & 1.857 & 1.861 & 1.857 & 2.639 & 0.437 & 0.437 & 0.436 & 0.439 & 0.44 & 0.439 & 0.439 & 0.443 & 0.439 & 0.851 \\ 
   &  & p=996 & 1.859 & 1.857 & 2.429 & 3.168 & 1.863 & 1.857 & 1.857 & 3.713 & 1.857 & 3.09 & 0.437 & 0.437 & 1.393 & 239.99 & 0.44 & 0.439 & 0.439 & 475.091 & 0.439 & 165.499 \\ 
  \cmidrule{2-23} & \multirow{3}[2]{*}{$\rho=0.9$} & p=12 & 0.211 & 0.211 & 0.211 & 0.211 & 0.211 & 0.211 & 0.211 & 0.211 & 0.211 & 0.215 & 0.061 & 0.061 & 0.061 & 0.061 & 0.061 & 0.061 & 0.061 & 0.061 & 0.061 & 0.064 \\ 
   &  & p=500 & 0.781 & 0.78 & 0.779 & 0.78 & 0.782 & 0.78 & 0.78 & 0.782 & 0.78 & 1.071 & 0.42 & 0.42 & 0.419 & 0.422 & 0.423 & 0.421 & 0.421 & 0.425 & 0.421 & 0.765 \\ 
   &  & p=996 & 0.781 & 0.78 & 1.028 & 1.345 & 0.782 & 0.78 & 0.78 & 1.573 & 0.78 & 1.268 & 0.42 & 0.42 & 1.385 & 246.247 & 0.423 & 0.421 & 0.421 & 466.027 & 0.421 & 159.91 \\ 
   \midrule 
 \multicolumn{1}{|c}{} & \multicolumn{1}{c}{} &       & \multicolumn{10}{c||}{Relative Test Error}                                    & \multicolumn{10}{c|}{Proportion of Variance Explained} \\
\midrule\multirow{9}[6]{*}{hsnr} & \multirow{3}[2]{*}{$\rho=0$} & p=12 & 1.012 & 1.012 & 1.012 & 1.012 & 1.012 & 1.012 & 1.012 & 1.012 & 1.012 & 1.012 & 0.894 & 0.894 & 0.894 & 0.894 & 0.894 & 0.894 & 0.894 & 0.894 & 0.894 & 0.894 \\ 
   &  & p=500 & 1.213 & 1.209 & 1.208 & 1.207 & 1.208 & 1.207 & 1.207 & 1.211 & 1.207 & 1.302 & 0.873 & 0.873 & 0.873 & 0.874 & 0.873 & 0.874 & 0.874 & 0.873 & 0.874 & 0.864 \\ 
   &  & p=996 & 1.213 & 1.209 & 1.375 & 49.91 & 1.208 & 1.207 & 1.207 & 104.096 & 1.207 & 74.91 & 0.873 & 0.873 & 0.856 & -4.227 & 0.873 & 0.874 & 0.874 & -9.902 & 0.874 & -6.845 \\ 
  \cmidrule{2-23} & \multirow{3}[2]{*}{$\rho=0.5$} & p=12 & 1.012 & 1.012 & 1.012 & 1.012 & 1.012 & 1.012 & 1.012 & 1.012 & 1.012 & 1.012 & 0.894 & 0.894 & 0.894 & 0.894 & 0.894 & 0.894 & 0.894 & 0.894 & 0.894 & 0.894 \\ 
   &  & p=500 & 1.212 & 1.208 & 1.208 & 1.207 & 1.208 & 1.207 & 1.207 & 1.21 & 1.207 & 1.303 & 0.873 & 0.873 & 0.873 & 0.873 & 0.873 & 0.873 & 0.873 & 0.873 & 0.873 & 0.863 \\ 
   &  & p=996 & 1.212 & 1.208 & 1.375 & 49.868 & 1.208 & 1.207 & 1.207 & 106.747 & 1.207 & 74.932 & 0.873 & 0.873 & 0.856 & -4.234 & 0.873 & 0.873 & 0.873 & -10.203 & 0.873 & -6.864 \\ 
  \cmidrule{2-23} & \multirow{3}[2]{*}{$\rho=0.9$} & p=12 & 1.012 & 1.012 & 1.012 & 1.012 & 1.012 & 1.012 & 1.012 & 1.012 & 1.012 & 1.012 & 0.894 & 0.894 & 0.894 & 0.894 & 0.894 & 0.894 & 0.894 & 0.894 & 0.894 & 0.894 \\ 
   &  & p=500 & 1.206 & 1.203 & 1.202 & 1.202 & 1.203 & 1.202 & 1.202 & 1.205 & 1.202 & 1.303 & 0.873 & 0.873 & 0.873 & 0.873 & 0.873 & 0.873 & 0.873 & 0.873 & 0.873 & 0.863 \\ 
   &  & p=996 & 1.206 & 1.203 & 1.371 & 50.868 & 1.203 & 1.202 & 1.202 & 103.534 & 1.202 & 72.665 & 0.873 & 0.873 & 0.855 & -4.368 & 0.873 & 0.873 & 0.873 & -9.925 & 0.873 & -6.668 \\ 
  \midrule\multirow{9}[6]{*}{msnr} & \multirow{3}[2]{*}{$\rho=0$} & p=12 & 1.012 & 1.012 & 1.012 & 1.012 & 1.012 & 1.012 & 1.012 & 1.012 & 1.012 & 1.012 & 0.496 & 0.496 & 0.496 & 0.496 & 0.496 & 0.496 & 0.496 & 0.496 & 0.496 & 0.496 \\ 
   &  & p=500 & 1.138 & 1.137 & 1.137 & 1.137 & 1.137 & 1.137 & 1.137 & 1.139 & 1.137 & 1.238 & 0.433 & 0.433 & 0.433 & 0.433 & 0.433 & 0.433 & 0.433 & 0.432 & 0.433 & 0.383 \\ 
   &  & p=996 & 1.138 & 1.137 & 1.318 & 49.285 & 1.137 & 1.137 & 1.137 & 100.229 & 1.137 & 49.339 & 0.433 & 0.433 & 0.343 & -23.573 & 0.433 & 0.433 & 0.433 & -48.972 & 0.433 & -23.6 \\ 
  \cmidrule{2-23} & \multirow{3}[2]{*}{$\rho=0.5$} & p=12 & 1.012 & 1.012 & 1.012 & 1.012 & 1.012 & 1.012 & 1.012 & 1.012 & 1.012 & 1.012 & 0.496 & 0.496 & 0.496 & 0.496 & 0.496 & 0.496 & 0.496 & 0.496 & 0.496 & 0.496 \\ 
   &  & p=500 & 1.137 & 1.137 & 1.136 & 1.137 & 1.136 & 1.137 & 1.137 & 1.139 & 1.137 & 1.237 & 0.432 & 0.433 & 0.433 & 0.432 & 0.433 & 0.433 & 0.432 & 0.432 & 0.432 & 0.382 \\ 
   &  & p=996 & 1.137 & 1.137 & 1.318 & 49.192 & 1.136 & 1.137 & 1.137 & 99.735 & 1.137 & 49.336 & 0.432 & 0.433 & 0.342 & -23.555 & 0.433 & 0.433 & 0.432 & -48.785 & 0.432 & -23.627 \\ 
  \cmidrule{2-23} & \multirow{3}[2]{*}{$\rho=0.9$} & p=12 & 1.012 & 1.012 & 1.012 & 1.012 & 1.012 & 1.012 & 1.012 & 1.012 & 1.012 & 1.012 & 0.496 & 0.496 & 0.496 & 0.496 & 0.496 & 0.496 & 0.496 & 0.496 & 0.496 & 0.496 \\ 
   &  & p=500 & 1.132 & 1.131 & 1.131 & 1.132 & 1.131 & 1.132 & 1.132 & 1.134 & 1.132 & 1.234 & 0.433 & 0.434 & 0.434 & 0.433 & 0.433 & 0.433 & 0.433 & 0.432 & 0.433 & 0.382 \\ 
   &  & p=996 & 1.132 & 1.131 & 1.315 & 50.15 & 1.131 & 1.132 & 1.132 & 98.71 & 1.132 & 46.808 & 0.433 & 0.434 & 0.342 & -24.109 & 0.433 & 0.433 & 0.433 & -48.423 & 0.433 & -22.436 \\ 
  \midrule\multirow{9}[6]{*}{lsnr} & \multirow{3}[2]{*}{$\rho=0$} & p=12 & 1.012 & 1.012 & 1.012 & 1.012 & 1.012 & 1.012 & 1.012 & 1.012 & 1.012 & 1.015 & 0.158 & 0.158 & 0.158 & 0.158 & 0.158 & 0.158 & 0.158 & 0.158 & 0.158 & 0.156 \\ 
   &  & p=500 & 1.088 & 1.088 & 1.088 & 1.088 & 1.089 & 1.088 & 1.088 & 1.089 & 1.088 & 1.174 & 0.094 & 0.094 & 0.094 & 0.094 & 0.094 & 0.094 & 0.094 & 0.093 & 0.094 & 0.023 \\ 
   &  & p=996 & 1.088 & 1.088 & 1.28 & 49.249 & 1.089 & 1.088 & 1.088 & 95.515 & 1.088 & 34.219 & 0.094 & 0.094 & -0.066 & -40.002 & 0.094 & 0.094 & 0.094 & -78.52 & 0.094 & -27.489 \\ 
  \cmidrule{2-23} & \multirow{3}[2]{*}{$\rho=0.5$} & p=12 & 1.012 & 1.012 & 1.012 & 1.012 & 1.012 & 1.012 & 1.012 & 1.012 & 1.012 & 1.013 & 0.158 & 0.158 & 0.158 & 0.158 & 0.158 & 0.158 & 0.158 & 0.158 & 0.158 & 0.157 \\ 
   &  & p=500 & 1.088 & 1.088 & 1.087 & 1.088 & 1.088 & 1.088 & 1.088 & 1.089 & 1.088 & 1.171 & 0.094 & 0.094 & 0.094 & 0.094 & 0.094 & 0.094 & 0.094 & 0.093 & 0.094 & 0.025 \\ 
   &  & p=996 & 1.088 & 1.088 & 1.28 & 49.158 & 1.088 & 1.088 & 1.088 & 96.334 & 1.088 & 34.21 & 0.094 & 0.094 & -0.066 & -39.942 & 0.094 & 0.094 & 0.094 & -79.234 & 0.094 & -27.492 \\ 
  \cmidrule{2-23} & \multirow{3}[2]{*}{$\rho=0.9$} & p=12 & 1.012 & 1.012 & 1.012 & 1.012 & 1.012 & 1.012 & 1.012 & 1.012 & 1.012 & 1.013 & 0.158 & 0.158 & 0.158 & 0.158 & 0.158 & 0.158 & 0.158 & 0.158 & 0.158 & 0.157 \\ 
   &  & p=500 & 1.084 & 1.084 & 1.084 & 1.084 & 1.084 & 1.084 & 1.084 & 1.085 & 1.084 & 1.153 & 0.097 & 0.096 & 0.097 & 0.096 & 0.096 & 0.096 & 0.096 & 0.096 & 0.096 & 0.039 \\ 
   &  & p=996 & 1.084 & 1.084 & 1.276 & 50.115 & 1.084 & 1.084 & 1.084 & 93.952 & 1.084 & 32.895 & 0.097 & 0.096 & -0.064 & -40.782 & 0.096 & 0.096 & 0.096 & -77.329 & 0.096 & -26.425 \\ 
   \midrule 
 \multicolumn{1}{|c}{} & \multicolumn{1}{c}{} &       & \multicolumn{10}{c||}{Kullback-Leibler Discrepancy}                                    & \multicolumn{10}{c|}{Number of Variables} \\
\midrule\multirow{9}[6]{*}{hsnr} & \multirow{3}[2]{*}{$\rho=0$} & p=12 & 1.098 & 1.098 & 1.098 & 1.098 & 1.098 & 1.098 & 1.098 & 1.098 & 1.098 & 1.098 & 12 & 12 & 12 & 12 & 12 & 12 & 12 & 12 & 12 & 12 \\ 
   &  & p=500 & 2.255 & 2.256 & 2.256 & 2.26 & 2.258 & 2.26 & 2.26 & 2.269 & 2.26 & 2.305 & 127.946 & 133.182 & 134.083 & 141.219 & 137.597 & 140.079 & 140.128 & 147.832 & 140.136 & 98.817 \\ 
   &  & p=996 & 2.255 & 2.256 & 2.497 & 391.623 & 2.258 & 2.26 & 2.26 & 637.167 & 2.26 & 603.846 & 127.946 & 133.182 & 193.613 & 363.155 & 137.593 & 140.071 & 140.125 & 530.445 & 140.136 & 352.479 \\ 
  \cmidrule{2-23} & \multirow{3}[2]{*}{$\rho=0.5$} & p=12 & 0.108 & 0.108 & 0.108 & 0.108 & 0.108 & 0.108 & 0.108 & 0.108 & 0.108 & 0.108 & 12 & 12 & 12 & 12 & 12 & 12 & 12 & 12 & 12 & 12 \\ 
   &  & p=500 & 1.184 & 1.185 & 1.186 & 1.19 & 1.188 & 1.189 & 1.189 & 1.198 & 1.189 & 1.236 & 127.639 & 132.845 & 133.758 & 140.938 & 137.397 & 139.66 & 139.727 & 147.45 & 139.734 & 98.096 \\ 
   &  & p=996 & 1.184 & 1.185 & 1.428 & 392.995 & 1.188 & 1.189 & 1.189 & 646.182 & 1.189 & 602.795 & 127.636 & 132.845 & 193.51 & 363.437 & 137.397 & 139.652 & 139.726 & 537.863 & 139.734 & 351.954 \\ 
  \cmidrule{2-23} & \multirow{3}[2]{*}{$\rho=0.9$} & p=12 & -1.385 & -1.385 & -1.385 & -1.385 & -1.385 & -1.385 & -1.385 & -1.385 & -1.385 & -1.385 & 12 & 12 & 12 & 12 & 12 & 12 & 12 & 12 & 12 & 12 \\ 
   &  & p=500 & -0.514 & -0.512 & -0.512 & -0.507 & -0.51 & -0.508 & -0.508 & -0.5 & -0.508 & -0.458 & 124.594 & 129.55 & 130.466 & 137.301 & 133.667 & 136.188 & 136.238 & 144.124 & 136.238 & 93.348 \\ 
   &  & p=996 & -0.514 & -0.512 & -0.269 & 397.997 & -0.51 & -0.508 & -0.508 & 633.828 & -0.508 & 591.61 & 124.594 & 129.55 & 190.882 & 360.624 & 133.667 & 136.184 & 136.238 & 521.693 & 136.238 & 339.768 \\ 
  \midrule\multirow{9}[6]{*}{msnr} & \multirow{3}[2]{*}{$\rho=0$} & p=12 & 3.238 & 3.238 & 3.238 & 3.238 & 3.238 & 3.238 & 3.238 & 3.238 & 3.238 & 3.238 & 12 & 12 & 12 & 12 & 12 & 12 & 12 & 12 & 12 & 12 \\ 
   &  & p=500 & 4.326 & 4.327 & 4.328 & 4.33 & 4.328 & 4.329 & 4.33 & 4.333 & 4.33 & 4.402 & 79.588 & 83.39 & 84.807 & 90.056 & 85.699 & 88.253 & 88.344 & 94.161 & 88.391 & 43.529 \\ 
   &  & p=996 & 4.326 & 4.327 & 4.584 & 392.892 & 4.328 & 4.329 & 4.33 & 628.527 & 4.33 & 492.85 & 79.588 & 83.39 & 156.249 & 323.679 & 85.699 & 88.253 & 88.344 & 476.579 & 88.391 & 214.034 \\ 
  \cmidrule{2-23} & \multirow{3}[2]{*}{$\rho=0.5$} & p=12 & 2.248 & 2.248 & 2.248 & 2.248 & 2.248 & 2.248 & 2.248 & 2.248 & 2.248 & 2.248 & 12 & 12 & 12 & 12 & 12 & 12 & 12 & 12 & 12 & 12 \\ 
   &  & p=500 & 3.256 & 3.257 & 3.257 & 3.26 & 3.257 & 3.259 & 3.259 & 3.263 & 3.259 & 3.331 & 79.447 & 83.023 & 84.602 & 89.743 & 85.164 & 87.865 & 88.055 & 93.597 & 88.06 & 43.16 \\ 
   &  & p=996 & 3.256 & 3.257 & 3.515 & 394.142 & 3.257 & 3.259 & 3.259 & 629.931 & 3.259 & 491.771 & 79.447 & 83.023 & 155.806 & 323.398 & 85.164 & 87.865 & 88.055 & 474.511 & 88.06 & 213.658 \\ 
  \cmidrule{2-23} & \multirow{3}[2]{*}{$\rho=0.9$} & p=12 & 0.755 & 0.755 & 0.755 & 0.755 & 0.755 & 0.755 & 0.755 & 0.755 & 0.755 & 0.755 & 12 & 12 & 12 & 12 & 12 & 12 & 12 & 12 & 12 & 12 \\ 
   &  & p=500 & 1.558 & 1.559 & 1.559 & 1.562 & 1.56 & 1.561 & 1.561 & 1.565 & 1.561 & 1.635 & 76.627 & 79.933 & 80.976 & 85.507 & 81.499 & 83.906 & 83.907 & 89.221 & 83.943 & 39.047 \\ 
   &  & p=996 & 1.558 & 1.559 & 1.818 & 399.137 & 1.56 & 1.561 & 1.561 & 624.227 & 1.561 & 475.861 & 76.627 & 79.933 & 153.247 & 320.689 & 81.499 & 83.906 & 83.907 & 470.493 & 83.943 & 199.693 \\ 
  \midrule\multirow{9}[6]{*}{lsnr} & \multirow{3}[2]{*}{$\rho=0$} & p=12 & 4.847 & 4.847 & 4.847 & 4.847 & 4.847 & 4.847 & 4.847 & 4.847 & 4.847 & 4.85 & 11.979 & 11.98 & 11.98 & 11.981 & 11.974 & 11.98 & 11.981 & 11.981 & 11.981 & 11.772 \\ 
   &  & p=500 & 5.89 & 5.89 & 5.89 & 5.891 & 5.891 & 5.891 & 5.891 & 5.892 & 5.891 & 5.961 & 45.12 & 47.327 & 48.841 & 51.993 & 47.917 & 50.47 & 50.506 & 53.685 & 50.518 & 5.059 \\ 
   &  & p=996 & 5.89 & 5.89 & 6.156 & 394.464 & 5.891 & 5.891 & 5.891 & 617.667 & 5.891 & 409.146 & 45.12 & 47.327 & 126.417 & 295.061 & 47.917 & 50.47 & 50.506 & 431.501 & 50.518 & 123.031 \\ 
  \cmidrule{2-23} & \multirow{3}[2]{*}{$\rho=0.5$} & p=12 & 3.857 & 3.857 & 3.857 & 3.857 & 3.857 & 3.857 & 3.857 & 3.857 & 3.857 & 3.858 & 11.999 & 11.999 & 11.999 & 11.999 & 12 & 11.999 & 11.999 & 11.999 & 11.999 & 11.97 \\ 
   &  & p=500 & 4.819 & 4.82 & 4.82 & 4.821 & 4.82 & 4.821 & 4.821 & 4.822 & 4.821 & 4.889 & 44.555 & 46.877 & 48.349 & 51.646 & 47.355 & 50.037 & 50.016 & 53.263 & 50.027 & 5.123 \\ 
   &  & p=996 & 4.819 & 4.82 & 5.087 & 395.717 & 4.82 & 4.821 & 4.821 & 621.622 & 4.821 & 408.064 & 44.555 & 46.877 & 126.215 & 295.48 & 47.355 & 50.037 & 50.016 & 432.04 & 50.027 & 123.114 \\ 
  \cmidrule{2-23} & \multirow{3}[2]{*}{$\rho=0.9$} & p=12 & 2.364 & 2.364 & 2.364 & 2.364 & 2.364 & 2.364 & 2.364 & 2.364 & 2.364 & 2.365 & 11.998 & 11.998 & 11.998 & 11.998 & 11.998 & 11.998 & 11.998 & 11.998 & 11.998 & 11.947 \\ 
   &  & p=500 & 3.123 & 3.123 & 3.123 & 3.124 & 3.124 & 3.124 & 3.124 & 3.125 & 3.124 & 3.181 & 40.942 & 43.413 & 44.561 & 47.079 & 43.446 & 45.892 & 45.862 & 48.67 & 45.935 & 4.012 \\ 
   &  & p=996 & 3.123 & 3.123 & 3.391 & 400.712 & 3.124 & 3.124 & 3.124 & 612.539 & 3.124 & 401.388 & 40.942 & 43.413 & 123.22 & 292.373 & 43.446 & 45.892 & 45.862 & 424.268 & 45.935 & 119.105 \\ 
   \bottomrule 
\end{tabular}
}
\end{table}


\clearpage
% latex table generated in R 4.0.2 by xtable 1.8-4 package
% Fri Jun 26 23:43:30 2020
\begin{table}[ht]
\centering
\caption{Fixed X, Ex1, p=6} 
\scalebox{0.52}{
\begin{tabular}{|c|c|c|cc|cc|cc|ccc|c||cc|cc|cc|ccc|c|}
  \toprule 
 \multicolumn{1}{|c}{} & \multicolumn{1}{c}{} &       & RAICc & AICc  & RC$_p$ & C$_p$ & 10F CV & LOO CV & S$_p$ & $\widehat{\text{C}}_p$ & GCV   & BIC   & RAICc & AICc  & RC$_p$ & C$_p$ & 10F CV & LOO CV & S$_p$ & $\widehat{\text{C}}_p$ & GCV   & BIC  \\
 \cmidrule{4-23}\multicolumn{1}{|c}{} & \multicolumn{1}{c}{} &       & \multicolumn{10}{c||}{RMSE}                                                   & \multicolumn{10}{c|}{Relative Risk}       \\
 \midrule\multirow{9}[6]{*}{$n=10$} & \multirow{3}[2]{*}{hsnr} & $\rho=0$ & 0.319 & 0.324 & 0.383 & 0.436 & 0.386 & 0.386 & 0.402 & 0.46 & 0.416 & 0.456 & 0.017 & 0.018 & 0.028 & 0.036 & 0.028 & 0.028 & 0.03 & 0.038 & 0.032 & 0.038 \\ 
   &  & $\rho=0.5$ & 0.497 & 0.508 & 0.614 & 0.689 & 0.625 & 0.625 & 0.641 & 0.73 & 0.665 & 0.721 & 0.015 & 0.016 & 0.028 & 0.037 & 0.03 & 0.03 & 0.031 & 0.041 & 0.034 & 0.04 \\ 
   &  & $\rho=0.9$ & 0.636 & 0.655 & 0.773 & 0.855 & 0.787 & 0.787 & 0.814 & 0.925 & 0.832 & 0.914 & 0.014 & 0.014 & 0.021 & 0.028 & 0.022 & 0.022 & 0.025 & 0.031 & 0.026 & 0.031 \\ 
  \cmidrule{2-23} & \multirow{3}[2]{*}{msnr} & $\rho=0$ & 0.974 & 0.985 & 1.157 & 1.302 & 1.203 & 1.203 & 1.217 & 1.387 & 1.258 & 1.366 & 0.149 & 0.155 & 0.232 & 0.304 & 0.259 & 0.259 & 0.263 & 0.334 & 0.276 & 0.327 \\ 
   &  & $\rho=0.5$ & 1.538 & 1.592 & 1.845 & 2.057 & 1.927 & 1.927 & 1.933 & 2.213 & 1.991 & 2.18 & 0.149 & 0.157 & 0.228 & 0.31 & 0.257 & 0.257 & 0.263 & 0.353 & 0.282 & 0.35 \\ 
   &  & $\rho=0.9$ & 1.877 & 1.92 & 2.225 & 2.512 & 2.311 & 2.311 & 2.362 & 2.678 & 2.446 & 2.656 & 0.132 & 0.135 & 0.182 & 0.246 & 0.196 & 0.196 & 0.209 & 0.27 & 0.222 & 0.267 \\ 
  \cmidrule{2-23} & \multirow{3}[2]{*}{lsnr} & $\rho=0$ & 2.187 & 2.245 & 2.631 & 2.941 & 2.647 & 2.647 & 2.787 & 3.179 & 2.86 & 3.116 & 0.759 & 0.8 & 1.175 & 1.504 & 1.2 & 1.2 & 1.338 & 1.694 & 1.412 & 1.662 \\ 
   &  & $\rho=0.5$ & 3.412 & 3.482 & 4.076 & 4.607 & 4.265 & 4.265 & 4.352 & 4.921 & 4.462 & 4.876 & 0.716 & 0.747 & 1.112 & 1.567 & 1.223 & 1.223 & 1.317 & 1.723 & 1.405 & 1.734 \\ 
   &  & $\rho=0.9$ & 4.133 & 4.264 & 4.953 & 5.573 & 5.092 & 5.092 & 5.279 & 5.974 & 5.461 & 5.877 & 0.593 & 0.625 & 0.87 & 1.19 & 0.932 & 0.932 & 1.001 & 1.32 & 1.096 & 1.302 \\ 
  \midrule\multirow{9}[6]{*}{$n=100$} & \multirow{3}[2]{*}{hsnr} & $\rho=0$ & 0.17 & 0.171 & 0.174 & 0.177 & 0.169 & 0.172 & 0.176 & 0.178 & 0.176 & 0.148 & 0.003 & 0.003 & 0.003 & 0.003 & 0.003 & 0.003 & 0.003 & 0.003 & 0.003 & 0.002 \\ 
   &  & $\rho=0.5$ & 0.255 & 0.258 & 0.262 & 0.264 & 0.253 & 0.259 & 0.263 & 0.266 & 0.263 & 0.224 & 0.003 & 0.003 & 0.003 & 0.003 & 0.003 & 0.003 & 0.003 & 0.003 & 0.003 & 0.002 \\ 
   &  & $\rho=0.9$ & 0.395 & 0.4 & 0.405 & 0.41 & 0.389 & 0.398 & 0.408 & 0.413 & 0.409 & 0.335 & 0.004 & 0.004 & 0.004 & 0.004 & 0.004 & 0.004 & 0.004 & 0.004 & 0.004 & 0.002 \\ 
  \cmidrule{2-23} & \multirow{3}[2]{*}{msnr} & $\rho=0$ & 0.526 & 0.529 & 0.535 & 0.543 & 0.523 & 0.529 & 0.54 & 0.546 & 0.541 & 0.46 & 0.028 & 0.028 & 0.029 & 0.03 & 0.03 & 0.029 & 0.029 & 0.03 & 0.03 & 0.019 \\ 
   &  & $\rho=0.5$ & 0.801 & 0.81 & 0.822 & 0.829 & 0.795 & 0.816 & 0.826 & 0.833 & 0.827 & 0.696 & 0.029 & 0.03 & 0.031 & 0.031 & 0.031 & 0.031 & 0.031 & 0.032 & 0.031 & 0.02 \\ 
   &  & $\rho=0.9$ & 1.156 & 1.167 & 1.18 & 1.195 & 1.12 & 1.162 & 1.192 & 1.202 & 1.193 & 0.973 & 0.029 & 0.03 & 0.031 & 0.032 & 0.03 & 0.03 & 0.031 & 0.032 & 0.031 & 0.019 \\ 
  \cmidrule{2-23} & \multirow{3}[2]{*}{lsnr} & $\rho=0$ & 1.237 & 1.249 & 1.265 & 1.285 & 1.243 & 1.246 & 1.283 & 1.293 & 1.285 & 1.066 & 0.153 & 0.156 & 0.16 & 0.166 & 0.163 & 0.16 & 0.165 & 0.169 & 0.166 & 0.103 \\ 
   &  & $\rho=0.5$ & 1.842 & 1.859 & 1.878 & 1.897 & 1.82 & 1.855 & 1.892 & 1.912 & 1.902 & 1.557 & 0.152 & 0.155 & 0.158 & 0.161 & 0.156 & 0.157 & 0.161 & 0.164 & 0.162 & 0.098 \\ 
   &  & $\rho=0.9$ & 2.558 & 2.59 & 2.603 & 2.629 & 2.497 & 2.571 & 2.619 & 2.644 & 2.62 & 2.118 & 0.148 & 0.152 & 0.154 & 0.157 & 0.152 & 0.154 & 0.156 & 0.159 & 0.156 & 0.092 \\ 
  \midrule\multirow{9}[6]{*}{$n=500$} & \multirow{3}[2]{*}{hsnr} & $\rho=0$ & 0.072 & 0.072 & 0.072 & 0.072 & 0.068 & 0.072 & 0.072 & 0.072 & 0.072 & 0.062 & 0.001 & 0.001 & 0.001 & 0.001 & 0.001 & 0.001 & 0.001 & 0.001 & 0.001 & 0 \\ 
   &  & $\rho=0.5$ & 0.11 & 0.11 & 0.11 & 0.11 & 0.102 & 0.108 & 0.11 & 0.11 & 0.11 & 0.094 & 0.001 & 0.001 & 0.001 & 0.001 & 0.001 & 0.001 & 0.001 & 0.001 & 0.001 & 0 \\ 
   &  & $\rho=0.9$ & 0.161 & 0.161 & 0.161 & 0.162 & 0.151 & 0.158 & 0.161 & 0.162 & 0.161 & 0.136 & 0.001 & 0.001 & 0.001 & 0.001 & 0.001 & 0.001 & 0.001 & 0.001 & 0.001 & 0 \\ 
  \cmidrule{2-23} & \multirow{3}[2]{*}{msnr} & $\rho=0$ & 0.212 & 0.212 & 0.212 & 0.213 & 0.198 & 0.21 & 0.213 & 0.213 & 0.213 & 0.183 & 0.005 & 0.005 & 0.005 & 0.005 & 0.005 & 0.005 & 0.005 & 0.005 & 0.005 & 0.003 \\ 
   &  & $\rho=0.5$ & 0.332 & 0.332 & 0.334 & 0.335 & 0.308 & 0.328 & 0.334 & 0.335 & 0.334 & 0.287 & 0.005 & 0.005 & 0.005 & 0.005 & 0.005 & 0.005 & 0.005 & 0.005 & 0.005 & 0.004 \\ 
   &  & $\rho=0.9$ & 0.5 & 0.502 & 0.503 & 0.503 & 0.478 & 0.498 & 0.503 & 0.503 & 0.503 & 0.424 & 0.006 & 0.006 & 0.006 & 0.006 & 0.006 & 0.006 & 0.006 & 0.006 & 0.006 & 0.004 \\ 
  \cmidrule{2-23} & \multirow{3}[2]{*}{lsnr} & $\rho=0$ & 0.497 & 0.498 & 0.5 & 0.501 & 0.469 & 0.495 & 0.501 & 0.501 & 0.501 & 0.428 & 0.026 & 0.026 & 0.026 & 0.027 & 0.025 & 0.026 & 0.027 & 0.027 & 0.027 & 0.019 \\ 
   &  & $\rho=0.5$ & 0.765 & 0.767 & 0.768 & 0.769 & 0.732 & 0.762 & 0.769 & 0.77 & 0.769 & 0.657 & 0.027 & 0.028 & 0.028 & 0.028 & 0.027 & 0.028 & 0.028 & 0.028 & 0.028 & 0.02 \\ 
   &  & $\rho=0.9$ & 1.106 & 1.106 & 1.109 & 1.112 & 1.033 & 1.098 & 1.111 & 1.114 & 1.111 & 0.918 & 0.028 & 0.028 & 0.028 & 0.028 & 0.025 & 0.027 & 0.028 & 0.028 & 0.028 & 0.018 \\ 
   \midrule 
 \multicolumn{1}{|c}{} & \multicolumn{1}{c}{} &       & \multicolumn{10}{c||}{Relative Test Error}                                    & \multicolumn{10}{c|}{Proportion of Variance Explained} \\
\midrule\multirow{9}[6]{*}{$n=10$} & \multirow{3}[2]{*}{hsnr} & $\rho=0$ & 1.327 & 1.345 & 1.548 & 1.702 & 1.544 & 1.544 & 1.593 & 1.751 & 1.624 & 1.739 & 0.936 & 0.935 & 0.925 & 0.918 & 0.925 & 0.925 & 0.923 & 0.915 & 0.921 & 0.916 \\ 
   &  & $\rho=0.5$ & 1.271 & 1.287 & 1.505 & 1.677 & 1.545 & 1.545 & 1.558 & 1.744 & 1.618 & 1.732 & 0.933 & 0.933 & 0.921 & 0.912 & 0.919 & 0.919 & 0.918 & 0.909 & 0.915 & 0.909 \\ 
   &  & $\rho=0.9$ & 1.329 & 1.345 & 1.518 & 1.667 & 1.544 & 1.544 & 1.598 & 1.761 & 1.619 & 1.762 & 0.947 & 0.947 & 0.94 & 0.934 & 0.939 & 0.939 & 0.937 & 0.93 & 0.936 & 0.93 \\ 
  \cmidrule{2-23} & \multirow{3}[2]{*}{msnr} & $\rho=0$ & 1.344 & 1.357 & 1.537 & 1.703 & 1.599 & 1.599 & 1.608 & 1.771 & 1.639 & 1.755 & 0.594 & 0.59 & 0.536 & 0.485 & 0.517 & 0.517 & 0.514 & 0.465 & 0.505 & 0.47 \\ 
   &  & $\rho=0.5$ & 1.318 & 1.333 & 1.485 & 1.661 & 1.547 & 1.547 & 1.561 & 1.751 & 1.6 & 1.746 & 0.579 & 0.574 & 0.526 & 0.469 & 0.506 & 0.506 & 0.501 & 0.44 & 0.489 & 0.442 \\ 
   &  & $\rho=0.9$ & 1.375 & 1.385 & 1.519 & 1.7 & 1.558 & 1.558 & 1.596 & 1.768 & 1.633 & 1.76 & 0.643 & 0.64 & 0.605 & 0.558 & 0.595 & 0.595 & 0.585 & 0.54 & 0.575 & 0.543 \\ 
  \cmidrule{2-23} & \multirow{3}[2]{*}{lsnr} & $\rho=0$ & 1.351 & 1.37 & 1.543 & 1.695 & 1.554 & 1.554 & 1.618 & 1.782 & 1.652 & 1.768 & 0.076 & 0.063 & -0.055 & -0.159 & -0.063 & -0.063 & -0.107 & -0.219 & -0.13 & -0.209 \\ 
   &  & $\rho=0.5$ & 1.305 & 1.318 & 1.473 & 1.667 & 1.521 & 1.521 & 1.561 & 1.734 & 1.598 & 1.738 & 0.085 & 0.076 & -0.033 & -0.169 & -0.067 & -0.067 & -0.095 & -0.216 & -0.121 & -0.219 \\ 
   &  & $\rho=0.9$ & 1.337 & 1.356 & 1.495 & 1.678 & 1.531 & 1.531 & 1.57 & 1.752 & 1.624 & 1.741 & 0.148 & 0.136 & 0.047 & -0.069 & 0.025 & 0.025 & 0 & -0.116 & -0.035 & -0.109 \\ 
  \midrule\multirow{9}[6]{*}{$n=100$} & \multirow{3}[2]{*}{hsnr} & $\rho=0$ & 1.026 & 1.027 & 1.028 & 1.029 & 1.029 & 1.029 & 1.028 & 1.029 & 1.029 & 1.018 & 0.9 & 0.899 & 0.899 & 0.899 & 0.899 & 0.899 & 0.899 & 0.899 & 0.899 & 0.9 \\ 
   &  & $\rho=0.5$ & 1.027 & 1.028 & 1.029 & 1.03 & 1.03 & 1.029 & 1.029 & 1.03 & 1.029 & 1.019 & 0.902 & 0.902 & 0.902 & 0.901 & 0.901 & 0.902 & 0.902 & 0.901 & 0.902 & 0.902 \\ 
   &  & $\rho=0.9$ & 1.033 & 1.034 & 1.035 & 1.036 & 1.035 & 1.034 & 1.035 & 1.036 & 1.035 & 1.022 & 0.9 & 0.9 & 0.9 & 0.9 & 0.9 & 0.9 & 0.9 & 0.9 & 0.9 & 0.901 \\ 
  \cmidrule{2-23} & \multirow{3}[2]{*}{msnr} & $\rho=0$ & 1.03 & 1.03 & 1.031 & 1.032 & 1.032 & 1.032 & 1.032 & 1.033 & 1.032 & 1.021 & 0.506 & 0.506 & 0.505 & 0.505 & 0.505 & 0.505 & 0.505 & 0.504 & 0.505 & 0.51 \\ 
   &  & $\rho=0.5$ & 1.032 & 1.033 & 1.034 & 1.035 & 1.034 & 1.034 & 1.035 & 1.035 & 1.035 & 1.022 & 0.511 & 0.511 & 0.51 & 0.51 & 0.51 & 0.51 & 0.51 & 0.51 & 0.51 & 0.516 \\ 
   &  & $\rho=0.9$ & 1.032 & 1.033 & 1.034 & 1.035 & 1.032 & 1.034 & 1.035 & 1.035 & 1.035 & 1.02 & 0.509 & 0.508 & 0.508 & 0.508 & 0.509 & 0.508 & 0.508 & 0.507 & 0.508 & 0.514 \\ 
  \cmidrule{2-23} & \multirow{3}[2]{*}{lsnr} & $\rho=0$ & 1.033 & 1.034 & 1.035 & 1.036 & 1.035 & 1.035 & 1.036 & 1.037 & 1.036 & 1.022 & 0.151 & 0.15 & 0.15 & 0.149 & 0.149 & 0.15 & 0.149 & 0.148 & 0.149 & 0.16 \\ 
   &  & $\rho=0.5$ & 1.034 & 1.034 & 1.035 & 1.036 & 1.035 & 1.035 & 1.036 & 1.036 & 1.036 & 1.022 & 0.154 & 0.154 & 0.153 & 0.153 & 0.154 & 0.153 & 0.153 & 0.152 & 0.152 & 0.164 \\ 
   &  & $\rho=0.9$ & 1.032 & 1.034 & 1.034 & 1.035 & 1.033 & 1.034 & 1.034 & 1.035 & 1.034 & 1.02 & 0.154 & 0.153 & 0.153 & 0.152 & 0.153 & 0.153 & 0.152 & 0.152 & 0.152 & 0.164 \\ 
  \midrule\multirow{9}[6]{*}{$n=500$} & \multirow{3}[2]{*}{hsnr} & $\rho=0$ & 1.005 & 1.005 & 1.005 & 1.005 & 1.005 & 1.005 & 1.005 & 1.005 & 1.005 & 1.004 & 0.906 & 0.906 & 0.906 & 0.906 & 0.906 & 0.906 & 0.906 & 0.906 & 0.906 & 0.907 \\ 
   &  & $\rho=0.5$ & 1.005 & 1.005 & 1.006 & 1.006 & 1.005 & 1.005 & 1.006 & 1.006 & 1.006 & 1.004 & 0.905 & 0.905 & 0.905 & 0.905 & 0.905 & 0.905 & 0.905 & 0.905 & 0.905 & 0.906 \\ 
   &  & $\rho=0.9$ & 1.006 & 1.006 & 1.006 & 1.006 & 1.006 & 1.006 & 1.006 & 1.006 & 1.006 & 1.004 & 0.904 & 0.904 & 0.904 & 0.904 & 0.904 & 0.904 & 0.904 & 0.904 & 0.904 & 0.904 \\ 
  \cmidrule{2-23} & \multirow{3}[2]{*}{msnr} & $\rho=0$ & 1.006 & 1.006 & 1.006 & 1.006 & 1.005 & 1.006 & 1.006 & 1.006 & 1.006 & 1.004 & 0.532 & 0.532 & 0.532 & 0.532 & 0.532 & 0.532 & 0.532 & 0.532 & 0.532 & 0.532 \\ 
   &  & $\rho=0.5$ & 1.006 & 1.006 & 1.006 & 1.006 & 1.006 & 1.006 & 1.006 & 1.006 & 1.006 & 1.004 & 0.528 & 0.528 & 0.528 & 0.528 & 0.528 & 0.528 & 0.528 & 0.528 & 0.528 & 0.529 \\ 
   &  & $\rho=0.9$ & 1.006 & 1.006 & 1.006 & 1.006 & 1.006 & 1.006 & 1.006 & 1.006 & 1.006 & 1.005 & 0.523 & 0.523 & 0.523 & 0.523 & 0.523 & 0.523 & 0.523 & 0.523 & 0.523 & 0.524 \\ 
  \cmidrule{2-23} & \multirow{3}[2]{*}{lsnr} & $\rho=0$ & 1.006 & 1.006 & 1.006 & 1.006 & 1.006 & 1.006 & 1.006 & 1.006 & 1.006 & 1.004 & 0.182 & 0.182 & 0.182 & 0.182 & 0.182 & 0.182 & 0.182 & 0.182 & 0.182 & 0.183 \\ 
   &  & $\rho=0.5$ & 1.006 & 1.006 & 1.006 & 1.006 & 1.006 & 1.006 & 1.006 & 1.006 & 1.006 & 1.004 & 0.18 & 0.18 & 0.18 & 0.18 & 0.18 & 0.18 & 0.18 & 0.18 & 0.18 & 0.181 \\ 
   &  & $\rho=0.9$ & 1.006 & 1.006 & 1.006 & 1.006 & 1.006 & 1.006 & 1.006 & 1.006 & 1.006 & 1.004 & 0.177 & 0.177 & 0.177 & 0.177 & 0.177 & 0.177 & 0.177 & 0.176 & 0.177 & 0.178 \\ 
   \midrule 
 \multicolumn{1}{|c}{} & \multicolumn{1}{c}{} &       & \multicolumn{10}{c||}{Kullback-Leibler Discrepancy}                                    & \multicolumn{10}{c|}{\% of Correct Restrictions Selected} \\
\midrule\multirow{9}[6]{*}{$n=10$} & \multirow{3}[2]{*}{hsnr} & $\rho=0$ & 1.314 & 1.679 & 3.766 & 5.359 & 4.343 & 4.343 & 4.505 & 5.702 & 4.928 & 5.73 & 89.2 & 88.3 & 77 & 65.4 & 71.6 & 71.6 & 72.2 & 59.6 & 69.4 & 61 \\ 
   &  & $\rho=0.5$ & 2.202 & 2.303 & 4.889 & 6.261 & 5.343 & 5.343 & 5.614 & 6.627 & 5.96 & 6.647 & 85.7 & 84.8 & 71.9 & 61.6 & 67.6 & 67.6 & 68.1 & 55.4 & 64.8 & 57 \\ 
   &  & $\rho=0.9$ & 2.779 & 3.221 & 5.159 & 6.506 & 5.827 & 5.827 & 5.839 & 7.007 & 6.176 & 7.026 & 75.7 & 73.5 & 61.6 & 53.6 & 59.3 & 59.3 & 58.1 & 46.2 & 56.2 & 48.1 \\ 
  \cmidrule{2-23} & \multirow{3}[2]{*}{msnr} & $\rho=0$ & 3.552 & 3.959 & 5.988 & 7.462 & 6.614 & 6.614 & 6.69 & 7.839 & 7.105 & 7.903 & 61.3 & 60.8 & 52.9 & 45.1 & 50.7 & 50.7 & 49.4 & 40.2 & 47.1 & 41.8 \\ 
   &  & $\rho=0.5$ & 4.507 & 5.156 & 7.033 & 8.288 & 7.614 & 7.614 & 7.758 & 8.707 & 8.034 & 8.766 & 56 & 54.4 & 46.2 & 39.6 & 44.8 & 44.8 & 44 & 34.1 & 41.9 & 35.5 \\ 
   &  & $\rho=0.9$ & 4.917 & 5.099 & 7.332 & 8.63 & 7.817 & 7.817 & 8.031 & 9.117 & 8.351 & 9.148 & 46.5 & 45.6 & 39.6 & 33.3 & 42 & 42 & 36 & 29 & 34.6 & 29.8 \\ 
  \cmidrule{2-23} & \multirow{3}[2]{*}{lsnr} & $\rho=0$ & 5.167 & 5.625 & 7.369 & 8.914 & 8.158 & 8.158 & 8.305 & 9.469 & 8.672 & 9.513 & 42.8 & 42.3 & 36 & 30.6 & 33.8 & 33.8 & 33 & 26.7 & 32.1 & 28 \\ 
   &  & $\rho=0.5$ & 6.098 & 6.292 & 8.41 & 9.874 & 9.106 & 9.106 & 9.189 & 10.307 & 9.556 & 10.364 & 37.6 & 37.3 & 31.1 & 26.2 & 29.3 & 29.3 & 29.1 & 23.1 & 27.7 & 23.9 \\ 
   &  & $\rho=0.9$ & 6.479 & 6.694 & 8.85 & 10.261 & 9.329 & 9.329 & 9.779 & 10.67 & 9.986 & 10.725 & 37.9 & 37.1 & 31.1 & 26.2 & 31.7 & 31.7 & 28.4 & 23.1 & 27.2 & 23.6 \\ 
  \midrule\multirow{9}[6]{*}{$n=100$} & \multirow{3}[2]{*}{hsnr} & $\rho=0$ & 1.526 & 1.526 & 1.527 & 1.528 & 1.528 & 1.528 & 1.528 & 1.529 & 1.528 & 1.517 & 91.4 & 91 & 89.9 & 88.8 & 87.3 & 89 & 89.3 & 88.2 & 89 & 99.6 \\ 
   &  & $\rho=0.5$ & 2.308 & 2.309 & 2.31 & 2.311 & 2.311 & 2.311 & 2.311 & 2.311 & 2.311 & 2.3 & 90.3 & 89.5 & 88.6 & 88 & 86.6 & 87.9 & 88.3 & 87.5 & 88.3 & 98.6 \\ 
   &  & $\rho=0.9$ & 3.021 & 3.022 & 3.023 & 3.024 & 3.022 & 3.022 & 3.023 & 3.024 & 3.023 & 3.009 & 74.3 & 73.5 & 72.6 & 71.7 & 72.7 & 73.5 & 72.1 & 71.2 & 72 & 84.6 \\ 
  \cmidrule{2-23} & \multirow{3}[2]{*}{msnr} & $\rho=0$ & 3.669 & 3.67 & 3.671 & 3.672 & 3.671 & 3.671 & 3.671 & 3.672 & 3.671 & 3.66 & 75.1 & 74.8 & 74.1 & 73.2 & 73.3 & 74.3 & 73.5 & 72.6 & 73.4 & 82.1 \\ 
   &  & $\rho=0.5$ & 4.453 & 4.454 & 4.455 & 4.456 & 4.455 & 4.455 & 4.455 & 4.456 & 4.456 & 4.443 & 66.5 & 65.8 & 65 & 64.6 & 66.3 & 66.2 & 64.8 & 64 & 64.7 & 74.1 \\ 
   &  & $\rho=0.9$ & 5.16 & 5.161 & 5.161 & 5.162 & 5.16 & 5.161 & 5.162 & 5.163 & 5.162 & 5.148 & 46.5 & 45.8 & 45.2 & 44.7 & 52.1 & 46.7 & 44.7 & 44.5 & 44.7 & 53.6 \\ 
  \cmidrule{2-23} & \multirow{3}[2]{*}{lsnr} & $\rho=0$ & 5.282 & 5.282 & 5.283 & 5.284 & 5.284 & 5.283 & 5.284 & 5.285 & 5.284 & 5.271 & 50.9 & 50.4 & 50 & 48.9 & 53 & 50.3 & 49 & 48.4 & 48.8 & 56.9 \\ 
   &  & $\rho=0.5$ & 6.063 & 6.064 & 6.065 & 6.066 & 6.064 & 6.065 & 6.065 & 6.066 & 6.066 & 6.052 & 45.3 & 44.9 & 44.5 & 44.1 & 49.6 & 45.1 & 44.2 & 43.8 & 44.2 & 52.1 \\ 
   &  & $\rho=0.9$ & 6.769 & 6.77 & 6.771 & 6.772 & 6.77 & 6.771 & 6.771 & 6.772 & 6.771 & 6.757 & 41 & 40.5 & 40.2 & 39.8 & 44.6 & 40.9 & 39.9 & 39.6 & 40 & 47.3 \\ 
  \midrule\multirow{9}[6]{*}{$n=500$} & \multirow{3}[2]{*}{hsnr} & $\rho=0$ & 1.438 & 1.439 & 1.439 & 1.439 & 1.438 & 1.439 & 1.439 & 1.439 & 1.439 & 1.437 & 91.1 & 91.1 & 91 & 90.9 & 89.7 & 90.5 & 90.9 & 90.9 & 90.9 & 99.8 \\ 
   &  & $\rho=0.5$ & 2.258 & 2.258 & 2.258 & 2.258 & 2.257 & 2.258 & 2.258 & 2.258 & 2.258 & 2.256 & 90.7 & 90.7 & 90.5 & 90.4 & 90.1 & 90.8 & 90.5 & 90.4 & 90.5 & 99.8 \\ 
   &  & $\rho=0.9$ & 2.977 & 2.977 & 2.977 & 2.977 & 2.976 & 2.977 & 2.977 & 2.977 & 2.977 & 2.975 & 88.8 & 88.8 & 88.6 & 88.5 & 87.9 & 89.1 & 88.6 & 88.4 & 88.6 & 98.9 \\ 
  \cmidrule{2-23} & \multirow{3}[2]{*}{msnr} & $\rho=0$ & 3.579 & 3.579 & 3.579 & 3.579 & 3.578 & 3.579 & 3.579 & 3.579 & 3.579 & 3.577 & 90 & 90 & 89.9 & 89.8 & 88.8 & 89.7 & 89.8 & 89.8 & 89.8 & 98.5 \\ 
   &  & $\rho=0.5$ & 4.398 & 4.398 & 4.398 & 4.399 & 4.398 & 4.398 & 4.398 & 4.399 & 4.398 & 4.397 & 85.7 & 85.6 & 85.3 & 85 & 85.9 & 85.6 & 85.2 & 85 & 85.2 & 94.5 \\ 
   &  & $\rho=0.9$ & 5.117 & 5.117 & 5.117 & 5.117 & 5.117 & 5.117 & 5.117 & 5.117 & 5.117 & 5.115 & 66.2 & 66.1 & 65.9 & 65.9 & 67.1 & 66.2 & 65.9 & 65.9 & 65.9 & 75.2 \\ 
  \cmidrule{2-23} & \multirow{3}[2]{*}{lsnr} & $\rho=0$ & 5.189 & 5.189 & 5.189 & 5.189 & 5.188 & 5.189 & 5.189 & 5.189 & 5.189 & 5.187 & 74.9 & 74.9 & 74.7 & 74.6 & 76.2 & 74.4 & 74.6 & 74.6 & 74.6 & 82.7 \\ 
   &  & $\rho=0.5$ & 6.008 & 6.008 & 6.008 & 6.008 & 6.008 & 6.008 & 6.008 & 6.008 & 6.008 & 6.006 & 68 & 67.9 & 67.8 & 67.7 & 68.6 & 66.7 & 67.8 & 67.6 & 67.8 & 75.1 \\ 
   &  & $\rho=0.9$ & 6.726 & 6.726 & 6.726 & 6.726 & 6.726 & 6.726 & 6.726 & 6.727 & 6.726 & 6.724 & 47.3 & 47.3 & 47.1 & 47 & 52.9 & 47.2 & 47.2 & 47.1 & 47.2 & 53.7 \\ 
   \bottomrule 
\end{tabular}
}
\end{table}

% latex table generated in R 4.0.2 by xtable 1.8-4 package
% Fri Jun 26 23:43:31 2020
\begin{table}[ht]
\centering
\caption{Fixed X, Ex2, p=6} 
\scalebox{0.52}{
\begin{tabular}{|c|c|c|cc|cc|cc|ccc|c||cc|cc|cc|ccc|c|}
  \toprule 
 \multicolumn{1}{|c}{} & \multicolumn{1}{c}{} &       & RAICc & AICc  & RC$_p$ & C$_p$ & 10F CV & LOO CV & S$_p$ & $\widehat{\text{C}}_p$ & GCV   & BIC   & RAICc & AICc  & RC$_p$ & C$_p$ & 10F CV & LOO CV & S$_p$ & $\widehat{\text{C}}_p$ & GCV   & BIC  \\
 \cmidrule{4-23}\multicolumn{1}{|c}{} & \multicolumn{1}{c}{} &       & \multicolumn{10}{c||}{RMSE}                                                   & \multicolumn{10}{c|}{Relative Risk}       \\
 \midrule\multirow{9}[6]{*}{$n=10$} & \multirow{3}[2]{*}{hsnr} & $\rho=0$ & 1.176 & 1.18 & 1.231 & 1.306 & 1.256 & 1.256 & 1.285 & 1.34 & 1.299 & 1.346 & 0.062 & 0.062 & 0.095 & 0.158 & 0.116 & 0.116 & 0.14 & 0.177 & 0.149 & 0.18 \\ 
   &  & $\rho=0.5$ & 0.838 & 0.842 & 0.873 & 0.922 & 0.893 & 0.893 & 0.908 & 0.949 & 0.915 & 0.956 & 0.062 & 0.067 & 0.099 & 0.148 & 0.108 & 0.108 & 0.13 & 0.172 & 0.137 & 0.182 \\ 
   &  & $\rho=0.9$ & 0.392 & 0.392 & 0.409 & 0.432 & 0.419 & 0.419 & 0.426 & 0.443 & 0.429 & 0.446 & 0.061 & 0.065 & 0.113 & 0.169 & 0.137 & 0.137 & 0.159 & 0.198 & 0.166 & 0.203 \\ 
  \cmidrule{2-23} & \multirow{3}[2]{*}{msnr} & $\rho=0$ & 3.369 & 3.388 & 3.544 & 3.765 & 3.593 & 3.593 & 3.699 & 3.877 & 3.745 & 3.896 & 0.5 & 0.536 & 0.81 & 1.225 & 0.913 & 0.913 & 1.087 & 1.442 & 1.164 & 1.474 \\ 
   &  & $\rho=0.5$ & 2.408 & 2.421 & 2.521 & 2.68 & 2.586 & 2.586 & 2.645 & 2.757 & 2.668 & 2.77 & 0.513 & 0.545 & 0.834 & 1.304 & 1.047 & 1.047 & 1.209 & 1.515 & 1.262 & 1.569 \\ 
   &  & $\rho=0.9$ & 1.12 & 1.123 & 1.175 & 1.247 & 1.211 & 1.211 & 1.232 & 1.286 & 1.242 & 1.296 & 0.496 & 0.528 & 0.952 & 1.419 & 1.223 & 1.223 & 1.323 & 1.653 & 1.403 & 1.724 \\ 
  \cmidrule{2-23} & \multirow{3}[2]{*}{lsnr} & $\rho=0$ & 7.503 & 7.532 & 7.899 & 8.413 & 8.044 & 8.044 & 8.256 & 8.666 & 8.342 & 8.738 & 2.489 & 2.53 & 4.072 & 6.353 & 4.814 & 4.814 & 5.568 & 7.348 & 5.792 & 7.664 \\ 
   &  & $\rho=0.5$ & 5.362 & 5.389 & 5.621 & 5.979 & 5.755 & 5.755 & 5.895 & 6.18 & 5.955 & 6.211 & 2.534 & 2.61 & 4.271 & 6.514 & 5.162 & 5.162 & 5.927 & 7.823 & 6.293 & 8.086 \\ 
   &  & $\rho=0.9$ & 2.496 & 2.507 & 2.621 & 2.786 & 2.701 & 2.701 & 2.755 & 2.884 & 2.783 & 2.9 & 2.483 & 2.628 & 4.62 & 7.135 & 6.212 & 6.212 & 6.692 & 8.482 & 7.083 & 8.639 \\ 
  \midrule\multirow{9}[6]{*}{$n=100$} & \multirow{3}[2]{*}{hsnr} & $\rho=0$ & 0.288 & 0.29 & 0.292 & 0.295 & 0.29 & 0.291 & 0.294 & 0.296 & 0.294 & 0.271 & 0.006 & 0.006 & 0.006 & 0.006 & 0.006 & 0.006 & 0.006 & 0.007 & 0.006 & 0.005 \\ 
   &  & $\rho=0.5$ & 0.2 & 0.202 & 0.203 & 0.205 & 0.2 & 0.202 & 0.204 & 0.205 & 0.205 & 0.189 & 0.006 & 0.006 & 0.006 & 0.006 & 0.006 & 0.006 & 0.006 & 0.006 & 0.006 & 0.005 \\ 
   &  & $\rho=0.9$ & 0.093 & 0.094 & 0.094 & 0.095 & 0.092 & 0.093 & 0.095 & 0.095 & 0.095 & 0.088 & 0.006 & 0.006 & 0.006 & 0.006 & 0.006 & 0.006 & 0.006 & 0.006 & 0.006 & 0.005 \\ 
  \cmidrule{2-23} & \multirow{3}[2]{*}{msnr} & $\rho=0$ & 0.862 & 0.867 & 0.871 & 0.876 & 0.872 & 0.869 & 0.875 & 0.882 & 0.876 & 0.815 & 0.054 & 0.055 & 0.056 & 0.056 & 0.056 & 0.056 & 0.056 & 0.057 & 0.056 & 0.048 \\ 
   &  & $\rho=0.5$ & 0.603 & 0.607 & 0.608 & 0.612 & 0.611 & 0.607 & 0.612 & 0.615 & 0.612 & 0.569 & 0.055 & 0.056 & 0.056 & 0.057 & 0.056 & 0.056 & 0.057 & 0.058 & 0.057 & 0.047 \\ 
   &  & $\rho=0.9$ & 0.28 & 0.282 & 0.283 & 0.284 & 0.282 & 0.282 & 0.284 & 0.286 & 0.284 & 0.266 & 0.055 & 0.056 & 0.056 & 0.057 & 0.055 & 0.056 & 0.056 & 0.057 & 0.056 & 0.048 \\ 
  \cmidrule{2-23} & \multirow{3}[2]{*}{lsnr} & $\rho=0$ & 1.92 & 1.935 & 1.94 & 1.95 & 1.916 & 1.928 & 1.949 & 1.961 & 1.949 & 1.812 & 0.269 & 0.273 & 0.275 & 0.278 & 0.27 & 0.274 & 0.278 & 0.281 & 0.278 & 0.235 \\ 
   &  & $\rho=0.5$ & 1.35 & 1.36 & 1.362 & 1.37 & 1.34 & 1.347 & 1.368 & 1.372 & 1.37 & 1.268 & 0.273 & 0.278 & 0.279 & 0.282 & 0.27 & 0.275 & 0.281 & 0.283 & 0.282 & 0.232 \\ 
   &  & $\rho=0.9$ & 0.628 & 0.63 & 0.633 & 0.637 & 0.63 & 0.627 & 0.637 & 0.641 & 0.637 & 0.59 & 0.274 & 0.277 & 0.28 & 0.284 & 0.28 & 0.275 & 0.284 & 0.288 & 0.284 & 0.232 \\ 
  \midrule\multirow{9}[6]{*}{$n=500$} & \multirow{3}[2]{*}{hsnr} & $\rho=0$ & 0.118 & 0.119 & 0.119 & 0.119 & 0.116 & 0.118 & 0.119 & 0.119 & 0.119 & 0.11 & 0.001 & 0.001 & 0.001 & 0.001 & 0.001 & 0.001 & 0.001 & 0.001 & 0.001 & 0.001 \\ 
   &  & $\rho=0.5$ & 0.084 & 0.084 & 0.084 & 0.084 & 0.082 & 0.084 & 0.084 & 0.084 & 0.084 & 0.078 & 0.001 & 0.001 & 0.001 & 0.001 & 0.001 & 0.001 & 0.001 & 0.001 & 0.001 & 0.001 \\ 
   &  & $\rho=0.9$ & 0.039 & 0.039 & 0.039 & 0.039 & 0.038 & 0.039 & 0.039 & 0.039 & 0.039 & 0.036 & 0.001 & 0.001 & 0.001 & 0.001 & 0.001 & 0.001 & 0.001 & 0.001 & 0.001 & 0.001 \\ 
  \cmidrule{2-23} & \multirow{3}[2]{*}{msnr} & $\rho=0$ & 0.349 & 0.349 & 0.35 & 0.351 & 0.345 & 0.351 & 0.35 & 0.351 & 0.351 & 0.324 & 0.008 & 0.008 & 0.008 & 0.008 & 0.008 & 0.008 & 0.008 & 0.008 & 0.008 & 0.007 \\ 
   &  & $\rho=0.5$ & 0.246 & 0.247 & 0.247 & 0.247 & 0.242 & 0.246 & 0.247 & 0.248 & 0.247 & 0.229 & 0.009 & 0.009 & 0.009 & 0.009 & 0.008 & 0.009 & 0.009 & 0.009 & 0.009 & 0.007 \\ 
   &  & $\rho=0.9$ & 0.114 & 0.114 & 0.114 & 0.114 & 0.112 & 0.114 & 0.114 & 0.114 & 0.114 & 0.106 & 0.009 & 0.009 & 0.009 & 0.009 & 0.008 & 0.009 & 0.009 & 0.009 & 0.009 & 0.007 \\ 
  \cmidrule{2-23} & \multirow{3}[2]{*}{lsnr} & $\rho=0$ & 0.792 & 0.794 & 0.795 & 0.795 & 0.788 & 0.796 & 0.795 & 0.795 & 0.795 & 0.738 & 0.042 & 0.042 & 0.043 & 0.043 & 0.042 & 0.043 & 0.043 & 0.043 & 0.043 & 0.035 \\ 
   &  & $\rho=0.5$ & 0.564 & 0.564 & 0.564 & 0.564 & 0.559 & 0.564 & 0.564 & 0.564 & 0.564 & 0.527 & 0.044 & 0.044 & 0.044 & 0.044 & 0.044 & 0.044 & 0.044 & 0.044 & 0.044 & 0.037 \\ 
   &  & $\rho=0.9$ & 0.261 & 0.261 & 0.262 & 0.262 & 0.258 & 0.261 & 0.262 & 0.261 & 0.262 & 0.246 & 0.045 & 0.045 & 0.045 & 0.045 & 0.044 & 0.045 & 0.045 & 0.045 & 0.045 & 0.039 \\ 
   \midrule 
 \multicolumn{1}{|c}{} & \multicolumn{1}{c}{} &       & \multicolumn{10}{c||}{Relative Test Error}                                    & \multicolumn{10}{c|}{Proportion of Variance Explained} \\
\midrule\multirow{9}[6]{*}{$n=10$} & \multirow{3}[2]{*}{hsnr} & $\rho=0$ & 1.271 & 1.274 & 1.421 & 1.698 & 1.511 & 1.511 & 1.616 & 1.779 & 1.659 & 1.794 & 0.765 & 0.764 & 0.737 & 0.686 & 0.721 & 0.721 & 0.701 & 0.671 & 0.693 & 0.668 \\ 
   &  & $\rho=0.5$ & 1.265 & 1.286 & 1.422 & 1.632 & 1.462 & 1.462 & 1.553 & 1.732 & 1.582 & 1.775 & 0.759 & 0.755 & 0.73 & 0.69 & 0.722 & 0.722 & 0.705 & 0.671 & 0.699 & 0.662 \\ 
   &  & $\rho=0.9$ & 1.258 & 1.272 & 1.474 & 1.711 & 1.578 & 1.578 & 1.669 & 1.833 & 1.699 & 1.855 & 0.759 & 0.756 & 0.717 & 0.672 & 0.697 & 0.697 & 0.68 & 0.648 & 0.674 & 0.644 \\ 
  \cmidrule{2-23} & \multirow{3}[2]{*}{msnr} & $\rho=0$ & 1.259 & 1.278 & 1.42 & 1.636 & 1.474 & 1.474 & 1.564 & 1.749 & 1.604 & 1.765 & 0.171 & 0.159 & 0.065 & -0.077 & 0.03 & 0.03 & -0.03 & -0.151 & -0.056 & -0.162 \\ 
   &  & $\rho=0.5$ & 1.257 & 1.273 & 1.418 & 1.653 & 1.524 & 1.524 & 1.606 & 1.759 & 1.632 & 1.786 & 0.163 & 0.152 & 0.055 & -0.102 & -0.016 & -0.016 & -0.07 & -0.172 & -0.087 & -0.19 \\ 
   &  & $\rho=0.9$ & 1.246 & 1.261 & 1.471 & 1.703 & 1.606 & 1.606 & 1.655 & 1.819 & 1.695 & 1.854 & 0.167 & 0.157 & 0.016 & -0.139 & -0.074 & -0.074 & -0.107 & -0.216 & -0.133 & -0.24 \\ 
  \cmidrule{2-23} & \multirow{3}[2]{*}{lsnr} & $\rho=0$ & 1.258 & 1.263 & 1.423 & 1.659 & 1.5 & 1.5 & 1.578 & 1.763 & 1.601 & 1.796 & -0.14 & -0.144 & -0.289 & -0.503 & -0.359 & -0.359 & -0.43 & -0.597 & -0.451 & -0.627 \\ 
   &  & $\rho=0.5$ & 1.254 & 1.262 & 1.428 & 1.653 & 1.517 & 1.517 & 1.594 & 1.784 & 1.63 & 1.81 & -0.14 & -0.147 & -0.298 & -0.502 & -0.379 & -0.379 & -0.449 & -0.621 & -0.482 & -0.645 \\ 
   &  & $\rho=0.9$ & 1.246 & 1.26 & 1.458 & 1.707 & 1.615 & 1.615 & 1.663 & 1.84 & 1.702 & 1.856 & -0.134 & -0.147 & -0.326 & -0.553 & -0.47 & -0.47 & -0.513 & -0.674 & -0.548 & -0.689 \\ 
  \midrule\multirow{9}[6]{*}{$n=100$} & \multirow{3}[2]{*}{hsnr} & $\rho=0$ & 1.05 & 1.051 & 1.052 & 1.053 & 1.052 & 1.051 & 1.052 & 1.053 & 1.052 & 1.043 & 0.885 & 0.885 & 0.885 & 0.885 & 0.885 & 0.885 & 0.885 & 0.885 & 0.885 & 0.886 \\ 
   &  & $\rho=0.5$ & 1.049 & 1.05 & 1.05 & 1.051 & 1.05 & 1.05 & 1.051 & 1.052 & 1.051 & 1.043 & 0.885 & 0.885 & 0.884 & 0.884 & 0.885 & 0.884 & 0.884 & 0.884 & 0.884 & 0.885 \\ 
   &  & $\rho=0.9$ & 1.048 & 1.049 & 1.05 & 1.051 & 1.048 & 1.049 & 1.051 & 1.051 & 1.051 & 1.041 & 0.883 & 0.883 & 0.883 & 0.883 & 0.883 & 0.883 & 0.883 & 0.882 & 0.883 & 0.884 \\ 
  \cmidrule{2-23} & \multirow{3}[2]{*}{msnr} & $\rho=0$ & 1.052 & 1.053 & 1.053 & 1.054 & 1.053 & 1.053 & 1.054 & 1.055 & 1.054 & 1.046 & 0.462 & 0.462 & 0.462 & 0.461 & 0.462 & 0.462 & 0.461 & 0.461 & 0.461 & 0.465 \\ 
   &  & $\rho=0.5$ & 1.052 & 1.053 & 1.053 & 1.054 & 1.053 & 1.053 & 1.054 & 1.055 & 1.054 & 1.045 & 0.461 & 0.461 & 0.46 & 0.46 & 0.46 & 0.46 & 0.46 & 0.46 & 0.46 & 0.465 \\ 
   &  & $\rho=0.9$ & 1.051 & 1.052 & 1.053 & 1.053 & 1.052 & 1.052 & 1.053 & 1.054 & 1.053 & 1.045 & 0.457 & 0.456 & 0.456 & 0.456 & 0.456 & 0.456 & 0.456 & 0.455 & 0.456 & 0.46 \\ 
  \cmidrule{2-23} & \multirow{3}[2]{*}{lsnr} & $\rho=0$ & 1.051 & 1.052 & 1.053 & 1.053 & 1.052 & 1.052 & 1.053 & 1.054 & 1.053 & 1.045 & 0.117 & 0.117 & 0.116 & 0.116 & 0.117 & 0.117 & 0.116 & 0.115 & 0.116 & 0.123 \\ 
   &  & $\rho=0.5$ & 1.052 & 1.053 & 1.053 & 1.054 & 1.051 & 1.052 & 1.054 & 1.054 & 1.054 & 1.044 & 0.116 & 0.116 & 0.115 & 0.115 & 0.117 & 0.116 & 0.115 & 0.115 & 0.115 & 0.123 \\ 
   &  & $\rho=0.9$ & 1.051 & 1.052 & 1.052 & 1.053 & 1.052 & 1.051 & 1.053 & 1.054 & 1.053 & 1.043 & 0.114 & 0.114 & 0.113 & 0.113 & 0.113 & 0.114 & 0.113 & 0.112 & 0.113 & 0.121 \\ 
  \midrule\multirow{9}[6]{*}{$n=500$} & \multirow{3}[2]{*}{hsnr} & $\rho=0$ & 1.009 & 1.009 & 1.009 & 1.009 & 1.008 & 1.009 & 1.009 & 1.009 & 1.009 & 1.007 & 0.901 & 0.901 & 0.901 & 0.901 & 0.901 & 0.901 & 0.901 & 0.901 & 0.901 & 0.901 \\ 
   &  & $\rho=0.5$ & 1.009 & 1.009 & 1.009 & 1.009 & 1.008 & 1.009 & 1.009 & 1.009 & 1.009 & 1.007 & 0.898 & 0.898 & 0.898 & 0.898 & 0.898 & 0.898 & 0.898 & 0.898 & 0.898 & 0.898 \\ 
   &  & $\rho=0.9$ & 1.009 & 1.009 & 1.009 & 1.009 & 1.008 & 1.009 & 1.009 & 1.009 & 1.009 & 1.007 & 0.897 & 0.897 & 0.897 & 0.897 & 0.897 & 0.897 & 0.897 & 0.897 & 0.897 & 0.897 \\ 
  \cmidrule{2-23} & \multirow{3}[2]{*}{msnr} & $\rho=0$ & 1.009 & 1.009 & 1.009 & 1.009 & 1.009 & 1.009 & 1.009 & 1.009 & 1.009 & 1.007 & 0.514 & 0.514 & 0.514 & 0.514 & 0.514 & 0.514 & 0.514 & 0.514 & 0.514 & 0.515 \\ 
   &  & $\rho=0.5$ & 1.009 & 1.009 & 1.009 & 1.009 & 1.009 & 1.009 & 1.009 & 1.009 & 1.009 & 1.007 & 0.506 & 0.506 & 0.506 & 0.506 & 0.506 & 0.506 & 0.506 & 0.505 & 0.506 & 0.506 \\ 
   &  & $\rho=0.9$ & 1.009 & 1.009 & 1.009 & 1.009 & 1.009 & 1.009 & 1.009 & 1.009 & 1.009 & 1.007 & 0.503 & 0.503 & 0.503 & 0.503 & 0.503 & 0.503 & 0.503 & 0.503 & 0.503 & 0.504 \\ 
  \cmidrule{2-23} & \multirow{3}[2]{*}{lsnr} & $\rho=0$ & 1.009 & 1.009 & 1.009 & 1.009 & 1.009 & 1.009 & 1.009 & 1.009 & 1.009 & 1.008 & 0.17 & 0.17 & 0.17 & 0.17 & 0.17 & 0.17 & 0.17 & 0.17 & 0.17 & 0.171 \\ 
   &  & $\rho=0.5$ & 1.009 & 1.009 & 1.009 & 1.009 & 1.009 & 1.009 & 1.009 & 1.009 & 1.009 & 1.008 & 0.165 & 0.165 & 0.165 & 0.165 & 0.165 & 0.165 & 0.165 & 0.165 & 0.165 & 0.166 \\ 
   &  & $\rho=0.9$ & 1.009 & 1.009 & 1.009 & 1.009 & 1.009 & 1.009 & 1.009 & 1.009 & 1.009 & 1.008 & 0.163 & 0.163 & 0.163 & 0.163 & 0.163 & 0.163 & 0.163 & 0.163 & 0.163 & 0.164 \\ 
   \midrule 
 \multicolumn{1}{|c}{} & \multicolumn{1}{c}{} &       & \multicolumn{10}{c||}{Kullback-Leibler Discrepancy}                                    & \multicolumn{10}{c|}{\% of Correct Restrictions Selected} \\
\midrule\multirow{9}[6]{*}{$n=10$} & \multirow{3}[2]{*}{hsnr} & $\rho=0$ & 4.121 & 4.387 & 6.154 & 7.487 & 6.749 & 6.749 & 7.17 & 7.851 & 7.315 & 7.944 & 71.3 & 70.9 & 65.1 & 54.1 & 58.3 & 58.3 & 57.5 & 48.1 & 55.3 & 46.4 \\ 
   &  & $\rho=0.5$ & 3.414 & 3.88 & 5.054 & 6.696 & 6.059 & 6.059 & 6.381 & 7.181 & 6.611 & 7.278 & 73.8 & 73.2 & 67.7 & 57 & 58.2 & 58.2 & 60.5 & 49.1 & 58.7 & 47.1 \\ 
   &  & $\rho=0.9$ & 1.902 & 2.051 & 3.918 & 5.157 & 4.553 & 4.553 & 4.929 & 5.607 & 5.067 & 5.74 & 70.9 & 70.8 & 64.9 & 55 & 55.3 & 55.3 & 58.2 & 48.8 & 56.4 & 46.7 \\ 
  \cmidrule{2-23} & \multirow{3}[2]{*}{msnr} & $\rho=0$ & 6.25 & 6.751 & 8.318 & 9.567 & 8.861 & 8.861 & 9.31 & 9.994 & 9.467 & 10.076 & 49.2 & 48.8 & 44.1 & 36.2 & 35.1 & 35.1 & 38.4 & 31.2 & 36.8 & 29.9 \\ 
   &  & $\rho=0.5$ & 6.034 & 6.427 & 7.8 & 8.903 & 8.292 & 8.292 & 8.646 & 9.312 & 8.828 & 9.405 & 46.3 & 46 & 42.2 & 34.5 & 32.7 & 32.7 & 36.5 & 30.1 & 35.2 & 29 \\ 
   &  & $\rho=0.9$ & 4.036 & 4.619 & 5.92 & 7.315 & 6.766 & 6.766 & 7.144 & 7.761 & 7.285 & 7.862 & 44.9 & 44.7 & 40.8 & 34.4 & 32.3 & 32.3 & 36.2 & 30.1 & 34.8 & 28.6 \\ 
  \cmidrule{2-23} & \multirow{3}[2]{*}{lsnr} & $\rho=0$ & 7.838 & 8.123 & 9.941 & 11.229 & 10.603 & 10.603 & 10.95 & 11.582 & 11.102 & 11.717 & 43.5 & 43.2 & 39.5 & 32.2 & 30.7 & 30.7 & 34.4 & 28.4 & 32.6 & 27 \\ 
   &  & $\rho=0.5$ & 7.161 & 7.807 & 9.141 & 10.581 & 9.991 & 9.991 & 10.333 & 10.96 & 10.498 & 11.051 & 38.2 & 37.7 & 34.4 & 28.5 & 25.4 & 25.4 & 30.3 & 25.3 & 29 & 24.2 \\ 
   &  & $\rho=0.9$ & 6.09 & 6.471 & 7.895 & 9.019 & 8.416 & 8.416 & 8.848 & 9.399 & 9.014 & 9.486 & 38.7 & 38.2 & 34.7 & 28.4 & 26.4 & 26.4 & 29.7 & 24.1 & 28.4 & 23.1 \\ 
  \midrule\multirow{9}[6]{*}{$n=100$} & \multirow{3}[2]{*}{hsnr} & $\rho=0$ & 1.828 & 1.829 & 1.829 & 1.83 & 1.829 & 1.83 & 1.83 & 1.831 & 1.83 & 1.821 & 88.4 & 86.8 & 85.4 & 83.9 & 82.7 & 84.1 & 84.6 & 83 & 84.4 & 97.4 \\ 
   &  & $\rho=0.5$ & 1.11 & 1.111 & 1.111 & 1.112 & 1.11 & 1.111 & 1.112 & 1.113 & 1.112 & 1.103 & 89.8 & 88.1 & 87.5 & 86 & 85.7 & 87 & 86.2 & 85.2 & 86 & 98.1 \\ 
   &  & $\rho=0.9$ & -0.415 & -0.414 & -0.414 & -0.413 & -0.415 & -0.414 & -0.413 & -0.412 & -0.413 & -0.422 & 90.1 & 88.9 & 88.1 & 86.6 & 87.4 & 88 & 86.9 & 86 & 86.9 & 97.9 \\ 
  \cmidrule{2-23} & \multirow{3}[2]{*}{msnr} & $\rho=0$ & 3.97 & 3.971 & 3.971 & 3.972 & 3.971 & 3.971 & 3.972 & 3.973 & 3.972 & 3.964 & 65.1 & 64.1 & 63.4 & 62.5 & 61.5 & 64.4 & 62.7 & 61.7 & 62.6 & 71.5 \\ 
   &  & $\rho=0.5$ & 3.253 & 3.253 & 3.254 & 3.255 & 3.254 & 3.254 & 3.255 & 3.255 & 3.255 & 3.246 & 68.8 & 67.8 & 67.5 & 66.1 & 64.3 & 67.3 & 66.2 & 65.4 & 66.1 & 75.4 \\ 
   &  & $\rho=0.9$ & 1.728 & 1.729 & 1.729 & 1.73 & 1.729 & 1.729 & 1.729 & 1.73 & 1.729 & 1.722 & 67.6 & 66.8 & 66.4 & 65.7 & 65.8 & 67.5 & 66 & 65.2 & 66 & 73.1 \\ 
  \cmidrule{2-23} & \multirow{3}[2]{*}{lsnr} & $\rho=0$ & 5.579 & 5.58 & 5.58 & 5.581 & 5.579 & 5.58 & 5.581 & 5.581 & 5.581 & 5.572 & 47.2 & 46.5 & 46.1 & 45.5 & 45.2 & 47.8 & 45.6 & 45 & 45.6 & 52 \\ 
   &  & $\rho=0.5$ & 4.862 & 4.863 & 4.863 & 4.863 & 4.861 & 4.862 & 4.863 & 4.864 & 4.863 & 4.855 & 45.9 & 45 & 44.7 & 44.5 & 45.6 & 46.8 & 44.5 & 44.2 & 44.5 & 50.7 \\ 
   &  & $\rho=0.9$ & 3.337 & 3.338 & 3.338 & 3.339 & 3.338 & 3.337 & 3.339 & 3.34 & 3.339 & 3.33 & 45.9 & 45.4 & 45.2 & 44.4 & 44.1 & 46.1 & 44.4 & 43.9 & 44.4 & 50.4 \\ 
  \midrule\multirow{9}[6]{*}{$n=500$} & \multirow{3}[2]{*}{hsnr} & $\rho=0$ & 1.659 & 1.659 & 1.659 & 1.659 & 1.658 & 1.659 & 1.659 & 1.659 & 1.659 & 1.657 & 89.7 & 89.2 & 89 & 88.8 & 88.5 & 89 & 88.9 & 88.7 & 88.8 & 100 \\ 
   &  & $\rho=0.5$ & 0.971 & 0.971 & 0.971 & 0.971 & 0.971 & 0.971 & 0.971 & 0.971 & 0.971 & 0.97 & 89.9 & 89.7 & 89.6 & 89.5 & 89.6 & 90.1 & 89.5 & 89.5 & 89.5 & 100 \\ 
   &  & $\rho=0.9$ & -0.562 & -0.562 & -0.562 & -0.562 & -0.563 & -0.562 & -0.562 & -0.562 & -0.562 & -0.564 & 90.9 & 90.8 & 90.7 & 90.6 & 91.4 & 90.7 & 90.6 & 90.6 & 90.6 & 99.9 \\ 
  \cmidrule{2-23} & \multirow{3}[2]{*}{msnr} & $\rho=0$ & 3.799 & 3.799 & 3.799 & 3.799 & 3.799 & 3.799 & 3.799 & 3.799 & 3.799 & 3.797 & 86.8 & 86.5 & 86.3 & 85.9 & 84.2 & 85.3 & 86.1 & 85.8 & 86 & 98 \\ 
   &  & $\rho=0.5$ & 3.112 & 3.112 & 3.112 & 3.112 & 3.111 & 3.112 & 3.112 & 3.112 & 3.112 & 3.11 & 89.1 & 88.8 & 88.6 & 88.5 & 87.9 & 89 & 88.5 & 88.4 & 88.5 & 98.8 \\ 
   &  & $\rho=0.9$ & 1.578 & 1.578 & 1.578 & 1.578 & 1.578 & 1.578 & 1.578 & 1.578 & 1.578 & 1.577 & 89.6 & 89.5 & 89.4 & 89.3 & 89.5 & 89.1 & 89.3 & 89.3 & 89.3 & 98.3 \\ 
  \cmidrule{2-23} & \multirow{3}[2]{*}{lsnr} & $\rho=0$ & 5.409 & 5.409 & 5.409 & 5.409 & 5.409 & 5.409 & 5.409 & 5.409 & 5.409 & 5.407 & 69.9 & 69.4 & 69.3 & 69.3 & 67 & 68.8 & 69.3 & 69.3 & 69.3 & 79.1 \\ 
   &  & $\rho=0.5$ & 4.721 & 4.721 & 4.721 & 4.721 & 4.721 & 4.721 & 4.721 & 4.721 & 4.721 & 4.72 & 71.5 & 71.4 & 71.4 & 71.4 & 70.7 & 70.9 & 71.4 & 71.4 & 71.4 & 80.2 \\ 
   &  & $\rho=0.9$ & 3.188 & 3.188 & 3.188 & 3.188 & 3.188 & 3.188 & 3.188 & 3.188 & 3.188 & 3.187 & 70.9 & 70.9 & 70.8 & 70.7 & 71.4 & 70.7 & 70.7 & 70.8 & 70.7 & 77.2 \\ 
   \bottomrule 
\end{tabular}
}
\end{table}

% latex table generated in R 4.0.2 by xtable 1.8-4 package
% Fri Jun 26 23:43:32 2020
\begin{table}[ht]
\centering
\caption{Random X, Ex3, p=6} 
\scalebox{0.52}{
\begin{tabular}{|c|c|c|cc|cc|cc|ccc|c||cc|cc|cc|ccc|c|}
  \toprule 
 \multicolumn{1}{|c}{} & \multicolumn{1}{c}{} &       & RAICc & AICc  & RC$_p$ & C$_p$ & 10F CV & LOO CV & S$_p$ & $\widehat{\text{C}}_p$ & GCV   & BIC   & RAICc & AICc  & RC$_p$ & C$_p$ & 10F CV & LOO CV & S$_p$ & $\widehat{\text{C}}_p$ & GCV   & BIC  \\
 \cmidrule{4-23}\multicolumn{1}{|c}{} & \multicolumn{1}{c}{} &       & \multicolumn{10}{c||}{RMSE}                                                   & \multicolumn{10}{c|}{Relative Risk}       \\
 \midrule\multirow{9}[6]{*}{$n=10$} & \multirow{3}[2]{*}{hsnr} & $\rho=0$ & 0.312 & 0.338 & 0.44 & 0.518 & 0.491 & 0.491 & 0.477 & 0.565 & 0.502 & 0.558 & 0.02 & 0.025 & 0.064 & 0.104 & 0.084 & 0.084 & 0.084 & 0.118 & 0.095 & 0.123 \\ 
   &  & $\rho=0.5$ & 0.204 & 0.221 & 0.284 & 0.339 & 0.323 & 0.323 & 0.31 & 0.377 & 0.328 & 0.368 & 0.019 & 0.024 & 0.062 & 0.109 & 0.084 & 0.084 & 0.089 & 0.129 & 0.097 & 0.126 \\ 
   &  & $\rho=0.9$ & 0.096 & 0.102 & 0.133 & 0.159 & 0.148 & 0.148 & 0.145 & 0.177 & 0.152 & 0.174 & 0.019 & 0.023 & 0.059 & 0.101 & 0.075 & 0.075 & 0.08 & 0.124 & 0.091 & 0.125 \\ 
  \cmidrule{2-23} & \multirow{3}[2]{*}{msnr} & $\rho=0$ & 0.918 & 1.01 & 1.32 & 1.537 & 1.511 & 1.511 & 1.442 & 1.712 & 1.518 & 1.673 & 0.174 & 0.239 & 0.55 & 0.967 & 0.807 & 0.807 & 0.749 & 1.145 & 0.929 & 1.139 \\ 
   &  & $\rho=0.5$ & 0.601 & 0.658 & 0.86 & 1.023 & 0.99 & 0.99 & 0.938 & 1.136 & 0.995 & 1.11 & 0.167 & 0.224 & 0.571 & 0.955 & 0.828 & 0.828 & 0.79 & 1.171 & 0.887 & 1.144 \\ 
   &  & $\rho=0.9$ & 0.282 & 0.306 & 0.402 & 0.481 & 0.46 & 0.46 & 0.441 & 0.535 & 0.463 & 0.522 & 0.168 & 0.216 & 0.557 & 0.963 & 0.718 & 0.718 & 0.748 & 1.139 & 0.857 & 1.109 \\ 
  \cmidrule{2-23} & \multirow{3}[2]{*}{lsnr} & $\rho=0$ & 2.051 & 2.275 & 2.988 & 3.5 & 3.48 & 3.48 & 3.331 & 3.919 & 3.484 & 3.817 & 0.844 & 1.172 & 2.842 & 4.943 & 3.874 & 3.874 & 3.803 & 5.735 & 4.246 & 5.649 \\ 
   &  & $\rho=0.5$ & 1.356 & 1.481 & 1.95 & 2.315 & 2.333 & 2.333 & 2.199 & 2.588 & 2.323 & 2.534 & 0.847 & 1.07 & 2.685 & 4.796 & 4.273 & 4.273 & 4.17 & 5.727 & 4.54 & 5.734 \\ 
   &  & $\rho=0.9$ & 0.631 & 0.692 & 0.911 & 1.076 & 1.085 & 1.085 & 1.014 & 1.209 & 1.066 & 1.185 & 0.82 & 1.088 & 2.95 & 4.532 & 3.616 & 3.616 & 3.687 & 5.611 & 4.15 & 5.571 \\ 
  \midrule\multirow{9}[6]{*}{$n=100$} & \multirow{3}[2]{*}{hsnr} & $\rho=0$ & 0.141 & 0.143 & 0.147 & 0.151 & 0.152 & 0.15 & 0.149 & 0.153 & 0.149 & 0.1 & 0.003 & 0.003 & 0.004 & 0.004 & 0.004 & 0.004 & 0.004 & 0.004 & 0.004 & 0.002 \\ 
   &  & $\rho=0.5$ & 0.093 & 0.095 & 0.096 & 0.098 & 0.098 & 0.098 & 0.098 & 0.099 & 0.098 & 0.066 & 0.003 & 0.003 & 0.004 & 0.004 & 0.004 & 0.004 & 0.004 & 0.004 & 0.004 & 0.002 \\ 
   &  & $\rho=0.9$ & 0.043 & 0.044 & 0.045 & 0.046 & 0.046 & 0.046 & 0.045 & 0.046 & 0.046 & 0.031 & 0.003 & 0.003 & 0.004 & 0.004 & 0.004 & 0.004 & 0.004 & 0.004 & 0.004 & 0.002 \\ 
  \cmidrule{2-23} & \multirow{3}[2]{*}{msnr} & $\rho=0$ & 0.436 & 0.441 & 0.448 & 0.456 & 0.47 & 0.459 & 0.451 & 0.46 & 0.453 & 0.312 & 0.03 & 0.031 & 0.032 & 0.032 & 0.033 & 0.033 & 0.032 & 0.033 & 0.032 & 0.017 \\ 
   &  & $\rho=0.5$ & 0.289 & 0.297 & 0.299 & 0.304 & 0.306 & 0.303 & 0.302 & 0.306 & 0.303 & 0.207 & 0.031 & 0.032 & 0.032 & 0.033 & 0.033 & 0.033 & 0.033 & 0.034 & 0.033 & 0.017 \\ 
   &  & $\rho=0.9$ & 0.134 & 0.136 & 0.139 & 0.14 & 0.144 & 0.142 & 0.14 & 0.142 & 0.14 & 0.096 & 0.031 & 0.032 & 0.033 & 0.033 & 0.034 & 0.033 & 0.033 & 0.034 & 0.033 & 0.017 \\ 
  \cmidrule{2-23} & \multirow{3}[2]{*}{lsnr} & $\rho=0$ & 0.985 & 0.998 & 1.01 & 1.032 & 1.061 & 1.049 & 1.023 & 1.044 & 1.025 & 0.706 & 0.154 & 0.158 & 0.163 & 0.168 & 0.173 & 0.172 & 0.166 & 0.171 & 0.167 & 0.087 \\ 
   &  & $\rho=0.5$ & 0.649 & 0.663 & 0.669 & 0.677 & 0.687 & 0.685 & 0.674 & 0.685 & 0.676 & 0.457 & 0.155 & 0.161 & 0.163 & 0.167 & 0.166 & 0.167 & 0.165 & 0.17 & 0.166 & 0.084 \\ 
   &  & $\rho=0.9$ & 0.301 & 0.305 & 0.312 & 0.314 & 0.32 & 0.32 & 0.315 & 0.317 & 0.314 & 0.216 & 0.155 & 0.159 & 0.166 & 0.168 & 0.167 & 0.172 & 0.168 & 0.17 & 0.167 & 0.087 \\ 
  \midrule\multirow{9}[6]{*}{$n=500$} & \multirow{3}[2]{*}{hsnr} & $\rho=0$ & 0.064 & 0.065 & 0.065 & 0.065 & 0.065 & 0.065 & 0.065 & 0.065 & 0.065 & 0.042 & 0.001 & 0.001 & 0.001 & 0.001 & 0.001 & 0.001 & 0.001 & 0.001 & 0.001 & 0 \\ 
   &  & $\rho=0.5$ & 0.042 & 0.042 & 0.042 & 0.043 & 0.043 & 0.042 & 0.042 & 0.043 & 0.042 & 0.027 & 0.001 & 0.001 & 0.001 & 0.001 & 0.001 & 0.001 & 0.001 & 0.001 & 0.001 & 0 \\ 
   &  & $\rho=0.9$ & 0.019 & 0.02 & 0.02 & 0.02 & 0.02 & 0.02 & 0.02 & 0.02 & 0.02 & 0.012 & 0.001 & 0.001 & 0.001 & 0.001 & 0.001 & 0.001 & 0.001 & 0.001 & 0.001 & 0 \\ 
  \cmidrule{2-23} & \multirow{3}[2]{*}{msnr} & $\rho=0$ & 0.19 & 0.192 & 0.193 & 0.193 & 0.193 & 0.193 & 0.193 & 0.193 & 0.193 & 0.126 & 0.006 & 0.006 & 0.006 & 0.006 & 0.006 & 0.006 & 0.006 & 0.006 & 0.006 & 0.003 \\ 
   &  & $\rho=0.5$ & 0.123 & 0.124 & 0.125 & 0.126 & 0.126 & 0.125 & 0.125 & 0.126 & 0.125 & 0.082 & 0.006 & 0.006 & 0.006 & 0.006 & 0.006 & 0.006 & 0.006 & 0.006 & 0.006 & 0.002 \\ 
   &  & $\rho=0.9$ & 0.058 & 0.058 & 0.059 & 0.059 & 0.059 & 0.059 & 0.059 & 0.059 & 0.059 & 0.038 & 0.006 & 0.006 & 0.006 & 0.006 & 0.006 & 0.006 & 0.006 & 0.006 & 0.006 & 0.002 \\ 
  \cmidrule{2-23} & \multirow{3}[2]{*}{lsnr} & $\rho=0$ & 0.446 & 0.448 & 0.45 & 0.451 & 0.459 & 0.451 & 0.45 & 0.451 & 0.45 & 0.284 & 0.029 & 0.03 & 0.03 & 0.03 & 0.031 & 0.03 & 0.03 & 0.03 & 0.03 & 0.013 \\ 
   &  & $\rho=0.5$ & 0.292 & 0.294 & 0.294 & 0.295 & 0.301 & 0.298 & 0.294 & 0.296 & 0.295 & 0.185 & 0.03 & 0.03 & 0.03 & 0.03 & 0.031 & 0.03 & 0.03 & 0.03 & 0.03 & 0.013 \\ 
   &  & $\rho=0.9$ & 0.137 & 0.138 & 0.139 & 0.139 & 0.141 & 0.14 & 0.139 & 0.139 & 0.139 & 0.086 & 0.03 & 0.03 & 0.03 & 0.03 & 0.031 & 0.031 & 0.03 & 0.03 & 0.03 & 0.013 \\ 
   \midrule 
 \multicolumn{1}{|c}{} & \multicolumn{1}{c}{} &       & \multicolumn{10}{c||}{Relative Test Error}                                    & \multicolumn{10}{c|}{Proportion of Variance Explained} \\
\midrule\multirow{9}[6]{*}{$n=10$} & \multirow{3}[2]{*}{hsnr} & $\rho=0$ & 1.189 & 1.241 & 1.577 & 1.939 & 1.756 & 1.756 & 1.763 & 2.068 & 1.857 & 2.095 & 0.88 & 0.875 & 0.841 & 0.806 & 0.824 & 0.824 & 0.823 & 0.793 & 0.814 & 0.789 \\ 
   &  & $\rho=0.5$ & 1.184 & 1.229 & 1.561 & 1.981 & 1.761 & 1.761 & 1.781 & 2.14 & 1.846 & 2.131 & 0.88 & 0.876 & 0.842 & 0.8 & 0.823 & 0.823 & 0.818 & 0.784 & 0.812 & 0.786 \\ 
   &  & $\rho=0.9$ & 1.185 & 1.218 & 1.53 & 1.927 & 1.695 & 1.695 & 1.701 & 2.116 & 1.813 & 2.145 & 0.879 & 0.876 & 0.844 & 0.807 & 0.83 & 0.83 & 0.826 & 0.787 & 0.816 & 0.786 \\ 
  \cmidrule{2-23} & \multirow{3}[2]{*}{msnr} & $\rho=0$ & 1.19 & 1.263 & 1.584 & 1.999 & 1.856 & 1.856 & 1.804 & 2.193 & 1.958 & 2.186 & 0.443 & 0.411 & 0.26 & 0.061 & 0.132 & 0.132 & 0.159 & -0.029 & 0.079 & -0.025 \\ 
   &  & $\rho=0.5$ & 1.19 & 1.256 & 1.616 & 2.008 & 1.854 & 1.854 & 1.823 & 2.234 & 1.937 & 2.211 & 0.442 & 0.413 & 0.243 & 0.06 & 0.128 & 0.128 & 0.141 & -0.046 & 0.093 & -0.034 \\ 
   &  & $\rho=0.9$ & 1.187 & 1.24 & 1.6 & 2.035 & 1.773 & 1.773 & 1.769 & 2.205 & 1.885 & 2.184 & 0.441 & 0.417 & 0.248 & 0.05 & 0.173 & 0.173 & 0.161 & -0.032 & 0.108 & -0.02 \\ 
  \cmidrule{2-23} & \multirow{3}[2]{*}{lsnr} & $\rho=0$ & 1.19 & 1.261 & 1.613 & 2.022 & 1.796 & 1.796 & 1.811 & 2.203 & 1.92 & 2.181 & 0.046 & -0.01 & -0.292 & -0.624 & -0.443 & -0.443 & -0.453 & -0.766 & -0.535 & -0.749 \\ 
   &  & $\rho=0.5$ & 1.194 & 1.242 & 1.591 & 2.019 & 1.874 & 1.874 & 1.874 & 2.189 & 1.949 & 2.189 & 0.042 & 0.004 & -0.275 & -0.62 & -0.509 & -0.509 & -0.506 & -0.761 & -0.567 & -0.761 \\ 
   &  & $\rho=0.9$ & 1.183 & 1.239 & 1.623 & 1.987 & 1.777 & 1.777 & 1.78 & 2.19 & 1.876 & 2.19 & 0.049 & 0.003 & -0.306 & -0.591 & -0.426 & -0.426 & -0.43 & -0.759 & -0.508 & -0.757 \\ 
  \midrule\multirow{9}[6]{*}{$n=100$} & \multirow{3}[2]{*}{hsnr} & $\rho=0$ & 1.029 & 1.03 & 1.031 & 1.032 & 1.032 & 1.032 & 1.031 & 1.033 & 1.032 & 1.015 & 0.892 & 0.892 & 0.892 & 0.891 & 0.891 & 0.891 & 0.892 & 0.891 & 0.892 & 0.893 \\ 
   &  & $\rho=0.5$ & 1.029 & 1.03 & 1.03 & 1.032 & 1.03 & 1.031 & 1.031 & 1.032 & 1.031 & 1.015 & 0.892 & 0.892 & 0.892 & 0.891 & 0.892 & 0.891 & 0.891 & 0.891 & 0.891 & 0.893 \\ 
   &  & $\rho=0.9$ & 1.029 & 1.03 & 1.031 & 1.032 & 1.032 & 1.032 & 1.031 & 1.032 & 1.031 & 1.016 & 0.892 & 0.892 & 0.892 & 0.891 & 0.891 & 0.891 & 0.891 & 0.891 & 0.891 & 0.893 \\ 
  \cmidrule{2-23} & \multirow{3}[2]{*}{msnr} & $\rho=0$ & 1.03 & 1.031 & 1.032 & 1.033 & 1.034 & 1.033 & 1.032 & 1.033 & 1.033 & 1.017 & 0.487 & 0.487 & 0.486 & 0.486 & 0.485 & 0.486 & 0.486 & 0.486 & 0.486 & 0.494 \\ 
   &  & $\rho=0.5$ & 1.031 & 1.032 & 1.033 & 1.034 & 1.033 & 1.033 & 1.033 & 1.034 & 1.033 & 1.017 & 0.487 & 0.486 & 0.486 & 0.485 & 0.485 & 0.485 & 0.485 & 0.485 & 0.485 & 0.493 \\ 
   &  & $\rho=0.9$ & 1.031 & 1.032 & 1.033 & 1.033 & 1.034 & 1.034 & 1.033 & 1.034 & 1.033 & 1.017 & 0.487 & 0.486 & 0.486 & 0.485 & 0.485 & 0.485 & 0.486 & 0.485 & 0.486 & 0.494 \\ 
  \cmidrule{2-23} & \multirow{3}[2]{*}{lsnr} & $\rho=0$ & 1.031 & 1.032 & 1.033 & 1.034 & 1.035 & 1.035 & 1.033 & 1.035 & 1.034 & 1.018 & 0.143 & 0.142 & 0.142 & 0.141 & 0.14 & 0.14 & 0.141 & 0.14 & 0.141 & 0.154 \\ 
   &  & $\rho=0.5$ & 1.031 & 1.032 & 1.033 & 1.034 & 1.033 & 1.034 & 1.033 & 1.034 & 1.033 & 1.017 & 0.143 & 0.142 & 0.141 & 0.141 & 0.141 & 0.141 & 0.141 & 0.14 & 0.141 & 0.155 \\ 
   &  & $\rho=0.9$ & 1.031 & 1.032 & 1.033 & 1.034 & 1.034 & 1.034 & 1.034 & 1.034 & 1.034 & 1.018 & 0.143 & 0.142 & 0.141 & 0.141 & 0.141 & 0.14 & 0.141 & 0.14 & 0.141 & 0.154 \\ 
  \midrule\multirow{9}[6]{*}{$n=500$} & \multirow{3}[2]{*}{hsnr} & $\rho=0$ & 1.005 & 1.006 & 1.006 & 1.006 & 1.006 & 1.006 & 1.006 & 1.006 & 1.006 & 1.002 & 0.894 & 0.894 & 0.894 & 0.894 & 0.894 & 0.894 & 0.894 & 0.894 & 0.894 & 0.895 \\ 
   &  & $\rho=0.5$ & 1.005 & 1.005 & 1.005 & 1.006 & 1.006 & 1.006 & 1.006 & 1.006 & 1.006 & 1.002 & 0.894 & 0.894 & 0.894 & 0.894 & 0.894 & 0.894 & 0.894 & 0.894 & 0.894 & 0.895 \\ 
   &  & $\rho=0.9$ & 1.005 & 1.006 & 1.006 & 1.006 & 1.006 & 1.006 & 1.006 & 1.006 & 1.006 & 1.002 & 0.894 & 0.894 & 0.894 & 0.894 & 0.894 & 0.894 & 0.894 & 0.894 & 0.894 & 0.895 \\ 
  \cmidrule{2-23} & \multirow{3}[2]{*}{msnr} & $\rho=0$ & 1.006 & 1.006 & 1.006 & 1.006 & 1.006 & 1.006 & 1.006 & 1.006 & 1.006 & 1.003 & 0.498 & 0.498 & 0.498 & 0.498 & 0.498 & 0.498 & 0.498 & 0.498 & 0.498 & 0.5 \\ 
   &  & $\rho=0.5$ & 1.006 & 1.006 & 1.006 & 1.006 & 1.006 & 1.006 & 1.006 & 1.006 & 1.006 & 1.002 & 0.498 & 0.498 & 0.498 & 0.498 & 0.498 & 0.498 & 0.498 & 0.498 & 0.498 & 0.5 \\ 
   &  & $\rho=0.9$ & 1.006 & 1.006 & 1.006 & 1.006 & 1.006 & 1.006 & 1.006 & 1.006 & 1.006 & 1.002 & 0.498 & 0.498 & 0.498 & 0.498 & 0.498 & 0.498 & 0.498 & 0.498 & 0.498 & 0.5 \\ 
  \cmidrule{2-23} & \multirow{3}[2]{*}{lsnr} & $\rho=0$ & 1.006 & 1.006 & 1.006 & 1.006 & 1.006 & 1.006 & 1.006 & 1.006 & 1.006 & 1.003 & 0.163 & 0.163 & 0.162 & 0.162 & 0.162 & 0.162 & 0.162 & 0.162 & 0.162 & 0.165 \\ 
   &  & $\rho=0.5$ & 1.006 & 1.006 & 1.006 & 1.006 & 1.006 & 1.006 & 1.006 & 1.006 & 1.006 & 1.003 & 0.163 & 0.162 & 0.162 & 0.162 & 0.162 & 0.162 & 0.162 & 0.162 & 0.162 & 0.165 \\ 
   &  & $\rho=0.9$ & 1.006 & 1.006 & 1.006 & 1.006 & 1.006 & 1.006 & 1.006 & 1.006 & 1.006 & 1.003 & 0.162 & 0.162 & 0.162 & 0.162 & 0.162 & 0.162 & 0.162 & 0.162 & 0.162 & 0.165 \\ 
   \midrule 
 \multicolumn{1}{|c}{} & \multicolumn{1}{c}{} &       & \multicolumn{10}{c||}{Kullback-Leibler Discrepancy}                                    & \multicolumn{10}{c|}{\% of Correct Restrictions Selected} \\
\midrule\multirow{9}[6]{*}{$n=10$} & \multirow{3}[2]{*}{hsnr} & $\rho=0$ & 1.874 & 2.157 & 5.253 & 8.972 & 6.709 & 6.709 & 7.38 & 9.734 & 8.136 & 9.947 & 93.6 & 89.8 & 74 & 62.2 & 66 & 66 & 68.5 & 53.9 & 64.6 & 55.7 \\ 
   &  & $\rho=0.5$ & 0.955 & 1.208 & 4.804 & 8.167 & 5.934 & 5.934 & 6.618 & 9.064 & 7.085 & 9.284 & 94.1 & 91.1 & 76 & 63.1 & 67.3 & 67.3 & 70.1 & 53.9 & 66.1 & 56.2 \\ 
   &  & $\rho=0.9$ & -0.578 & -0.415 & 3.273 & 6.573 & 4.137 & 4.137 & 4.709 & 7.617 & 5.495 & 7.852 & 94.1 & 91 & 75 & 62.5 & 67.5 & 67.5 & 69.9 & 53.1 & 65.7 & 55 \\ 
  \cmidrule{2-23} & \multirow{3}[2]{*}{msnr} & $\rho=0$ & 3.939 & 4.532 & 7.328 & 11.166 & 9.218 & 9.218 & 9.658 & 12.29 & 10.584 & 12.366 & 92.8 & 87.8 & 70.6 & 59.3 & 58.6 & 58.6 & 64 & 49.7 & 59.9 & 52.1 \\ 
   &  & $\rho=0.5$ & 3.104 & 3.63 & 6.835 & 10.065 & 8.352 & 8.352 & 8.863 & 11.369 & 9.504 & 11.55 & 93.1 & 87.9 & 70.6 & 57.8 & 57.4 & 57.4 & 64 & 48 & 59.7 & 51.2 \\ 
   &  & $\rho=0.9$ & 1.679 & 1.905 & 5.491 & 8.817 & 6.495 & 6.495 & 7.067 & 9.852 & 7.887 & 9.855 & 93.2 & 88.4 & 70.4 & 58.1 & 57.1 & 57.1 & 64.8 & 47.7 & 61.1 & 50.9 \\ 
  \cmidrule{2-23} & \multirow{3}[2]{*}{lsnr} & $\rho=0$ & 5.704 & 6.314 & 9.17 & 12.737 & 10.598 & 10.598 & 11.185 & 13.712 & 11.957 & 13.734 & 92.4 & 86.5 & 67.8 & 56.5 & 51.4 & 51.4 & 59.8 & 44.5 & 56.1 & 48 \\ 
   &  & $\rho=0.5$ & 4.743 & 5.228 & 8.103 & 11.748 & 9.944 & 9.944 & 10.591 & 12.921 & 11.102 & 13.079 & 91.8 & 86.8 & 68.7 & 55.6 & 48.8 & 48.8 & 59.2 & 44.4 & 54.6 & 47.1 \\ 
   &  & $\rho=0.9$ & 3.177 & 3.584 & 6.777 & 10.33 & 7.943 & 7.943 & 8.726 & 11.439 & 9.367 & 11.527 & 92.3 & 86.6 & 68.5 & 56.1 & 49.7 & 49.7 & 60.8 & 45.8 & 57 & 48.1 \\ 
  \midrule\multirow{9}[6]{*}{$n=100$} & \multirow{3}[2]{*}{hsnr} & $\rho=0$ & 1.247 & 1.248 & 1.249 & 1.25 & 1.25 & 1.25 & 1.25 & 1.251 & 1.25 & 1.23 & 92.4 & 91.8 & 90.8 & 89.3 & 87.7 & 89 & 90.1 & 88.8 & 89.9 & 98.8 \\ 
   &  & $\rho=0.5$ & 0.414 & 0.415 & 0.416 & 0.418 & 0.416 & 0.417 & 0.417 & 0.418 & 0.417 & 0.398 & 92.3 & 90.8 & 90.4 & 88.7 & 89.4 & 89.3 & 89.4 & 88.4 & 89.2 & 99 \\ 
   &  & $\rho=0.9$ & -1.114 & -1.113 & -1.111 & -1.11 & -1.111 & -1.11 & -1.111 & -1.11 & -1.111 & -1.13 & 91.9 & 91.4 & 90.2 & 89.5 & 88.6 & 89 & 89.8 & 89 & 89.8 & 98.7 \\ 
  \cmidrule{2-23} & \multirow{3}[2]{*}{msnr} & $\rho=0$ & 3.388 & 3.389 & 3.39 & 3.391 & 3.392 & 3.391 & 3.391 & 3.392 & 3.391 & 3.373 & 75.3 & 74.7 & 74 & 72.8 & 69.6 & 71.9 & 73.4 & 72.2 & 73.1 & 92.8 \\ 
   &  & $\rho=0.5$ & 2.556 & 2.558 & 2.559 & 2.56 & 2.559 & 2.559 & 2.559 & 2.56 & 2.559 & 2.54 & 76.4 & 74.5 & 74.2 & 73.2 & 72 & 72.3 & 73.5 & 72.7 & 73.4 & 92.6 \\ 
   &  & $\rho=0.9$ & 1.028 & 1.029 & 1.031 & 1.031 & 1.032 & 1.032 & 1.031 & 1.032 & 1.031 & 1.012 & 75.2 & 74.5 & 73.2 & 72.5 & 70.4 & 71.7 & 72.6 & 71.8 & 72.4 & 92.9 \\ 
  \cmidrule{2-23} & \multirow{3}[2]{*}{lsnr} & $\rho=0$ & 4.999 & 4.999 & 5.001 & 5.002 & 5.003 & 5.003 & 5.001 & 5.003 & 5.002 & 4.983 & 73.8 & 72.7 & 72 & 70.6 & 63.9 & 67.6 & 71 & 69.7 & 71 & 93.3 \\ 
   &  & $\rho=0.5$ & 4.166 & 4.168 & 4.168 & 4.169 & 4.169 & 4.169 & 4.169 & 4.17 & 4.169 & 4.15 & 73.3 & 71.7 & 71.3 & 70 & 65.1 & 68 & 70.3 & 69.2 & 70.1 & 93.7 \\ 
   &  & $\rho=0.9$ & 2.638 & 2.639 & 2.641 & 2.641 & 2.641 & 2.642 & 2.641 & 2.642 & 2.641 & 2.622 & 74.3 & 73.5 & 72.1 & 71.4 & 65.4 & 68.6 & 71.1 & 70.3 & 71 & 93.6 \\ 
  \midrule\multirow{9}[6]{*}{$n=500$} & \multirow{3}[2]{*}{hsnr} & $\rho=0$ & 1.215 & 1.215 & 1.215 & 1.215 & 1.215 & 1.215 & 1.215 & 1.215 & 1.215 & 1.211 & 91.7 & 91.4 & 91 & 90.9 & 90.4 & 91.1 & 90.9 & 90.8 & 90.9 & 100 \\ 
   &  & $\rho=0.5$ & 0.381 & 0.381 & 0.381 & 0.381 & 0.381 & 0.381 & 0.381 & 0.382 & 0.381 & 0.378 & 91.6 & 91.2 & 91 & 90.7 & 90 & 91.1 & 90.9 & 90.5 & 90.9 & 100 \\ 
   &  & $\rho=0.9$ & -1.146 & -1.146 & -1.146 & -1.146 & -1.146 & -1.146 & -1.146 & -1.146 & -1.146 & -1.149 & 91.6 & 91.2 & 91.1 & 90.9 & 89.6 & 90.7 & 90.9 & 90.9 & 90.9 & 100 \\ 
  \cmidrule{2-23} & \multirow{3}[2]{*}{msnr} & $\rho=0$ & 3.355 & 3.355 & 3.355 & 3.355 & 3.355 & 3.355 & 3.355 & 3.355 & 3.355 & 3.352 & 89.3 & 89 & 88.6 & 88.5 & 88.1 & 88.7 & 88.6 & 88.5 & 88.5 & 98.6 \\ 
   &  & $\rho=0.5$ & 2.522 & 2.522 & 2.522 & 2.522 & 2.522 & 2.522 & 2.522 & 2.522 & 2.522 & 2.518 & 90 & 89.6 & 89.4 & 89.2 & 88.4 & 89.6 & 89.4 & 89 & 89.4 & 99 \\ 
   &  & $\rho=0.9$ & 0.995 & 0.995 & 0.995 & 0.995 & 0.995 & 0.995 & 0.995 & 0.995 & 0.995 & 0.991 & 89.5 & 89.1 & 88.9 & 88.8 & 87.7 & 88.7 & 88.8 & 88.8 & 88.8 & 98.8 \\ 
  \cmidrule{2-23} & \multirow{3}[2]{*}{lsnr} & $\rho=0$ & 4.965 & 4.965 & 4.965 & 4.965 & 4.965 & 4.965 & 4.965 & 4.965 & 4.965 & 4.961 & 74.6 & 74.4 & 74 & 73.8 & 72.6 & 74.8 & 74 & 73.8 & 73.9 & 97.4 \\ 
   &  & $\rho=0.5$ & 4.131 & 4.131 & 4.131 & 4.131 & 4.132 & 4.132 & 4.131 & 4.131 & 4.131 & 4.128 & 76.2 & 75.8 & 75.7 & 75.6 & 74.3 & 74.8 & 75.7 & 75.4 & 75.6 & 97.4 \\ 
   &  & $\rho=0.9$ & 2.604 & 2.604 & 2.604 & 2.604 & 2.604 & 2.604 & 2.604 & 2.604 & 2.604 & 2.601 & 74.8 & 74.3 & 74.2 & 74.1 & 73.1 & 73.5 & 74 & 74 & 74 & 97.4 \\ 
   \bottomrule 
\end{tabular}
}
\end{table}

% latex table generated in R 4.0.2 by xtable 1.8-4 package
% Fri Jun 26 23:43:30 2020
\begin{table}[ht]
\centering
\caption{Fixed X, Ex1, p=6} 
\scalebox{0.52}{
\begin{tabular}{|c|c|c|cc|cc|cc|ccc|c||cc|cc|cc|ccc|c|}
  \toprule 
 \multicolumn{1}{|c}{} & \multicolumn{1}{c}{} &       & RAICc & AICc  & RC$_p$ & C$_p$ & 10F CV & LOO CV & S$_p$ & $\widehat{\text{C}}_p$ & GCV   & BIC   & RAICc & AICc  & RC$_p$ & C$_p$ & 10F CV & LOO CV & S$_p$ & $\widehat{\text{C}}_p$ & GCV   & BIC  \\
 \cmidrule{4-23}\multicolumn{1}{|c}{} & \multicolumn{1}{c}{} &       & \multicolumn{10}{c||}{RMSE}                                                   & \multicolumn{10}{c|}{Relative Risk}       \\
 \midrule\multirow{9}[6]{*}{$n=10$} & \multirow{3}[2]{*}{hsnr} & $\rho=0$ & 0.319 & 0.324 & 0.383 & 0.436 & 0.386 & 0.386 & 0.402 & 0.46 & 0.416 & 0.456 & 0.017 & 0.018 & 0.028 & 0.036 & 0.028 & 0.028 & 0.03 & 0.038 & 0.032 & 0.038 \\ 
   &  & $\rho=0.5$ & 0.497 & 0.508 & 0.614 & 0.689 & 0.625 & 0.625 & 0.641 & 0.73 & 0.665 & 0.721 & 0.015 & 0.016 & 0.028 & 0.037 & 0.03 & 0.03 & 0.031 & 0.041 & 0.034 & 0.04 \\ 
   &  & $\rho=0.9$ & 0.636 & 0.655 & 0.773 & 0.855 & 0.787 & 0.787 & 0.814 & 0.925 & 0.832 & 0.914 & 0.014 & 0.014 & 0.021 & 0.028 & 0.022 & 0.022 & 0.025 & 0.031 & 0.026 & 0.031 \\ 
  \cmidrule{2-23} & \multirow{3}[2]{*}{msnr} & $\rho=0$ & 0.974 & 0.985 & 1.157 & 1.302 & 1.203 & 1.203 & 1.217 & 1.387 & 1.258 & 1.366 & 0.149 & 0.155 & 0.232 & 0.304 & 0.259 & 0.259 & 0.263 & 0.334 & 0.276 & 0.327 \\ 
   &  & $\rho=0.5$ & 1.538 & 1.592 & 1.845 & 2.057 & 1.927 & 1.927 & 1.933 & 2.213 & 1.991 & 2.18 & 0.149 & 0.157 & 0.228 & 0.31 & 0.257 & 0.257 & 0.263 & 0.353 & 0.282 & 0.35 \\ 
   &  & $\rho=0.9$ & 1.877 & 1.92 & 2.225 & 2.512 & 2.311 & 2.311 & 2.362 & 2.678 & 2.446 & 2.656 & 0.132 & 0.135 & 0.182 & 0.246 & 0.196 & 0.196 & 0.209 & 0.27 & 0.222 & 0.267 \\ 
  \cmidrule{2-23} & \multirow{3}[2]{*}{lsnr} & $\rho=0$ & 2.187 & 2.245 & 2.631 & 2.941 & 2.647 & 2.647 & 2.787 & 3.179 & 2.86 & 3.116 & 0.759 & 0.8 & 1.175 & 1.504 & 1.2 & 1.2 & 1.338 & 1.694 & 1.412 & 1.662 \\ 
   &  & $\rho=0.5$ & 3.412 & 3.482 & 4.076 & 4.607 & 4.265 & 4.265 & 4.352 & 4.921 & 4.462 & 4.876 & 0.716 & 0.747 & 1.112 & 1.567 & 1.223 & 1.223 & 1.317 & 1.723 & 1.405 & 1.734 \\ 
   &  & $\rho=0.9$ & 4.133 & 4.264 & 4.953 & 5.573 & 5.092 & 5.092 & 5.279 & 5.974 & 5.461 & 5.877 & 0.593 & 0.625 & 0.87 & 1.19 & 0.932 & 0.932 & 1.001 & 1.32 & 1.096 & 1.302 \\ 
  \midrule\multirow{9}[6]{*}{$n=100$} & \multirow{3}[2]{*}{hsnr} & $\rho=0$ & 0.17 & 0.171 & 0.174 & 0.177 & 0.169 & 0.172 & 0.176 & 0.178 & 0.176 & 0.148 & 0.003 & 0.003 & 0.003 & 0.003 & 0.003 & 0.003 & 0.003 & 0.003 & 0.003 & 0.002 \\ 
   &  & $\rho=0.5$ & 0.255 & 0.258 & 0.262 & 0.264 & 0.253 & 0.259 & 0.263 & 0.266 & 0.263 & 0.224 & 0.003 & 0.003 & 0.003 & 0.003 & 0.003 & 0.003 & 0.003 & 0.003 & 0.003 & 0.002 \\ 
   &  & $\rho=0.9$ & 0.395 & 0.4 & 0.405 & 0.41 & 0.389 & 0.398 & 0.408 & 0.413 & 0.409 & 0.335 & 0.004 & 0.004 & 0.004 & 0.004 & 0.004 & 0.004 & 0.004 & 0.004 & 0.004 & 0.002 \\ 
  \cmidrule{2-23} & \multirow{3}[2]{*}{msnr} & $\rho=0$ & 0.526 & 0.529 & 0.535 & 0.543 & 0.523 & 0.529 & 0.54 & 0.546 & 0.541 & 0.46 & 0.028 & 0.028 & 0.029 & 0.03 & 0.03 & 0.029 & 0.029 & 0.03 & 0.03 & 0.019 \\ 
   &  & $\rho=0.5$ & 0.801 & 0.81 & 0.822 & 0.829 & 0.795 & 0.816 & 0.826 & 0.833 & 0.827 & 0.696 & 0.029 & 0.03 & 0.031 & 0.031 & 0.031 & 0.031 & 0.031 & 0.032 & 0.031 & 0.02 \\ 
   &  & $\rho=0.9$ & 1.156 & 1.167 & 1.18 & 1.195 & 1.12 & 1.162 & 1.192 & 1.202 & 1.193 & 0.973 & 0.029 & 0.03 & 0.031 & 0.032 & 0.03 & 0.03 & 0.031 & 0.032 & 0.031 & 0.019 \\ 
  \cmidrule{2-23} & \multirow{3}[2]{*}{lsnr} & $\rho=0$ & 1.237 & 1.249 & 1.265 & 1.285 & 1.243 & 1.246 & 1.283 & 1.293 & 1.285 & 1.066 & 0.153 & 0.156 & 0.16 & 0.166 & 0.163 & 0.16 & 0.165 & 0.169 & 0.166 & 0.103 \\ 
   &  & $\rho=0.5$ & 1.842 & 1.859 & 1.878 & 1.897 & 1.82 & 1.855 & 1.892 & 1.912 & 1.902 & 1.557 & 0.152 & 0.155 & 0.158 & 0.161 & 0.156 & 0.157 & 0.161 & 0.164 & 0.162 & 0.098 \\ 
   &  & $\rho=0.9$ & 2.558 & 2.59 & 2.603 & 2.629 & 2.497 & 2.571 & 2.619 & 2.644 & 2.62 & 2.118 & 0.148 & 0.152 & 0.154 & 0.157 & 0.152 & 0.154 & 0.156 & 0.159 & 0.156 & 0.092 \\ 
  \midrule\multirow{9}[6]{*}{$n=500$} & \multirow{3}[2]{*}{hsnr} & $\rho=0$ & 0.072 & 0.072 & 0.072 & 0.072 & 0.068 & 0.072 & 0.072 & 0.072 & 0.072 & 0.062 & 0.001 & 0.001 & 0.001 & 0.001 & 0.001 & 0.001 & 0.001 & 0.001 & 0.001 & 0 \\ 
   &  & $\rho=0.5$ & 0.11 & 0.11 & 0.11 & 0.11 & 0.102 & 0.108 & 0.11 & 0.11 & 0.11 & 0.094 & 0.001 & 0.001 & 0.001 & 0.001 & 0.001 & 0.001 & 0.001 & 0.001 & 0.001 & 0 \\ 
   &  & $\rho=0.9$ & 0.161 & 0.161 & 0.161 & 0.162 & 0.151 & 0.158 & 0.161 & 0.162 & 0.161 & 0.136 & 0.001 & 0.001 & 0.001 & 0.001 & 0.001 & 0.001 & 0.001 & 0.001 & 0.001 & 0 \\ 
  \cmidrule{2-23} & \multirow{3}[2]{*}{msnr} & $\rho=0$ & 0.212 & 0.212 & 0.212 & 0.213 & 0.198 & 0.21 & 0.213 & 0.213 & 0.213 & 0.183 & 0.005 & 0.005 & 0.005 & 0.005 & 0.005 & 0.005 & 0.005 & 0.005 & 0.005 & 0.003 \\ 
   &  & $\rho=0.5$ & 0.332 & 0.332 & 0.334 & 0.335 & 0.308 & 0.328 & 0.334 & 0.335 & 0.334 & 0.287 & 0.005 & 0.005 & 0.005 & 0.005 & 0.005 & 0.005 & 0.005 & 0.005 & 0.005 & 0.004 \\ 
   &  & $\rho=0.9$ & 0.5 & 0.502 & 0.503 & 0.503 & 0.478 & 0.498 & 0.503 & 0.503 & 0.503 & 0.424 & 0.006 & 0.006 & 0.006 & 0.006 & 0.006 & 0.006 & 0.006 & 0.006 & 0.006 & 0.004 \\ 
  \cmidrule{2-23} & \multirow{3}[2]{*}{lsnr} & $\rho=0$ & 0.497 & 0.498 & 0.5 & 0.501 & 0.469 & 0.495 & 0.501 & 0.501 & 0.501 & 0.428 & 0.026 & 0.026 & 0.026 & 0.027 & 0.025 & 0.026 & 0.027 & 0.027 & 0.027 & 0.019 \\ 
   &  & $\rho=0.5$ & 0.765 & 0.767 & 0.768 & 0.769 & 0.732 & 0.762 & 0.769 & 0.77 & 0.769 & 0.657 & 0.027 & 0.028 & 0.028 & 0.028 & 0.027 & 0.028 & 0.028 & 0.028 & 0.028 & 0.02 \\ 
   &  & $\rho=0.9$ & 1.106 & 1.106 & 1.109 & 1.112 & 1.033 & 1.098 & 1.111 & 1.114 & 1.111 & 0.918 & 0.028 & 0.028 & 0.028 & 0.028 & 0.025 & 0.027 & 0.028 & 0.028 & 0.028 & 0.018 \\ 
   \midrule 
 \multicolumn{1}{|c}{} & \multicolumn{1}{c}{} &       & \multicolumn{10}{c||}{Relative Test Error}                                    & \multicolumn{10}{c|}{Proportion of Variance Explained} \\
\midrule\multirow{9}[6]{*}{$n=10$} & \multirow{3}[2]{*}{hsnr} & $\rho=0$ & 1.327 & 1.345 & 1.548 & 1.702 & 1.544 & 1.544 & 1.593 & 1.751 & 1.624 & 1.739 & 0.936 & 0.935 & 0.925 & 0.918 & 0.925 & 0.925 & 0.923 & 0.915 & 0.921 & 0.916 \\ 
   &  & $\rho=0.5$ & 1.271 & 1.287 & 1.505 & 1.677 & 1.545 & 1.545 & 1.558 & 1.744 & 1.618 & 1.732 & 0.933 & 0.933 & 0.921 & 0.912 & 0.919 & 0.919 & 0.918 & 0.909 & 0.915 & 0.909 \\ 
   &  & $\rho=0.9$ & 1.329 & 1.345 & 1.518 & 1.667 & 1.544 & 1.544 & 1.598 & 1.761 & 1.619 & 1.762 & 0.947 & 0.947 & 0.94 & 0.934 & 0.939 & 0.939 & 0.937 & 0.93 & 0.936 & 0.93 \\ 
  \cmidrule{2-23} & \multirow{3}[2]{*}{msnr} & $\rho=0$ & 1.344 & 1.357 & 1.537 & 1.703 & 1.599 & 1.599 & 1.608 & 1.771 & 1.639 & 1.755 & 0.594 & 0.59 & 0.536 & 0.485 & 0.517 & 0.517 & 0.514 & 0.465 & 0.505 & 0.47 \\ 
   &  & $\rho=0.5$ & 1.318 & 1.333 & 1.485 & 1.661 & 1.547 & 1.547 & 1.561 & 1.751 & 1.6 & 1.746 & 0.579 & 0.574 & 0.526 & 0.469 & 0.506 & 0.506 & 0.501 & 0.44 & 0.489 & 0.442 \\ 
   &  & $\rho=0.9$ & 1.375 & 1.385 & 1.519 & 1.7 & 1.558 & 1.558 & 1.596 & 1.768 & 1.633 & 1.76 & 0.643 & 0.64 & 0.605 & 0.558 & 0.595 & 0.595 & 0.585 & 0.54 & 0.575 & 0.543 \\ 
  \cmidrule{2-23} & \multirow{3}[2]{*}{lsnr} & $\rho=0$ & 1.351 & 1.37 & 1.543 & 1.695 & 1.554 & 1.554 & 1.618 & 1.782 & 1.652 & 1.768 & 0.076 & 0.063 & -0.055 & -0.159 & -0.063 & -0.063 & -0.107 & -0.219 & -0.13 & -0.209 \\ 
   &  & $\rho=0.5$ & 1.305 & 1.318 & 1.473 & 1.667 & 1.521 & 1.521 & 1.561 & 1.734 & 1.598 & 1.738 & 0.085 & 0.076 & -0.033 & -0.169 & -0.067 & -0.067 & -0.095 & -0.216 & -0.121 & -0.219 \\ 
   &  & $\rho=0.9$ & 1.337 & 1.356 & 1.495 & 1.678 & 1.531 & 1.531 & 1.57 & 1.752 & 1.624 & 1.741 & 0.148 & 0.136 & 0.047 & -0.069 & 0.025 & 0.025 & 0 & -0.116 & -0.035 & -0.109 \\ 
  \midrule\multirow{9}[6]{*}{$n=100$} & \multirow{3}[2]{*}{hsnr} & $\rho=0$ & 1.026 & 1.027 & 1.028 & 1.029 & 1.029 & 1.029 & 1.028 & 1.029 & 1.029 & 1.018 & 0.9 & 0.899 & 0.899 & 0.899 & 0.899 & 0.899 & 0.899 & 0.899 & 0.899 & 0.9 \\ 
   &  & $\rho=0.5$ & 1.027 & 1.028 & 1.029 & 1.03 & 1.03 & 1.029 & 1.029 & 1.03 & 1.029 & 1.019 & 0.902 & 0.902 & 0.902 & 0.901 & 0.901 & 0.902 & 0.902 & 0.901 & 0.902 & 0.902 \\ 
   &  & $\rho=0.9$ & 1.033 & 1.034 & 1.035 & 1.036 & 1.035 & 1.034 & 1.035 & 1.036 & 1.035 & 1.022 & 0.9 & 0.9 & 0.9 & 0.9 & 0.9 & 0.9 & 0.9 & 0.9 & 0.9 & 0.901 \\ 
  \cmidrule{2-23} & \multirow{3}[2]{*}{msnr} & $\rho=0$ & 1.03 & 1.03 & 1.031 & 1.032 & 1.032 & 1.032 & 1.032 & 1.033 & 1.032 & 1.021 & 0.506 & 0.506 & 0.505 & 0.505 & 0.505 & 0.505 & 0.505 & 0.504 & 0.505 & 0.51 \\ 
   &  & $\rho=0.5$ & 1.032 & 1.033 & 1.034 & 1.035 & 1.034 & 1.034 & 1.035 & 1.035 & 1.035 & 1.022 & 0.511 & 0.511 & 0.51 & 0.51 & 0.51 & 0.51 & 0.51 & 0.51 & 0.51 & 0.516 \\ 
   &  & $\rho=0.9$ & 1.032 & 1.033 & 1.034 & 1.035 & 1.032 & 1.034 & 1.035 & 1.035 & 1.035 & 1.02 & 0.509 & 0.508 & 0.508 & 0.508 & 0.509 & 0.508 & 0.508 & 0.507 & 0.508 & 0.514 \\ 
  \cmidrule{2-23} & \multirow{3}[2]{*}{lsnr} & $\rho=0$ & 1.033 & 1.034 & 1.035 & 1.036 & 1.035 & 1.035 & 1.036 & 1.037 & 1.036 & 1.022 & 0.151 & 0.15 & 0.15 & 0.149 & 0.149 & 0.15 & 0.149 & 0.148 & 0.149 & 0.16 \\ 
   &  & $\rho=0.5$ & 1.034 & 1.034 & 1.035 & 1.036 & 1.035 & 1.035 & 1.036 & 1.036 & 1.036 & 1.022 & 0.154 & 0.154 & 0.153 & 0.153 & 0.154 & 0.153 & 0.153 & 0.152 & 0.152 & 0.164 \\ 
   &  & $\rho=0.9$ & 1.032 & 1.034 & 1.034 & 1.035 & 1.033 & 1.034 & 1.034 & 1.035 & 1.034 & 1.02 & 0.154 & 0.153 & 0.153 & 0.152 & 0.153 & 0.153 & 0.152 & 0.152 & 0.152 & 0.164 \\ 
  \midrule\multirow{9}[6]{*}{$n=500$} & \multirow{3}[2]{*}{hsnr} & $\rho=0$ & 1.005 & 1.005 & 1.005 & 1.005 & 1.005 & 1.005 & 1.005 & 1.005 & 1.005 & 1.004 & 0.906 & 0.906 & 0.906 & 0.906 & 0.906 & 0.906 & 0.906 & 0.906 & 0.906 & 0.907 \\ 
   &  & $\rho=0.5$ & 1.005 & 1.005 & 1.006 & 1.006 & 1.005 & 1.005 & 1.006 & 1.006 & 1.006 & 1.004 & 0.905 & 0.905 & 0.905 & 0.905 & 0.905 & 0.905 & 0.905 & 0.905 & 0.905 & 0.906 \\ 
   &  & $\rho=0.9$ & 1.006 & 1.006 & 1.006 & 1.006 & 1.006 & 1.006 & 1.006 & 1.006 & 1.006 & 1.004 & 0.904 & 0.904 & 0.904 & 0.904 & 0.904 & 0.904 & 0.904 & 0.904 & 0.904 & 0.904 \\ 
  \cmidrule{2-23} & \multirow{3}[2]{*}{msnr} & $\rho=0$ & 1.006 & 1.006 & 1.006 & 1.006 & 1.005 & 1.006 & 1.006 & 1.006 & 1.006 & 1.004 & 0.532 & 0.532 & 0.532 & 0.532 & 0.532 & 0.532 & 0.532 & 0.532 & 0.532 & 0.532 \\ 
   &  & $\rho=0.5$ & 1.006 & 1.006 & 1.006 & 1.006 & 1.006 & 1.006 & 1.006 & 1.006 & 1.006 & 1.004 & 0.528 & 0.528 & 0.528 & 0.528 & 0.528 & 0.528 & 0.528 & 0.528 & 0.528 & 0.529 \\ 
   &  & $\rho=0.9$ & 1.006 & 1.006 & 1.006 & 1.006 & 1.006 & 1.006 & 1.006 & 1.006 & 1.006 & 1.005 & 0.523 & 0.523 & 0.523 & 0.523 & 0.523 & 0.523 & 0.523 & 0.523 & 0.523 & 0.524 \\ 
  \cmidrule{2-23} & \multirow{3}[2]{*}{lsnr} & $\rho=0$ & 1.006 & 1.006 & 1.006 & 1.006 & 1.006 & 1.006 & 1.006 & 1.006 & 1.006 & 1.004 & 0.182 & 0.182 & 0.182 & 0.182 & 0.182 & 0.182 & 0.182 & 0.182 & 0.182 & 0.183 \\ 
   &  & $\rho=0.5$ & 1.006 & 1.006 & 1.006 & 1.006 & 1.006 & 1.006 & 1.006 & 1.006 & 1.006 & 1.004 & 0.18 & 0.18 & 0.18 & 0.18 & 0.18 & 0.18 & 0.18 & 0.18 & 0.18 & 0.181 \\ 
   &  & $\rho=0.9$ & 1.006 & 1.006 & 1.006 & 1.006 & 1.006 & 1.006 & 1.006 & 1.006 & 1.006 & 1.004 & 0.177 & 0.177 & 0.177 & 0.177 & 0.177 & 0.177 & 0.177 & 0.176 & 0.177 & 0.178 \\ 
   \midrule 
 \multicolumn{1}{|c}{} & \multicolumn{1}{c}{} &       & \multicolumn{10}{c||}{Kullback-Leibler Discrepancy}                                    & \multicolumn{10}{c|}{\% of Correct Restrictions Selected} \\
\midrule\multirow{9}[6]{*}{$n=10$} & \multirow{3}[2]{*}{hsnr} & $\rho=0$ & 1.314 & 1.679 & 3.766 & 5.359 & 4.343 & 4.343 & 4.505 & 5.702 & 4.928 & 5.73 & 89.2 & 88.3 & 77 & 65.4 & 71.6 & 71.6 & 72.2 & 59.6 & 69.4 & 61 \\ 
   &  & $\rho=0.5$ & 2.202 & 2.303 & 4.889 & 6.261 & 5.343 & 5.343 & 5.614 & 6.627 & 5.96 & 6.647 & 85.7 & 84.8 & 71.9 & 61.6 & 67.6 & 67.6 & 68.1 & 55.4 & 64.8 & 57 \\ 
   &  & $\rho=0.9$ & 2.779 & 3.221 & 5.159 & 6.506 & 5.827 & 5.827 & 5.839 & 7.007 & 6.176 & 7.026 & 75.7 & 73.5 & 61.6 & 53.6 & 59.3 & 59.3 & 58.1 & 46.2 & 56.2 & 48.1 \\ 
  \cmidrule{2-23} & \multirow{3}[2]{*}{msnr} & $\rho=0$ & 3.552 & 3.959 & 5.988 & 7.462 & 6.614 & 6.614 & 6.69 & 7.839 & 7.105 & 7.903 & 61.3 & 60.8 & 52.9 & 45.1 & 50.7 & 50.7 & 49.4 & 40.2 & 47.1 & 41.8 \\ 
   &  & $\rho=0.5$ & 4.507 & 5.156 & 7.033 & 8.288 & 7.614 & 7.614 & 7.758 & 8.707 & 8.034 & 8.766 & 56 & 54.4 & 46.2 & 39.6 & 44.8 & 44.8 & 44 & 34.1 & 41.9 & 35.5 \\ 
   &  & $\rho=0.9$ & 4.917 & 5.099 & 7.332 & 8.63 & 7.817 & 7.817 & 8.031 & 9.117 & 8.351 & 9.148 & 46.5 & 45.6 & 39.6 & 33.3 & 42 & 42 & 36 & 29 & 34.6 & 29.8 \\ 
  \cmidrule{2-23} & \multirow{3}[2]{*}{lsnr} & $\rho=0$ & 5.167 & 5.625 & 7.369 & 8.914 & 8.158 & 8.158 & 8.305 & 9.469 & 8.672 & 9.513 & 42.8 & 42.3 & 36 & 30.6 & 33.8 & 33.8 & 33 & 26.7 & 32.1 & 28 \\ 
   &  & $\rho=0.5$ & 6.098 & 6.292 & 8.41 & 9.874 & 9.106 & 9.106 & 9.189 & 10.307 & 9.556 & 10.364 & 37.6 & 37.3 & 31.1 & 26.2 & 29.3 & 29.3 & 29.1 & 23.1 & 27.7 & 23.9 \\ 
   &  & $\rho=0.9$ & 6.479 & 6.694 & 8.85 & 10.261 & 9.329 & 9.329 & 9.779 & 10.67 & 9.986 & 10.725 & 37.9 & 37.1 & 31.1 & 26.2 & 31.7 & 31.7 & 28.4 & 23.1 & 27.2 & 23.6 \\ 
  \midrule\multirow{9}[6]{*}{$n=100$} & \multirow{3}[2]{*}{hsnr} & $\rho=0$ & 1.526 & 1.526 & 1.527 & 1.528 & 1.528 & 1.528 & 1.528 & 1.529 & 1.528 & 1.517 & 91.4 & 91 & 89.9 & 88.8 & 87.3 & 89 & 89.3 & 88.2 & 89 & 99.6 \\ 
   &  & $\rho=0.5$ & 2.308 & 2.309 & 2.31 & 2.311 & 2.311 & 2.311 & 2.311 & 2.311 & 2.311 & 2.3 & 90.3 & 89.5 & 88.6 & 88 & 86.6 & 87.9 & 88.3 & 87.5 & 88.3 & 98.6 \\ 
   &  & $\rho=0.9$ & 3.021 & 3.022 & 3.023 & 3.024 & 3.022 & 3.022 & 3.023 & 3.024 & 3.023 & 3.009 & 74.3 & 73.5 & 72.6 & 71.7 & 72.7 & 73.5 & 72.1 & 71.2 & 72 & 84.6 \\ 
  \cmidrule{2-23} & \multirow{3}[2]{*}{msnr} & $\rho=0$ & 3.669 & 3.67 & 3.671 & 3.672 & 3.671 & 3.671 & 3.671 & 3.672 & 3.671 & 3.66 & 75.1 & 74.8 & 74.1 & 73.2 & 73.3 & 74.3 & 73.5 & 72.6 & 73.4 & 82.1 \\ 
   &  & $\rho=0.5$ & 4.453 & 4.454 & 4.455 & 4.456 & 4.455 & 4.455 & 4.455 & 4.456 & 4.456 & 4.443 & 66.5 & 65.8 & 65 & 64.6 & 66.3 & 66.2 & 64.8 & 64 & 64.7 & 74.1 \\ 
   &  & $\rho=0.9$ & 5.16 & 5.161 & 5.161 & 5.162 & 5.16 & 5.161 & 5.162 & 5.163 & 5.162 & 5.148 & 46.5 & 45.8 & 45.2 & 44.7 & 52.1 & 46.7 & 44.7 & 44.5 & 44.7 & 53.6 \\ 
  \cmidrule{2-23} & \multirow{3}[2]{*}{lsnr} & $\rho=0$ & 5.282 & 5.282 & 5.283 & 5.284 & 5.284 & 5.283 & 5.284 & 5.285 & 5.284 & 5.271 & 50.9 & 50.4 & 50 & 48.9 & 53 & 50.3 & 49 & 48.4 & 48.8 & 56.9 \\ 
   &  & $\rho=0.5$ & 6.063 & 6.064 & 6.065 & 6.066 & 6.064 & 6.065 & 6.065 & 6.066 & 6.066 & 6.052 & 45.3 & 44.9 & 44.5 & 44.1 & 49.6 & 45.1 & 44.2 & 43.8 & 44.2 & 52.1 \\ 
   &  & $\rho=0.9$ & 6.769 & 6.77 & 6.771 & 6.772 & 6.77 & 6.771 & 6.771 & 6.772 & 6.771 & 6.757 & 41 & 40.5 & 40.2 & 39.8 & 44.6 & 40.9 & 39.9 & 39.6 & 40 & 47.3 \\ 
  \midrule\multirow{9}[6]{*}{$n=500$} & \multirow{3}[2]{*}{hsnr} & $\rho=0$ & 1.438 & 1.439 & 1.439 & 1.439 & 1.438 & 1.439 & 1.439 & 1.439 & 1.439 & 1.437 & 91.1 & 91.1 & 91 & 90.9 & 89.7 & 90.5 & 90.9 & 90.9 & 90.9 & 99.8 \\ 
   &  & $\rho=0.5$ & 2.258 & 2.258 & 2.258 & 2.258 & 2.257 & 2.258 & 2.258 & 2.258 & 2.258 & 2.256 & 90.7 & 90.7 & 90.5 & 90.4 & 90.1 & 90.8 & 90.5 & 90.4 & 90.5 & 99.8 \\ 
   &  & $\rho=0.9$ & 2.977 & 2.977 & 2.977 & 2.977 & 2.976 & 2.977 & 2.977 & 2.977 & 2.977 & 2.975 & 88.8 & 88.8 & 88.6 & 88.5 & 87.9 & 89.1 & 88.6 & 88.4 & 88.6 & 98.9 \\ 
  \cmidrule{2-23} & \multirow{3}[2]{*}{msnr} & $\rho=0$ & 3.579 & 3.579 & 3.579 & 3.579 & 3.578 & 3.579 & 3.579 & 3.579 & 3.579 & 3.577 & 90 & 90 & 89.9 & 89.8 & 88.8 & 89.7 & 89.8 & 89.8 & 89.8 & 98.5 \\ 
   &  & $\rho=0.5$ & 4.398 & 4.398 & 4.398 & 4.399 & 4.398 & 4.398 & 4.398 & 4.399 & 4.398 & 4.397 & 85.7 & 85.6 & 85.3 & 85 & 85.9 & 85.6 & 85.2 & 85 & 85.2 & 94.5 \\ 
   &  & $\rho=0.9$ & 5.117 & 5.117 & 5.117 & 5.117 & 5.117 & 5.117 & 5.117 & 5.117 & 5.117 & 5.115 & 66.2 & 66.1 & 65.9 & 65.9 & 67.1 & 66.2 & 65.9 & 65.9 & 65.9 & 75.2 \\ 
  \cmidrule{2-23} & \multirow{3}[2]{*}{lsnr} & $\rho=0$ & 5.189 & 5.189 & 5.189 & 5.189 & 5.188 & 5.189 & 5.189 & 5.189 & 5.189 & 5.187 & 74.9 & 74.9 & 74.7 & 74.6 & 76.2 & 74.4 & 74.6 & 74.6 & 74.6 & 82.7 \\ 
   &  & $\rho=0.5$ & 6.008 & 6.008 & 6.008 & 6.008 & 6.008 & 6.008 & 6.008 & 6.008 & 6.008 & 6.006 & 68 & 67.9 & 67.8 & 67.7 & 68.6 & 66.7 & 67.8 & 67.6 & 67.8 & 75.1 \\ 
   &  & $\rho=0.9$ & 6.726 & 6.726 & 6.726 & 6.726 & 6.726 & 6.726 & 6.726 & 6.727 & 6.726 & 6.724 & 47.3 & 47.3 & 47.1 & 47 & 52.9 & 47.2 & 47.2 & 47.1 & 47.2 & 53.7 \\ 
   \bottomrule 
\end{tabular}
}
\end{table}

% latex table generated in R 4.0.2 by xtable 1.8-4 package
% Fri Jun 26 23:43:31 2020
\begin{table}[ht]
\centering
\caption{Fixed X, Ex2, p=6} 
\scalebox{0.52}{
\begin{tabular}{|c|c|c|cc|cc|cc|ccc|c||cc|cc|cc|ccc|c|}
  \toprule 
 \multicolumn{1}{|c}{} & \multicolumn{1}{c}{} &       & RAICc & AICc  & RC$_p$ & C$_p$ & 10F CV & LOO CV & S$_p$ & $\widehat{\text{C}}_p$ & GCV   & BIC   & RAICc & AICc  & RC$_p$ & C$_p$ & 10F CV & LOO CV & S$_p$ & $\widehat{\text{C}}_p$ & GCV   & BIC  \\
 \cmidrule{4-23}\multicolumn{1}{|c}{} & \multicolumn{1}{c}{} &       & \multicolumn{10}{c||}{RMSE}                                                   & \multicolumn{10}{c|}{Relative Risk}       \\
 \midrule\multirow{9}[6]{*}{$n=10$} & \multirow{3}[2]{*}{hsnr} & $\rho=0$ & 1.176 & 1.18 & 1.231 & 1.306 & 1.256 & 1.256 & 1.285 & 1.34 & 1.299 & 1.346 & 0.062 & 0.062 & 0.095 & 0.158 & 0.116 & 0.116 & 0.14 & 0.177 & 0.149 & 0.18 \\ 
   &  & $\rho=0.5$ & 0.838 & 0.842 & 0.873 & 0.922 & 0.893 & 0.893 & 0.908 & 0.949 & 0.915 & 0.956 & 0.062 & 0.067 & 0.099 & 0.148 & 0.108 & 0.108 & 0.13 & 0.172 & 0.137 & 0.182 \\ 
   &  & $\rho=0.9$ & 0.392 & 0.392 & 0.409 & 0.432 & 0.419 & 0.419 & 0.426 & 0.443 & 0.429 & 0.446 & 0.061 & 0.065 & 0.113 & 0.169 & 0.137 & 0.137 & 0.159 & 0.198 & 0.166 & 0.203 \\ 
  \cmidrule{2-23} & \multirow{3}[2]{*}{msnr} & $\rho=0$ & 3.369 & 3.388 & 3.544 & 3.765 & 3.593 & 3.593 & 3.699 & 3.877 & 3.745 & 3.896 & 0.5 & 0.536 & 0.81 & 1.225 & 0.913 & 0.913 & 1.087 & 1.442 & 1.164 & 1.474 \\ 
   &  & $\rho=0.5$ & 2.408 & 2.421 & 2.521 & 2.68 & 2.586 & 2.586 & 2.645 & 2.757 & 2.668 & 2.77 & 0.513 & 0.545 & 0.834 & 1.304 & 1.047 & 1.047 & 1.209 & 1.515 & 1.262 & 1.569 \\ 
   &  & $\rho=0.9$ & 1.12 & 1.123 & 1.175 & 1.247 & 1.211 & 1.211 & 1.232 & 1.286 & 1.242 & 1.296 & 0.496 & 0.528 & 0.952 & 1.419 & 1.223 & 1.223 & 1.323 & 1.653 & 1.403 & 1.724 \\ 
  \cmidrule{2-23} & \multirow{3}[2]{*}{lsnr} & $\rho=0$ & 7.503 & 7.532 & 7.899 & 8.413 & 8.044 & 8.044 & 8.256 & 8.666 & 8.342 & 8.738 & 2.489 & 2.53 & 4.072 & 6.353 & 4.814 & 4.814 & 5.568 & 7.348 & 5.792 & 7.664 \\ 
   &  & $\rho=0.5$ & 5.362 & 5.389 & 5.621 & 5.979 & 5.755 & 5.755 & 5.895 & 6.18 & 5.955 & 6.211 & 2.534 & 2.61 & 4.271 & 6.514 & 5.162 & 5.162 & 5.927 & 7.823 & 6.293 & 8.086 \\ 
   &  & $\rho=0.9$ & 2.496 & 2.507 & 2.621 & 2.786 & 2.701 & 2.701 & 2.755 & 2.884 & 2.783 & 2.9 & 2.483 & 2.628 & 4.62 & 7.135 & 6.212 & 6.212 & 6.692 & 8.482 & 7.083 & 8.639 \\ 
  \midrule\multirow{9}[6]{*}{$n=100$} & \multirow{3}[2]{*}{hsnr} & $\rho=0$ & 0.288 & 0.29 & 0.292 & 0.295 & 0.29 & 0.291 & 0.294 & 0.296 & 0.294 & 0.271 & 0.006 & 0.006 & 0.006 & 0.006 & 0.006 & 0.006 & 0.006 & 0.007 & 0.006 & 0.005 \\ 
   &  & $\rho=0.5$ & 0.2 & 0.202 & 0.203 & 0.205 & 0.2 & 0.202 & 0.204 & 0.205 & 0.205 & 0.189 & 0.006 & 0.006 & 0.006 & 0.006 & 0.006 & 0.006 & 0.006 & 0.006 & 0.006 & 0.005 \\ 
   &  & $\rho=0.9$ & 0.093 & 0.094 & 0.094 & 0.095 & 0.092 & 0.093 & 0.095 & 0.095 & 0.095 & 0.088 & 0.006 & 0.006 & 0.006 & 0.006 & 0.006 & 0.006 & 0.006 & 0.006 & 0.006 & 0.005 \\ 
  \cmidrule{2-23} & \multirow{3}[2]{*}{msnr} & $\rho=0$ & 0.862 & 0.867 & 0.871 & 0.876 & 0.872 & 0.869 & 0.875 & 0.882 & 0.876 & 0.815 & 0.054 & 0.055 & 0.056 & 0.056 & 0.056 & 0.056 & 0.056 & 0.057 & 0.056 & 0.048 \\ 
   &  & $\rho=0.5$ & 0.603 & 0.607 & 0.608 & 0.612 & 0.611 & 0.607 & 0.612 & 0.615 & 0.612 & 0.569 & 0.055 & 0.056 & 0.056 & 0.057 & 0.056 & 0.056 & 0.057 & 0.058 & 0.057 & 0.047 \\ 
   &  & $\rho=0.9$ & 0.28 & 0.282 & 0.283 & 0.284 & 0.282 & 0.282 & 0.284 & 0.286 & 0.284 & 0.266 & 0.055 & 0.056 & 0.056 & 0.057 & 0.055 & 0.056 & 0.056 & 0.057 & 0.056 & 0.048 \\ 
  \cmidrule{2-23} & \multirow{3}[2]{*}{lsnr} & $\rho=0$ & 1.92 & 1.935 & 1.94 & 1.95 & 1.916 & 1.928 & 1.949 & 1.961 & 1.949 & 1.812 & 0.269 & 0.273 & 0.275 & 0.278 & 0.27 & 0.274 & 0.278 & 0.281 & 0.278 & 0.235 \\ 
   &  & $\rho=0.5$ & 1.35 & 1.36 & 1.362 & 1.37 & 1.34 & 1.347 & 1.368 & 1.372 & 1.37 & 1.268 & 0.273 & 0.278 & 0.279 & 0.282 & 0.27 & 0.275 & 0.281 & 0.283 & 0.282 & 0.232 \\ 
   &  & $\rho=0.9$ & 0.628 & 0.63 & 0.633 & 0.637 & 0.63 & 0.627 & 0.637 & 0.641 & 0.637 & 0.59 & 0.274 & 0.277 & 0.28 & 0.284 & 0.28 & 0.275 & 0.284 & 0.288 & 0.284 & 0.232 \\ 
  \midrule\multirow{9}[6]{*}{$n=500$} & \multirow{3}[2]{*}{hsnr} & $\rho=0$ & 0.118 & 0.119 & 0.119 & 0.119 & 0.116 & 0.118 & 0.119 & 0.119 & 0.119 & 0.11 & 0.001 & 0.001 & 0.001 & 0.001 & 0.001 & 0.001 & 0.001 & 0.001 & 0.001 & 0.001 \\ 
   &  & $\rho=0.5$ & 0.084 & 0.084 & 0.084 & 0.084 & 0.082 & 0.084 & 0.084 & 0.084 & 0.084 & 0.078 & 0.001 & 0.001 & 0.001 & 0.001 & 0.001 & 0.001 & 0.001 & 0.001 & 0.001 & 0.001 \\ 
   &  & $\rho=0.9$ & 0.039 & 0.039 & 0.039 & 0.039 & 0.038 & 0.039 & 0.039 & 0.039 & 0.039 & 0.036 & 0.001 & 0.001 & 0.001 & 0.001 & 0.001 & 0.001 & 0.001 & 0.001 & 0.001 & 0.001 \\ 
  \cmidrule{2-23} & \multirow{3}[2]{*}{msnr} & $\rho=0$ & 0.349 & 0.349 & 0.35 & 0.351 & 0.345 & 0.351 & 0.35 & 0.351 & 0.351 & 0.324 & 0.008 & 0.008 & 0.008 & 0.008 & 0.008 & 0.008 & 0.008 & 0.008 & 0.008 & 0.007 \\ 
   &  & $\rho=0.5$ & 0.246 & 0.247 & 0.247 & 0.247 & 0.242 & 0.246 & 0.247 & 0.248 & 0.247 & 0.229 & 0.009 & 0.009 & 0.009 & 0.009 & 0.008 & 0.009 & 0.009 & 0.009 & 0.009 & 0.007 \\ 
   &  & $\rho=0.9$ & 0.114 & 0.114 & 0.114 & 0.114 & 0.112 & 0.114 & 0.114 & 0.114 & 0.114 & 0.106 & 0.009 & 0.009 & 0.009 & 0.009 & 0.008 & 0.009 & 0.009 & 0.009 & 0.009 & 0.007 \\ 
  \cmidrule{2-23} & \multirow{3}[2]{*}{lsnr} & $\rho=0$ & 0.792 & 0.794 & 0.795 & 0.795 & 0.788 & 0.796 & 0.795 & 0.795 & 0.795 & 0.738 & 0.042 & 0.042 & 0.043 & 0.043 & 0.042 & 0.043 & 0.043 & 0.043 & 0.043 & 0.035 \\ 
   &  & $\rho=0.5$ & 0.564 & 0.564 & 0.564 & 0.564 & 0.559 & 0.564 & 0.564 & 0.564 & 0.564 & 0.527 & 0.044 & 0.044 & 0.044 & 0.044 & 0.044 & 0.044 & 0.044 & 0.044 & 0.044 & 0.037 \\ 
   &  & $\rho=0.9$ & 0.261 & 0.261 & 0.262 & 0.262 & 0.258 & 0.261 & 0.262 & 0.261 & 0.262 & 0.246 & 0.045 & 0.045 & 0.045 & 0.045 & 0.044 & 0.045 & 0.045 & 0.045 & 0.045 & 0.039 \\ 
   \midrule 
 \multicolumn{1}{|c}{} & \multicolumn{1}{c}{} &       & \multicolumn{10}{c||}{Relative Test Error}                                    & \multicolumn{10}{c|}{Proportion of Variance Explained} \\
\midrule\multirow{9}[6]{*}{$n=10$} & \multirow{3}[2]{*}{hsnr} & $\rho=0$ & 1.271 & 1.274 & 1.421 & 1.698 & 1.511 & 1.511 & 1.616 & 1.779 & 1.659 & 1.794 & 0.765 & 0.764 & 0.737 & 0.686 & 0.721 & 0.721 & 0.701 & 0.671 & 0.693 & 0.668 \\ 
   &  & $\rho=0.5$ & 1.265 & 1.286 & 1.422 & 1.632 & 1.462 & 1.462 & 1.553 & 1.732 & 1.582 & 1.775 & 0.759 & 0.755 & 0.73 & 0.69 & 0.722 & 0.722 & 0.705 & 0.671 & 0.699 & 0.662 \\ 
   &  & $\rho=0.9$ & 1.258 & 1.272 & 1.474 & 1.711 & 1.578 & 1.578 & 1.669 & 1.833 & 1.699 & 1.855 & 0.759 & 0.756 & 0.717 & 0.672 & 0.697 & 0.697 & 0.68 & 0.648 & 0.674 & 0.644 \\ 
  \cmidrule{2-23} & \multirow{3}[2]{*}{msnr} & $\rho=0$ & 1.259 & 1.278 & 1.42 & 1.636 & 1.474 & 1.474 & 1.564 & 1.749 & 1.604 & 1.765 & 0.171 & 0.159 & 0.065 & -0.077 & 0.03 & 0.03 & -0.03 & -0.151 & -0.056 & -0.162 \\ 
   &  & $\rho=0.5$ & 1.257 & 1.273 & 1.418 & 1.653 & 1.524 & 1.524 & 1.606 & 1.759 & 1.632 & 1.786 & 0.163 & 0.152 & 0.055 & -0.102 & -0.016 & -0.016 & -0.07 & -0.172 & -0.087 & -0.19 \\ 
   &  & $\rho=0.9$ & 1.246 & 1.261 & 1.471 & 1.703 & 1.606 & 1.606 & 1.655 & 1.819 & 1.695 & 1.854 & 0.167 & 0.157 & 0.016 & -0.139 & -0.074 & -0.074 & -0.107 & -0.216 & -0.133 & -0.24 \\ 
  \cmidrule{2-23} & \multirow{3}[2]{*}{lsnr} & $\rho=0$ & 1.258 & 1.263 & 1.423 & 1.659 & 1.5 & 1.5 & 1.578 & 1.763 & 1.601 & 1.796 & -0.14 & -0.144 & -0.289 & -0.503 & -0.359 & -0.359 & -0.43 & -0.597 & -0.451 & -0.627 \\ 
   &  & $\rho=0.5$ & 1.254 & 1.262 & 1.428 & 1.653 & 1.517 & 1.517 & 1.594 & 1.784 & 1.63 & 1.81 & -0.14 & -0.147 & -0.298 & -0.502 & -0.379 & -0.379 & -0.449 & -0.621 & -0.482 & -0.645 \\ 
   &  & $\rho=0.9$ & 1.246 & 1.26 & 1.458 & 1.707 & 1.615 & 1.615 & 1.663 & 1.84 & 1.702 & 1.856 & -0.134 & -0.147 & -0.326 & -0.553 & -0.47 & -0.47 & -0.513 & -0.674 & -0.548 & -0.689 \\ 
  \midrule\multirow{9}[6]{*}{$n=100$} & \multirow{3}[2]{*}{hsnr} & $\rho=0$ & 1.05 & 1.051 & 1.052 & 1.053 & 1.052 & 1.051 & 1.052 & 1.053 & 1.052 & 1.043 & 0.885 & 0.885 & 0.885 & 0.885 & 0.885 & 0.885 & 0.885 & 0.885 & 0.885 & 0.886 \\ 
   &  & $\rho=0.5$ & 1.049 & 1.05 & 1.05 & 1.051 & 1.05 & 1.05 & 1.051 & 1.052 & 1.051 & 1.043 & 0.885 & 0.885 & 0.884 & 0.884 & 0.885 & 0.884 & 0.884 & 0.884 & 0.884 & 0.885 \\ 
   &  & $\rho=0.9$ & 1.048 & 1.049 & 1.05 & 1.051 & 1.048 & 1.049 & 1.051 & 1.051 & 1.051 & 1.041 & 0.883 & 0.883 & 0.883 & 0.883 & 0.883 & 0.883 & 0.883 & 0.882 & 0.883 & 0.884 \\ 
  \cmidrule{2-23} & \multirow{3}[2]{*}{msnr} & $\rho=0$ & 1.052 & 1.053 & 1.053 & 1.054 & 1.053 & 1.053 & 1.054 & 1.055 & 1.054 & 1.046 & 0.462 & 0.462 & 0.462 & 0.461 & 0.462 & 0.462 & 0.461 & 0.461 & 0.461 & 0.465 \\ 
   &  & $\rho=0.5$ & 1.052 & 1.053 & 1.053 & 1.054 & 1.053 & 1.053 & 1.054 & 1.055 & 1.054 & 1.045 & 0.461 & 0.461 & 0.46 & 0.46 & 0.46 & 0.46 & 0.46 & 0.46 & 0.46 & 0.465 \\ 
   &  & $\rho=0.9$ & 1.051 & 1.052 & 1.053 & 1.053 & 1.052 & 1.052 & 1.053 & 1.054 & 1.053 & 1.045 & 0.457 & 0.456 & 0.456 & 0.456 & 0.456 & 0.456 & 0.456 & 0.455 & 0.456 & 0.46 \\ 
  \cmidrule{2-23} & \multirow{3}[2]{*}{lsnr} & $\rho=0$ & 1.051 & 1.052 & 1.053 & 1.053 & 1.052 & 1.052 & 1.053 & 1.054 & 1.053 & 1.045 & 0.117 & 0.117 & 0.116 & 0.116 & 0.117 & 0.117 & 0.116 & 0.115 & 0.116 & 0.123 \\ 
   &  & $\rho=0.5$ & 1.052 & 1.053 & 1.053 & 1.054 & 1.051 & 1.052 & 1.054 & 1.054 & 1.054 & 1.044 & 0.116 & 0.116 & 0.115 & 0.115 & 0.117 & 0.116 & 0.115 & 0.115 & 0.115 & 0.123 \\ 
   &  & $\rho=0.9$ & 1.051 & 1.052 & 1.052 & 1.053 & 1.052 & 1.051 & 1.053 & 1.054 & 1.053 & 1.043 & 0.114 & 0.114 & 0.113 & 0.113 & 0.113 & 0.114 & 0.113 & 0.112 & 0.113 & 0.121 \\ 
  \midrule\multirow{9}[6]{*}{$n=500$} & \multirow{3}[2]{*}{hsnr} & $\rho=0$ & 1.009 & 1.009 & 1.009 & 1.009 & 1.008 & 1.009 & 1.009 & 1.009 & 1.009 & 1.007 & 0.901 & 0.901 & 0.901 & 0.901 & 0.901 & 0.901 & 0.901 & 0.901 & 0.901 & 0.901 \\ 
   &  & $\rho=0.5$ & 1.009 & 1.009 & 1.009 & 1.009 & 1.008 & 1.009 & 1.009 & 1.009 & 1.009 & 1.007 & 0.898 & 0.898 & 0.898 & 0.898 & 0.898 & 0.898 & 0.898 & 0.898 & 0.898 & 0.898 \\ 
   &  & $\rho=0.9$ & 1.009 & 1.009 & 1.009 & 1.009 & 1.008 & 1.009 & 1.009 & 1.009 & 1.009 & 1.007 & 0.897 & 0.897 & 0.897 & 0.897 & 0.897 & 0.897 & 0.897 & 0.897 & 0.897 & 0.897 \\ 
  \cmidrule{2-23} & \multirow{3}[2]{*}{msnr} & $\rho=0$ & 1.009 & 1.009 & 1.009 & 1.009 & 1.009 & 1.009 & 1.009 & 1.009 & 1.009 & 1.007 & 0.514 & 0.514 & 0.514 & 0.514 & 0.514 & 0.514 & 0.514 & 0.514 & 0.514 & 0.515 \\ 
   &  & $\rho=0.5$ & 1.009 & 1.009 & 1.009 & 1.009 & 1.009 & 1.009 & 1.009 & 1.009 & 1.009 & 1.007 & 0.506 & 0.506 & 0.506 & 0.506 & 0.506 & 0.506 & 0.506 & 0.505 & 0.506 & 0.506 \\ 
   &  & $\rho=0.9$ & 1.009 & 1.009 & 1.009 & 1.009 & 1.009 & 1.009 & 1.009 & 1.009 & 1.009 & 1.007 & 0.503 & 0.503 & 0.503 & 0.503 & 0.503 & 0.503 & 0.503 & 0.503 & 0.503 & 0.504 \\ 
  \cmidrule{2-23} & \multirow{3}[2]{*}{lsnr} & $\rho=0$ & 1.009 & 1.009 & 1.009 & 1.009 & 1.009 & 1.009 & 1.009 & 1.009 & 1.009 & 1.008 & 0.17 & 0.17 & 0.17 & 0.17 & 0.17 & 0.17 & 0.17 & 0.17 & 0.17 & 0.171 \\ 
   &  & $\rho=0.5$ & 1.009 & 1.009 & 1.009 & 1.009 & 1.009 & 1.009 & 1.009 & 1.009 & 1.009 & 1.008 & 0.165 & 0.165 & 0.165 & 0.165 & 0.165 & 0.165 & 0.165 & 0.165 & 0.165 & 0.166 \\ 
   &  & $\rho=0.9$ & 1.009 & 1.009 & 1.009 & 1.009 & 1.009 & 1.009 & 1.009 & 1.009 & 1.009 & 1.008 & 0.163 & 0.163 & 0.163 & 0.163 & 0.163 & 0.163 & 0.163 & 0.163 & 0.163 & 0.164 \\ 
   \midrule 
 \multicolumn{1}{|c}{} & \multicolumn{1}{c}{} &       & \multicolumn{10}{c||}{Kullback-Leibler Discrepancy}                                    & \multicolumn{10}{c|}{\% of Correct Restrictions Selected} \\
\midrule\multirow{9}[6]{*}{$n=10$} & \multirow{3}[2]{*}{hsnr} & $\rho=0$ & 4.121 & 4.387 & 6.154 & 7.487 & 6.749 & 6.749 & 7.17 & 7.851 & 7.315 & 7.944 & 71.3 & 70.9 & 65.1 & 54.1 & 58.3 & 58.3 & 57.5 & 48.1 & 55.3 & 46.4 \\ 
   &  & $\rho=0.5$ & 3.414 & 3.88 & 5.054 & 6.696 & 6.059 & 6.059 & 6.381 & 7.181 & 6.611 & 7.278 & 73.8 & 73.2 & 67.7 & 57 & 58.2 & 58.2 & 60.5 & 49.1 & 58.7 & 47.1 \\ 
   &  & $\rho=0.9$ & 1.902 & 2.051 & 3.918 & 5.157 & 4.553 & 4.553 & 4.929 & 5.607 & 5.067 & 5.74 & 70.9 & 70.8 & 64.9 & 55 & 55.3 & 55.3 & 58.2 & 48.8 & 56.4 & 46.7 \\ 
  \cmidrule{2-23} & \multirow{3}[2]{*}{msnr} & $\rho=0$ & 6.25 & 6.751 & 8.318 & 9.567 & 8.861 & 8.861 & 9.31 & 9.994 & 9.467 & 10.076 & 49.2 & 48.8 & 44.1 & 36.2 & 35.1 & 35.1 & 38.4 & 31.2 & 36.8 & 29.9 \\ 
   &  & $\rho=0.5$ & 6.034 & 6.427 & 7.8 & 8.903 & 8.292 & 8.292 & 8.646 & 9.312 & 8.828 & 9.405 & 46.3 & 46 & 42.2 & 34.5 & 32.7 & 32.7 & 36.5 & 30.1 & 35.2 & 29 \\ 
   &  & $\rho=0.9$ & 4.036 & 4.619 & 5.92 & 7.315 & 6.766 & 6.766 & 7.144 & 7.761 & 7.285 & 7.862 & 44.9 & 44.7 & 40.8 & 34.4 & 32.3 & 32.3 & 36.2 & 30.1 & 34.8 & 28.6 \\ 
  \cmidrule{2-23} & \multirow{3}[2]{*}{lsnr} & $\rho=0$ & 7.838 & 8.123 & 9.941 & 11.229 & 10.603 & 10.603 & 10.95 & 11.582 & 11.102 & 11.717 & 43.5 & 43.2 & 39.5 & 32.2 & 30.7 & 30.7 & 34.4 & 28.4 & 32.6 & 27 \\ 
   &  & $\rho=0.5$ & 7.161 & 7.807 & 9.141 & 10.581 & 9.991 & 9.991 & 10.333 & 10.96 & 10.498 & 11.051 & 38.2 & 37.7 & 34.4 & 28.5 & 25.4 & 25.4 & 30.3 & 25.3 & 29 & 24.2 \\ 
   &  & $\rho=0.9$ & 6.09 & 6.471 & 7.895 & 9.019 & 8.416 & 8.416 & 8.848 & 9.399 & 9.014 & 9.486 & 38.7 & 38.2 & 34.7 & 28.4 & 26.4 & 26.4 & 29.7 & 24.1 & 28.4 & 23.1 \\ 
  \midrule\multirow{9}[6]{*}{$n=100$} & \multirow{3}[2]{*}{hsnr} & $\rho=0$ & 1.828 & 1.829 & 1.829 & 1.83 & 1.829 & 1.83 & 1.83 & 1.831 & 1.83 & 1.821 & 88.4 & 86.8 & 85.4 & 83.9 & 82.7 & 84.1 & 84.6 & 83 & 84.4 & 97.4 \\ 
   &  & $\rho=0.5$ & 1.11 & 1.111 & 1.111 & 1.112 & 1.11 & 1.111 & 1.112 & 1.113 & 1.112 & 1.103 & 89.8 & 88.1 & 87.5 & 86 & 85.7 & 87 & 86.2 & 85.2 & 86 & 98.1 \\ 
   &  & $\rho=0.9$ & -0.415 & -0.414 & -0.414 & -0.413 & -0.415 & -0.414 & -0.413 & -0.412 & -0.413 & -0.422 & 90.1 & 88.9 & 88.1 & 86.6 & 87.4 & 88 & 86.9 & 86 & 86.9 & 97.9 \\ 
  \cmidrule{2-23} & \multirow{3}[2]{*}{msnr} & $\rho=0$ & 3.97 & 3.971 & 3.971 & 3.972 & 3.971 & 3.971 & 3.972 & 3.973 & 3.972 & 3.964 & 65.1 & 64.1 & 63.4 & 62.5 & 61.5 & 64.4 & 62.7 & 61.7 & 62.6 & 71.5 \\ 
   &  & $\rho=0.5$ & 3.253 & 3.253 & 3.254 & 3.255 & 3.254 & 3.254 & 3.255 & 3.255 & 3.255 & 3.246 & 68.8 & 67.8 & 67.5 & 66.1 & 64.3 & 67.3 & 66.2 & 65.4 & 66.1 & 75.4 \\ 
   &  & $\rho=0.9$ & 1.728 & 1.729 & 1.729 & 1.73 & 1.729 & 1.729 & 1.729 & 1.73 & 1.729 & 1.722 & 67.6 & 66.8 & 66.4 & 65.7 & 65.8 & 67.5 & 66 & 65.2 & 66 & 73.1 \\ 
  \cmidrule{2-23} & \multirow{3}[2]{*}{lsnr} & $\rho=0$ & 5.579 & 5.58 & 5.58 & 5.581 & 5.579 & 5.58 & 5.581 & 5.581 & 5.581 & 5.572 & 47.2 & 46.5 & 46.1 & 45.5 & 45.2 & 47.8 & 45.6 & 45 & 45.6 & 52 \\ 
   &  & $\rho=0.5$ & 4.862 & 4.863 & 4.863 & 4.863 & 4.861 & 4.862 & 4.863 & 4.864 & 4.863 & 4.855 & 45.9 & 45 & 44.7 & 44.5 & 45.6 & 46.8 & 44.5 & 44.2 & 44.5 & 50.7 \\ 
   &  & $\rho=0.9$ & 3.337 & 3.338 & 3.338 & 3.339 & 3.338 & 3.337 & 3.339 & 3.34 & 3.339 & 3.33 & 45.9 & 45.4 & 45.2 & 44.4 & 44.1 & 46.1 & 44.4 & 43.9 & 44.4 & 50.4 \\ 
  \midrule\multirow{9}[6]{*}{$n=500$} & \multirow{3}[2]{*}{hsnr} & $\rho=0$ & 1.659 & 1.659 & 1.659 & 1.659 & 1.658 & 1.659 & 1.659 & 1.659 & 1.659 & 1.657 & 89.7 & 89.2 & 89 & 88.8 & 88.5 & 89 & 88.9 & 88.7 & 88.8 & 100 \\ 
   &  & $\rho=0.5$ & 0.971 & 0.971 & 0.971 & 0.971 & 0.971 & 0.971 & 0.971 & 0.971 & 0.971 & 0.97 & 89.9 & 89.7 & 89.6 & 89.5 & 89.6 & 90.1 & 89.5 & 89.5 & 89.5 & 100 \\ 
   &  & $\rho=0.9$ & -0.562 & -0.562 & -0.562 & -0.562 & -0.563 & -0.562 & -0.562 & -0.562 & -0.562 & -0.564 & 90.9 & 90.8 & 90.7 & 90.6 & 91.4 & 90.7 & 90.6 & 90.6 & 90.6 & 99.9 \\ 
  \cmidrule{2-23} & \multirow{3}[2]{*}{msnr} & $\rho=0$ & 3.799 & 3.799 & 3.799 & 3.799 & 3.799 & 3.799 & 3.799 & 3.799 & 3.799 & 3.797 & 86.8 & 86.5 & 86.3 & 85.9 & 84.2 & 85.3 & 86.1 & 85.8 & 86 & 98 \\ 
   &  & $\rho=0.5$ & 3.112 & 3.112 & 3.112 & 3.112 & 3.111 & 3.112 & 3.112 & 3.112 & 3.112 & 3.11 & 89.1 & 88.8 & 88.6 & 88.5 & 87.9 & 89 & 88.5 & 88.4 & 88.5 & 98.8 \\ 
   &  & $\rho=0.9$ & 1.578 & 1.578 & 1.578 & 1.578 & 1.578 & 1.578 & 1.578 & 1.578 & 1.578 & 1.577 & 89.6 & 89.5 & 89.4 & 89.3 & 89.5 & 89.1 & 89.3 & 89.3 & 89.3 & 98.3 \\ 
  \cmidrule{2-23} & \multirow{3}[2]{*}{lsnr} & $\rho=0$ & 5.409 & 5.409 & 5.409 & 5.409 & 5.409 & 5.409 & 5.409 & 5.409 & 5.409 & 5.407 & 69.9 & 69.4 & 69.3 & 69.3 & 67 & 68.8 & 69.3 & 69.3 & 69.3 & 79.1 \\ 
   &  & $\rho=0.5$ & 4.721 & 4.721 & 4.721 & 4.721 & 4.721 & 4.721 & 4.721 & 4.721 & 4.721 & 4.72 & 71.5 & 71.4 & 71.4 & 71.4 & 70.7 & 70.9 & 71.4 & 71.4 & 71.4 & 80.2 \\ 
   &  & $\rho=0.9$ & 3.188 & 3.188 & 3.188 & 3.188 & 3.188 & 3.188 & 3.188 & 3.188 & 3.188 & 3.187 & 70.9 & 70.9 & 70.8 & 70.7 & 71.4 & 70.7 & 70.7 & 70.8 & 70.7 & 77.2 \\ 
   \bottomrule 
\end{tabular}
}
\end{table}

% latex table generated in R 4.0.2 by xtable 1.8-4 package
% Fri Jun 26 23:43:32 2020
\begin{table}[ht]
\centering
\caption{Random X, Ex3, p=6} 
\scalebox{0.52}{
\begin{tabular}{|c|c|c|cc|cc|cc|ccc|c||cc|cc|cc|ccc|c|}
  \toprule 
 \multicolumn{1}{|c}{} & \multicolumn{1}{c}{} &       & RAICc & AICc  & RC$_p$ & C$_p$ & 10F CV & LOO CV & S$_p$ & $\widehat{\text{C}}_p$ & GCV   & BIC   & RAICc & AICc  & RC$_p$ & C$_p$ & 10F CV & LOO CV & S$_p$ & $\widehat{\text{C}}_p$ & GCV   & BIC  \\
 \cmidrule{4-23}\multicolumn{1}{|c}{} & \multicolumn{1}{c}{} &       & \multicolumn{10}{c||}{RMSE}                                                   & \multicolumn{10}{c|}{Relative Risk}       \\
 \midrule\multirow{9}[6]{*}{$n=10$} & \multirow{3}[2]{*}{hsnr} & $\rho=0$ & 0.312 & 0.338 & 0.44 & 0.518 & 0.491 & 0.491 & 0.477 & 0.565 & 0.502 & 0.558 & 0.02 & 0.025 & 0.064 & 0.104 & 0.084 & 0.084 & 0.084 & 0.118 & 0.095 & 0.123 \\ 
   &  & $\rho=0.5$ & 0.204 & 0.221 & 0.284 & 0.339 & 0.323 & 0.323 & 0.31 & 0.377 & 0.328 & 0.368 & 0.019 & 0.024 & 0.062 & 0.109 & 0.084 & 0.084 & 0.089 & 0.129 & 0.097 & 0.126 \\ 
   &  & $\rho=0.9$ & 0.096 & 0.102 & 0.133 & 0.159 & 0.148 & 0.148 & 0.145 & 0.177 & 0.152 & 0.174 & 0.019 & 0.023 & 0.059 & 0.101 & 0.075 & 0.075 & 0.08 & 0.124 & 0.091 & 0.125 \\ 
  \cmidrule{2-23} & \multirow{3}[2]{*}{msnr} & $\rho=0$ & 0.918 & 1.01 & 1.32 & 1.537 & 1.511 & 1.511 & 1.442 & 1.712 & 1.518 & 1.673 & 0.174 & 0.239 & 0.55 & 0.967 & 0.807 & 0.807 & 0.749 & 1.145 & 0.929 & 1.139 \\ 
   &  & $\rho=0.5$ & 0.601 & 0.658 & 0.86 & 1.023 & 0.99 & 0.99 & 0.938 & 1.136 & 0.995 & 1.11 & 0.167 & 0.224 & 0.571 & 0.955 & 0.828 & 0.828 & 0.79 & 1.171 & 0.887 & 1.144 \\ 
   &  & $\rho=0.9$ & 0.282 & 0.306 & 0.402 & 0.481 & 0.46 & 0.46 & 0.441 & 0.535 & 0.463 & 0.522 & 0.168 & 0.216 & 0.557 & 0.963 & 0.718 & 0.718 & 0.748 & 1.139 & 0.857 & 1.109 \\ 
  \cmidrule{2-23} & \multirow{3}[2]{*}{lsnr} & $\rho=0$ & 2.051 & 2.275 & 2.988 & 3.5 & 3.48 & 3.48 & 3.331 & 3.919 & 3.484 & 3.817 & 0.844 & 1.172 & 2.842 & 4.943 & 3.874 & 3.874 & 3.803 & 5.735 & 4.246 & 5.649 \\ 
   &  & $\rho=0.5$ & 1.356 & 1.481 & 1.95 & 2.315 & 2.333 & 2.333 & 2.199 & 2.588 & 2.323 & 2.534 & 0.847 & 1.07 & 2.685 & 4.796 & 4.273 & 4.273 & 4.17 & 5.727 & 4.54 & 5.734 \\ 
   &  & $\rho=0.9$ & 0.631 & 0.692 & 0.911 & 1.076 & 1.085 & 1.085 & 1.014 & 1.209 & 1.066 & 1.185 & 0.82 & 1.088 & 2.95 & 4.532 & 3.616 & 3.616 & 3.687 & 5.611 & 4.15 & 5.571 \\ 
  \midrule\multirow{9}[6]{*}{$n=100$} & \multirow{3}[2]{*}{hsnr} & $\rho=0$ & 0.141 & 0.143 & 0.147 & 0.151 & 0.152 & 0.15 & 0.149 & 0.153 & 0.149 & 0.1 & 0.003 & 0.003 & 0.004 & 0.004 & 0.004 & 0.004 & 0.004 & 0.004 & 0.004 & 0.002 \\ 
   &  & $\rho=0.5$ & 0.093 & 0.095 & 0.096 & 0.098 & 0.098 & 0.098 & 0.098 & 0.099 & 0.098 & 0.066 & 0.003 & 0.003 & 0.004 & 0.004 & 0.004 & 0.004 & 0.004 & 0.004 & 0.004 & 0.002 \\ 
   &  & $\rho=0.9$ & 0.043 & 0.044 & 0.045 & 0.046 & 0.046 & 0.046 & 0.045 & 0.046 & 0.046 & 0.031 & 0.003 & 0.003 & 0.004 & 0.004 & 0.004 & 0.004 & 0.004 & 0.004 & 0.004 & 0.002 \\ 
  \cmidrule{2-23} & \multirow{3}[2]{*}{msnr} & $\rho=0$ & 0.436 & 0.441 & 0.448 & 0.456 & 0.47 & 0.459 & 0.451 & 0.46 & 0.453 & 0.312 & 0.03 & 0.031 & 0.032 & 0.032 & 0.033 & 0.033 & 0.032 & 0.033 & 0.032 & 0.017 \\ 
   &  & $\rho=0.5$ & 0.289 & 0.297 & 0.299 & 0.304 & 0.306 & 0.303 & 0.302 & 0.306 & 0.303 & 0.207 & 0.031 & 0.032 & 0.032 & 0.033 & 0.033 & 0.033 & 0.033 & 0.034 & 0.033 & 0.017 \\ 
   &  & $\rho=0.9$ & 0.134 & 0.136 & 0.139 & 0.14 & 0.144 & 0.142 & 0.14 & 0.142 & 0.14 & 0.096 & 0.031 & 0.032 & 0.033 & 0.033 & 0.034 & 0.033 & 0.033 & 0.034 & 0.033 & 0.017 \\ 
  \cmidrule{2-23} & \multirow{3}[2]{*}{lsnr} & $\rho=0$ & 0.985 & 0.998 & 1.01 & 1.032 & 1.061 & 1.049 & 1.023 & 1.044 & 1.025 & 0.706 & 0.154 & 0.158 & 0.163 & 0.168 & 0.173 & 0.172 & 0.166 & 0.171 & 0.167 & 0.087 \\ 
   &  & $\rho=0.5$ & 0.649 & 0.663 & 0.669 & 0.677 & 0.687 & 0.685 & 0.674 & 0.685 & 0.676 & 0.457 & 0.155 & 0.161 & 0.163 & 0.167 & 0.166 & 0.167 & 0.165 & 0.17 & 0.166 & 0.084 \\ 
   &  & $\rho=0.9$ & 0.301 & 0.305 & 0.312 & 0.314 & 0.32 & 0.32 & 0.315 & 0.317 & 0.314 & 0.216 & 0.155 & 0.159 & 0.166 & 0.168 & 0.167 & 0.172 & 0.168 & 0.17 & 0.167 & 0.087 \\ 
  \midrule\multirow{9}[6]{*}{$n=500$} & \multirow{3}[2]{*}{hsnr} & $\rho=0$ & 0.064 & 0.065 & 0.065 & 0.065 & 0.065 & 0.065 & 0.065 & 0.065 & 0.065 & 0.042 & 0.001 & 0.001 & 0.001 & 0.001 & 0.001 & 0.001 & 0.001 & 0.001 & 0.001 & 0 \\ 
   &  & $\rho=0.5$ & 0.042 & 0.042 & 0.042 & 0.043 & 0.043 & 0.042 & 0.042 & 0.043 & 0.042 & 0.027 & 0.001 & 0.001 & 0.001 & 0.001 & 0.001 & 0.001 & 0.001 & 0.001 & 0.001 & 0 \\ 
   &  & $\rho=0.9$ & 0.019 & 0.02 & 0.02 & 0.02 & 0.02 & 0.02 & 0.02 & 0.02 & 0.02 & 0.012 & 0.001 & 0.001 & 0.001 & 0.001 & 0.001 & 0.001 & 0.001 & 0.001 & 0.001 & 0 \\ 
  \cmidrule{2-23} & \multirow{3}[2]{*}{msnr} & $\rho=0$ & 0.19 & 0.192 & 0.193 & 0.193 & 0.193 & 0.193 & 0.193 & 0.193 & 0.193 & 0.126 & 0.006 & 0.006 & 0.006 & 0.006 & 0.006 & 0.006 & 0.006 & 0.006 & 0.006 & 0.003 \\ 
   &  & $\rho=0.5$ & 0.123 & 0.124 & 0.125 & 0.126 & 0.126 & 0.125 & 0.125 & 0.126 & 0.125 & 0.082 & 0.006 & 0.006 & 0.006 & 0.006 & 0.006 & 0.006 & 0.006 & 0.006 & 0.006 & 0.002 \\ 
   &  & $\rho=0.9$ & 0.058 & 0.058 & 0.059 & 0.059 & 0.059 & 0.059 & 0.059 & 0.059 & 0.059 & 0.038 & 0.006 & 0.006 & 0.006 & 0.006 & 0.006 & 0.006 & 0.006 & 0.006 & 0.006 & 0.002 \\ 
  \cmidrule{2-23} & \multirow{3}[2]{*}{lsnr} & $\rho=0$ & 0.446 & 0.448 & 0.45 & 0.451 & 0.459 & 0.451 & 0.45 & 0.451 & 0.45 & 0.284 & 0.029 & 0.03 & 0.03 & 0.03 & 0.031 & 0.03 & 0.03 & 0.03 & 0.03 & 0.013 \\ 
   &  & $\rho=0.5$ & 0.292 & 0.294 & 0.294 & 0.295 & 0.301 & 0.298 & 0.294 & 0.296 & 0.295 & 0.185 & 0.03 & 0.03 & 0.03 & 0.03 & 0.031 & 0.03 & 0.03 & 0.03 & 0.03 & 0.013 \\ 
   &  & $\rho=0.9$ & 0.137 & 0.138 & 0.139 & 0.139 & 0.141 & 0.14 & 0.139 & 0.139 & 0.139 & 0.086 & 0.03 & 0.03 & 0.03 & 0.03 & 0.031 & 0.031 & 0.03 & 0.03 & 0.03 & 0.013 \\ 
   \midrule 
 \multicolumn{1}{|c}{} & \multicolumn{1}{c}{} &       & \multicolumn{10}{c||}{Relative Test Error}                                    & \multicolumn{10}{c|}{Proportion of Variance Explained} \\
\midrule\multirow{9}[6]{*}{$n=10$} & \multirow{3}[2]{*}{hsnr} & $\rho=0$ & 1.189 & 1.241 & 1.577 & 1.939 & 1.756 & 1.756 & 1.763 & 2.068 & 1.857 & 2.095 & 0.88 & 0.875 & 0.841 & 0.806 & 0.824 & 0.824 & 0.823 & 0.793 & 0.814 & 0.789 \\ 
   &  & $\rho=0.5$ & 1.184 & 1.229 & 1.561 & 1.981 & 1.761 & 1.761 & 1.781 & 2.14 & 1.846 & 2.131 & 0.88 & 0.876 & 0.842 & 0.8 & 0.823 & 0.823 & 0.818 & 0.784 & 0.812 & 0.786 \\ 
   &  & $\rho=0.9$ & 1.185 & 1.218 & 1.53 & 1.927 & 1.695 & 1.695 & 1.701 & 2.116 & 1.813 & 2.145 & 0.879 & 0.876 & 0.844 & 0.807 & 0.83 & 0.83 & 0.826 & 0.787 & 0.816 & 0.786 \\ 
  \cmidrule{2-23} & \multirow{3}[2]{*}{msnr} & $\rho=0$ & 1.19 & 1.263 & 1.584 & 1.999 & 1.856 & 1.856 & 1.804 & 2.193 & 1.958 & 2.186 & 0.443 & 0.411 & 0.26 & 0.061 & 0.132 & 0.132 & 0.159 & -0.029 & 0.079 & -0.025 \\ 
   &  & $\rho=0.5$ & 1.19 & 1.256 & 1.616 & 2.008 & 1.854 & 1.854 & 1.823 & 2.234 & 1.937 & 2.211 & 0.442 & 0.413 & 0.243 & 0.06 & 0.128 & 0.128 & 0.141 & -0.046 & 0.093 & -0.034 \\ 
   &  & $\rho=0.9$ & 1.187 & 1.24 & 1.6 & 2.035 & 1.773 & 1.773 & 1.769 & 2.205 & 1.885 & 2.184 & 0.441 & 0.417 & 0.248 & 0.05 & 0.173 & 0.173 & 0.161 & -0.032 & 0.108 & -0.02 \\ 
  \cmidrule{2-23} & \multirow{3}[2]{*}{lsnr} & $\rho=0$ & 1.19 & 1.261 & 1.613 & 2.022 & 1.796 & 1.796 & 1.811 & 2.203 & 1.92 & 2.181 & 0.046 & -0.01 & -0.292 & -0.624 & -0.443 & -0.443 & -0.453 & -0.766 & -0.535 & -0.749 \\ 
   &  & $\rho=0.5$ & 1.194 & 1.242 & 1.591 & 2.019 & 1.874 & 1.874 & 1.874 & 2.189 & 1.949 & 2.189 & 0.042 & 0.004 & -0.275 & -0.62 & -0.509 & -0.509 & -0.506 & -0.761 & -0.567 & -0.761 \\ 
   &  & $\rho=0.9$ & 1.183 & 1.239 & 1.623 & 1.987 & 1.777 & 1.777 & 1.78 & 2.19 & 1.876 & 2.19 & 0.049 & 0.003 & -0.306 & -0.591 & -0.426 & -0.426 & -0.43 & -0.759 & -0.508 & -0.757 \\ 
  \midrule\multirow{9}[6]{*}{$n=100$} & \multirow{3}[2]{*}{hsnr} & $\rho=0$ & 1.029 & 1.03 & 1.031 & 1.032 & 1.032 & 1.032 & 1.031 & 1.033 & 1.032 & 1.015 & 0.892 & 0.892 & 0.892 & 0.891 & 0.891 & 0.891 & 0.892 & 0.891 & 0.892 & 0.893 \\ 
   &  & $\rho=0.5$ & 1.029 & 1.03 & 1.03 & 1.032 & 1.03 & 1.031 & 1.031 & 1.032 & 1.031 & 1.015 & 0.892 & 0.892 & 0.892 & 0.891 & 0.892 & 0.891 & 0.891 & 0.891 & 0.891 & 0.893 \\ 
   &  & $\rho=0.9$ & 1.029 & 1.03 & 1.031 & 1.032 & 1.032 & 1.032 & 1.031 & 1.032 & 1.031 & 1.016 & 0.892 & 0.892 & 0.892 & 0.891 & 0.891 & 0.891 & 0.891 & 0.891 & 0.891 & 0.893 \\ 
  \cmidrule{2-23} & \multirow{3}[2]{*}{msnr} & $\rho=0$ & 1.03 & 1.031 & 1.032 & 1.033 & 1.034 & 1.033 & 1.032 & 1.033 & 1.033 & 1.017 & 0.487 & 0.487 & 0.486 & 0.486 & 0.485 & 0.486 & 0.486 & 0.486 & 0.486 & 0.494 \\ 
   &  & $\rho=0.5$ & 1.031 & 1.032 & 1.033 & 1.034 & 1.033 & 1.033 & 1.033 & 1.034 & 1.033 & 1.017 & 0.487 & 0.486 & 0.486 & 0.485 & 0.485 & 0.485 & 0.485 & 0.485 & 0.485 & 0.493 \\ 
   &  & $\rho=0.9$ & 1.031 & 1.032 & 1.033 & 1.033 & 1.034 & 1.034 & 1.033 & 1.034 & 1.033 & 1.017 & 0.487 & 0.486 & 0.486 & 0.485 & 0.485 & 0.485 & 0.486 & 0.485 & 0.486 & 0.494 \\ 
  \cmidrule{2-23} & \multirow{3}[2]{*}{lsnr} & $\rho=0$ & 1.031 & 1.032 & 1.033 & 1.034 & 1.035 & 1.035 & 1.033 & 1.035 & 1.034 & 1.018 & 0.143 & 0.142 & 0.142 & 0.141 & 0.14 & 0.14 & 0.141 & 0.14 & 0.141 & 0.154 \\ 
   &  & $\rho=0.5$ & 1.031 & 1.032 & 1.033 & 1.034 & 1.033 & 1.034 & 1.033 & 1.034 & 1.033 & 1.017 & 0.143 & 0.142 & 0.141 & 0.141 & 0.141 & 0.141 & 0.141 & 0.14 & 0.141 & 0.155 \\ 
   &  & $\rho=0.9$ & 1.031 & 1.032 & 1.033 & 1.034 & 1.034 & 1.034 & 1.034 & 1.034 & 1.034 & 1.018 & 0.143 & 0.142 & 0.141 & 0.141 & 0.141 & 0.14 & 0.141 & 0.14 & 0.141 & 0.154 \\ 
  \midrule\multirow{9}[6]{*}{$n=500$} & \multirow{3}[2]{*}{hsnr} & $\rho=0$ & 1.005 & 1.006 & 1.006 & 1.006 & 1.006 & 1.006 & 1.006 & 1.006 & 1.006 & 1.002 & 0.894 & 0.894 & 0.894 & 0.894 & 0.894 & 0.894 & 0.894 & 0.894 & 0.894 & 0.895 \\ 
   &  & $\rho=0.5$ & 1.005 & 1.005 & 1.005 & 1.006 & 1.006 & 1.006 & 1.006 & 1.006 & 1.006 & 1.002 & 0.894 & 0.894 & 0.894 & 0.894 & 0.894 & 0.894 & 0.894 & 0.894 & 0.894 & 0.895 \\ 
   &  & $\rho=0.9$ & 1.005 & 1.006 & 1.006 & 1.006 & 1.006 & 1.006 & 1.006 & 1.006 & 1.006 & 1.002 & 0.894 & 0.894 & 0.894 & 0.894 & 0.894 & 0.894 & 0.894 & 0.894 & 0.894 & 0.895 \\ 
  \cmidrule{2-23} & \multirow{3}[2]{*}{msnr} & $\rho=0$ & 1.006 & 1.006 & 1.006 & 1.006 & 1.006 & 1.006 & 1.006 & 1.006 & 1.006 & 1.003 & 0.498 & 0.498 & 0.498 & 0.498 & 0.498 & 0.498 & 0.498 & 0.498 & 0.498 & 0.5 \\ 
   &  & $\rho=0.5$ & 1.006 & 1.006 & 1.006 & 1.006 & 1.006 & 1.006 & 1.006 & 1.006 & 1.006 & 1.002 & 0.498 & 0.498 & 0.498 & 0.498 & 0.498 & 0.498 & 0.498 & 0.498 & 0.498 & 0.5 \\ 
   &  & $\rho=0.9$ & 1.006 & 1.006 & 1.006 & 1.006 & 1.006 & 1.006 & 1.006 & 1.006 & 1.006 & 1.002 & 0.498 & 0.498 & 0.498 & 0.498 & 0.498 & 0.498 & 0.498 & 0.498 & 0.498 & 0.5 \\ 
  \cmidrule{2-23} & \multirow{3}[2]{*}{lsnr} & $\rho=0$ & 1.006 & 1.006 & 1.006 & 1.006 & 1.006 & 1.006 & 1.006 & 1.006 & 1.006 & 1.003 & 0.163 & 0.163 & 0.162 & 0.162 & 0.162 & 0.162 & 0.162 & 0.162 & 0.162 & 0.165 \\ 
   &  & $\rho=0.5$ & 1.006 & 1.006 & 1.006 & 1.006 & 1.006 & 1.006 & 1.006 & 1.006 & 1.006 & 1.003 & 0.163 & 0.162 & 0.162 & 0.162 & 0.162 & 0.162 & 0.162 & 0.162 & 0.162 & 0.165 \\ 
   &  & $\rho=0.9$ & 1.006 & 1.006 & 1.006 & 1.006 & 1.006 & 1.006 & 1.006 & 1.006 & 1.006 & 1.003 & 0.162 & 0.162 & 0.162 & 0.162 & 0.162 & 0.162 & 0.162 & 0.162 & 0.162 & 0.165 \\ 
   \midrule 
 \multicolumn{1}{|c}{} & \multicolumn{1}{c}{} &       & \multicolumn{10}{c||}{Kullback-Leibler Discrepancy}                                    & \multicolumn{10}{c|}{\% of Correct Restrictions Selected} \\
\midrule\multirow{9}[6]{*}{$n=10$} & \multirow{3}[2]{*}{hsnr} & $\rho=0$ & 1.874 & 2.157 & 5.253 & 8.972 & 6.709 & 6.709 & 7.38 & 9.734 & 8.136 & 9.947 & 93.6 & 89.8 & 74 & 62.2 & 66 & 66 & 68.5 & 53.9 & 64.6 & 55.7 \\ 
   &  & $\rho=0.5$ & 0.955 & 1.208 & 4.804 & 8.167 & 5.934 & 5.934 & 6.618 & 9.064 & 7.085 & 9.284 & 94.1 & 91.1 & 76 & 63.1 & 67.3 & 67.3 & 70.1 & 53.9 & 66.1 & 56.2 \\ 
   &  & $\rho=0.9$ & -0.578 & -0.415 & 3.273 & 6.573 & 4.137 & 4.137 & 4.709 & 7.617 & 5.495 & 7.852 & 94.1 & 91 & 75 & 62.5 & 67.5 & 67.5 & 69.9 & 53.1 & 65.7 & 55 \\ 
  \cmidrule{2-23} & \multirow{3}[2]{*}{msnr} & $\rho=0$ & 3.939 & 4.532 & 7.328 & 11.166 & 9.218 & 9.218 & 9.658 & 12.29 & 10.584 & 12.366 & 92.8 & 87.8 & 70.6 & 59.3 & 58.6 & 58.6 & 64 & 49.7 & 59.9 & 52.1 \\ 
   &  & $\rho=0.5$ & 3.104 & 3.63 & 6.835 & 10.065 & 8.352 & 8.352 & 8.863 & 11.369 & 9.504 & 11.55 & 93.1 & 87.9 & 70.6 & 57.8 & 57.4 & 57.4 & 64 & 48 & 59.7 & 51.2 \\ 
   &  & $\rho=0.9$ & 1.679 & 1.905 & 5.491 & 8.817 & 6.495 & 6.495 & 7.067 & 9.852 & 7.887 & 9.855 & 93.2 & 88.4 & 70.4 & 58.1 & 57.1 & 57.1 & 64.8 & 47.7 & 61.1 & 50.9 \\ 
  \cmidrule{2-23} & \multirow{3}[2]{*}{lsnr} & $\rho=0$ & 5.704 & 6.314 & 9.17 & 12.737 & 10.598 & 10.598 & 11.185 & 13.712 & 11.957 & 13.734 & 92.4 & 86.5 & 67.8 & 56.5 & 51.4 & 51.4 & 59.8 & 44.5 & 56.1 & 48 \\ 
   &  & $\rho=0.5$ & 4.743 & 5.228 & 8.103 & 11.748 & 9.944 & 9.944 & 10.591 & 12.921 & 11.102 & 13.079 & 91.8 & 86.8 & 68.7 & 55.6 & 48.8 & 48.8 & 59.2 & 44.4 & 54.6 & 47.1 \\ 
   &  & $\rho=0.9$ & 3.177 & 3.584 & 6.777 & 10.33 & 7.943 & 7.943 & 8.726 & 11.439 & 9.367 & 11.527 & 92.3 & 86.6 & 68.5 & 56.1 & 49.7 & 49.7 & 60.8 & 45.8 & 57 & 48.1 \\ 
  \midrule\multirow{9}[6]{*}{$n=100$} & \multirow{3}[2]{*}{hsnr} & $\rho=0$ & 1.247 & 1.248 & 1.249 & 1.25 & 1.25 & 1.25 & 1.25 & 1.251 & 1.25 & 1.23 & 92.4 & 91.8 & 90.8 & 89.3 & 87.7 & 89 & 90.1 & 88.8 & 89.9 & 98.8 \\ 
   &  & $\rho=0.5$ & 0.414 & 0.415 & 0.416 & 0.418 & 0.416 & 0.417 & 0.417 & 0.418 & 0.417 & 0.398 & 92.3 & 90.8 & 90.4 & 88.7 & 89.4 & 89.3 & 89.4 & 88.4 & 89.2 & 99 \\ 
   &  & $\rho=0.9$ & -1.114 & -1.113 & -1.111 & -1.11 & -1.111 & -1.11 & -1.111 & -1.11 & -1.111 & -1.13 & 91.9 & 91.4 & 90.2 & 89.5 & 88.6 & 89 & 89.8 & 89 & 89.8 & 98.7 \\ 
  \cmidrule{2-23} & \multirow{3}[2]{*}{msnr} & $\rho=0$ & 3.388 & 3.389 & 3.39 & 3.391 & 3.392 & 3.391 & 3.391 & 3.392 & 3.391 & 3.373 & 75.3 & 74.7 & 74 & 72.8 & 69.6 & 71.9 & 73.4 & 72.2 & 73.1 & 92.8 \\ 
   &  & $\rho=0.5$ & 2.556 & 2.558 & 2.559 & 2.56 & 2.559 & 2.559 & 2.559 & 2.56 & 2.559 & 2.54 & 76.4 & 74.5 & 74.2 & 73.2 & 72 & 72.3 & 73.5 & 72.7 & 73.4 & 92.6 \\ 
   &  & $\rho=0.9$ & 1.028 & 1.029 & 1.031 & 1.031 & 1.032 & 1.032 & 1.031 & 1.032 & 1.031 & 1.012 & 75.2 & 74.5 & 73.2 & 72.5 & 70.4 & 71.7 & 72.6 & 71.8 & 72.4 & 92.9 \\ 
  \cmidrule{2-23} & \multirow{3}[2]{*}{lsnr} & $\rho=0$ & 4.999 & 4.999 & 5.001 & 5.002 & 5.003 & 5.003 & 5.001 & 5.003 & 5.002 & 4.983 & 73.8 & 72.7 & 72 & 70.6 & 63.9 & 67.6 & 71 & 69.7 & 71 & 93.3 \\ 
   &  & $\rho=0.5$ & 4.166 & 4.168 & 4.168 & 4.169 & 4.169 & 4.169 & 4.169 & 4.17 & 4.169 & 4.15 & 73.3 & 71.7 & 71.3 & 70 & 65.1 & 68 & 70.3 & 69.2 & 70.1 & 93.7 \\ 
   &  & $\rho=0.9$ & 2.638 & 2.639 & 2.641 & 2.641 & 2.641 & 2.642 & 2.641 & 2.642 & 2.641 & 2.622 & 74.3 & 73.5 & 72.1 & 71.4 & 65.4 & 68.6 & 71.1 & 70.3 & 71 & 93.6 \\ 
  \midrule\multirow{9}[6]{*}{$n=500$} & \multirow{3}[2]{*}{hsnr} & $\rho=0$ & 1.215 & 1.215 & 1.215 & 1.215 & 1.215 & 1.215 & 1.215 & 1.215 & 1.215 & 1.211 & 91.7 & 91.4 & 91 & 90.9 & 90.4 & 91.1 & 90.9 & 90.8 & 90.9 & 100 \\ 
   &  & $\rho=0.5$ & 0.381 & 0.381 & 0.381 & 0.381 & 0.381 & 0.381 & 0.381 & 0.382 & 0.381 & 0.378 & 91.6 & 91.2 & 91 & 90.7 & 90 & 91.1 & 90.9 & 90.5 & 90.9 & 100 \\ 
   &  & $\rho=0.9$ & -1.146 & -1.146 & -1.146 & -1.146 & -1.146 & -1.146 & -1.146 & -1.146 & -1.146 & -1.149 & 91.6 & 91.2 & 91.1 & 90.9 & 89.6 & 90.7 & 90.9 & 90.9 & 90.9 & 100 \\ 
  \cmidrule{2-23} & \multirow{3}[2]{*}{msnr} & $\rho=0$ & 3.355 & 3.355 & 3.355 & 3.355 & 3.355 & 3.355 & 3.355 & 3.355 & 3.355 & 3.352 & 89.3 & 89 & 88.6 & 88.5 & 88.1 & 88.7 & 88.6 & 88.5 & 88.5 & 98.6 \\ 
   &  & $\rho=0.5$ & 2.522 & 2.522 & 2.522 & 2.522 & 2.522 & 2.522 & 2.522 & 2.522 & 2.522 & 2.518 & 90 & 89.6 & 89.4 & 89.2 & 88.4 & 89.6 & 89.4 & 89 & 89.4 & 99 \\ 
   &  & $\rho=0.9$ & 0.995 & 0.995 & 0.995 & 0.995 & 0.995 & 0.995 & 0.995 & 0.995 & 0.995 & 0.991 & 89.5 & 89.1 & 88.9 & 88.8 & 87.7 & 88.7 & 88.8 & 88.8 & 88.8 & 98.8 \\ 
  \cmidrule{2-23} & \multirow{3}[2]{*}{lsnr} & $\rho=0$ & 4.965 & 4.965 & 4.965 & 4.965 & 4.965 & 4.965 & 4.965 & 4.965 & 4.965 & 4.961 & 74.6 & 74.4 & 74 & 73.8 & 72.6 & 74.8 & 74 & 73.8 & 73.9 & 97.4 \\ 
   &  & $\rho=0.5$ & 4.131 & 4.131 & 4.131 & 4.131 & 4.132 & 4.132 & 4.131 & 4.131 & 4.131 & 4.128 & 76.2 & 75.8 & 75.7 & 75.6 & 74.3 & 74.8 & 75.7 & 75.4 & 75.6 & 97.4 \\ 
   &  & $\rho=0.9$ & 2.604 & 2.604 & 2.604 & 2.604 & 2.604 & 2.604 & 2.604 & 2.604 & 2.604 & 2.601 & 74.8 & 74.3 & 74.2 & 74.1 & 73.1 & 73.5 & 74 & 74 & 74 & 97.4 \\ 
   \bottomrule 
\end{tabular}
}
\end{table}

\fi
